\input macros
\begindescriptions
\begindesc
\bix^^{insertions//commands for}
\cts topinsert {\<vertical mode material> {\bt \\endinsert}}
\cts midinsert {\<vertical mode material> {\bt \\endinsert}}
\cts pageinsert {\<vertical mode material> {\bt \\endinsert}}
\explain
These commands produce different forms of insertions that
instruct
(or allow) \TeX\ to relocate the \<vertical mode material>:
\ulist
\li |\topinsert| attempts to put the material at the top of the current page.
If it won't fit there, |\topinsert|
will move the material to the next available top of page.
\li |\midinsert| attempts to put the material at the current position.
If it won't fit there, |\midinsert|
will move the material to the next available top of page.
\li |\pageinsert| puts the material by itself on the next page.
To avoid an underfull page, be sure to end the inserted material with
|\vfil| or fill out the excess space some other way.
% Knuth doesn't say this, but I tried an experiment that verified it.
% Nor does he say explicitly that an insertion does a \par.
\endulist
\noindent
The \<vertical mode material>
is said to be ``floating'' ^^{floating material} because \TeX\
can move it from one place to another.
Insertions are very useful for material such as figures and tables because
you can position such material where you want it without knowing where the
page breaks will fall.

Each of these commands implicitly ends the current paragraph, so 
you should use them only between paragraphs.
You should not use them within a box or within another insertion.  
If you have several insertions competing for
the same space, \TeX\ will retain their relative order.
\example
\pageinsert
% This text will appear on the following page, by itself.
This page is reserved for a picture of the Queen of Hearts
sharing a plate of oysters with the Walrus and
the Carpenter.
\endinsert
|
\endexample
\enddesc
\enddescriptions
\end