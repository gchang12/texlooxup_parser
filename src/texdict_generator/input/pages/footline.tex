\input macros
\begindescriptions
\begindesc
secondprinting{\vglue-.75\baselineskip\vskip0pt}
\cts headline {\param{token list}}
\cts footline {\param{token list}}
\explain
These parameters
contain, respectively, the current headline (header) and the current
footline (footer).
The \plainTeX\ output routine
places the headline at the top of each page and the footline
at the bottom of each page.
The default headline is empty and the default footline is a centered
page number.

The headline and footline should both
be as wide as |\hsize| (use |\hfil|, \xref{\hfil}, for this
if necessary).
You should always include a font-setting command in these lines, since
the current font is unpredictable when \TeX\ is calling the
output routine.  If you don't set the font explicitly,
you'll get whatever font \TeX\ was using when it broke the page.

You shouldn't try to use |\headline| or |\footline|
to produce multiline headers or footers.
Although \TeX\ won't complain, it will give you something that's very ugly.
See \xrefpg{bighead} for a method of creating multiline headers or
footers.
\example
\headline = {\tenrm My First Reader\hfil Page \folio}
|
\produces
\pageno = 10
\line{\tenrm \noindent My First Reader\hfil Page \folio}
\par ({\it at the top of page \folio}\/)
\nextexample
\footline = {\tenit\ifodd\pageno\hfil\folio
           \else\folio\hfil\fi}
% Produce the page number in ten-point italic at 
% the outside bottom corner of each page.
|
\endexample
\enddesc
\enddescriptions
\end