\input macros
\begindescriptions
\begindesc
\cts openout {\<number> {\bt =} \<filename>}
\explain
^^{output streams//opening}
This command tells \TeX\ to open the file named \<filename>
and make it available for writing  via the \minref{output stream}
designated by \<number>.
\<number> must be between $0$ and $15$.
Once you've opened a file and connected it to an output stream,
you can write to the file using the |\write| command
with the output stream's number.  

An |\openout| generates a whatsit that becomes part of a box.
The |\openout| does not take effect until \TeX\ ships out that box
to the \dvifile,
unless you've preceded the |\openout| with ^|\immediate|.

\TeX\ won't complain if you associate more than one output stream with the
same file, but you'll get garbage in the file if you try it!

You should allocate stream numbers for |\openout| using
|\newwrite| (\xref{\@newwrite}).
\example
\newwrite\auxfile  \openout\auxfile = addenda.aux
% \auxfile now denotes the number of this opening
% of addenda.aux.
|
\endexample
\enddesc
\enddescriptions
\end