\input macros
\begindescriptions
\begindesc
\cts edef {\<control sequence> \<parameter text> \rqbraces{\<replacement text>}}
\explain
This command defines a macro in the same general way as |\def|.
The difference is that \TeX\ expands the \<replacement text> 
of an |\edef| immediately (but still without executing anything).
Thus any definitions within the \<replacement text> are expanded, but 
assignments and commands that produce things such as boxes and glue
are left as is.  For example, an |\hbox| command within
the \<replacement text> of an |\edef| remains as a command and is not
turned into a box as \TeX\ is processing the definition.
It isn't always obvious what's expanded and what isn't, but you'll
find a complete list of expandable control sequences on
\knuth{pages~212--215}.

You can inhibit the expansion of a control sequence that would otherwise
be expanded by using |\no!-expand| (\xref\noexpand). ^^|\noexpand|
You can postpone the expansion of a control sequence by using
^|\expandafter| (\xref\expandafter).

The |\write|, |\message|, |\errmessage|, |\wlog|, and |\csname|
commands expand their
token lists using the same rules that |\edef| uses to expand its
replacement text.
^^|\write//expanded by {\tt\\edef} rules|
^^|\message//expanded by {\tt\\edef} rules|
^^|\errmessage//expanded by {\tt\\edef} rules|
^^|\wlog//expanded by {\tt\\edef} rules|
^^|\csname//expanded by {\tt\\edef} rules|
\example
\def\aa{xy} \count255 = 1
\edef\bb{w\ifnum \count255 > 0\aa\fi z}
% equivalent to \def\bb{wxyz}
\def\aa{} \count255 = 0 % leaves \bb unaffected
\bb
|
\produces
\def\aa{xy} \count255 = 1
\edef\bb{w\ifnum \count255 > 0\aa\fi z}
% equivalent to \def\bb{wxyz}
\def\aa{} \count255 = 0 % leaves \bb unaffected
\bb
\endexample
\enddesc
\enddescriptions
\end