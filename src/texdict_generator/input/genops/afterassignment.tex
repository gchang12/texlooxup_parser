\input macros
\begindescriptions
\begindesc
\cts afterassignment {\<token>}
\explain
When \TeX\ encounters this command it saves \<token> in a special
place.  After it next performs an \minref{assignment}, it inserts
\<token> into the input and expands it.  If you call |\afterassignment|
more than once before an assignment, only the last call has any effect.
One use of |\afterassignment|
is in writing \minref{macro}s for commands intended to be written
in the
form of assignments, as in the example below.

See \knuth{page~279} for a precise description
of the behavior of |\afterassignment|.
\example
\def\setme{\afterassignment\setmeA\count255}
\def\setmeA{$\number\count255\advance\count255 by 10
   +10=\number\count255$}
Some arithmetic: \setme = 27
% After expanding \setme, TeX sets \count255 to 27 and
% then calls \setmeA.
|
\produces
\def\setme{\afterassignment\setmeA\count255}
\def\setmeA{$\number\count255\advance\count255 by 10
+10=\number\count255$}
Some arithmetic: \setme = 27
% After expanding \setme, TeX sets \count255 to 27 and
% then calls \setmeA.
\endexample
\enddesc
\enddescriptions
\end