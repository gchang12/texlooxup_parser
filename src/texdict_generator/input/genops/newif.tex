\input macros
\begindescriptions
\begindesc
\ctspecial newif {{\bt \\if}\<test name>}\ctsxrdef{@newif}
\explain
This command names a trio of control sequences with names |\alpha!-true|,
|\alphafalse|,
and |\ifalpha|, where |alpha| is \<test name>.
You can use them to define your own tests by
creating a logical variable that records
true\slash false information:
\ulist\compact
\li |\alphatrue| sets the  logical variable |alpha| true.
\li |\alphafalse| sets the logical variable |alpha| false
\li |\ifalpha| is a conditional test that is true if the logical
variable |alpha| is true and false otherwise.
\endulist
The logical variable |alpha| doesn't really exist, but \TeX\ behaves as
though it did.  After |\newif\ifalpha|, the logical variable is initially
false.

|\newif| is an outer command, so you can't use it inside a macro
definition.
\example
\newif\iflong  \longtrue
\iflong Rabbits have long ears.
\else Rabbits don't have long ears.\fi
|
\produces
\newif\iflong
\longtrue
\iflong Rabbits have long ears.\else Rabbits don't have long ears.\fi
\endexample
\eix^^{conditional tests}
\enddesc
\enddescriptions
\end