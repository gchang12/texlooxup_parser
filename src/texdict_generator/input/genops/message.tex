\input macros
\begindescriptions
\begindesc
\bix^^{messages, sending}
\bix^^{error messages}
\cts message {\rqbraces{\<token list>}}
\cts errmessage {\rqbraces{\<token list>}}
\explain
These commands display the message given by \<token list> on your
terminal and also enter it into the log.  Any \minref{macro}s in the
message are expanded, but no commands are executed.  This is the same rule
that \TeX\ uses for |\edef| (\xref \edef).

For |\errmessage|, \TeX\ pauses 
in the same way that it does for one of its own error messages
and displays the |\errhelp| tokens if you ask for help.

You can generate multiline messages by using the ^|\newlinechar|
character (\xref \newlinechar).
\example
\message{Starting a new section.}
|
\endexample
\enddesc
\enddescriptions
\end