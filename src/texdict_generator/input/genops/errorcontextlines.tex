\input macros
\begindescriptions
\begindesc
\cts errorcontextlines {\param{number}}
\explain
This parameter determines the number of pairs of context lines,
not counting the top and bottom pairs, that \TeX\ prints when it
encounters an error.  By setting it to $0$ you can get rid of long
error messages.  
You can still force out the full context by typing something like:
\csdisplay
I\errorcontextlines=100\oops
|
in response to an error,
since the undefined control sequence |\oops| will cause another error.
\PlainTeX\ sets |\error!-context!-lines| to $5$.
\enddesc
\enddescriptions
\end