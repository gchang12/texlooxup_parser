\input macros
\begindescriptions
\begindesc
\ctspecial ifdim {\<dimen$_1$> \<relation> \<dimen$_2$>}\ctsxrdef{@ifdim}
\explain
^^{dimensions//comparing}
This command tests if \<dimen$_1$> and \<dimen$_2$>
satisfy \<relation>, which must be either `|<|', `|=|', or `|>|'.
The dimensions can be constants such as |1in|, dimension registers
such as |\dimen6|, or dimension parameters such as |\parindent|.
Before performing the test, \TeX\ expands tokens following the |\ifdim|
until it obtains a sequence of tokens having
the form \<dimen$_1$> \<relation> \<dimen$_2$>, followed by a token
that can't be part of \<dimen$_2$>.

\example
\dimen0 = 1000pt \ifdim \dimen0 > 3in true\else false\fi
|
\produces
\dimen0 = 1000pt \ifdim \dimen0 > 3in true\else false\fi
\endexample
\enddesc
\enddescriptions
\end