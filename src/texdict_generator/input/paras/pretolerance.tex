\input macros
\begindescriptions
\begindesc
\bix^^{line breaks//parameters affecting}
%
\cts pretolerance {\param{number}}
\cts tolerance {\param{number}}
\explain
These parameters determine the \minref{badness} that \TeX\ will tolerate
on each line when it is choosing line breaks
for a paragraph.
The badness is a measure of how far the interword spacing deviates from
the ideal.
|\pretolerance| specifies the  tolerable badness for
line breaks without hyphenation;
|\tolerance| specifies the tolerable badness for line breaks with
hyphenation.
The tolerable badness can be exceeded in either of two ways:
a line is too tight (the interword spaces are too
small) or it is too loose (the interword spaces are too big).

\ulist
\li If \TeX\ must set a line too loosely, it
complains about an ``underfull hbox''.
\li If \TeX\ must set a line too rightly, 
it lets the line run past the right margin and
complains about an ``overfull \minref{hbox}''.
\endulist

\noindent \TeX\ chooses line breaks in the following steps:
\olist
\li It attempts to choose line breaks without hyphenating.
If none of the
resulting lines have a badness exceeding |\pretolerance|, the
line breaks are acceptable and the paragraph can now be set.
\li Otherwise, it tries another set of line breaks, this
time allowing hyphenation.  If none of the resulting lines have a badness
exceeding |\tolerance|, the new set of line breaks is
acceptable and the paragraph can now be set.
\li Otherwise, it adds ^|\emergencystretch| (see below) to the stretch
of each line and tries again.
\li If none of these attempts have produced an acceptable
set of line breaks,
it sets the paragraph with one or more overfull hboxes
and complains about them.
\endolist

\PlainTeX\ sets |\tolerance| to $200$ and |\pretolerance| to $100$.
If you set |\tolerance| to $10000$, \TeX\
becomes infinitely tolerant and accepts any spacing, no matter how bad
(unless it encounters a word that won't fit on a line, even with
hyphenation).  Thus by changing |\tolerance| you can avoid
overfull and underfull hboxes, but at the cost of making the spacing worse.
By making |\pretolerance| larger you can get \TeX\ to avoid hyphenation
(and also run faster),
again at the cost of possibly worse spacing.
If you set |\pretolerance| to $-1$,
\TeX\ will not even try to set the paragraph without hyphenation.

The  ^|\hbadness| parameter (\xref \hbadness) determines the level of badness
that \TeX\ will tolerate before it complains, but |\hbadness| does not affect
the way that \TeX\ typesets your document.
The ^|\hfuzz| parameter (\xref \hfuzz) determines the amount that
an hbox can exceed its specified width before \TeX\ considers it to be
erroneous.
\enddesc
\enddescriptions
\end