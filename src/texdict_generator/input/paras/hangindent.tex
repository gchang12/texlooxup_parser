\input macros
\begindescriptions
\begindesc
\cts hangafter {\param{number}}
\cts hangindent {\param{dimen}}

\explain
These two \minref{parameter}s jointly
specify  ``^{hanging indentation}'' for a paragraph.
The hanging indentation indicates to \TeX\ that certain lines
of the paragraph should
be indented and the remaining lines should have their normal width.
^^{indentation}
|\hangafter| determines which lines
are indented, while |\hangindent| determines the amount of indentation
and whether it occurs on the left or on the right: 

\ulist
\li Let $n$ be the value of |\hangafter|.  If $n < 0$, 
the first $-n$ lines of the paragraph will be indented.
If $n\ge0$, all but the first $n$ lines of the paragraph will be
indented.

\li Let $x$ be the value of |\hangindent|.
If $x\ge0$, the lines will be indented
by $x$ on the left. If $x<0$ the lines will be indented by $-x$ on
the right.  
\endulist

When you specify hanging indentation, it applies 
only to the next paragraph (if you're in vertical mode) or to
the current paragraph (if you're in horizontal mode).
\TeX\ uses the values of |\hangafter| and |\hangindent| at the end of a
paragraph, when it breaks that paragraph into lines.\minrefs{line
break}
 
Unlike most of the other paragraph-shaping parameters,
|\hangafter| and |\hangindent| are reset to their default values
at the start of each paragraph, namely,
$1$ for |\hangafter| and $0$ for |\hangindent|.
If you want to typeset a sequence of paragraphs with hanging
indentation, use |\everypar| (\xref{\everypar}).
^^|\everypar//for hanging indentation|
If you specify |\hangafter| and |\hangindent| as well as ^|\parshape|,
\TeX\ ignores the |\hangafter| and |\hangindent|.

\example
\hangindent=6pc \hangafter=-2
This is an example of a paragraph with hanging indentation. 
In this case, the first two lines are indented on the left,
but after that we return to unindented text.
|
\produces
\hangindent=6pc \hangafter=-2
This is an example of a paragraph with hanging indentation. 
In this case, the first two lines are indented on the left,
but after that we return to unindented text.
\nextexample
\hangindent=-6pc \hangafter=1
This is another example of a paragraph with hanging
indentation.  Here, all lines after the first have been
indented on the right. The first line, on the other
hand, has been left unindented.
|
\produces
\hangindent=-6pc \hangafter=1
This is another example of a paragraph with hanging
indentation.  Here, all lines after the first have been
indented on the right. The first line, on the other
hand, has been left unindented.
\endexample
\enddesc
\enddescriptions
\end