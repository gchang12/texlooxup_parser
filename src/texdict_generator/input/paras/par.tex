\input macros
\begindescriptions
\begindesc
\bix^^{paragraphs//shaping}
\ctspecial par \ctsxrdef{@par}
\explain
This command ends a paragraph and puts \TeX\ into \minref{vertical mode},
ready to add more items to the page.  Since \TeX\ converts a blank line in
your input file into a |\par| \minref{token}, you don't ordinarily need to
type an explicit |\par| in order to end a paragraph.

An important point is that |\par| doesn't tell
\TeX\ to start a paragraph; it only tells \TeX\ to end a paragraph.
\TeX\ starts a paragraph when it is in ordinary vertical mode (which it
is after a |\par|) and encounters an inherently horizontal item such as
a letter.  As part of its ceremony for starting a paragraph, \TeX\
^^{paragraphs//starting}
inserts an amount of vertical space given by the parameter |\parskip|
(\xref{\parskip}) and indents the paragraph by a horizontal space given
by |\parindent| (\xref{\parindent}).

You can usually cancel any interparagraph space produced by a |\par| by giving
the command |\vskip -\lastskip|.  It can often
be helpful to do this when you're writing a \minref{macro} that is
supposed to work the same way whether or not it is preceded by a blank
line.

You can get \TeX\ to take some special action at the start of each paragraph
by placing the instructions in ^|\everypar| (\xref\everypar).

See \knuth{pages~283 and 286} for the precise effect of |\par|.

\example
\parindent = 2em
``Can you row?'' the Sheep asked, handing Alice a pair of
knitting-needles as she was speaking.\par ``Yes, a little%
---but not on land---and not with needles---'' Alice was
starting to say, when suddenly the needles turned into oars.
|
\produces
\parindent = 2em
``Can you row?'' the Sheep asked, handing Alice a pair of
knitting-needles as she was speaking.\par ``Yes, a little%
---but not on land---and not with needles---'' Alice was
starting to say, when suddenly the needles turned into oars.
\endexample
\enddesc
\enddescriptions
\end