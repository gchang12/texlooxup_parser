\input macros
\begindescriptions
\begindesc
\bix^^{fonts//hyphenation characters for}
\cts hyphenchar {\<font>\param{number}}
\explain
\TeX\ doesn't necessarily use the `-' character at hyphenation points.
Instead, it uses the |\hyphenchar| of the current font, which is usually
`-' but need not be.   If a font has a negative |\hyphenchar| value,
\TeX\ won't hyphenate words in that font.

Note that \<font> is a control sequence
that names a font, not a \<font\-name> that names font files.
Beware: 
an assignment to |\hyphenchar| is \emph{not} undone at the end
of a group.
If you want to change |\hyphenchar| locally, you'll need to
save and restore its original value explicitly.

\example
\hyphenchar\tenrm = `- 
   % Set hyphenation for tenrm font to `-'.
\hyphenchar\tentt = -1
   % Don't hyphenate words in font tentt.
|
\endexample
\enddesc
\enddescriptions
\end