\input macros
\begindescriptions
\begindesc
^^{accents}
\xrdef{accents}
%
\ctspecialx ' \ctsxrdef{@prime} {^{acute accent} as in \'e}
\ctspecialx . \ctsxrdef{@dot} {^{dot accent} as in \.n}
\ctspecialx = \ctsxrdef{@equal} {^{macron accent} as in \=r}
\ctspecialx ^ \ctsxrdef{@hat} {^{circumflex accent} as in \^o}
\ctspecialx ` \ctsxrdef{@lquote} {^{grave accent} as in \`e}
\ctspecialx " \ctsxrdef{@quote} {^{umlaut accent} as in \"o}
\ctspecialx ~ \ctsxrdef{@not} {^{tilde accent} as in \~a}
\ctsx c {^{cedilla accent} as in \c c}
\ctsx d {^{underdot accent} as in \d r}
\ctsx H {^{Hungarian umlaut accent} as in \H o}
\ctsx t {^{tie-after accent} as in \t uu}
\ctsx u {^{breve accent} as in \u r}
\ctsx v {^{check accent} as in \v o}
\explain
These commands produce accent marks in ordinary text.  You'll usually
need to leave a space after the ones denoted by a single letter
(see ``Spaces'', \xref{spaces}).

\example
Add a soup\c con of \'elan to my pin\~a colada.
|
\produces
Add a soup\c con of \'elan to my pin\~a colada.
\endexample

\margin{`see also' moved to end of group, replacing the one there.}
\enddesc
\enddescriptions
\end