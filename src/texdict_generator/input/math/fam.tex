\input macros
\begindescriptions
\begindesc
\cts fam {\param{number}}
\explain
When \TeX\ is in \minref{math mode}, it ordinarily typesets a character
using the font family ^^{class} given in its \minref{mathcode}.
^^{family//given by \b\tt\\fam\e}
However, when \TeX\ is in math mode and encounters a character whose
\minref{class} is $7$ (Variable), it typesets that character using
the font \minref{family} given by the value of |\fam|, provided that the
value of |\fam| is between $0$ and $15$.
If the value of |\fam| isn't in that range, \TeX\ uses the family in
the character's mathcode as in the ordinary case.
\TeX\ sets |\fam| to $-1$ whenever it enters math mode.
Outside of math mode, |\fam| has no effect.

By assigning a value to
|\fam| you can change the way that \TeX\ typesets ordinary
characters such as variables.    
For instance, by setting |\fam| to |\ttfam|, you cause \TeX\ to typeset
variables using a typewriter font.
\PlainTeX\ defines |\tt| as a \minref{macro} that, among other things,
sets |\fam| to |\ttfam|.
\example
\def\bf{\fam\bffam\tenbf} % As in plain TeX.
|
\endexample
\enddesc
\enddescriptions
\end