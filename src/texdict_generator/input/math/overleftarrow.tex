\input macros
\begindescriptions
\begindesc
\cts underbrace {\<argument>}
\cts overbrace {\<argument>}
\cts underline {\<argument>}
\cts overline {\<argument>}
\cts overleftarrow {\<argument>}
\cts overrightarrow {\<argument>}
\explain
These commands place extensible ^{braces}, lines, or ^{arrows}
over or under the subformula given by \<argument>.
\TeX\ will make these constructs as wide as they need to be for
the context.
When \TeX\ produces the extended braces, lines, or arrows, it considers
only the dimensions of the \minref{box} containing \<argument>.
If you use more than one of these commands in a single formula, the
braces, lines, or arrows they produce
may not line up properly with each other.
You can use the |\mathstrut| command (\xref \mathstrut)
to overcome this difficulty.
\example
$$\displaylines{
\underbrace{x \circ y}\qquad \overbrace{x \circ y}\qquad
\underline{x \circ y}\qquad \overline{x \circ y}\qquad
\overleftarrow{x \circ y}\qquad
\overrightarrow{x \circ y}\cr
{\overline r + \overline t}\qquad
{\overline {r \mathstrut} + \overline {t \mathstrut}}\cr
}$$
|
\dproduces
$$\displaylines{
\underbrace{x \circ y}\qquad \overbrace{x \circ y}\qquad
\underline{x \circ y}\qquad \overline{x \circ y}\qquad
\overleftarrow{x \circ y}\qquad
\overrightarrow{x \circ y}\cr
{\overline r + \overline t}\qquad
{\overline {r \mathstrut} + \overline {t \mathstrut}}\cr
}$$
\endexample
\enddesc
\enddescriptions
\end