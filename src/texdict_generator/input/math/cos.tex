\input macros
\begindescriptions
\begindesc
\xrdef{namedfns}
\bix^^{functions, names of}
{\symbolspace = 36pt
\threecolumns 32
\easy\ctsdisplay cos {}
\ctsdisplay sin {}
\ctsdisplay tan {}
\ctsdisplay cot {}
\ctsdisplay csc {}
\ctsdisplay sec {}
\ctsdisplay arccos {}
\ctsdisplay arcsin {}
\ctsdisplay arctan {}
\ctsdisplay cosh {}
\ctsdisplay coth {}
\ctsdisplay sinh {}
\ctsdisplay tanh {}
\ctsdisplay det {}
\ctsdisplay dim {}
\ctsdisplay exp {}
\ctsdisplay ln {}
\ctsdisplay log {}
\ctsdisplay lg {}
\ctsdisplay arg {}
\ctsdisplay deg {}
\ctsdisplay gcd {}
\ctsdisplay hom {}
\ctsdisplay ker {}
\ctsdisplay inf {}
\ctsdisplay sup {}
\ctsdisplay lim {}
\ctsdisplay liminf {}
\ctsdisplay limsup {}
\ctsdisplay max {}
\ctsdisplay min {}
\ctsdisplay Pr {}
\egroup}
\explain
These commands set the names of various mathematical functions
in roman type, as is customary.
If you apply a superscript or subscript to one of these commands,
\TeX\ will in most cases typeset it in the usual place.
In display style, \TeX\ typesets superscripts and subscripts 
on |\det|, |\gcd|, |\inf|, |\lim|, |\liminf|,
|\limsup|, |\max|, |\min|, |\Pr|, and |\sup|
as though they were limits,
i.e., directly above or directly below the function name.

\example
$\cos^2 x + \sin^2 x = 1\qquad\max_{a \in A} g(a) = 1$
|
\produces
$\cos^2 x + \sin^2 x = 1\qquad\max_{a \in A} g(a) = 1$
\endexample
\enddesc
\enddescriptions
\end