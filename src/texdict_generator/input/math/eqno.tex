\input macros
\begindescriptions
\begindesc
\easy\cts eqno {}
\cts leqno {}
\explain
These commands attach an equation number to a displayed formula.
|\eqno| puts the equation number on the right and |\leqno| puts it on
the left.
The commands must be given at the end of the formula.
If you have a multiline display and you want to number more than one
of the lines, use the |\eq!-alignno| or |\leq!-alignno| command
(\xref \eqalignno).

These commands are valid only in display math mode.

\example
$$e^{i\theta} = \cos \theta + i \sin \theta\eqno{(11)}$$
|
\produces
$$e^{i\theta} = \cos \theta + i \sin \theta\eqno{(11)}$$
\endexample
\example
$$\cos^2 \theta + \sin^2 \theta = 1\leqno{(12)}$$
|
\produces
\abovedisplayskip = -\baselineskip
$$\cos^2 \theta + \sin^2 \theta = 1\leqno{(12)}$$
\endexample
\enddesc
\enddescriptions
\end