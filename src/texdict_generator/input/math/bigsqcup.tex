\input macros
\begindescriptions
\begindesc

\bix^^{operators//large}
\threecolumns 15
\easy\ctsdoubledisplay bigcap {}
\ctsdoubledisplay bigcup {}
\ctsdoubledisplay bigodot {}
\ctsdoubledisplay bigoplus {}
\ctsdoubledisplay bigotimes {}
\ctsdoubledisplay bigsqcup {}
\ctsdoubledisplay biguplus {}
\ctsdoubledisplay bigvee {}
\ctsdoubledisplay bigwedge {}
\ctsdoubledisplay coprod {}
{\symbolspace = 42pt\basicdisplay {\hskip 26pt$\smallint$}%
   {\\smallint}\ddstrut}%
   \xrdef{\smallint} \pix\ctsidxref{smallint}
\ctsdoubledisplay int {}
\ctsdoubledisplay oint {}
\ctsdoubledisplay prod {}
\ctsdoubledisplay sum {}
}
\explain
These commands produce various large operator symbols.  
\TeX\ produces the smaller size when it's in ^{text style}
\minrefs{math mode} and the larger size when it's in ^{display style}.
Operators are one of \TeX's \minref{class}es of math symbols.
\TeX\ puts different amounts of space
around different classes of math symbols.

The large operator symbols with `|big|' in their names are different
from the corresponding binary operations (see \xref{binops}) such as
|\cap| ($\cap$) since they usually appear at the beginning
of a formula.  \TeX\ uses different spacing for a large operator
than it does for a binary operation.

Don't confuse `$\sum$' (|\sum|) with `$\Sigma$'^^|\Sigma| (|\Sigma|)
or confuse `$\prod$' (|\prod|) with `$\Pi$' ^^|\Pi| (|\Pi|).
|\Sigma| and |\Pi| produce capital Greek letters, which are smaller and
have a different appearance.

A large operator can have ^{limits}.  The lower limit is specified as a
subscript and the upper limit as a superscript.

\example
$$\bigcap_{k=1}^r (a_k \cup b_k)$$
|
\dproduces
$$\bigcap_{k=1}^r (a_k \cup b_k)$$
\endexample
\interexampleskip
\example
$${\int_0^\pi \sin^2 ax\,dx} = {\pi \over 2}$$
|
\dproduces
$${\int_0^\pi \sin^2 ax\,dx} = {\pi \over 2}$$
\endexample
\enddesc
\enddescriptions
\end