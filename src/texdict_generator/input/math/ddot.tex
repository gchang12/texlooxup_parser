\input macros
\begindescriptions
\begindesc
\xrdef{mathaccent}
^^{accents}
^^{math//accents}
%
\easy\ctsx acute {^{acute accent} as in $\acute x$}
\ctsx b {^{bar-under accent} as in $\b x$}
\ctsx bar {^{bar accent} as in $\bar x$}
\ctsx breve {^{breve accent} as in $\breve x$}
\ctsx check {^{check accent} as in $\check x$}
\ctsx ddot {^{double dot accent} as in $\ddot x$}
\ctsx dot {^{dot accent} as in $\dot x$}
\ctsx grave {^{grave accent} as in $\grave x$}
\ctsx hat {^{hat accent} as in $\hat x$}
\ctsx widehat {^{wide hat accent} as in $\widehat {x+y}$}
\ctsx tilde {^{tilde accent} as in $\tilde x$}
\ctsx widetilde {^{wide tilde accent} as in $\widetilde {z+a}$}
\ctsx vec {^{vector accent} as in $\vec x$}
\explain
These commands produce accent marks in math formulas.  You'll ordinarily
need to leave a space after any one of them.
A wide accent can be applied to a multicharacter subformula;
\TeX\ will center the accent over the subformula.
The other accents are usefully applied only to a single character.

\example
$\dot t^n \qquad \widetilde{v_1 + v_2}$
|
\produces
$\dot t^n \qquad \widetilde{v_1 + v_2}$
\endexample

\enddesc
\enddescriptions
\end