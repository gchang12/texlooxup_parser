\input macros
\beginconcepts
\concept alignment

\bix^^{tables}
An \defterm{alignment} is a construct for aligning material, such as a
table, in columns or rows.  To form an alignment you need to
(a)~describe the layout of the columns or rows and (b)~tell \TeX\ what
material goes into the columns or rows.  A tabbing alignment or a
horizontal alignment is organized as a sequence of rows; a vertical
alignment is organized as a sequence of columns.  We first describe
tabbing and horizontal alignments and then more briefly describe
vertical alignments.

Tabbing alignments are defined by \plainTeX.  They are simpler but less
flexible than horizontal alignments.  Tabbing and horizontal alignments
differ principally in how you describe their layouts.

\bix^^|\settabs|
\bix\ctsidxref{+}
\bix\ctsidxref{cr}

To construct a tabbing alignment you first issue a |\settabs| command
\ctsref{\settabs} that specifies how \TeX\ should divide the available
horizontal space into columns.  Then you provide a sequence of rows for
the table.  Each row consists of a |\+| control sequence \ctsref{\@plus}
followed by a list of ``entries'', i.e., row\slash column intersections.
^^{entry (column or row)}
Adjacent entries in a row are separated by an ampersand (|&|).
\xrdef{@and}
\ttidxref{&}
The end of a row is indicated by ^|\cr| after its
last entry. 
If a row has fewer entries than there are columns in the alignment,
\TeX\ effectively fills out the row with blank entries.

As long as it's preceded by a |\settabs| command, you can put a row of a
tabbing alignment anywhere in your document.  In particular, you can put
other things between the rows of a tabbing alignment or describe several
tabbing alignments with a single |\settabs|.  Here's an example of a
tabbing alignment:

\xrdef{tabbedexample}\csdisplay
{\hsize = 1.7 in \settabs 2 \columns
\+cattle&herd\cr
\+fish&school\cr
\+lions&pride\cr}
|
The |\settabs 2 \columns| command in this example \ctsref{\settabs}
tells \TeX\ to produce two equally wide columns.
The line length is $1.7$ inches.
The typeset alignment looks like this:

{\def\+{\tabalign}% so it isn't \outer.
\vdisplay{%
\hsize 1.7 in \settabs 2 \columns
\+cattle&herd\cr
\+fish&school\cr
\+lions&pride\cr}
}%

\margin{Missing explanation added here.}
There's another form of tabbing alignment in which you specify the column
widths with a template.  The column widths in the template
determine the column widths in the rest of the alignment:
\csdisplay
{\settabs\+cattle\quad&school\cr
\+cattle&herd\cr
\+fish&school\cr
\+lions&pride\cr}
|
Here's the result:
{\def\+{\tabalign}% so it isn't \outer.
\vdisplay{%
\settabs\+cattle\quad&school\cr
\+cattle&herd\cr
\+fish&school\cr
\+lions&pride\cr}
}%

\eix^^|\settabs|
\eix\ctsidxref{+}
\bix^^|\halign|
Horizontal alignments are constructed with |\halign| \ctsref\halign.
\TeX\ adjusts the column widths of a horizontal alignment according to
what is in the columns.  When \TeX\ encounters the |\halign| command
that begins a horizontal alignment, it first examines all the rows of
the alignment to see how wide the entries are.  It then sets each column
width to accommodate the widest entry in that column.

A horizontal alignment governed by |\halign| consists of a
``\pix^{preamble}'' that indicates the row layout followed by the rows
themselves.
\ulist
\li The preamble consists of a sequence of \pix^{template}s, one for each
column.  The template for a column specifies how the text for that
column should be typeset.  Each template must include a single |#|
character
\ttidxref{#}\xrdef{@asharp}
to indicate where \TeX\ should substitute the text of an entry into the
template.  The templates are separated by ampersands (|&|), \ttidxref{&}
and the end of the preamble is indicated by |\cr|.  By providing an
appropriate template you can obtain effects such as centering a column,
left or right justifying a column, or setting a column in a particular
\refterm{font}.

\li The rows have the same form as in a tabbing alignment, except that
you omit the |\+| at the beginning of each row.
As before, entries are separated by |&| and the end of the row
is indicated by |\cr|.
\TeX\ treats each entry as a 
\refterm{group}, so any
font-setting command or other \refterm{assignment}
in a column template is in effect only for the entries in that column.
\endulist
\noindent The preamble and the rows must all be enclosed in the braces
that follow |\halign|.  Each |\halign| alignment must include
its own preamble.

For example, the horizontal alignment:
\csdisplay
\tabskip=2pc
\halign{\hfil#\hfil &\hfil#\hfil &\hfil#\hfil \cr
  &&\it Table\cr
\noalign{\kern -2pt}
  \it Creature&\it Victual&\it Position\cr
\noalign{\kern 2pt}
  Alice&crumpet&left\cr
  Dormouse&muffin&middle\cr
  Hatter&tea&right\cr}
|

\noindent produces the result:

\xrdef{halignexample}
\vdisplay{%
\tabskip=2pc \halign{\hfil#\hfil &\hfil#\hfil &\hfil#\hfil \cr
  &&\it Table\cr
\noalign{\kern -2pt}
  \it Creature&\it Victual&\it Position\cr
\noalign{\kern 2pt}
  Alice&crumpet&left\cr
  Dormouse&muffin&middle\cr
  Hatter&tea&right\cr}
}%
\noindent The ^|\tabskip| \ctsref{\tabskip} in this example 
tells \TeX\ to insert |2pc| of 
\refterm{glue} between the columns.
The ^|\noalign| \ctsref{\noalign} commands tell \TeX\ to insert
\refterm{vertical mode} material between two rows.
In this example we've
used |\noalign| to produce some extra space between the title rows and
the data rows, and also to bring ``Table'' and ``Position'' closer together.
(You can also use |\noalign| before the first row or after the
last row.)
\eix^^|\halign|

You can construct a vertical alignment with the ^|\valign| command
\ctsref{\valign}.  A vertical alignment is organized as a series of
columns rather than as a series of rows.  A vertical alignment follows
the same rules as a horizontal alignment except that the roles of rows
and columns are interchanged.  For example, the vertical alignment:

\csdisplay
{\hsize=0.6in \parindent=0pt
\valign{#\strut&#\strut&#\strut\cr
  one&two&three\cr
  four&five&six\cr
  seven&eight&nine\cr
  ten&eleven\cr}}
|
\noindent yields:
\vdisplay{%
{\hsize=0.6in \parindent=\listleftindent % Because lists and displays
                                         % are not indented just by \parindent.
\valign{#\strut&#\strut&#\strut\cr
  one&two&three\cr
  four&five&six\cr
  seven&eight&nine\cr
  ten&eleven\cr}}
}
The ^|\strut| commands \ctsref{\strut} 
in the template are necessary to get the entries in each row
to line up properly, i.e., to have a common \refterm{baseline},
and to keep the distance between baselines uniform.
\eix\ctsidxref{cr}
\eix^^{tables}

\endconcept


\endconcepts
\end