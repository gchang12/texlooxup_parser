\input macros
\beginconcepts
\concept family

A \defterm{family} is a group of three related \refterm{fonts:font} used
when \TeX\ is in \refterm{math mode}.
^^{fonts//families of}
Outside of math mode, families
have no effect.  The three fonts in a family are used for normal symbols
(\refterm{text size}), subscripts and superscripts (\refterm{script
size}), and sub-subscripts, super-superscripts, etc.\
(\refterm{scriptscript size}).
^^{text size}
^^{script size}
^^{scriptscript size}
For example, the numeral `|2|' set in
these three fonts would give you `$2$', `$\scriptstyle 2$', and
`$\scriptscriptstyle 2$' (in \plainTeX).
Ordinarily you would set up the
three fonts in a family as different point sizes of the same typeface,
but nothing prevents you from using different typefaces for the three
fonts as well or using the same font twice in a family.

{\tighten
\TeX\ provides for up to sixteen families, numbered $0$--$15$.  For
example, family $0$ in \refterm{\plainTeX} consists of $10$-point roman
for text, $7$-point roman for script, and $5$-point roman for
scriptscript.
^^{\plainTeX//font families in}
\PlainTeX\ also defines family $1$ to consist of math
italic fonts and reserves families $2$ and $3$ for ^{special symbols} and
^{math extensions} respectively.\footnote{Families $2$ and $3$ are special
in that their font metric files must include parameters for math
spacing.} If you need to define a family for yourself, you should use
the ^|\newfam| command \ctsref{\@newfam} to get the number of a family that
isn't in use, and the ^|\textfont|, ^|\scriptfont|,
and ^|\scriptscriptfont| commands \ctsref{\scriptscriptfont}
to assign fonts to that family.
\par}

\endconcept



\endconcepts
\end