\input macros
\beginconcepts
\concept {hyphenation}

\TeX\ automatically hyphenates words as it is processing your document.
\TeX\ is not eager to insert hyphens, preferring instead to find good
line breaks by adjusting the spacing between words and moving words
from one line to another.
\TeX\ is clever enough to understand
hyphens that are already in words.

You can control \TeX's hyphenation in several ways:
\ulist
\li You can tell \TeX\ to
allow a hyphen in a particular place by inserting a
discretionary hyphen
^^{discretionary hyphens}
with the ^|\-| command \ctsref{\@minus}.
\li You can tell \TeX\ how to
hyphenate particular words throughout your document with the ^|\hyphenation|
command \ctsref{\hyphenation}.  
\li You can enclose a word in an \refterm{hbox}, thus preventing \TeX\
from hyphenating it.
\li You can set the value of penalties such as |\hyphenpenalty|
\ctsref\hyphenpenalty.
\endulist
\noindent If a word contains an explicit or discretionary hyphen,
\TeX\ will never break it elsewhere.
\endconcept



\endconcepts
\end