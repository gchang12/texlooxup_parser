\input macros
\beginconcepts
\concept {horizontal mode}

^^{hboxes//horizontal mode for}
When \TeX\ is assembling a paragraph or an \refterm{hbox}, it is in one
of two \defterm{horizontal modes}: ^{ordinary horizontal
mode} for assembling paragraphs and ^{restricted horizontal mode} for
assembling hboxes.  Whenever \TeX\ is in a horizontal mode its stomach
\seeconcept{\anatomy} is constructing a \refterm{horizontal
list} of items (boxes, glue, penalties, etc.).
\TeX\ typesets the items in the list
one after another, left to right.

A horizontal list can't contain any
items produced by inherently vertical commands, e.g., |\vskip|.
^^{horizontal lists//can't contain vertical commands}

\ulist
\li If \TeX\ is  assembling a horizontal list in ordinary horizontal mode and
encounters an inherently vertical command, \TeX\ ends the paragraph and
enters \refterm{vertical mode}.

\li If \TeX\ is assembling a horizontal list in restricted horizontal
mode and encounters an inherently vertical command, it complains.
\endulist Two commands that you might at first think are inherently
horizontal are in fact inherently vertical: |\halign| \ctsref{\halign}
and |\hrule| \ctsref{\hrule}.
^^|\hrule//inherently vertical|
^^|\halign//inherently vertical|
See \knuth{page~286} for a list 
of the inherently vertical commands.

{\tighten
You should be aware of a subtle but important property of restricted
horizontal mode: \emph{you can't enter ordinary horizontal mode
when you're in restricted horizontal mode}.  What this means in practice is that
when \TeX\ is assembling an hbox it
won't handle paragraph-like text, i.e., text for which it does
\refterm{line breaking}.  You can get
around this restriction by enclosing the paragraph-like text in a
\refterm{vbox} within the hbox.  The same method works if you want to
put, say, a horizontal \refterm{alignment} inside an~hbox.
}% end scope of tighten

\endconcept


\endconcepts
\end