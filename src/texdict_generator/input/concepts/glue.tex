\input macros
\beginconcepts
\concept glue

\bix^^{stretch}
\bix^^{shrink}
\defterm{Glue} is blank space that can stretch or shrink. 
Glue gives \TeX\ the flexibility that it needs in order to produce
handsome
documents.  Glue comes in two flavors: horizontal glue and vertical glue.  
Horizontal glue occurs within \refterm{horizontal lists:horizontal list},
while vertical glue occurs within \refterm{vertical lists:vertical list}.
^^{horizontal lists}
^^{vertical lists}
You
can produce a glue item either implicitly, e.g., with an interword space, or
explicitly, e.g., with the ^|\hskip| command.
^^{spaces//interword}
\TeX\ itself produces many glue
items as it typesets your document.
We'll just describe horizontal glue---vertical glue is analogous.

When \TeX\ assembles a list of boxes and glue into a larger
unit,
^^{boxes//glue with}
it adjusts the size of the glue to meet the space requirements of the
larger unit.  For instance, \TeX\ ensures that the ^{right margin} of a page
is uniform by adjusting the horizontal glue within lines.
Similarly, it ensures that different pages have the
same ^{bottom margin}
by adjusting the glue between blocks of text such as paragraphs and
math displays
(where the change is least likely to be conspicuous).

A glue item has a natural space---the size it ``wants to be''.  Glue
also has two other attributes: its \refterm{stretch} and its
\refterm{shrink}.  You can produce a specific amount of horizontal glue
with the ^|\hskip| \refterm{command} \ctsref{\hskip}.  The command 
|\hskip 6pt plus 2pt minus 3pt|
produces a horizontal glue item whose natural
size is $6$ points, whose stretch is $2$ points, and whose shrink is
$3$ points.  Similarly, you can produce a specific amount of vertical
glue with the ^|\vskip| command \ctsref{\vskip}.

The best way to understand what stretch and shrink are about
is to see an example of glue at work.
Suppose you're constructing an \refterm{hbox} from three boxes and two glue
items, as in this picture:
\gluepicture 
   29 {\picbox 4 \gluebox 6 4 1 6 \picbox 5 \gluebox 10 8 3 10 \picbox 4 }
\noindent
The units of measurement here could be points, millimeters, or anything else.
If the hbox is allowed to assume its natural width, then each glue item in the
box also assumes its natural width.  The total width of the hbox is then the
sum of the widths of its parts, namely, $29$ units.

Next, suppose that the hbox is required to be wider than $29$ units, say
$35$ units.  This
could happen, for example, if the hbox is required to occupy an entire
line and the line width is $35$ units.
Since the boxes can't change their width, 
\TeX\ produces the necessary extra space by making the glue items wider.
The picture now looks like this:
\gluepicture 
   35 {\picbox 4 \gluebox 6 4 2 8 \picbox 5 \gluebox 10 8 6 14 \picbox 4 }
The glue items don't become wider equally; they became wider in proportion to
their stretch.  Since the second glue item 
has twice as much stretch as the first one, 
it gets wider by four units while the first glue item gets wider by only 
two units.
Glue can be stretched as far as necessary, although \TeX\ is
somewhat reluctant to
stretch it beyond the amount of stretch given in its definition.

Finally, suppose that the hbox is required to be narrower than $29$ units, say
$25$ units.  In this case \TeX\ makes the glue items narrower.
The picture looks like this:
\gluepicture 
   25 {\picbox 4 \gluebox 6 4 2 5 \picbox 5 \gluebox 10 8 6 7 \picbox 4 }
The glue items become narrower in proportion to their shrink.
The first glue item becomes narrower by one unit, while the second glue item
becomes narrower by three units.  Glue cannot shrink by a distance
less than the amount of shrink
given in its definition even though the distance it can stretch is
unlimited.  In this important sense the shrink and 
the stretch behave differently.

A good rule of thumb for glue is to set the natural size to the amount
of space that looks best, the stretch to the largest amount of space
that \TeX\ can add before the document starts to look bad, and the
shrink to the largest amount of space that \TeX\ can take away before
the document starts to look bad.  You may need to set the values by
experiment.

You can produce glue that is infinitely stretchable  by specifying
its stretch in units of `^|fil|', `^|fill|', or `^|filll|'.   Glue measured in
`|fill|' is infinitely more stretchable than glue measured in `|fil|', and
glue measured in `|filll|'  is infinitely more stretchable than glue measured
in `|fill|'.  You should rarely have any need for `|filll|' glue.  Glue whose
stretch is |2fil| has twice as much stretch as glue whose stretch is |1fil|,
and similarly for the other kinds of infinitely stretchable glue.

When \TeX\ is
apportioning extra space among glue items, the infinitely stretchable
ones, if there
are any, get all of it.  Infinitely stretchable glue is particularly useful for
setting text flush left, flush right, or centered:

\ulist\compact
\li To set text ^{flush left}, put infinitely stretchable
horizontal glue to the right of it.
That glue will consume all the 
extra space that's available on the line.
You can use the ^|\leftline| command \ctsref{\leftline} 
or the |\raggedright| command \ctsref{\raggedright} to do~this.
\li To set text ^{flush right}, put infinitely
stretchable horizontal glue to the left of it.
As before, that glue will consume all the extra space on the line.
You can use the ^|\rightline| command \ctsref{\rightline} to do~this.
\li To set ^{centered text}, put identical infinitely stretchable
horizontal glue items
on both sides of it.
These two glue items will divide all the extra space on the line
equally between them.
You can use the ^|\centerline| command \ctsref{\centerline} to do~this.
\endulist

You can also specify infinitely shrinkable glue
^^{glue//infinitely shrinkable}
in a similar way.  Infinitely shrinkable glue can act as negative space.
Note that |fil|, etc., can be used only 
to specify the stretch and shrink of glue---they can't be used to specify
its natural size.
\eix^^{stretch}
\eix^^{shrink}
\endconcept



\endconcepts
\end