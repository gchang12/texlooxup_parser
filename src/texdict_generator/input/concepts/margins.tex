\input macros
\beginconcepts
\concept margins

The \refterm{margins}
of a page define a rectangle that normally
contains the printed matter on the page.
You can get \TeX\ to print material outside of this rectangle,
but only by taking some explicit action that moves the material there.
\TeX\ considers headers and footers to lie outside the margins.

The rectangle is defined in terms of its upper-left corner, its width, and
its depth.  The location of the upper-left corner is defined by
the ^|\hoffset|
and ^|\voffset| parameters
\ctsref\voffset.  The default is to place that corner one inch from the top
and one inch from the left side of the page, corresponding to a value of
zero for both |\hoffset| and |\voffset|.%
\footnote{This seems to us to be an odd convention.
It would have been more natural to have the $(0,0)$
point for |\hoffset| and |\voffset| be at the upper-left corner of the
paper and to have set their default values to one inch.}
The width of the rectangle is given by ^|\hsize| and the depth by ^|\vsize|.

The implications of these conventions are:
\ulist\compact
\li The left margin is given by |\hoffset|\tplus|1in|.
\li The right margin is given by the width of the paper minus 
    |\hoffset|\tplus|1in|\tplus|\hsize|.
\li The top margin is given by |\voffset|\tplus|1in|.
\li The bottom margin is given by the length of the paper minus
    |\voff!-set|\tplus|1in|\tplus|\vsize|.
\endulist
From this information you can see what parameters you need to
change in order to change the margins.

Any changes that you make to |\hoffset|, |\voffset|, or |\vsize| become
effective the next time \TeX\ starts a page.  In other words, if you change
them within a page, the change will affect only the \emph{following} pages.
If you change |\hsize|, the change will become effective immediately.
\endconcept



\endconcepts
\end