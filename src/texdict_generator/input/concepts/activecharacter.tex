\input macros
\beginconcepts
\concept {active character}

An \defterm{active character} is a \refterm{character}
that has a definition, e.g., a macro definition, associated with it.  
^^{macros//named by active characters}
You can think of an active character as a special kind of control sequence.
When \TeX\ encounters an active character, it
executes the definition associated with the character.
If \TeX\ encounters an active character that does not have
an associated definition, it will complain about an
undefined control sequence.

An active character has a \refterm{category code} of $13$ (the value
of ^|\active|).
To define an active character, you should first
use the ^|\catcode| command
\ctsref{\catcode} to make it active
and then provide the definition of the character, using
a command such as |\def|, |\let|, or |\chardef|.
The definition of an active character has the same form as
the definition of a \refterm{control sequence}.
^^{category codes//of active characters}
If you try to define the macro for an active character
before you make the character active, \TeX\ will complain about a 
missing control sequence.  

For example, the tilde character (|~|) is defined as an active character
in \plainTeX.  It
produces a space between two words but links those words so that 
\TeX\ will not turn the space into a \refterm{line break}.
\refterm{\PlainTeX:\plainTeX} defines `|~|' by the commands:

\csdisplay
\catcode `~ = \active \def~{\penalty10000\!visiblespace}
|
(The |\penalty| inhibits a line break and the `|\!visiblespace|'
inserts a space.)
\endconcept



\endconcepts
\end