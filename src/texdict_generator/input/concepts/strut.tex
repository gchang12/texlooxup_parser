\input macros
\beginconcepts
\concept strut

{\tighten
A \defterm{strut} is an invisible \refterm{box}
^^{boxes//invisible}
whose width is zero and whose height and depth are slightly more than
those of a ``normal''
line of type in the context.  Struts are useful for obtaining
uniform vertical spacing when \TeX's
usual line spacing is disabled, e.g., within a math formula
or within a horizontal alignment where you've specified ^|\offinterlineskip|.
Because a strut is taller and deeper than everything else on its line,
it determines the height and depth of the line.
You can produce a strut with
the ^|\strut| command \ctsref{\strut} or the ^|\mathstrut| command
\ctsref\mathstrut.
You can use |\strut| anywhere, but you can only use |\mathstrut| when
\TeX\ is in math \refterm{mode}.  A strut in \plainTeX\ has height 8.5\pt\ and
depth 3.5\pt, while a math strut has the height and depth of a left
parenthesis in the current \refterm{style} (so it's smaller for
subscripts and superscripts).
\par}

Here's an example showing how you might use a strut:
\csdisplay
\vbox{\hsize = 3in \raggedright
   \strut Here is the first of two paragraphs that we're
   setting in a much narrower line length.\strut}
\vbox{\hsize = 3in \raggedright
   \strut Here is the second of two paragraphs that we're
   setting in a much narrower line length.\strut}
|
This input yields:
\display{\vbox{
\vbox{\hsize = 3in \raggedright
\strut Here is the first of two paragraphs that we're setting
in a much narrower line length.\strut}
\vbox{\hsize = 3in \raggedright
\strut Here is the second of two paragraphs that we're setting
in a much narrower line length.\strut}
}}
\noindent
Without the struts the \refterm{vboxes:vbox} would be too close 
together.  Similarly, in the formula:
\csdisplay
$\overline{x\mathstrut} \otimes \overline{t\mathstrut}$
|
the math struts cause both bars to be set at the same height even
though the `$x$' and the `$t$' have different heights:
\display{
$\overline{x\mathstrut} \otimes \overline{t\mathstrut}$
}
\vskip -\belowdisplayskip
\endconcept
\nobreak


\endconcepts
\end