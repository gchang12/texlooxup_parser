\input macros
\beginconcepts
\concept {justified text} 

\defterm{Justified text} is text that has been typeset so that both
margins are even.  Unjustified text, on the other hand, has been typeset
with ``ragged'' margins on one or both sides.
Documents typed on old-fashioned typewriters almost always have 
ragged right margins.
Although documents produced by \TeX\ are
justified by default, you can if you wish produce documents (or
sequences of lines) that have ^{ragged right}---or ^{ragged left}---margins.
You can also get \TeX\ to center a sequence of lines, thus making both
margins ragged.
^^{centered text}
You can use the 
^|\leftskip|, ^|\rightskip|, and  ^|\raggedright| commands
(\pp \xrefn{\leftskip},~\xrefn{\raggedright}) for these purposes.

When \TeX\ is producing justified text, it usually
needs to stretch or shrink the glue within each line to make the margins
come out even.  When \TeX\ is producing unjustified text, on the other
hand, it usually leaves the glue within each line at its natural width.
Many typographers prefer unjustified text because its interword
spacing is more uniform.
\endconcept



\endconcepts
\end