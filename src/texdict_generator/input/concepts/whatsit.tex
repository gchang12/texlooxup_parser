\input macros
\beginconcepts
\concept whatsit

A \defterm{whatsit} is an item of information that 
tells \TeX\ to carry out some action
that doesn't fit into its ordinary scheme of things.
A whatsit can appear in a horizontal or vertical list, just like a box
or a glue item.
\TeX\ typesets a whatsit
as a \refterm{box} having zero width, height, and depth---in other
words, a box that contains nothing and occupies no space.

Three sorts of whatsits are built into \TeX: 
\ulist
\li The |\openout|, |\closeout|, and |\write| commands
(\p\xrefn{\openout})
% (2nd) removed \xref to \write, since it's on the same page
produce a whatsit for operating on an output file.
^^|\openout//whatsit produced by|
^^|\write//whatsit produced by|
^^|\closeout//whatsit produced by|
\TeX\ postpones the operation until it next ships out a page
to the {\dvifile}
^^{\dvifile//whatsits in}
(unless the operation is preceded by ^|\immediate|).
\TeX\ uses a whatsit for these commands because they don't have anything
to do with what it's typesetting when it encounters them.
\li The ^|\special| command \ctsref{\special} tells \TeX\ to
insert certain text directly into the \dvifile.
As with the |\write| command, \TeX\ postpones the insertion
until it next ships out a page to the {\dvifile}.
^^{\dvifile//material inserted by \b\tt\\special\e}
A typical use of |\special| would be to 
name a graphics file that the device driver should incorporate into
your final output.
\li When you change languages with the ^|\language| or ^|\setlanguage|
commands \ctsref{\language},
\TeX\ inserts a whatsit that instructs it to use a
certain set of hyphenation rules later on when it's breaking a paragraph
into lines.
\endulist
\noindent
A particular implementation of \TeX\ may provide additional whatsits.
\endconcept


\endconcepts
\end