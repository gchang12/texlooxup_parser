\input macros
\beginconcepts
\concept delimiter

A \defterm{delimiter} is a character
that is intended to be used as
a visible boundary of a math formula.
The essential property of a delimiter is that \TeX\ can adjust
its size according
to the vertical size (\refterm{height} plus \refterm{depth})
of the subformula.
However, \TeX\ performs the adjustment only if the delimiter
appears in a ``delimiter context'', namely, as an argument to 
one of the commands ^|\left|,
^|\right|,
|\over!-with!-delims|,
|\atop!-with!-delims|,
or |\above!-with!-delims|
^^|\overwithdelims|
^^|\atopwithdelims|
^^|abovewithdelims|
\margin{Footnote deleted}
(see \pp\xrefn{\overwithdelims},~\xrefn{\left}).
The delimiter contexts also include any \refterm{argument}
to a \refterm{macro} that uses the argument in a delimiter context.

For example, the left and right
parentheses are delimiters.
If you use ^{parentheses} in a delimiter context
around a formula, \TeX\ makes the parentheses big
enough to enclose the \refterm{box} that contains the formula (as long as the
\refterm{fonts:font} you're using have big enough parentheses).
For example:
\csdisplay
$$ \left( a \over b \right) $$
|
yields:
\centereddisplays $$\left (a \over b \right ) $$
Here \TeX\ has made the parentheses big enough to accommodate the fraction.
But if you write, instead:
\csdisplay
$$({a \over b})$$
|
you'll get:
$$({a \over b})$$
Since the parentheses aren't in a delimiter context,
they are \emph{not} enlarged.  

Delimiters come in pairs:
an opening delimiter at the left of the subformula
and a closing delimiter at its right.
You can explicitly choose a larger height for a
delimiter with the commands ^|\bigl|, ^|\bigr|, and their
relatives \ctsref{\bigl}.\footnote
{\PlainTeX\ defines the various |\big| commands by using |\left| and |\right|
to provide a delimiter context.  It sets the size by
constructing an empty formula with the desired height.}
For instance, in order to get the
displayed formula:
$$\bigl(f(x) - x \bigr) \bigl(f(y) - y \bigr)$$

\noindent in which the outer parentheses are a little bigger than the inner
ones, you should write:

\csdisplay
$$\bigl( f(x) - x \bigr) \bigl( f(y) - y \bigr)$$
|

The $22$ \plainTeX\ delimiters, shown at their normal size, are:
\display{%
$( \>) \>[ \>] \>\{ \>\}
\>\lfloor \>\rfloor \>\lceil \>\rceil
\>\langle \>\rangle \>/ \>\backslash
\>\vert \>\Vert
\>\uparrow \>\downarrow \>\updownarrow
\>\Uparrow \>\Downarrow \>\Updownarrow$}
^^|)| ^^|)| ^^|[| ^^|]| ^^|\lbrack| ^^|\rbrack|
^^|\{| ^^|\}| ^^|\lbrace| ^^|\rbrace|
^^|\lfloor| ^^|\rfloor|  ^^|\lceil|  ^^|\rceil|
^^|\langle|  ^^|\rangle|  ^^|/|  ^^|\backslash|
^^|\vert|  ^^|\Vert|
^^|\uparrow|  ^^|\downarrow|  ^^|\updownarrow|
^^|\Uparrow|  ^^|\Downarrow|  ^^|\Updownarrow|
\noindent
Here they are at the largest size provided explicitly by \plainTeX\
 (the |\Biggl|, |\Biggr|, etc., versions):
\nobreak\vskip .5\abovedisplayskip
\display{%
$\Biggl( \>\Biggr) \>\Biggl[ \>\Biggr]
\>\Biggl\{ \>\Biggr\} \>\Biggl\lfloor \>\Biggr\rfloor
\>\Biggl\lceil \>\Biggr\rceil
\>\Biggl\langle \>\Biggr\rangle
\>\Biggm/ \>\Biggm\backslash
\>\Biggm\vert \>\Biggm\Vert
\>\Biggm\uparrow \>\Biggm\downarrow \>\Biggm\updownarrow
\>\Biggm\Uparrow \>\Biggm\Downarrow \>\Biggm\Updownarrow$}
\vskip .5\belowdisplayskip
\noindent
The delimiters (except for `|(|', `|)|', and
`|/|')
are among the symbols listed on
pages~\xrefn{\lbrace}--\xrefn{\Uparrow}.
They are listed in one place on \knuth{page~146}.

A delimiter can belong to any class.
^^{class//of a delimiter}
For a delimiter that you enlarge with
|\bigl|, |\bigr|, etc., the class is determined by the command:
``opener'' for |l|-commands, ``closer'' for |r|-commands, 
``relation'' for |m|-commands, and ``ordinary symbol'' for |g|-commands,
e.g., |\Big|.

You can obtain a delimiter in two different ways:
\olist
\li You can make a character be a delimiter by assigning it a
nonnegative delimiter code
\bix^^{delimiter codes}
(see below) with the ^|\delcode| command (\xref\delcode).
Thereafter the character acts as a delimiter whenever you use it in a
delimiter context.\footnote{%
It's possible to use a character with a nonnegative delimiter code in
a context where it isn't a delimiter. In this case \TeX\ doesn't perform the
search; instead it just uses the character in the ordinary way
(see \knuth{page~156}).}
\li You can produce a delimiter explicitly with the ^|\delimiter| command
(\xref\delimiter), in analogy to the way that you can produce an ordinary
character with the |\char| command or a math character with the |\mathchar|
command.
The |\delimiter| command uses the same delimiter codes that are used in a
|\delcode| table entry, but with an extra digit in front to indicate a
class.
It's rare to use |\delimiter| outside of a macro definition.
\endolist

A delimiter code tells
\TeX\ how to search for an appropriate output character to represent
a delimiter.
The rules for this search are rather complicated
(see \knuth{pages~156 and 442}).
A complete understanding of these rules requires knowing
about the organization of font ^{metrics file}s, a topic that is not just beyond
the scope of this book but beyond the scope of \texbook\ as well.

In essence the search works like this.  The delimiter code specifies a
``small'' output character and a ``large'' output character by
providing a \refterm{font} position and a font \refterm{family} for each
(see \xref\delcode).
Using this information, \TeX\ can find (or construct)
larger and larger versions of the delimiter.  \TeX\ first tries 
different sizes (from small to large) 
of the ``small'' character in the ``small'' font
and then
different sizes (also from small to large)
of the ``large'' character in the ``large'' font,
seeking one whose height plus depth is sufficiently large.
If none of the characters it finds are large
enough, it uses the largest one that it finds.
It's possible that
the small character, the large character, or both have been left unspecified
(indicated by a zero in the appropriate part of the delimiter code).
If only one character
has been specified, \TeX\ uses that one. If neither has been specified, 
it replaces the delimiter by a space of width ^|\nulldelimiterspace|.
\eix^^{delimiter codes}

\endconcept


\endconcepts
\end