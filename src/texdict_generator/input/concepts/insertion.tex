\input macros
\beginconcepts
\concept insertion

\looseness = -1 
An \defterm{insertion} is a vertical list containing material
to be inserted into
a page when \TeX\ has finished building that page.\footnote
{\tighten 
\TeX\ itself doesn't
insert the material---it just makes the material available to
the output routine, which is then responsible for transferring
it to the composed page.
^^{output routine//insertions, treatment of}
The only immediate effect of the ^|\insert| command
\ctsref{\insert} is to change \TeX's page break calculations so that it
will leave room on the page for the inserted material.  Later, when
\TeX\ actually breaks the page, it divides the inserted material into
two groups: the material that fits on the current page and the material
that doesn't.
^^{page breaks//insertions at}
The material that fits on the page is placed into box registers,
one per insertion,
and the material that doesn't fit is carried over to the next page.
This procedure allows \TeX\ to do such
things as distributing parts of a long footnote over several consecutive
pages.} Examples of such insertions are footnotes and figures.  The
\refterm{\plainTeX} commands for
creating insertions are ^|\footnote|, ^|\topinsert|, |\mid!-insert|,
^^|\midinsert|
and ^|\pageinsert|, as well as the primitive ^|\insert| command
itself (\pp\xrefn\footnote--\xrefn{endofinsert}).
\TeX's mechanism for handling insertions is rather complicated;
see \knuth{pages~122--125} for the details.
\endconcept


\endconcepts
\end