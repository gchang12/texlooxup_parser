\input macros
\beginconcepts
\concept {line break}

A \defterm{line break} is a place in your document where \TeX\ ends
a line as it typesets a paragraph.
When \TeX\ processes your document, it collects the contents of each
paragraph in a \refterm{horizontal list}.
When it has collected an entire paragraph,
it analyzes the list to find what it considers to be the best possible
line breaks.  \TeX\ associates ``^{demerits}'' with various symptoms of
unattractive line breaks---lines that have too much or too little
space between words, consecutive lines that end in hyphens, and so forth.  It
then chooses the line breaks so as to minimize the total number of demerits.
See \knuth{pages~96--101} for a full description of \TeX's line-breaking rules.

You can control \TeX's choice of line breaks in several ways:
\ulist

\li You can insert a \refterm{penalty} (\xref{hpenalty}) somewhere in 
the horizontal list that \TeX\ builds as it forms a paragraph.
^^{penalties//in horizontal lists}
A
positive penalty discourages \TeX\ from breaking the line there, while a
negative penalty---a bonus, in other words---encourages \TeX\ to break
the line there.  A penalty of $10000$ or more prevents a line break,
while a penalty of $-10000$ or less forces a line break.  You can get
the same effects with the ^|\break| and
^|\nobreak| commands (\pp\xrefn{hbreak},~\xrefn{hnobreak}).

\li You can tell \TeX\ to allow a hyphen in a particular place by
inserting a discretionary hyphen
^^{discretionary hyphens}
with the |\-| command \ctsref{\@minus}, or
otherwise control how \TeX\ hyphenates your document \seeconcept
{hyphenation}.
^^|\-//in line breaking|

\li You can tell \TeX\ to allow a line break after a ^{solidus} (/) between
two words by inserting ^|\slash| \ctsref{\slash}
between them, e.g., `|fur!-longs\slash fortnight|'.

\li You can tell \TeX\ not to break a line between two particular words by
inserting a ^{tie} (|~|) between those words.
^^|~//in line breaking|

\li You can adjust the penalties associated with line breaking by
assigning different values to \TeX's line-breaking
\refterm{parameters:parameter}.

\li You can enclose a word or sequence of words in an \refterm{hbox}, 
thus preventing \TeX\ from breaking the line anywhere within the hbox.
^^{hboxes//controlling line breaks}
\endulist

It's useful to know the places where \TeX\ can break a line:
\ulist
\li at glue, provided that:
\olist
\li the item preceding the glue is one of the following:
a box, a discretionary item (e.g., a discretionary hyphen),
the end of a math formula,
a whatsit, 
or vertical material produced by |\mark| or |\vadjust|
or |\insert|
\li the glue is not part of a math formula
\endolist
\noindent
When \TeX\ breaks a line at glue, it makes the break at the left edge
of the glue space and forgets about the rest of the glue.
\li at a kern that's immediately followed by glue,
provided that this kern isn't within a math formula
\li at the end of a math formula that's immediately followed by glue
\li at a penalty, even one within a math formula
\li at a discretionary break
\endulist
When \TeX\ breaks a line, it discards any 
sequence of glue, kerns, and penalty items that follows the break point.
If such a sequence is followed by the beginning of a math formula, it
also discards any kern produced by the beginning of the formula.
\endconcept



\endconcepts
\end