\input macros
\beginconcepts
\concept {page layout}

\bix^^{margins}
\bix^^{headers}
\bix^^{footers}
When you're designing a document, you need to decide on its
\defterm{page layout}: the page size,
the margins on all four sides, the headers and footers, if any,
that appear at the top and bottom of the page,
and the amount of space between the body of the text and the headers or
footers.  \TeX\ has defaults for all of these.  It assumes an $8 \frac1/2$-%
by-$11$-inch page with margins of approximately one inch
on all four sides, no header,
and a footer consisting of a centered page number.

The margins are determined jointly by the four parameters
^|\hoffset|, ^|\voffset|, ^|\hsize|, and ^|\vsize| (see
``margins'', \xrefpg{margins},
for advice on how to adjust them).
\eix^^{margins}
The header normally consists of a single line that appears at the top of each
page, within the top margin area.  You can set it by assigning
a \refterm{token} list to the ^|\headline| parameter (\xref{\headline}).
Similarly,
the footer normally consists of a single line that appears at the bottom
of each
page, within the bottom margin area.  You can set it by assigning
a \refterm{token} list to the ^|\footline| parameter (\xref{\footline}).
For example, the input:
\csdisplay
\headline = {Baby's First Document\dotfill Page\folio}
\footline = {\hfil}
|
produces a header line like this on each page:
\vdisplay{
\dimen0 = \hsize
\advance \dimen0 by -\parindent
\hbox to \dimen0{Baby's First Document\dotfill Page 19}}
\noindent
and no footer line.

You can use marks to place the current topic of a section
of text into the header or footer.
^^{marks//with headers or footers}
See \conceptcit{mark} for an explanation of how to do this.
\eix^^{headers}
\eix^^{footers}
\endconcept


\endconcepts
\end