\input macros
\beginconcepts
\concept font

A \defterm{font} in \TeX\ is a collection of up to $256$ output
characters, usually having the same typeface design, style (roman,
italic, bold, condensed, etc.),
and point size.\footnote{\PlainTeX\ uses a special
font for constructing ^{math symbols} in which the characters have
different sizes.  Other special fonts are often useful for applications
such as typesetting ^{logos}.} The ^{Computer Modern fonts} that
generally come with \TeX\ have only $128$ characters. The colophon on
the last page of this book describes the typefaces that we used to set
this book.

For instance, here is the alphabet in the Palatino Roman $10$ point font:
^^{Palatino fonts}
\vskip\abovedisplayskip{\narrower\tenpal
\noindent ABCDEFGHIJKLMNOPQRSTUVWXYZ\hfil\break
abcdefghijklmnopqrstuvwxyz\par
}\vskip\belowdisplayskip
\noindent
And here it is in the Computer Modern Bold Extended $12$
point font:
^^{Computer Modern fonts}
\vskip\abovedisplayskip{\narrower\font\twelvebf=cmbx12\twelvebf
\noindent ABCDEFGHIJKLMNOPQRSTUVWXYZ\hfil\break
abcdefghijklmnopqrstuvwxyz\par
}\vskip\belowdisplayskip
The characters in a font are numbered.
The numbering usually agrees with the ^{\ascii} numbering
for those characters that exist in the \ascii\ character set.
The code table for each font indicates what the character
with code $n$ looks like in that font.
Some fonts, such as the ones used for mathematical symbols, have no
letters at all in them.  You can produce a \refterm{box} containing the
character numbered $n$, typeset in the current font, by writing `|\char |$n$'
 \ctsref{\char}.

In order to use a font in your document,
you must first name it with a control sequence and load it.  Thereafter you
can select it by typing
that control sequence whenever you want to use it.
\PlainTeX\ provides a number of fonts that are already named and~loaded.

You name and load a font as a single operation, using a
command such as `|\font\twelvebf=cmbx12|'.  Here `|\twelvebf|' is the
control sequence that you use to name the font
and `|cmbx12|' identifies the font metrics file
|cmbx12.tfm| 
in your computer's file system.
You then can start to use the font by typing
`|\twelvebf|'.  After that, the font will be in effect until
either (a)~you select another font or (b)~you terminate the
\refterm{group}, if any, in which you started the
font.  For example, the input:

\csdisplay
{\twelvebf white rabbits like carrots}
|
will cause the |cmbx12| font to be in effect just for the
text `|white rabbits like carrots|'.

You can use \TeX\ with fonts other than
Computer Modern (look at the example on \xrefpg{palatino} and
at the page headers). 
The files for such fonts need to be
installed in your computer's file system in a place where \TeX\ can find
them.  \TeX\ and its companion programs generally need two files for each font:
one to give its metrics (|cmbx12.tfm|,
^^{\tfmfile}
for example) and another to
give the shape of the characters (|cmbx12.pk|, for example).
\TeX\ itself uses only the metrics
file.  Another program, the device driver,
^^{device drivers}
converts the \dvifile\
^^{\dvifile//converted by driver}
produced by \TeX\ to a form that your printer
or other output device can handle.  The driver
uses the shape file (if it exists).

The font metrics file contains the information that \TeX\ needs in
order to allocate space for each typeset character.
Thus it includes the size of each character, the ligatures and
kerns that pertain to sequences of adjacent characters, and so on.
What the metrics file
\emph{doesn't} include is any information about the shapes
of the characters in the font.

{\tighten
The shape (pixel) file \xrdef{shape}
^^{pixel file}^^{shape file}
may be in any of several
formats. The extension part of the name (the part after the dot)
tells the driver which format the shape file is in.  For example,
|cmbx12.pk| ^^{\pkfile} might be the shape file for font |cmbx12| in
packed format, while |cmbx12.gf| ^^{\gffile} might be the shape file
for font |cmbx12| in generic font format.
A shape file may not be needed for a font that's resident in your
output device.
\par}

\endconcept



\endconcepts
\end