\input macros
\beginconcepts
\concept {\anatomy}

\texbook\ describes the way that {\TeX} processes its input in terms of \TeX's
``digestive tract''---its ``^{eyes}'', ``^{mouth}'',
``^{gullet}'', ``^{stomach}'', and ``^{intestines}''.  Knowing how this
processing works can be helpful when you're trying to understand subtle
aspects of \TeX's behavior as it's digesting a document.

\ulist

\li Using its ``\pix^{eyes}'', \TeX\ reads \refterm{characters:character} from
^{input files} and passes them to its mouth.  Since an input file
can contain ^|\input| commands \ctsref{\input},
\TeX\ can in effect ``shift its gaze'' from one file to another.

\li Using its ``\pix^{mouth}'', {\TeX} assembles the characters into
\refterm{tokens:token} and passes them to its gullet.
Each token is either a \refterm{control sequence} or a single
character.  A control sequence always starts with an \refterm{escape
character}.  Note that spaces and ends-of-line are characters in their
own right, although \TeX\ compresses a sequence of input spaces into a single
space token.  See \knuth{pages~46--47} for the rules by which \TeX\ assembles
characters into tokens.
^^{tokens//assembled from characters}

\li Using its ``\pix^{gullet}'', {\TeX} expands any macros, conditionals, and
^^{macros//expanded in \TeX's stomach}
^^{tokens//passed to \TeX's stomach}
similar constructs that it finds (see \knuth{pages~212--216}) and passes
the resulting sequence of \refterm{tokens:token}
to \TeX's stomach.  Expanding one token
may yield other tokens that in turn need to be expanded.  {\TeX} carries
out this expansion from left to right unless the order is modified by
a command such as |\expandafter| \ctsref{\expandafter}.
In other words, \TeX's gullet always expands the leftmost un\-ex\-panded
token that it has not yet sent to \TeX's stomach.

\li Using its ``\pix^{stomach}'', {\TeX} processes the tokens in
groups.
Each group contains a primitive command followed by its arguments, if any.
Most of the commands are of the ``typeset this character'' variety,
so their groups consist of just one token.
Obeying the instructions given by the commands,
\TeX's stomach assembles larger and larger 
units, starting with 
characters and ending with pages,
and passes the pages to \TeX's intestines.
^^{pages//assembled in \TeX's stomach}
\TeX's stomach handles the tasks of \refterm{line break}ing---%
^^{line breaking}
i.e., breaking each paragraph into a sequence of lines---%
and of \refterm{page break}ing---i.e., breaking a continuous sequence of lines
and other vertical mode material
into pages. 

\li Using its ``\pix^{intestines}'', \TeX\ transforms the pages produced by its
stomach into a form intended for processing
by other programs.  It then sends the transformed output to the 
\dvifile.
^^{\dvifile//created by \TeX's intestines}

\endulist

Most of the time you can think of the processes that take place in \TeX's
eyes, mouth, gullet, stomach, and intestines 
as happening one after the other.  But the 
truth of the matter is that commands executed in \TeX's stomach can
influence the earlier stages of digestion.  For instance, when \TeX's stomach
encounters the |\input| command \ctsref{\input}, 
its eyes start reading from a different
file; when \TeX's stomach encounters a |\catcode| command
 \ctsref{\catcode} specifying a category code
for a character $c$, the interpretation of $c$ by \TeX's mouth is affected.
And when
\TeX's stomach encounters a \refterm{macro} definition, the expansions carried
out in \TeX's gullet are affected.

You can understand how the processes interact by imagining that each
process eagerly gobbles up the output of its predecessor as soon as it
becomes available.  For instance, once \TeX's stomach has seen
the last character of the filename in an |\input| command, \TeX's gaze
immediately shifts to the first character of the specified input file.
\endconcept



\endconcepts
\end