\input macros
\beginconcepts
\concept paragraph

Intuitively, a \defterm{paragraph} is a sequence of input lines that's
ended by a blank line, by a ^|\par| command \ctsref{\@par},
^^|\par//ending a paragraph with|
or by an intrinsically vertical command, such as |\vskip|.
More precisely, a paragraph is a sequence of commands that \TeX\ processes
in ordinary horizontal mode.
When \TeX\ has collected an entire paragraph, it forms it into a sequence of
lines by choosing line breaks \seeconcept{line break}.
The result is a list of hboxes with glue, interline penalties,
and interspersed vertical material between them.
Each hbox is a single line, and the glue is the interline glue.

\eject
\TeX\ starts a paragraph when it's in a vertical mode
and encounters an inherently horizontal command.
In particular, it's in a vertical mode when it's just finished a paragraph,
so the horizontal material on the line after a blank input line starts the
next paragraph in a natural way.
There are many kinds of inherently horizontal commands, but the most common
kind is an ordinary character, e.g., a letter.

\looseness = -1
The ^|\indent| and ^|\noindent| commands
(\pp\xrefn{\indent},~\xrefn{\noindent})
are also inherently horizontal commands that tell
\TeX\ either to indent or not to indent the beginning of a paragraph.
Any other horizontal command in vertical
mode causes \TeX\ to do an implicit |\indent|.
Once \TeX\ has started a paragraph, it's in ordinary horizontal mode.
It first obeys any commands that are in ^|\everypar|.
It then proceeds to collect items for the paragraph until it gets a signal
that the paragraph is ended.
At the end of the paragraph it
resets the paragraph shape parameters ^|\parshape|, |\hang!-indent|,
^^|\hangindent| 
and ^|\looseness|.

\TeX\ ordinarily translates a blank line into |\par|.  
It also
inserts a |\par| into the input whenever it's in horizontal mode and
sees an intrinsically vertical command.
So ultimately the thing that ends a paragraph is always a |\par| command.

When \TeX\ receives a |\par| command, it first
fills out\footnote{%
More precisely, it executes the commands:
\csdisplay
\unskip \penalty10000 \hskip\parfillskip
|
thus appending items for these commands
to the end of the current horizontal list.}
the paragraph it's working on.
It then breaks the paragraph into lines,
adds the resulting list of items to the enclosing vertical list,
and exercises the page builder 
(in the case where the enclosing vertical list is the main vertical list).
If the paragraph was ended by an intrinsically vertical command,
\TeX\ then executes that command.

\endconcept


\endconcepts
\end