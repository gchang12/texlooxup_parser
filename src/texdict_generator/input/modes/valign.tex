\input macros
\begindescriptions
\begindesc
{\tighten
\cts valign {{\bt \rqbraces{\<preamble>\\cr \<column>\\cr $\ldots$
   \<column>\\cr}}}
\aux\cts valign {{\bt to \<dimen>%
   \rqbraces{\<preamble>\\cr \<column>\\cr $\ldots$ \<column>\\cr}}}
\aux\cts valign {{\bt spread \<dimen>%
   \rqbraces{\<preamble>\\cr \<column>\\cr $\ldots$ \<column>\\cr}}}
\par}
\explain
This command produces a vertical \minref{alignment}
consisting of
a sequence of columns, where each column in turn contains a sequence
of row entries.
\TeX\ adjusts the heights of the row entries to accommodate the tallest one
in each row.

A vertical alignment
can only appear when \TeX\ is in a horizontal \minref{mode}.
Because vertical alignments are (a)~conceptually somewhat difficult and
(b)~not often used, we recommend that you learn about
alignments in general
(\xref{alignment}) and the |\halign| command (see above) before
you attempt to use the |\valign| command.

An alignment consists of a ^{preamble}
followed by the text to be aligned. The preamble,
which describes the layout of the columns that follow,
consists of a
sequence of row templates, separated by `|&|' and ended
by |\cr|.
\bix^^{template}
Each column consists of a sequence of ^{row}
entries, also separated by `|&|' and ended by |\cr|. Within
a template, `|#|' indicates where \TeX\ should insert the
corresponding text of a row entry. 

\TeX\ typesets each row entry in internal vertical
mode, i.e., as the contents of a \minref{vbox},
and implicitly encloses the entry in a group.
It always gives the vbox zero depth.
Any text or other horizontal mode material in a row entry then puts \TeX\
into ordinary horizontal mode.
(This is just an application of the general rules for \TeX's behavior
in internal vertical mode.)
The usual paragraphing parameters apply in this case:
the row entry has an initial
indentation of |\parindent| (\xref\parindent) and 
its lines have the
|\leftskip| and |\rightskip| (\xref\leftskip) \minref{glue}
appended to~them.

Note in particular that a row entry containing text has
a width of |\hsize| (\xref\hsize).  Unless you reset
|\hsize| to the row width that you want, you're likely to
encounter overfull \minref{hbox}es, or find that
the first column takes up the width of the entire page, or both.
\eix^^{entry (column or row)}

Normally, you need to include a \minref{strut}
^^{struts//in vertical alignments}
in each
template so that the rows don't come out crooked as a result
of the varying heights of the entries in the alignment.  You
can produce a strut with the |\strut| command.

The |to| form of this command instructs \TeX\
to make the vertical extent of the alignment be \<dimen>,
adjusting the space between rows as necessary.
The |spread| form of this command instructs \TeX\
to make the alignment taller by \<dimen> than its natural height.
These forms are like the corresponding forms of |\vbox| \ctsref\vbox.

\example
{\hsize=1in \parindent=0pt
\valign{#\strut&#\strut&#\strut&#\strut\cr
   bernaise&curry&hoisin&hollandaise\cr
   ketchup&marinara&mayonnaise&mustard\cr
   rarebit&tartar\cr}}
|
\produces
{\hsize=1in \parindent=0pt \leftskip=0pt
\valign{#\strut&#\strut&#\strut&#\strut\cr
   bernaise&curry&hoisin&hollandaise\cr
   ketchup&marinara&mayonnaise&mustard\cr
   rarebit&tartar\cr}}
\nextexample
% same thing but without struts (shows why you need them)
{\hsize=1in \parindent=0pt
\valign{#&#&#&#\cr
   bernaise&curry&hoisin&hollandaise\cr
   ketchup&marinara&mayonnaise&mustard\cr
   rarebit&tartar\cr}}
|
\produces
{\hsize=1in \parindent=0pt \leftskip=0pt
\valign{#&#&#&#\cr
   bernaise&curry&hoisin&hollandaise\cr
   ketchup&marinara&mayonnaise&mustard\cr
   rarebit&tartar\cr}}
\endexample
\enddesc
\enddescriptions
\end