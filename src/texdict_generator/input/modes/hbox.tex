\input macros
\begindescriptions
\begindesc
\bix^^{box commands}
^^{hboxes//constructing with \b\tt\\hbox\e}
%
\cts hbox {\rqbraces{\<horizontal mode material>}}
\aux\cts hbox {{\bt to} \<dimen> \rqbraces{\<horizontal mode material>}}
\aux\cts hbox {{\bt spread} \<dimen> \rqbraces{\<horizontal mode material>}}
\explain
This command produces an hbox
(horizontal \minref{box}) containing \<horizontal mode material>.
The braces around \<horizontal mode material> define a group.
\TeX\ doesn't break the \<horizontal mode material> into lines,
since it's in restricted horizontal mode when it's assembling the box.
\TeX\ won't change the size of the box once it's been produced.

|\hbox| is often useful when you want to keep some text all on one line.
If your use of |\hbox| prevents \TeX\ from breaking lines in an acceptable
way, \TeX\ will complain about an overfull hbox.

The width of the hbox depends on the arguments to |\hbox|:
\ulist\compact
\li If you specify only \<horizontal mode material>,
the hbox will have its natural \minref{width}.
\li If you specify |to| \<dimen>, the width of the hbox will be \<dimen>.
\li If you specify |spread| \<dimen>, the width of the hbox will be
its natural width plus \<dimen>, i.e., the hbox will be spread out by
\<dimen>.
\endulist

The ^|\hfil| command (\xref\hfil) is useful for filling out an
hbox with empty space when the material in the box isn't as wide as
the width of the box.
\example
\hbox{ugly suburban sprawl}
\hbox to 2in{ugly \hfil suburban \hfil sprawl}
\hbox spread 1in {ugly \hfil suburban \hfil sprawl}
% Without \hfil in the two preceding lines,
% you'd get `underfull hbox'es.
|
\produces
\hbox{ugly suburban sprawl}
\hbox to 2in{ugly \hfil suburban \hfil sprawl}
\hbox spread 1in {ugly \hfil suburban \hfil sprawl}
% Without \hfil in the two preceding lines,
% you'd get `underfull hbox'es.

\doruler{\8\8\8}3{in}
\endexample
\enddesc
\enddescriptions
\end