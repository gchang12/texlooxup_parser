\input macros
\begindescriptions
\begindesc
\cts settabs {\<number> {\bt \\columns}}
\aux\cts settabs {{\bt \\+} \<sample line> {\bt \\cr}}
\explain
The first form of
this command defines a set of tab stops ^^{tabs}
for a tabbing \minref{alignment}.
It tells \TeX\ to set the tab stops so as to divide each line into
\<number> equal parts.  \TeX\ takes the length of a line to be
|\hsize|, as usual.  You can make the alignment narrower by decreasing
|\hsize|.
\margin{paragraph ``The tab settings $\ldots$'' moved to below.}
\example
{\hsize = 3in \settabs 3 \columns
\+$1$&one&first\cr
\+$2$&two&second\cr
\+$3$&three&third\cr}
|
\produces
{\hsize = 3in \settabs 3 \columns
\+$1$ & one & first\cr
\+$2$ & two & second\cr
\+$3$ & three & third\cr}

\noindent\doruler{\8\8\8}3{in}
\smallskip
\endexample

The second form of this command defines tab stops by setting the tab stops
at the positions indicated by the `|&|'s in the sample line.
The sample line itself does not appear in the output.  When you
use this form you'll usually want to put material into the sample line that is
somewhat wider than the widest corresponding material in the alignment,
in order to produce space between the columns.
That's what we've done in the 
example below.  The material following the last tab
stop is irrelevant, since \TeX\ does not need to position
anything at the place where the |\cr|~appears.

The tab settings established by |\settabs| remain in effect until
you issue a new |\settabs| command or end a group containing
the |\settabs| command.
This is true for both forms of the command.

\example
% The first line establishes the template.
\settabs \+$1$\qquad & three\quad & seventh\cr
\+$1$&one&first\cr
\+$2$&two&second\cr
\+$3$&three&third\cr
|
\produces
\settabs \+$1$\qquad & three\quad & seventh\cr % the sample line
\+$1$&one&first\cr
\+$2$&two&second\cr
\+$3$&three&third\cr
\smallskip
\endexample
\enddesc
\enddescriptions
\end