\input macros
\begindescriptions
\begindesc
\bix^^{rules}
\bix^^{horizontal rules}
\bix^^{vertical rules}
%
\cts hrule {}
\aux\cts hrule {\bt height \<dimen> width \<dimen> depth \<dimen>}
\cts vrule {}
\aux\cts vrule {{\bt width \<dimen> height \<dimen> depth \<dimen>}}
\explain
The |\hrule| command produces a horizontal rule; the |\vrule| command
produces a vertical rule.  You can specify any or all of the width,
height, and depth of the rule---\TeX\ supplies default values for those
that you omit.  You can give the dimensions of the rule in any order;
the forms listed above show just two of the possible combinations.  You
can even give a dimension of a given kind more than once---if you do,
the last one is the one that counts.

If you don't specify the width of a horizontal rule, the rule is
extended horizontally to the boundaries of the innermost \minref{box} or
\minref{alignment} that contains the rule.  If you don't specify the
height of a horizontal rule, it defaults to |0.4pt|; if you don't
specify the depth of a horizontal rule, it defaults to |0pt|.

If you don't specify the width of a vertical rule, it defaults to
|0.4pt|.  If you don't specify the height or the depth of a vertical
rule, the rule is extended to the boundary of the innermost \minref{box}
or \minref{alignment} that contains the rule.

\TeX\ treats a horizontal rule as an inherently vertical item
and a vertical rule as an inherently horizontal item.
Thus a horizontal rule is legal only in a \minref{vertical mode},
while a vertical rule is legal only in a \minref{horizontal mode}.
^^{horizontal mode//rules in}
^^{vertical mode//rules in}
If this seems surprising, visualize it---a horizontal rule runs
from left to right and separates vertical items in a sequence, while
a vertical rule runs up and down and
separates horizontal items in a sequence.

\example
\hrule\smallskip
\hrule width 2in \smallskip
\hrule width 3in height 2pt \smallskip
\hrule width 3in depth 2pt
|
\produces
\medskip
\hrule\smallskip
\hrule width 2in \smallskip
\hrule width 3in height 2pt \smallskip
\hrule width 3in depth 2pt
\nextexample
% Here you can see how the baseline relates to the
% height and depth of an \hrule.
\leftline{
   \vbox{\hrule width .6in height 5pt depth 0pt}
   \vbox{\hrule width .6in height 0pt depth 8pt}
   \vbox{\hrule width .6in height 5pt depth 8pt}
   \vbox{\hbox{ baseline}\kern 3pt \hrule width .6in}
}
|
\produces
\medskip
\leftline{
   \vbox{\hrule width .6in height 5pt depth 0pt}
   \vbox{\hrule width .6in height 0pt depth 8pt}
   \vbox{\hrule width .6in height 5pt depth 8pt}
   \vbox{\hbox{ baseline}\kern 3pt \hrule width .6in}
}
\nextexample
\hbox{( {\vrule} {\vrule width 8pt} )}
\hbox {( {\vrule height 13pt depth 0pt}
   {\vrule height 13pt depth 7pt} x)}
% the parentheses define the height and depth of each of the 
% two preceding boxes; the `x' sits on the baseline
|
\produces
\medskip
\hbox{( {\vrule} {\vrule width 8pt} )}
\hbox {( {\vrule height 13pt depth 0pt}
   {\vrule height 13pt depth 7pt} x)}
\endexample

\eix^^{rules}
\eix^^{horizontal rules}
\eix^^{vertical rules}
\enddesc
\enddescriptions
\end