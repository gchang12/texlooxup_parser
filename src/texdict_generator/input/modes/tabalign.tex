\input macros
\begindescriptions
\begindesc
\bix^^{tabbing alignments}
\bix^^{alignments//commands for}
%
\ctspecial + {{\bt \<text>\thinspace\&\thinspace\<text>%
   \thinspace\& $\cdots$ \\cr}} 
\ctsxrdef{@plus}
\cts tabalign {}
\explain
These commands begin a single line in a  tabbed \minref{alignment}.
The only difference between |\+| and |\tabalign| is that 
|\+| is an outer macro---you can't use it when \TeX\ is reading tokens at
high speed \seeconcept{outer}.

If you place an `|&|' at a position
to the right of all existing tabs in a tabbing alignment,
the `|&|' establishes a new tab at that position.
\example
\cleartabs % Nullify any previous \settabs.
\+ {\bf if }$a[i] < a[i+1]$ &{\bf then}&\cr
\+&&$a[i] := a[i+1]$;\cr
\+&&{\it found }$:=$ {\bf true};\cr
\+&{\bf else}\cr
\+&&{\it found }$:=$ {\bf false};\cr
\+&{\bf end if};\cr
|
\produces
\cleartabs % Nullify any previous \settabs.
\+ {\bf if }$a[i] < a[i+1]$ &{\bf then}&\cr
\+&&$a[i] := a[i+1]$;\cr
\+&&{\it found }$:=$ {\bf true};\cr
\+&{\bf else}\cr
\+&&{\it found }$:=$ {\bf false};\cr
\+&{\bf end if};\cr
\endexample
\enddesc
\enddescriptions
\end