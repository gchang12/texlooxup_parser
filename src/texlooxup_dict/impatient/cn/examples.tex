% -*- coding: utf-8 -*-
% This is part of the book TeX for the Impatient.
% Copyright (C) 2003 Paul W. Abrahams, Kathryn A. Hargreaves, Karl Berry.
% See file fdl.tex for copying conditions.

\input macros
%\chapter{Examples}
\chapter{例子}

\chapterdef{examples}

%This section of the book contains a set of examples
%to help get you started and to show you how to do various things with \TeX.
%Each example has \TeX\ output on the left-hand page and the \TeX \
%input that led to that output on the right-hand page.
%You can use these examples both as forms to imitate and
%as a way of finding the
%\TeX\ commands that you need in order to achieve a particular effect.
%However, these examples can illustrate only a few of the
%about $900$ \TeX\ commands.
这一章中包括了一组例子,来帮助你熟悉 \TeX ,
同时这些例子还展示了如何使用 \TeX\ 来完成各种排版工作。
每个例子都有一个放在左页的 \TeX\ 排版的结果和放在右页的相对应的 \TeX\ 输入文本。
你可以把这些例子作为模仿的样式,也可以用来找到你想要的效果的实现命令。
不过要注意的是,这些例子仅能展示 \TeX\ $900$ 条左右的命令的一小部分。

%Some of the examples are self-descriptive---that is, they discuss the very
%features of \TeX\ that they are illustrating.  These discussions are
%necessarily sketchy because there isn't room in the examples for all the
%information you'd need.  The capsule summary of commands
%(\chapterref{capsule})
%and the index will help you
%locate the complete explanation of every \TeX\ feature shown in the
%examples.
这里的某些例子是其义自现的——也就是说,它们在介绍每个所排印出来的功能。
这个介绍很粗略,因为没有足够的篇幅来讲术所有你想得到的信息。
命令速查表(\chapterref{capsule})和索引可以帮你来找到例子中的每个 \TeX\ 功能。

%Because we've designed the examples to illustrate
%many things at once, some examples contain a great variety of
%typographical effects.  These examples generally are \emph{not} good
%models of typographical practice.  For instance, Example~8 has some of its
%equation numbers on the left and some on the right.  You'd never want to
%do that in a real publication.
因为我们在设计这些例子时,把很多的东西放在一起描述,
因此,这些例子展示了很多的排版效果。
这些例子一般并\emph{不是}好的排版实践模版。
比如例~8 把有些公式编号放在左边,又把另一些放在了右边。
你永远不会在一个实际的科学出版物中使用这样的公式编号。

%\xrdef{xmphead}
%Each example except for the first one starts with a macro (see
%\xref{macro}) named |\xmpheader|.  We've used |\xmpheader| in order to
%conserve space in the input, since without it each example would have
%several lines of material you'd already seen.
%|\xmpheader| produces the title of an example and the
%extra space that goes with it.  You can see in the first example
%what |\xmpheader| does, so you can imitate it if you wish.
%Except for |\xmpheader|, every command that we use in these examples is
%defined in \plainTeX.
\xrdef{xmphead}
除了第一个例子以外,每个例子都由一个叫 |\xmpheader| 的宏开始(见\xref{宏})。
我们这样做是为了节省输入文本的篇幅,
否则每个例子开头你都会看到几行你先前已经看到的内容。
|\xmpheader| 会排印出例子的标题和标题后的空白。
你可以参见没有使用 |\xmpheader| 的第一个例子是如何实现这一点的,
然后你就能模仿它了。
除了 |\xmpheader|,在这里使用的每个命令都是在 \PlainTeX\ 中定义过的。

% The first example does the necessary eject here.
{%
   \let\bye = \relax % We don't want to obey \bye in the example input.
   % These switches can't be done by a macro since \bye is outer.
   \doexamples {xmptext}% Typeset the actual examples.
}%

\ifoldeplain\else\ifcompletebook\else
\vfil\eject{\sectionfonts\leftline{本章索引}}
\readindexfile{i}
\fi\fi

\endchapter
\byebye
