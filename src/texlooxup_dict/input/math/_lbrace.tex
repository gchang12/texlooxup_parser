\input macros
\begindescriptions
\begindesc
\bix^^{delimiters}
%
\dothreecolumns 12
\easy\ctsdisplay lbrace {}
\ctsydisplay { @lbrace {}
\ctsdisplay rbrace {}
\ctsydisplay } @rbrace {}
\ctsdisplay lbrack {}
\ctsdisplay rbrack {}
\ctsdisplay langle {}
\ctsdisplay rangle {}
\ctsdisplay lceil {}
\ctsdisplay rceil {}
\ctsdisplay lfloor {}
\ctsdisplay rfloor {}
\egroup
\explain
These commands produce left and right \minref{delimiter}s.
Mathematicians use delimiters to indicate the boundaries between parts
of a formula.  Left delimiters are also called ``^{opening}s'', and
right delimiters are also called ``^{closing}s''.  Openings and closings
are two of \TeX's \minref{class}es of math symbols.  \TeX\ puts
different amounts of space around different \minref{class}es of math
symbols. You might expect the space that \TeX\ puts around openings and
closings to be symmetrical, but in fact it isn't.

Some left and right delimiters have more than one command that you can
use to produce them:

\ulist\compact
\li `$\{$' (|\lbrace| and |\{|)
\li `$\}$' (|\rbrace| and |\}|)
\li `$[$' (|\lbrack| and `|[|')
\li `$]$' (|\rbrack| and `|]|')
\endulist
\noindent You can also use the left and right bracket characters
(in either form) outside of math mode.

In addition to these commands, \TeX\ treats `|(|' as a left
delimiter and `|)|' as a right delimiter.

You can have \TeX\
choose the size for a delimiter by using |\left| and |\right| (\xref\left).
Alternatively,
you can get a delimiter of a specific size by using one of the |\big|$x$
commands (see |\big| et al., \xref{\big}).

\example
The set $\{\,x \mid x>0\,\}$ is empty.
|
\produces
The set $\{\,x \mid x>0\,\}$ is empty.

\eix^^{delimiters}
\endexample
\enddesc
\enddescriptions
\end