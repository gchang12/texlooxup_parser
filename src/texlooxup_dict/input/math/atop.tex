\input macros
\begindescriptions
\begindesc
\bix^^{fractions}
\bix^^{stacking subformulas}
\easy\cts over {}
\cts atop {}
\cts above {\<dimen>}
\cts choose {}
\cts brace {}
\cts brack {}
\explain
{\def\fri{\<formula$_1$>}%
\def\frii{\<formula$_2$>}%
These commands stack one subformula on top of another one.  We will explain how
|\over| works, and then relate the other commands to it.

|\over| is the command that you'd normally use to produce a fraction.
^^{fractions//produced by \b\tt\\over\e} 
If you write something in one of the following forms:
\csdisplay
$$!fri\over!frii$$
$!fri\over!frii$
\left!<delim>!fri\over!frii\right!<delim>
{!fri\over!frii}
|
you'll get a fraction with numerator \fri\  and denominator \<for\-mu\-la$_2$>,
i.e., \fri\ over \frii.
In the first three of
these forms the |\over| is not implicitly contained in a group;
it absorbs
everything to its left and to its right until it comes to a boundary,
namely, the beginning or end of a group.

You can't use |\over| or any of the other commands in this group
more than once in a formula.
Thus a formula such as:
\csdisplay
$$a \over n \choose k$$
|
isn't legal.
This is not a severe restriction because
you can always enclose one of the commands in braces.
The reason for the restriction is that if you had two of these commands
in a single formula, \TeX\ wouldn't know how to group them.

The other commands are similar to |\over|, with the following exceptions:
\ulist\compact
\li |\atop| leaves out the fraction bar. 
\li |\above| provides a fraction bar of thickness \<dimen>.
\li |\choose|
leaves out the fraction bar and encloses the construct in parentheses.
(It's called ``choose'' because $n \choose k$ is the notation for the
number of ways of choosing $k$ things out of $n$ things.)
\li |\brace| leaves out the fraction bar and encloses the construct in braces.
\li |\brack|
leaves out the fraction bar and encloses the construct in brackets.
\endulist
}%
\example
$${n+1 \over n-1}      \qquad {n+1 \atop n-1}   \qquad
  {n+1 \above 2pt n-1} \qquad {n+1 \choose n-1} \qquad
  {n+1 \brace n-1}     \qquad {n+1 \brack n-1}$$
|
\dproduces
$${n+1 \over n-1}      \qquad {n+1 \atop n-1}   \qquad
  {n+1 \above 2pt n-1} \qquad {n+1 \choose n-1} \qquad
  {n+1 \brace n-1}     \qquad {n+1 \brack n-1}$$
\endexample
\enddesc
\enddescriptions
\end