\input macros
\beginconcepts
\concept {escape character}

An \defterm{escape character} introduces a control sequence.  The escape
character in \refterm{\plainTeX} is the backslash (|\|).
\indexchar \
You can change the escape character from $c_1$ to $c_2$
by reassigning the category codes of $c_1$ and $c_2$
with the ^|\catcode| command \ctsref{\catcode}.
You can also define additional escape characters similarly.
If you want to typeset material containing literal escape characters, you must
either 
(a) define a control sequence that stands for the printed escape character or
(b) temporarily
disable the escape character by changing its category code, using the
method shown on \xrefpg{verbatim}.  The definition:

\csdisplay
\def\\{$\backslash$}
|
is one way of creating a control sequence that stands for `$\backslash$'
(a backslash typeset in a math font).

You can use the ^|\escapechar| parameter \ctsref{\escapechar} to specify
how the escape character is represented in synthesized control sequences,
e.g., those created by |\string| and |\message|.


\endconcept


\endconcepts
\end