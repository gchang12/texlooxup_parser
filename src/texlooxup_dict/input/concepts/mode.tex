\input macros
\beginconcepts
\concept mode

When \TeX\ is processing your input in its stomach \seeconcept{\anatomy},
it is in one of six \defterm{modes}:
\ulist\compact
\li ^{ordinary horizontal mode} (assembling a paragraph)
\li ^{restricted horizontal mode} (assembling an \refterm{hbox})
\li ^{ordinary vertical mode} (assembling a page)
\li ^{internal vertical mode} (assembling a \refterm{vbox})
\li ^{text math mode} (assembling a formula that appears in text)
\li ^{display math mode}
(assembling a formula that appears on a line by~itself)
^^{horizontal mode}^^{vertical mode}^^{math mode}
\endulist
The mode describes the kind of entity that \TeX\ is putting together.

Because you can embed one kind of entity within another, e.g., a vbox
within a math formula, \TeX\ keeps track not just of one mode but of a
whole list of modes (what computer scientists call a ``stack'').
Suppose that \TeX\ is in mode $M$ and encounters something that
puts it into a new mode \Mprimeperiod.  When it finishes its work in
mode \Mprimecomma, it resumes what it was doing in mode \Mperiod.

\endconcept


\endconcepts
\end