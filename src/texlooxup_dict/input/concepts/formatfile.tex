\input macros
\beginconcepts
\concept {format file}

{\tighten
A \defterm{format file} is a file that contains an image of
\TeX's memory in a form in which it can be reloaded quickly.
A format file can be created with the ^|\dump| command \ctsref\dump.
The image contains a complete record
of the definitions (of \refterm{fonts:font}, \refterm{macros:macro}, etc.)
that were present when the dump took place.
By using ^|virtex|, a special ``virgin'' form of \TeX,
you can then reload the format file at high speed and 
continue in the same state that \TeX\ was in at the time of the dump.
The advantage of a format file over an ordinary input
file containing the same information is that \TeX\ can load it much
faster.
\par}

Format files can only be created by ^|initex|, another special
form of \TeX\ intended just for that purpose.
Neither |virtex| nor |initex| has any
facilities other than the primitives built into the
\TeX\ program itself.

A ^{preloaded} form of \TeX\ is one that has a format file already 
loaded and is ready to accept user input.
The form of \TeX\ that's called |tex|
often has the \plainTeX\ definitions preloaded.
(\PlainTeX\ is ordinarily  available in two other forms as well:
as a format file and as a \TeX\ source file.
In some environments, |tex| is equivalent to calling |virtex|
and then loading |plain|.)
Creating preloaded forms of \TeX\ requires a special program;
it cannot be done using only the facilities of \TeX\ itself.

\endconcept

\endconcepts
\end