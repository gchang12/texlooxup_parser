\input macros
\beginconcepts
\concept {page break}

A \defterm{page break} is a place in your document where \TeX\ ends a
page and (except at the end of the document) starts a new one.  
See \conceptcit{page} for the process that \TeX\ goes through in choosing
a page break.

You can control \TeX's choice of page breaks in several ways:
\ulist
\li You can insert a penalty (\xref{vpenalty}) 
^^{penalties//in vertical lists}
between two items in the main vertical list.  A positive
penalty discourages \TeX\ from breaking the page 
there, while a negative penalty---a bonus, in other words---%
encourages \TeX\ to break the page there.  A penalty of $10000$
or more prevents a
page break, while a penalty of $-10000$ or less forces a page break.
You can get the same effects with the |\break| and
|\nobreak| commands \ctsref{vbreak}.

\li You can adjust the penalties associated with page breaking
by assigning different values to \TeX's page-breaking
\refterm{parameters:parameter}.

\li You can enclose a sequence of paragraphs 
or other items in the main vertical list within a \refterm{vbox}, 
thus preventing \TeX\ from breaking the page anywhere within the sequence.
\endulist

Once \TeX\ has chosen a page break, it places the portion of the main vertical
list that precedes the break into |\box255|.
It then calls the current \refterm{output routine}
to process |\box255| and eventually ships its contents to the \dvifile.
^^{\dvifile//material from output routine}
The output routine must
also handle \refterm{insertions}, such as footnotes, that \TeX\ has accumulated
while processing the page.

It's useful to know the places where \TeX\ can break a page:
\ulist
\li At glue, provided that the item preceding the glue is
a box, a whatsit, a mark, or an insertion.
When \TeX\ breaks a page at glue, it makes the break at the top
of the glue space and forgets about the rest of the glue.
\li At a kern that's immediately followed by glue.
\li At a penalty, possibly between the lines of a paragraph.
\endulist
When \TeX\ breaks a page, it discards any 
sequence of glue, kerns, and penalty items that follows the break point.


\endconcept


\endconcepts
\end