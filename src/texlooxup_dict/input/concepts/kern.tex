\input macros
\beginconcepts
\concept kern

^^{spacing//adjusting with kerns}
A \defterm{kern} indicates a change to the normal spacing between
the items of a vertical or horizontal list.
A kern can be either positive or negative.  By
putting a positive kern between two items, you push them further apart
by the amount of the kern.  By putting a negative kern between two
items, you bring them closer together by the amount of the kern.  For
instance, this text:
\csdisplay
11\quad 1\kern1pt 1\quad 1\kern-.75pt 1
|
produces letter pairs that look like this:
\display{11\quad 1\kern1pt 1\quad 1\kern-.75pt 1}
You can use kerns in vertical mode to adjust the space between 
particular pairs of lines.

A kern of size $d$ is very similar to a \refterm{glue} item that has
size $d$ and no stretch or shrink.  Both the kern and the glue insert or
remove space between neighboring items.  The essential difference is
that \TeX\ considers two boxes with only kerns between them to be tied
together.  That is, \TeX\ won't break a line or a page at a kern unless
the kern is immediately followed by glue.  Bear this difference in mind
when you're deciding whether to use a kern or a glue item for a
particular purpose.

\TeX\ automatically inserts kerns between particular pairs of adjacent
letters, thus adjusting the space between those letters and enhancing
the appearance of your typeset document.  
For instance, the Computer Modern $10$-point roman font contains a kern
for the pair `To' that brings the left edge of the `o' under the
`T'.  Without the kern, you'd get \hbox{``T{o}p''} rather than ``Top''---%
the difference is slight but noticeable.
The metrics file
(^{\tfmfile})
for each \refterm{font} specifies the placement and size of the
kerns that \TeX\ automatically inserts when it is setting text in that~font.
\margin{paragraph deleted to save space; most of the material was
already in this section.}

\endconcept


\endconcepts
\end