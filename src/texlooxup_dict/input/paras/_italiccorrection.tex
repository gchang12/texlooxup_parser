\input macros
\begindescriptions
\begindesc
\easy\ctspecial / \ctsxrdef{@slash}
\explain
Every character in a \TeX\ \minref{font}
has an ``^{italic correction}'' associated with it, although
the italic correction
is normally zero for a character in an unslanted (upright) font.
The italic correction specifies the extra space that's needed
when you're switching from a slanted font (not necessarily
an italic font) to an unslanted font.  
The extra
space is needed because a slanted character projects into the
space that follows it, making the space look too small when the
next character is unslanted.
The metrics file for a font includes the italic correction of each 
character in the font.
^^{metrics file//italic correction in}

The |\/| command
produces an ^{italic correction} for the preceding character.
You should insert an italic correction when you're switching from
a slanted font to an unslanted font,
except when the next character is a period or comma.
\example
However, {\it somebody} ate {\it something}: that's clear.

However, {\it somebody\/} ate {\it something\/}:
that's clear.
|
\produces
However, {\it somebody} ate {\it something}: that's clear.

However, {\it somebody\/} ate {\it something\/}:
that's clear.
\endexample
\enddesc
\enddescriptions
\end