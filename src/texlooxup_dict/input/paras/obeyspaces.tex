\input macros
\begindescriptions
\begindesc
\cts obeyspaces {}
\explain
\TeX\ normally condenses a sequence of several spaces to a single space.
|\obeyspaces| instructs \TeX\ to produce a space in the output
for each space in the input.
|\obeyspaces| does not cause spaces at the beginning of a line
to show up, however; for that we recommend the |\obey!-white!-space|
command defined in |eplain.tex| 
(\xref{ewhitesp}).
^^|\obeywhitespace|
|\obeyspaces| is often useful when you're typesetting something,
computer input for example,
in a monospaced font (one in which each character takes up the
same amount of space)
and you want to show exactly what each line of input looks like.

You can use the |\obeylines| command (\xref{\obeylines}) to get \TeX\
to follow the line boundaries of your input.  |\obeylines| is often
used in combination with |\obeyspaces|.
\example 
These     spaces    are    closed   up
{\obeyspaces but   these  are     not   }.
|
\produces
These     spaces    are    closed   up
{\obeyspaces but   these  are     not   }.
\endexample
\enddesc
\enddescriptions
\end