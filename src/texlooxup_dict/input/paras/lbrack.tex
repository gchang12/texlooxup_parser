\input macros
\begindescriptions
\begindesc
^^{special characters}
%
\easy\ctspecialx # \ctsxrdef{@pound} {pound sign \#}
\ctspecialx $ \ctsxrdef{@bucks} {dollar sign \$}
\ctspecialx % \ctsxrdef{@percent} {percent sign \%}
\ctspecialx & \ctsxrdef{@and} {ampersand \&}
\ctspecialx _ \ctsxrdef{@underscore} {underscore \_}
\ctsx lq {left quote \lq}
\ctsx rq {right quote \rq}
\aux\ctsx lbrack left bracket [
\aux\ctsx rbrack right bracket ]
\ctsx dag {dagger symbol \dag}
\ctsx ddag {double dagger symbol \ddag}
\ctsx copyright {copyright symbol \copyright}
\ctsx P {paragraph symbol \P}
\ctsx S {section symbol \S}
\explain
These commands produce various special characters and marks.  The first
five commands are necessary because \TeX\ by default
attaches special meanings to
the characters (|#|, |$|, |%|, |&|, |_|).
You needn't be in \minref{math mode} to use these commands.

You can use the dollar sign in the Computer Modern
italic fonts to get the ^{pound
sterling} symbol, as shown in the example below.

\example
\dag It'll only cost you \$9.98 over here, but in England
it's {\it \$}24.98.
|
\produces
\dag It'll only cost you \$9.98 over here, but in England
it's {\it \$}24.98.
\endexample
\enddesc
\enddescriptions
\end