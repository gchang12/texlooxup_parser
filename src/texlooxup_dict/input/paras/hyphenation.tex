\input macros
\begindescriptions
\begindesc
\cts hyphenation {\rqbraces{\<word>\thinspace\vs\ $\ldots$\ \vs
   \thinspace\<word>}}
\explain
\TeX\ keeps a dictionary of exceptions to its ^{hyphenation} rules.
Each dictionary entry indicates how a particular word should
be hyphenated.  
The |\hyphenation| command adds words to the dictionary.
Its argument is a sequence of words separated by blanks. 
Uppercase and lowercase letters are equivalent.
The hyphens in each word indicate the places
where \TeX\ can hyphenate that word.
A word with no hyphens in it will never be hyphenated.
However, you can still override the hyphenation dictionary by
using |\-| in a particular occurrence of a word.
You need to provide all the grammatical forms of a word
that you want \TeX\ to handle, e.g., both the singular and the plural.

\example
\hyphenation{Gry-phon my-co-phagy}
\hyphenation{man-u-script man-u-scripts piz-za}
|
\endexample
\enddesc
\enddescriptions
\end