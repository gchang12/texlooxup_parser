\input macros
\begindescriptions
\begindesc
\cts discretionary {\rqbraces{\<pre-break text>}
   \rqbraces{\<post-break text>}
   \rqbraces{\<no-break text>}}
\explain
\minrefs{line break}
^^{hyphenation}
This command specifies a ``discretionary break'', namely,
a place where \TeX\ can break a line.
It also tells \TeX\ what text to put on either side of the break.
\ulist
\li If \TeX\ does not break there, it uses the \<no-break text>.
\li If \TeX\ does break there, it puts the \<pre-break text> just before
the break and the \<post-break text> just after the break.
\endulist
\noindent
Just as with |\-|,
\TeX\ isn't obligated to break a line at a discretionary break.
In fact, |\-| is ordinarily equivalent to |\discretionary!allowbreak{-}{}{}|.

\TeX\ sometimes inserts discretionary breaks on its own.
For example, it inserts |\discretionary!allowbreak{}{}{}| after
an explicit hyphen or dash.

{\hyphenchar\tentt=-1 % needed to avoid weirdnesses
\example
% An ordinary discretionary hyphen (equivalent to \-):
\discretionary{-}{}{}
% A place where TeX can break a line, but should not
% insert a space if the line isn't broken there, e.g.,
% after a dash:
\discretionary{}{}{}
% Accounts for German usage: `flicken', but `flik-
% ken':
German ``fli\discretionary{k-}{k}{ck}en''
|
^^{hyphenation//German}
\endexample}

\enddesc
\enddescriptions
\end