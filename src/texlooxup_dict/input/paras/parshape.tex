\input macros
\begindescriptions
\begindesc
\cts parshape {$n\; i_1 l_1\; i_2 l_2\; \ldots \;i_n l_n$}
\explain
This command specifies the shape of the first $n$ lines of a paragraph---
the next paragraph if you're in vertical mode and the current paragraph
if you're in horizontal mode.
The $i$'s and $l$'s are all
dimensions.  The first line is indented by $i_1$ and has length $l_1$,
the second line is indented by $i_2$ and has length $l_2$, and so forth.
If the paragraph has more than $n$ lines, the last indentation\slash
length pair is used for the extra lines.
To achieve special effects such as the one
shown here, you usually have to experiment a lot, insert kerns here and
there, and choose your words to fit the shape.

|\parshape|, like ^|\hangafter| and ^|\hangindent|, is effective only for one
paragraph.
If you specify |\hangafter| and |\hangindent| as well as |\par!-shape|,
\TeX\ ignores the ^|\hangafter| and ^|\hangindent|.

By the way, the following example saves and restores |\fontdimen| values
explicitly, using temporary registers, since |\fontdimen| changes are
always global (see \xref\fontdimen).

\ifodd\pageno\vfill\eject\fi % so the wineglass is on a single page.

\example
% A small font and close interline spacing make this work
\smallskip\font\sixrm=cmr6 \sixrm \baselineskip=7pt
\dimen0=\fontdimen3\font \dimen2=\fontdimen4\font
\fontdimen3\font=1.8pt \fontdimen4\font=.9pt
\noindent \hfuzz=.1pt
\parshape 30 0pt 120pt 1pt 118pt 2pt 116pt 4pt 112pt 6pt
108pt 9pt 102pt 12pt 96pt 15pt 90pt 19pt 84pt 23pt 77pt
27pt 68pt 30.5pt 60pt 35pt 52pt 39pt 45pt 43pt 36pt 48pt
27pt 51.5pt 21pt 53pt 16.75pt 53pt 16.75pt 53pt 16.75pt 53pt
16.75pt 53pt 16.75pt 53pt 16.75pt 53pt 16.75pt 53pt 16.75pt
53pt 14.6pt 48pt 24pt 45pt 30.67pt 36.5pt 51pt 23pt 76.3pt
The wines of France and California may be the best known,
but they are not the only fine wines. Spanish wines are
often underestimated, and quite old ones may be available at
reasonable prices. For Spanish wines the vintage is not so
critical, but  the climate of the Bordeaux region varies
greatly from year to year. Some vintages are not as good as
others, so these years ought to be s\kern -.1pt p\kern -.1pt
e\kern -.1pt c\hfil ially n\kern .1pt o\kern .1pt
t\kern .1pt e\kern .1pt d\hfil: 1962, 1964, 1966.  1958,
1959, 1960, 1961, 1964, 1966 are also good California
vintages. Good luck finding them!!
\fontdimen3\font=\dimen0 \fontdimen4\font=\dimen2
|
%\margin{Wineglass text replaced because of permissions problem.}
\produces
% A small font and close interline spacing make this work
\smallskip\font\sixrm=cmr6 \sixrm \baselineskip=7pt
\dimen0=\fontdimen3\font \dimen2=\fontdimen4\font
\fontdimen3\font=1.8pt \fontdimen4\font=0.9pt
\noindent \hfuzz=0.1pt % reordered to save a line
\parshape 30 0pt 120pt 1pt 118pt 2pt 116pt 4pt 112pt 6pt 108pt 9pt 102pt
12pt 96pt 15pt 90pt 19pt 84pt 23pt 77pt 27pt 68pt 30.5pt 60pt 35pt 52pt
39pt 45pt 43pt 36pt 48pt 27pt 51.5pt 21pt 53pt 16.75pt 53pt 16.75pt
53pt 16.75pt 53pt 16.75pt 53pt 16.75pt 53pt 16.75pt 53pt 16.75pt
53pt 16.75pt 53pt 14.6pt 48pt 24pt 45pt 30.67pt 36.5pt 51pt 23pt 76.3pt
The wines of France and California may be the best
known, but they are not the only fine wines. Spanish
wines are often underestimated, and quite old ones may
be available at reasonable prices. For Spanish wines
the vintage is not so critical, but  the climate of the
Bordeaux region varies greatly from year to year. Some
vintages are not as good as others,
so these years ought to be
s\kern -.1pt p\kern -.1pt e\kern -.1pt c\hfil ially
n\kern .1pt o\kern .1pt t\kern .1pt e\kern .1pt d\hfil:
1962, 1964, 1966.  1958, 1959, 1960, 1961, 1964,
1966 are also good California vintages.
Good luck finding them!
\fontdimen3\font=\dimen0 \fontdimen4\font=\dimen2
\endexample
\eix^^{indentation}
\enddesc
\enddescriptions
\end