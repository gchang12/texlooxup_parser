\input macros
\begindescriptions
\begindesc
\bix^^{case conversion}
\bix^^{uppercase//conversion to}
\bix^^{lowercase//conversion to}
\cts lccode {\<charcode> \tblentry{number}}
\cts uccode {\<charcode> \tblentry{number}}
\explain
The |\lccode| and |\uccode| values for the $256$ possible input
characters specify the correspondence between the lowercase and
uppercase forms of letters.  These values are used by the |\lowercase|
and |\uppercase| commands respectively and by \TeX's hyphenation
algorithm.

\TeX\ initializes the values of |\lccode| and |\uccode| as follows:

\ulist\compact
\li The |\lccode| of a lowercase letter is the {\ascii} code for that letter.
\li The |\lccode| of an uppercase letter is the {\ascii} code for the
corresponding lowercase letter.
\li The |\uccode| of an uppercase letter is the {\ascii} code for that letter.
\li The |\uccode| of a lowercase letter is the {\ascii} code for the
corresponding uppercase letter.
\li The |\lccode| and |\uccode| of a nonletter are both zero.
\endulist

Most of the time there's no reason to change these values,
but you might want to change them if you're using a  language
that has more letters than English.
\example
\char\uccode`s \char\lccode`a \char\lccode`M
|
\produces
\char\uccode`s \char\lccode`a \char\lccode`M
\endexample
\enddesc
\enddescriptions
\end