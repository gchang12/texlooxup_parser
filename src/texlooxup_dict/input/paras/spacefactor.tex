\input macros
\begindescriptions
\begindesc
\cts spacefactor {\param{number}}
\cts spaceskip {\param{glue}}
\cts xspaceskip {\param{glue}}
\cts sfcode {\<charcode> \tblentry{number}}
\explain
These primitive \minref{parameter}s affect how much space \TeX\
puts between two adjacent words, i.e., the ^{interword spacing}.
The normal interword spacing is supplied by the current font.
As \TeX\ is processing a \minref{horizontal list}, it keeps track of the
^{space factor} $f$ in |\spacefactor|.
As it processes each input character $c$, it updates $f$ according to the
value of $f_c$, the space factor code of $c$ (see below).
For most characters, $f_c$ is $1000$ and \TeX\ sets $f$ to $1000$.
(The initial value of $f$ is also $1000$.)
When \TeX\ sees an interword space, it adjusts the size of that space 
by multiplying the stretch and shrink of that space by 
$f/1000$ and $1000/f$ respectively.
Thus:
\olist\compact
\li If $f=1000$, the interword space keeps its normal value.
\li If $f<1000$, the interword space gets less \minref{stretch}
and more \minref{shrink}.
\li If $f>1000$, the interword space gets more \minref{stretch}
and less \minref{shrink}.
\endolist
% > changed to \ge on the next line after second edition was typeset. 
% Correction made by A-W production.
In addition, if $f\ge2000$ the interword space is further increased by the
``extra space'' parameter associated with the current font.

Each 
input character $c$ has an entry in the |\sfcode| (space factor code)
table.
The |\sfcode| table entry is independent of the font.
Usually \TeX\ just sets $f$ to $f_c$ after it processes $c$.
However:
\ulist
\li If $f_c$ is zero, \TeX\ leaves $f$ unchanged.
Thus a character such as `|)|' in \plainTeX,
for which $f_c$ is zero, is essentially transparent to
the interword space calculation.
\li If $f<1000<f_c$, \TeX\ sets $f$ to $1000$ rather than to $f_c$,
i.e., it refuses to raise $f$ very rapidly.
\endulist
The |\sfcode| value for a period is normally $3000$, 
which is why \TeX\ usually puts extra space after a period
% > to \ge here, too, as above.
(see the rule above for the case $f\ge2000$).
Noncharacter items in a horizontal list, e.g., vertical rules,
generally act like characters with a space factor of $1000$.

You can change the space factor explicitly by assigning
a different numerical value to |\spacefactor|.
You can also override the normal
interword spacing by assigning a different numerical
value to |\xspaceskip| or to |\spaceskip|:
\ulist
\li |\xspaceskip| specifies the glue to be used when $f\ge2000$;
in the case where
|\xspaceskip| is zero, the normal rules apply.
\li |\spaceskip| specifies the glue to be used when $f<2000$ or when
\hbox{|\xspaceskip|} is zero; if |\spaceskip| is zero, the normal rules apply.
The stretch and shrink of
the |\spaceskip| glue, like that of the ordinary interword glue, 
is modified according to the value of $f$.
\endulist

See \knuth{page~76} for the precise rules that \TeX\ uses in calculating
interword \minref{glue}, and \knuth{pages~285--287} for the adjustments
made to |\spacefactor| after various items in a horizontal list.
\eix^^{spaces//interword}
\enddesc
\enddescriptions
\end