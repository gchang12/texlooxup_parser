\input macros
\begindescriptions
\begindesc
\cts looseness {\param{number}}
\explain
\minrefs{line break}
This parameter gives you a way 
to change the total number of lines in a paragraph from what they
optimally would be.
|\looseness| is so named because it's a
measure of how loose the paragraph is, i.e., how much extra space there is in
it. 

Normally, |\looseness| is $0$ and
\TeX\ chooses line breaks in its usual way.  But if
|\looseness| is, say, $3$, \TeX\ does the following:
\olist
\li It chooses line breaks normally, resulting in a paragraph of $n$ lines.
\li It discards these line breaks and
tries to find a new set of line breaks that gives the paragraph $n+3$ lines.
(Without the previous step, \TeX\ wouldn't know the value of $n$.)
\li If the previous attempt results in lines whose badness exceeds
|\tol!-er!-ance|,
^^|\tolerance|
it tries to get $n+2$ lines---and if that also fails,
$n+1$ lines, and finally $n$ lines again.
\endolist
\noindent
Similarly, if looseness is $-n$,
\TeX\ attempts to set the paragraph with $n$
fewer lines  than normal.
The easiest way for \TeX\ to make a paragraph one line longer is to put
a single word on the excess line.  You can prevent this by
putting a tie (\xref{@not}) between the last two words of the paragraph. 

Setting |\looseness| is the best way to force a paragraph
to occupy a given number of lines.
Setting it to a negative value is useful when you're trying to
increase the amount of text you can fit on a page.
Similarly, setting it to a positive
value is useful when you're trying to 
decrease the amount of text on a page.

\TeX\ sets |\looseness| to $0$ when it ends a paragraph, after breaking
the paragraph into lines.
If you want to change the looseness of several paragraphs, you must do it
individually for each one or put the change into |\everypar|
\ctsref\everypar.
^^|\everypar//for setting \b\tt\\looseness\e|
\enddesc
\enddescriptions
\end