\input macros
\begindescriptions
\begindesc
\cts obeylines {}
\explain
\TeX\ normally treats an end of line as a space.
|\obeylines| instructs \TeX\ to treat each end of line as
an end of paragraph, thus forcing a line break.
|\obeylines| is often useful when you're typesetting verse or
computer programs.
^^{verse, typesetting}^^{poetry, typesetting}^^{computer programs, typesetting}
If any of your lines are longer than the effective line length
(|\hsize|\tminus|\parindent|),
however,
you may get an extra line break within those lines.

Because \TeX\ inserts the |\parskip| glue (\xref\parskip)
between lines controlled by |\obeylines| (since it thinks each line is a
paragraph), you should normally set |\parskip| to zero when you're using
|\obeylines|.

You can use the ^|\obeyspaces| command (\xref{\obeyspaces}) to get
\TeX\ to take spaces within a line literally.  |\obeylines| and |\obeyspaces|
are often used together.
\example 
\obeylines
``Beware the Jabberwock, my son!!
\quad The jaws that bite, the claws that catch!!
Beware the Jubjub bird, and shun
\quad The frumious Bandersnatch!!''
|
\produces
\obeylines
``Beware the Jabberwock, my son!
\quad The jaws that bite, the claws that catch!
Beware the Jubjub bird, and shun
\quad The frumious Bandersnatch!''
\endexample
\enddesc
\enddescriptions
\end