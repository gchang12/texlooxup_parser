\input macros
\begindescriptions
\begindesc
\bix^^{fonts//naming and modifying}
\cts font {}
\aux\cts font {\<control sequence> = \<fontname>}
\aux\cts font {\<control sequence> = \<fontname> {\bt scaled} \<number>}
\aux\cts font {\<control sequence> = \<fontname> {\bt at} \<dimen>}
\explain
Used alone, the |\font| control sequence designates the current font.
|\font| isn't a true command when it's used alone, 
since it then can appear only as an argument to another command.

For the other three forms of |\font|,
\<font\-name> names a set of files that define a font.
These forms of |\font|  are commands.  Each of these forms has two effects:
{\tighten
\olist
\li It defines \<control sequence> as a name that selects
the font \<font\-name>, possibly magnified (see below).
\li It causes \TeX\ to load the font ^{metrics file}
(^{\tfmfile}) for \<fontname>.
\endolist
}% end tighten

\noindent
The name of a font file usually indicates its design size.
For example, |cmr10| indicates Computer Modern roman with a
design size of $10$ points.
The design size of a font is recorded in its metrics file.

If neither |scaled| \<number> nor |at| \<dimen>
is present, the font is used 
at its design size---the size at which it usually looks best.
Otherwise, a magnified version of the font is loaded:
\ulist
\li If |scaled| \<number>
is present, the font is magnified by a factor of $\hbox{\<number>}/1000$.
\li If |at| \<dimen> is present, the font is scaled to \<dimen> by magnifying
it by $\hbox{\<dimen>}/ds$, where $ds$ is the design size of
\<fontname>.
\<dimen> and $ds$ are nearly always given in points.
\endulist
\noindent
Magnifications of less than $1$ are possible; they reduce the size.

You usually need to provide a shape file (\xref{shape}) for each
magnification of a font that you load.
However, some ^{device drivers} can utilize fonts that are resident
in a printer. ^^{resident fonts}
Such fonts don't need shape files.

See \conceptcit{font} and
\conceptcit{magnification} for further information.

\example
\font\tentt = cmtt10
\font\bigttfont = cmtt10 scaled \magstep2
\font\eleventtfont = cmtt10 at 11pt
First we use {\tentt regular CM typewriter}.
Then we use {\eleventtfont eleven-point CM typewriter}.
Finally we use {\bigttfont big CM typewriter}.
|
\produces
\font\regttfont = cmtt10
\font\bigttfont = cmtt10 scaled \magstep 2
\font\eleventtfont = cmtt10 at 11pt
First we use {\regttfont regular CM typewriter}.
Then we use {\eleventtfont eleven-point CM typewriter}.
Finally we use {\bigttfont big CM typewriter}.
\endexample
\enddesc
\enddescriptions
\end