\input macros
\begindescriptions
\begindesc
\cts magnification {{\bt =} \<number>}
\cts mag {\param{number}}
\explain
\margin{{\tt\\mag} and {\tt\\magnification} combined.}
An assignment to |\magnification| establishes 
the ``^{scale factor}'' $f$ that determines
the \minref{magnification} ratio of your document \seeconcept{magnification}.
The assignment to |\magni!-fication| must occur before the first page
of your document has been shipped out.

The assignment sets $f$ to \<number> and also
sets |\hsize| and |\vsize|
^^|\hsize//set by {\tt\\magnification}|
^^|\vsize//set by {\tt\\magnification}|
respectively to |6.5true in| and |8.9true in|,
the values appropriate for an $8 \frac1/2$-%
by-$11$-inch page.
$f$ must be between $0$ and $32768$.
The \minref{magnification} ratio of the
document is $f/1000$.
A scale factor
of $1000$ provides unit magnification, i.e., it leaves the size of your
document
unchanged.  It's customary to use powers of $1.2$ as scale factors, and
most libraries of fonts are based on such factors.  You can use the
^|\magstep| and ^|\magstephalf| commands to specify magnifications by
these factors.

|\magnification| is not a parameter. You can't use it
to \emph{retrieve} the scale factor.  If you write something like
|\dimen0 = \mag!-nifi!-cation|, \TeX\ will complain about it.

The |\mag| parameter contains the scale factor.
Changing the value of |\mag| rescales the page dimensions, which is not
usually what you want.
Therefore it's usually better to change the magnification by
assigning to |\magnification| rather than to |\mag|.

\example
\magnification = \magstep2 
% magnify fonts by 1.44 (=1.2x1.2)
|
\endexample
\enddesc
\enddescriptions
\end