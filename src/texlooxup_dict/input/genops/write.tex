\input macros
\begindescriptions
\begindesc
\cts write {\<number> \rqbraces{\<token list>}}
\explain
^^{output streams//writing}
^^{writing a file}
This command tells \TeX\ to write \<token list> to the file
associated with the \minref{output stream}
designated by \<number>.
It generates a whatsit that becomes part of a box.
The actual writing does not take place until \TeX\ ships out that box
to the \dvifile,
unless you've preceded the |\write| with ^|\immediate|.

For a |\write| that is not immediate, \TeX\ does not expand macros in
\<token list> until the token list is actually written to the file.
The macro expansions follow the same rules as |\edef| (\xref\edef).
In particular, any control sequence that is not 
the name of a macro is written as
^|\escapechar| followed by the control sequence name
and a space.  Any `|#|' tokens
in \<token list> are doubled, i.e., written as `|##|'.

If \<number> is not in the range from $0$ to $15$, \TeX\ writes
\<token list> to the log file.
^^{log file//written by \b\tt\\write\e}
If \<number> is greater than $15$ or isn't associated with an output
stream, \TeX\ also writes \<token list> to the terminal.
\example
\def\aa{a a}
\write\auxfile{\hbox{$x#y$} \aa}
% Writes the string `\hbox {$x##y$} a a' to \auxfile.
|
\endexample
\enddesc
\enddescriptions
\end