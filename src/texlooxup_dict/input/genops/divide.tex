\input macros
\begindescriptions
\begindesc
\cts multiply {\<register> {\bt by} \<number>}
\cts divide {\<register> {\bt by} \<number>}
\explain
These commands multiply and divide the value in \<register>
by \<number> (which can be negative).
The register can be a ^|\count|, ^|\dimen|, ^|\skip|, or ^|\muskip|
register.
For a ^|\skip| or ^|\muskip| register (\xref\skip),
all three components of the \minref{glue} in the register are modified.
You can omit the word |by| in these commands---\TeX\ will understand them
anyway.

You can also obtain a multiple of a \<dimen> by preceding it by a \<number>
\minrefs{number}
or decimal constant, e.g.,
|-2.5\dimen2|.
You can also use this notation for \<glue>, but watch out---the result
is a \<dimen>, not \<glue>.
Thus |2\baselineskip| yields a \<dimen> that is twice the natural size
of |\baselineskip|, with no stretch or shrink.
\example
\count0 = 9\multiply \count0 by 8 \number\count0 ;
\divide \count0 by 12 \number\count0 \par
\skip0 = 20pt plus 2pt minus 3pt \multiply \skip0 by 3
Multiplied value of skip0 is \the\skip0.\par
\dimen0 = .5in \multiply\dimen0 by 6
\hbox to \dimen0{a\hfil b}
|
\produces
\count0 = 9\multiply \count0 by 8 \number\count0 ;
\divide \count0 by 12 \number\count0 \par
\skip0 = 20pt plus 2pt minus 3pt \multiply \skip0 by 3
Multiplied value of skip0 is \the\skip0.\par
\dimen0 = .5in \multiply\dimen0 by 6
\hbox to \dimen0{a\hfil b}
\doruler{\8\8\8}3{in}
\endexample

\eix^^{arithmetic}
\eix^^{registers//arithmetic in}
\eix^^{registers}
\enddesc
\enddescriptions
\end