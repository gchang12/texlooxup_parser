\input macros

\begindescriptions
{\def\test{{\bt \\if}$\Omega$}%
\begindesc
\bix^^{repeated actions}
\bix^^{loops}
\cts loop {$\alpha$ {\test} $\beta$ {\bt \\repeat}}
\ctspecial repeat {}\ctsxrdef{@repeat}
\explain
These commands provide a looping construct for \TeX.
Here $\alpha$ and $\beta$ are arbitrary sequences of commands
and \test\ is any of the conditional tests described in
\headcit{Conditional tests}{conds}.
The |\repeat| replaces the |\fi| corresponding to the test,
so you must not write an explicit |\fi| to terminate the test.
Nor, unfortunately, can you associate an |\else| with the test.
If you want to use the test in the opposite sense, you need to
rearrange the test or
define an auxiliary test with |\newif| (see above) and use that
test in the sense you want (see the second example below).

\TeX\ expands |\loop| as follows:
\olist
\li $\alpha$ is expanded.
\li {\test} is performed.  If the result is false, the loop is terminated.
\li $\beta$ is expanded.
\li The cycle is repeated.
\endolist
\example
\count255 = 6
\loop
\number\count255\
\ifnum\count255 > 0
\advance\count255 by -1
\repeat
|
\produces
\count255 = 6
\loop
\number\count255\
\ifnum\count255 > 0
\advance\count255 by -1
\repeat
\nextexample
\newif\ifnotdone % \newif uses \count255 in its definition
\count255=6
\loop
\number\count255\
\ifnum\count255 < 1 \notdonefalse\else\notdonetrue\fi
\ifnotdone
\advance\count255 by -1
\repeat
|
\produces
\newif\ifnotdone
\count255=6
\loop
\number\count255\
\ifnum\count255 < 1 \notdonefalse\else\notdonetrue\fi
\ifnotdone
\advance\count255 by -1
\repeat
%
\eix^^{repeated actions}
\eix^^{loops}
%
\endexample
\enddesc
}
\enddescriptions
\end
