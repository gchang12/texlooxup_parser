\input macros

\begindescriptions

\begindesc
\cts openup {\<dimen>}
\explain
This command increases ^|\baselineskip| by \<dimen>.
An |\openup| command before the end of a paragraph affects
the entire paragraph, so you shouldn't use |\openup| to
change |\baseline!-skip| within a paragraph.  |\openup| is
most useful for typesetting tables and math displays---a
little extra space between rows often makes them more readable.
^^{alignments//space between rows of}
\example
Alice picked up the White King very gently, and lifted him
across more slowly than she had lifted the Queen; but before
she put him on the table, she thought she might well dust
him a little, he was so covered with ashes.
\openup .5\baselineskip % 1.5 linespacing.
|
\produces
Alice picked up the White King very gently, and lifted him
across more slowly than she had lifted the Queen; but before
she put him on the table, she thought she might well dust
him a little, he was so covered with ashes.
\openup .5\baselineskip %1.5 linespacing
\endexample\enddesc

%==========================================================================
\section {Page breaks}

%==========================================================================
\subsection {Encouraging or discouraging page breaks}

\begindesc
\bix^^{page breaks}
\bix^^{page breaks//encouraging or discouraging}
\ctspecial break {} \xrdef{vbreak}
\explain
%
\margin{Four commands identical to ones for line breaks (\xref{hbreak})
have been added to correct an omission. The descriptions are exactly parallel.}
%
This command forces a page break.
Unless you do something to fill out the page, you're likely to
get an underfull vbox.
|\break| can also be used in horizontal mode.
\enddesc

\enddescriptions
\end