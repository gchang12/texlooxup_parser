\input macros

\begindescriptions

\begindesc
\cts baselineskip {\param{glue}}
\cts lineskiplimit {\param{dimen}}
\cts lineskip {\param{glue}}
\explain
^^{line spacing}
\bix^^{interline glue}
\bix^^|\baselineskip|
\bix^^|\lineskip|
\bix^^|\lineskiplimit|
These three parameters jointly determine how much space \TeX\ leaves between
consecutive \minref{box}es of an ordinary \minref{vertical list},
e.g., the lines of a paragraph.
This space is called ``\minref{interline glue}''.
It is also inserted between the component boxes of a vbox constructed in
internal vertical mode.
^^{vboxes//interline glue for}

In the usual case, when the boxes aren't abnormally high or deep, \TeX\
makes the distance from the baseline of one box to the baseline of the
next one equal to |\baselineskip|.  It does this by inserting interline
glue equal to |\baselineskip| minus the depth of the upper box (as given
by ^|\prevdepth|) and the height of the lower box.  But if this
interline glue would be less than |\lineskiplimit|, indicating that the
two boxes are too close together, \TeX\ inserts the |\lineskip| glue
instead.\footnote
{\TeX\ actually accounts for the beginning of a
vertical list by setting |\prevdepth| to $-1000$\pt\ and testing
|\prevdepth| before \emph{every} box.  If |\prevdepth|$\>\le-1000$\pt\
it does not insert any interline glue.} See \knuth{pages~79--80} for a
precise description.

Note that |\baselineskip| and |\lineskip| measure \emph{different
things}: the distance between baselines on the one hand and the distance
between the bottom of one box and the top of the next box on the other
hand.  See \knuth{page~78} for further details.  The first example below
shows the effects of |\lineskiplimit|.

You can obtain the effect of ^{double spacing} by doubling the value
of |\baselineskip| as illustrated in the second example below.
A change to |\baselineskip| at any point before the end of a paragraph affects
the entire paragraph.

\example
\baselineskip = 11pt \lineskiplimit = 1pt
\lineskip = 2pt plus .5pt
Sometimes you'll need to typeset a paragraph that has
tall material, such as a mathematical formula,  embedded
within it.  An example of such a formula is $n \choose k$.
Note the extra space above and below this line as
compared with the other lines.
(If the formula didn't project below the line,
we'd only get extra space above the line.)
|
\produces
\baselineskip = 11pt \lineskiplimit = 1pt
\lineskip = 2pt plus .5pt
Sometimes you'll need to typeset a paragraph that has
tall material, such as a mathematical formula,  embedded
within it.  An example of such a formula is $n \choose k$.
Note the extra space above and below this line as
compared with the other lines.
(If the formula didn't project below the line,
we'd only get extra space above the line.)
\endexample

\example
\baselineskip = 2\baselineskip % Start double spacing.
|
\endexample

\eix^^{interline glue}
\eix^^|\baselineskip|
\eix^^|\lineskip|
\eix^^|\lineskiplimit|
\enddesc

\enddescriptions
\end