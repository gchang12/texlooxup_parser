\input macros

\begindescriptions

\begindesc
\cts jobname {}
\explain
This command produces the base
name of the file with which \TeX\ was invoked.
For example, if your main input file is |hatter.tex|,
|\jobname|
{\parfillskip=0pt\par\eject\noindent}
will expand to |hatter|.
|\jobname| is most useful when you're
creating an auxiliary file to be associated with a document.
^^{auxiliary files}
\example
\newwrite\indexfile  \openout\indexfile = \jobname.idx
% For input file `hatter.tex', open index file `hatter.idx'.
|
\endexample\enddesc

%==========================================================================
\subsection {Values of variables}

\begindesc
\cts meaning {\<token>}
\explain
^^{tokens//showing the meaning of}
This command produces
the meaning of \<token>.  It is useful for diagnostic output.
You can use the ^|\the| command (\xref\the) in a similar way
to get information about the values of \minref{register}s and other
\TeX\ entities.
\example
[{\tt \meaning\eject}] [\meaning\tenrm] [\meaning Y]
|
\produces
[{\tt \meaning\eject}] [\meaning\tenrm] [\meaning Y]
\endexample\enddesc

\begindesc
\cts string {\<control sequence>}
\explain
^^{control sequences//converting to strings}
This command produces
the characters that form the name of \<control sequence>,
including the \minref{escape character}.
The escape character is represented by the current value of
^|\escapechar|.
^^{escape character//represented by \b\tt\\escapechar\e}
\TeX\ gives the characters in the list a category code of $12$ (other).

You can perform the reverse operation with
the ^|\csname| command (\xref \csname),
which turns a string into a control sequence.
\example
the control sequence {\tt \string\bigbreak}
|
\produces
the control sequence {\tt \string\bigbreak}
\endexample\enddesc

\begindesc
\cts escapechar {\param{number}}
\explain
This parameter specifies the \ascii\ code \minrefs{\ascii} of the
character that \TeX\ uses to represent the \minref{escape character}
^^{escape character//represented by \b\tt\\escapechar\e}
when it's
converting a control sequence name to a sequence of character tokens.
This conversion occurs when you use the |\string| command and also when
\TeX\ is producing diagnostic messages.  The default value of the escape
character is $92$, the {\ascii} character code for a ^{backslash}.
If |\escapechar| is not in the range $0$--$255$,
\TeX\ does not include an escape character in the result of the conversion.
\example
\escapechar = `!!
the control sequence {\tt \string\bigbreak}
|
\produces
\escapechar = `!
the control sequence {\tt \string\bigbreak}
\endexample
\enddesc

\enddescriptions
\end