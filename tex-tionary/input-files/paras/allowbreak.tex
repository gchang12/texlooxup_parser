\input macros

\begindescriptions

\begindesc
\ctspecial allowbreak {} \xrdef{hallowbreak}
\explain
This command tells \TeX\ to
allow a line break where one could not ordinarily occur.
It's most often useful within a math formula, since \TeX\
is reluctant to break lines there.  ^^{line breaks//in math formulas}
|\allowbreak| can also be used in vertical mode.
\example
Under most circumstances we can state with some confidence
that $2+2\allowbreak=4$, but skeptics may disagree.
\par For such moronic automata, it is not difficult to
analyze the input/\allowbreak output behavior in the limit.
|
\produces
Under most circumstances we can state with some confidence
that $2+2\allowbreak=4$, but skeptics may disagree.
\par For such moronic automata, it is not difficult to
analyze the input/\allowbreak output behavior in the limit.
\endexample\enddesc

\begindesc
\ctspecial penalty {\<number>} \xrdef{hpenalty}
\explain
This command produces a \minref{penalty} item.
The penalty item makes \TeX\ more or less willing to break a line
at the point where that item occurs.
A negative penalty, i.e., a bonus, encourages a line break;
a positive penalty discourages a line break.
A penalty of $10000$ or more prevents a break altogether,
while a penalty of $-10000$ or less forces a break.
|\penalty| can also be used in vertical mode.
\secondprinting{\vfill\eject}
\example
\def\break{\penalty -10000 } % as in plain TeX
\def\nobreak{\penalty 10000 } % as in plain TeX
\def\allowbreak{\penalty 0 } % as in plain TeX
|
\endexample
\enddesc

\enddescriptions
\end