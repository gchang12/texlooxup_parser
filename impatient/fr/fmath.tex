% This is part of the book TeX for the Impatient.
% Copyright (C) 2003 Paul W. Abrahams, Kathryn A. Hargreaves, Karl Berry.
% Copyright (C) 2004 Marc Chaudemanche pour la traduction fran�aise.
% See file fdl.tex for copying conditions.

\input fmacros
\chapter {Commandes pour \linebreak composer des \linebreak formules math\'ematiques}

\bix^^{math\'ematiques}
\chapterdef{math}

Cette section couvre les commandes pour construire des formules math\'e\-matiques.
Pour une explication des conventions utilis\'ees dans cette section,
voir \headcit{Description des commandes}{cmddesc}.

\begindescriptions
%==========================================================================
\section {Parties simples de formules}

%==========================================================================
\subsection {Lettres grecques}

\begindesc
\bix^^{lettres grecques}
\dothreecolumns 40
\easy\ctsdisplay alpha {}
\ctsdisplay beta {}
\ctsdisplay chi {}
\ctsdisplay delta {}
\ctsdisplay Delta {}
\ctsdisplay epsilon {}
\ctsdisplay varepsilon {}
\ctsdisplay eta {}
\ctsdisplay gamma {}
\ctsdisplay Gamma {}
\ctsdisplay iota {}
\ctsdisplay kappa {}
\ctsdisplay lambda {}
\ctsdisplay Lambda {}
\ctsdisplay mu {}
\ctsdisplay nu {}
\ctsdisplay omega {}
\ctsdisplay Omega {}
\ctsdisplay phi {}
\ctsdisplay varphi {}
\ctsdisplay Phi {}
\ctsdisplay pi {}
\ctsdisplay varpi {}
\ctsdisplay Pi {}
\ctsdisplay psi {}
\ctsdisplay Psi {}
\ctsdisplay rho {}
\ctsdisplay varrho {}
\ctsdisplay sigma {}
\ctsdisplay varsigma {}
\ctsdisplay Sigma {}
\ctsdisplay tau {}
\ctsdisplay theta {}
\ctsdisplay vartheta {}
\ctsdisplay Theta {}
\ctsdisplay upsilon {}
\ctsdisplay Upsilon {}
\ctsdisplay xi {}
\ctsdisplay Xi {}
\ctsdisplay zeta {}
\egroup
\explain
Ces commandes produisent les lettres grecques appropri\'ees aux math\'e\-matiques. 
Vous ne pouvez les utiliser que dans une formule math\'ema\-tique, ainsi si vous 
avez besoin d'une lettre grecque dans du texte ordinaire, vous devez l'englober 
entre des signes dollar (|$|). \TeX\ n'a pas de commandes pour des lettres 
grecques qui ressemblent \`a leurs contre-parties romaines, puisque vous pouvez 
les obtenir en employant ces contre-parties romaines. Par exemple, vous pouvez 
obtenir un ^{omicron} minuscule dans une formule en \'ecrivant la lettre `o', 
c'est-\`a-dire, `|{\rm o}|' ou un ^{beta} majuscule (`B') en \'ecrivant `|{\rm B}|'.

Ne confondez pas les lettres suivantes~:
\ulist \compact
\li |\upsilon| (`$\upsilon$'), |{\rm v}| (`v'), et |\nu| (`$\nu$').
\li |\varsigma| (`$\varsigma$') and |\zeta| (`$\zeta$').
\endulist

Vous pouvez obtenir des lettres grecques capitales inclin\'ees en utilisant 
la \minref{police} math\'ematique italique (|\mit|).

\TeX \ traite les lettres grecques comme des symboles ordinaires quand 
il calcule combien d'espace mettre autour d'eux.

\example
If $\rho$ and $\theta$ are both positive, then $f(\theta)
-{\mit \Gamma}_{\theta} < f(\rho)-{\mit \Gamma}_{\rho}$.
|
\produces
If $\rho$ and $\theta$ are both positive, then
$f(\theta)-{\mit \Gamma}_{\theta} < f(\rho)-{\mit \Gamma}_{\rho}$.
\endexample
\eix^^{lettres grecques}
\enddesc

%==========================================================================
\subsection {Divers Symboles math\'ematiques ordinaires}

\begindesc
\xrdef{specsyms}
\dothreecolumns 34
\easy\ctsdisplay infty {}
\ctsdisplay Re {}
\ctsdisplay Im {}
\ctsdisplay angle {}
\ctsdisplay triangle {}
\ctsdisplay backslash {}
\ctsdisplay vert {}
\ctsydisplay | @bar {}
\ctsdisplay Vert {}
\ctsdisplay emptyset {}
\ctsdisplay bot {}
\ctsdisplay top {}
\ctsdisplay exists {}
\ctsdisplay forall {}
\ctsdisplay hbar {}
\ctsdisplay ell {}
\ctsdisplay aleph {}
\ctsdisplay imath {}
\ctsdisplay jmath {}
\ctsdisplay nabla {}
\ctsdisplay neg {}
\ctsdisplay lnot {}
\actdisplay ' @prime \ (apostrophe)
\ctsdisplay prime {}
\ctsdisplay partial {}
\ctsdisplay surd {}
\ctsdisplay wp {}
\ctsdisplay flat {}
\ctsdisplay sharp {}
\ctsdisplay natural {}
\ctsdisplay clubsuit {}
\ctsdisplay diamondsuit {}
\ctsdisplay heartsuit {}
\ctsdisplay spadesuit {}
\egroup
\explain
^^{symboles musicaux} ^^{couleurs de carte \`a jouer}
Ces commandes produisent des symboles vari\'es.  On les appelle
``^{symboles ordinaires}'' pour les distinguer des autres classes de
symboles comme les relations. Vous ne pouvez utiliser un 
symbole ordinaire que dans une formule math\'ematique,
donc si vous avez besoin d'un symbole ordinaire dans du texte ordinaire
vous devez l'englober entre des signes dollar (|$|).

Les commandes |\imath| et |\jmath| sont utiles quand vous devez mettre un
accent sur un `$i$' ou un `$j$'.

Une apostrophe (|'|) est un raccourci pour \'ecrire un |\prime| en exposant.  (La
commande |\prime| de son c\^ot\'e, g\'en\`ere un gros prime laid.)

Les commandes |\!|| et ^|\Vert| sont synonymes, comme
le sont les commandes ^|\neg| et ^|\lnot|.
\margin{explanation of {\tt\\vert} added}
La commande |\vert| produit le m\^eme r\'esultat que `|!||'.
\indexchar |

Les symboles produits par |\backslash|, |\vert| et |\Vert|
sont des \minref{d\'eli\-mi\-teur}s.  ces symboles peuvent \^etre produit en grandes taille
en utilisant ^|\bigm| etc.\ (\xref \bigm).  

\example
The Knave of $\heartsuit$s, he stole some tarts.
|
\produces
The Knave of $\heartsuit$s, he stole some tarts.
\nextexample
If $\hat\imath < \hat\jmath$ then $i' \leq j^\prime$.
|
\produces
If $\hat\imath < \hat\jmath$ then $i' \leq j^\prime$.
\nextexample
$${{x-a}\over{x+a}}\biggm\backslash{{y-b}\over{y+b}}$$
|
\dproduces
$${{x-a}\over{x+a}}\biggm\backslash{{y-b}\over{y+b}}$$
\endexample
\enddesc

%==========================================================================
\subsection {Op\'erations binaires}

\begindesc
\bix^^{op\'erations}
\xrdef{binops}
\dothreecolumns 34
\easy\ctsdisplay vee {}
\ctsdisplay wedge {}
\ctsdisplay amalg {}
\ctsdisplay cap {}
\ctsdisplay cup {}
\ctsdisplay uplus {}
\ctsdisplay sqcap {}
\ctsdisplay sqcup {}
\ctsdisplay dagger {}
\ctsdisplay ddagger {}
\ctsdisplay land {}
\ctsdisplay lor {}
\ctsdisplay cdot {}
\ctsdisplay diamond {}
\ctsdisplay bullet {}
\ctsdisplay circ {}
\ctsdisplay bigcirc {}
\ctsdisplay odot {}
\ctsdisplay ominus {}
\ctsdisplay oplus {}
\ctsdisplay oslash {}
\ctsdisplay otimes {}
\ctsdisplay pm {}
\ctsdisplay mp {}
\ctsdisplay triangleleft {}
\ctsdisplay triangleright {}
\ctsdisplay bigtriangledown {}
\ctsdisplay bigtriangleup {}
\ctsdisplay ast {}
\ctsdisplay star {}
\ctsdisplay times {}
\ctsdisplay div {}
\ctsdisplay setminus {}
\ctsdisplay wr {}
\egroup
\explain
Ces commandes produisent les symboles pour diff\'erentes op\'erations binaires.
Les op\'erations binaires sont une des \minref{classe}s des symboles math\'e\-matiques de \TeX.
\TeX\ met diff\'erents montants d'espace autour de dif\-f\'e\-rentes classes de symboles
math\'ematiques.  Quand \TeX\ a besoin de couper une ligne de texte dans une formule math\'ematique, 
\minrefs{line break} il devra placer la coupure apr\`es une op\'eration binaire---mais seulement si 
l'op\'eration est \`a un niveau sup\'erieur \`a la formule, c'est-\`a-dire, non incluse dans un groupe.

En plus de ces commandes, \TeX\ traite \'egalement `|+|' et `|-|' 
en tant qu'op\'erations binaires. Il consid\`ere `|/|' comme un symbole ordinaire, 
en d\'epit du fait que, math\'ematiquement, ce soit une op\'eration binaire, 
parce qu'il semble mieux avec moins d'espace autour de lui.

\example
$$z = x \div y \quad \hbox{if and only if} \quad
z \times y = x \;\hbox{and}\; y \neq 0$$
|
\dproduces
$$z = x \div y \quad \hbox{if and only if} \quad
z \times y = x \;\hbox{and}\; y \neq 0$$
\endexample
\enddesc

\begindesc
\ctspecial * \ctsxrdef{@star}
\explain
La commande |\*| indique un symbole de multiplication optionnel
($\times$), qui est une op\'eration binaire.  Ce symbole de multiplication
agit comme une c\'esure optionnelle quand il appara\^\i t dans une formule dans un
texte\minrefs{texte math\'ematique}.  Cela \'etant, \TeX\ ne composera le symbole |\times|
\emph{que} si la formule doit \^etre coup\'ee en ce point.
Il n'y a aucun point en utilisant |\*| dans des formules affich\'ees \minrefs{math\'ematique 
affich\'ee} car \TeX\ ne coupe jamais de formules affich\'ees de lui-m\^eme.

\example
Let $c = a\*b$. In the case that $c=0$ or $c=1$, let
$\Delta$ be $(\hbox{the smallest $q$})\*(\hbox{the
largest $q$})$ in the set of approximate $\tau$-values.
|
\produces
Let $c = a\*b$. In the case that $c=0$ or $c=1$, let
$\Delta$ be $(\hbox{the smallest $q$})\*(\hbox{the
largest $q$})$ in the set of approximate $\tau$-values.

\eix^^{op\'erations}
\endexample
\enddesc

%==========================================================================
\subsection {Relations}

\begindesc
\xrdef {relations}
\bix^^{relations}
\dothreecolumns 39
\easy\ctsdisplay asymp {}
\ctsdisplay cong {}
\ctsdisplay dashv {}
\ctsdisplay vdash {}
\ctsdisplay perp {}
\ctsdisplay mid {}
\ctsdisplay parallel {}
\ctsdisplay doteq {}
\ctsdisplay equiv {}
\ctsdisplay ge {}
\ctsdisplay geq {}
\ctsdisplay le {}
\ctsdisplay leq {}
\ctsdisplay gg {}
\ctsdisplay ll {}
\ctsdisplay models {}
\ctsdisplay ne {}
\ctsdisplay neq {}
\ctsdisplay notin {}
\ctsdisplay in {}
\ctsdisplay ni {}
\ctsdisplay owns {}
\ctsdisplay prec {}
\ctsdisplay preceq {}
\ctsdisplay succ {}
\ctsdisplay succeq {}
\ctsdisplay bowtie {}
\ctsdisplay propto {}
\ctsdisplay approx {}
\ctsdisplay sim {}
\ctsdisplay simeq {}
\ctsdisplay frown {}
\ctsdisplay smile {}
\ctsdisplay subset {}
\ctsdisplay subseteq {}
\ctsdisplay supset {}
\ctsdisplay supseteq {}
\ctsdisplay sqsubseteq {}
\ctsdisplay sqsupseteq {}
\egroup
\explain
Ces commandes produisent des symboles pour diff\'erentes relations.
Les relations sont une des \minref{classe}s des symboles math\'ematiques.
\TeX\ met diff\'erents montants d'espace autour de diff\'erentes classes de symboles
math\'ematiques.
Quand \TeX\ a besoin de couper une ligne de texte dans une formule math\'ematique, 
\minrefs{line break} il devra placer la coupure apr\`es une relation---mais seulement si 
l'op\'eration est \`a un niveau sup\'erieur \`a la formule, c'est-\`a-dire, non incluse dans un groupe.

En plus des commandes list\'ees ici, \TeX\ traite  `^|=|' et les 
commandes ``fl\`eche'' (\xref{arrows}) comme des relations.

Certaines relations ont plus d'une commande que vous pouvez utiliser
pour les produire~:
\ulist \compact
\li `$\ge$' (|\ge| et |\geq|).
\li `$\le$' (|\le| et |\leq|).
\li `$\ne$' (|\ne|, |\neq| et |\not=|).
\li `$\ni$' (|\ni| et |\owns|).
\endulist

\xrdef{\not}
Vous pouvez produire des relations n\'egatives en les pr\'efixant avec |\not|, comme cela~:

\nobreak
\threecolumns 21
\basicdisplay {$\not\asymp$}{\\not\\asymp}\ctsidxref{asymp}
\basicdisplay {$\not\cong$}{\\not\\cong}\ctsidxref{cong}
\basicdisplay {$\not\equiv$}{\\not\\equiv}\ctsidxref{equiv}
\basicdisplay {$\not=$}{\\not=}\ttidxref{=}
\basicdisplay {$\not\ge$}{\\not\\ge}\ctsidxref{ge}
\basicdisplay {$\not\geq$}{\\not\\geq}\ctsidxref{geq}
\basicdisplay {$\not\le$}{\\not\\le}\ctsidxref{le}
\basicdisplay {$\not\leq$}{\\not\\leq}\ctsidxref{leq}
\basicdisplay {$\not\prec$}{\\not\\prec}\ctsidxref{prec}
\basicdisplay {$\not\preceq$}{\\not\\preceq}\ctsidxref{preceq}
\basicdisplay {$\not\succ$}{\\not\\succ}\ctsidxref{succ}
\basicdisplay {$\not\succeq$}{\\not\\succeq}\ctsidxref{succeq}
\basicdisplay {$\not\approx$}{\\not\\approx}\ctsidxref{approx}
\basicdisplay {$\not\sim$}{\\not\\sim}\ctsidxref{sim}
\basicdisplay {$\not\simeq$}{\\not\\simeq}\ctsidxref{simeq}
\basicdisplay {$\not\subset$}{\\not\\subset}\ctsidxref{subset}
\basicdisplay {$\not\subseteq$}{\\not\\subseteq}\ctsidxref{subseteq}
\basicdisplay {$\not\supset$}{\\not\\supset}\ctsidxref{supset}
\basicdisplay {$\not\supseteq$}{\\not\\supseteq}\ctsidxref{supseteq}
\basicdisplay {$\not\sqsubseteq$}{\\not\\sqsubseteq}%
   \ctsidxref{sqsubseteq}
\basicdisplay {$\not\sqsupseteq$}{\\not\\sqsupseteq}%
   \ctsidxref{sqsupseteq}
\egroup

\example
We can show that $AB \perp AC$, and that
$\triangle ABF \not\sim \triangle ACF$.
|
\produces
We can show that $AB \perp AC$, and that
$\triangle ABF \not\sim \triangle ACF$.

\eix^^{relations}
\endexample
\enddesc

%==========================================================================
\subsection {D\'elimiteurs gauches et droits}

\begindesc
\bix^^{d\'elimiteurs}
%
\dothreecolumns 12
\easy\ctsdisplay lbrace {}
\ctsydisplay { @lbrace {}
\ctsdisplay rbrace {}
\ctsydisplay } @rbrace {}
\ctsdisplay lbrack {}
\ctsdisplay rbrack {}
\ctsdisplay langle {}
\ctsdisplay rangle {}
\ctsdisplay lceil {}
\ctsdisplay rceil {}
\ctsdisplay lfloor {}
\ctsdisplay rfloor {}
\egroup
\explain
Ces commandes produisent des \minref{d\'elimiteur}s gauches et droits.
Les math\'e\-maticiens utilisent les d\'elimiteurs pour indiquer les fronti\`eres entre les parties
d'une formule.  Les d\'elimiteurs gauches sont aussi appel\'es ``^{ouvrant}s'' et
les d\'elimiteurs droits sont aussi appel\'es ``^{fermant}s''.  Ouvrants et Fermants
sont deux des \minref{classe}s de symboles math\'ematiques de \TeX.  \TeX\ met
diff\'erents montants d'espace autour des diff\'erentes \minref{classe}s de symboles math\'ematiques.
Vous devriez vous attendre \`a ce que l'espacement que \TeX\ met autour des ouvrants et
des fermants soit sym\'etrique, mais en fait, il ne l'est pas.

Certains d\'elimiteurs gauches et droits ont plus d'une commande utilisable pour les produire~:

\ulist\compact
\li `$\{$' (|\lbrace| and |\{|)
\li `$\}$' (|\rbrace| and |\}|)
\li `$[$' (|\lbrack| and `|[|')
\li `$]$' (|\rbrack| and `|]|')
\endulist
\noindent vous pouvez aussi utiliser les caract\`eres crochets gauches et droits
(sous toutes ses formes) en dehors du mode math\'ematique.

En plus de ces commandes, \TeX\ traite `|(|' comme un d\'elimiteur gauche
et `|)|' comme un d\'elimiteur droit.

Vous pouvez laisser \TeX\
choisir la taille d'un d\'elimiteur en utilisant |\left| et |\right| (\xref\left).
Alternativement,
vous pouvez obtenir un d\'eli\-miteur d'une taille sp\'ecifique en utilisant une des commandes |\big|$x$
(voir |\big| et al., \xref{\big}).

\example
The set $\{\,x \mid x>0\,\}$ is empty.
|
\produces
The set $\{\,x \mid x>0\,\}$ is empty.

\eix^^{d\'elimiteurs}
\endexample
\enddesc

%==========================================================================
\subsection {Fl\`eches}

\begindesc
\bix^^{fl\`eches}
\xrdef{arrows}
%
{\symbolspace=24pt \makecolumns 34/2:
\easy%
\ctsdisplay leftarrow {}
\ctsdisplay gets {}
\ctsdisplay Leftarrow {}
\ctsdisplay rightarrow {}
\ctsdisplay to {}
\ctsdisplay Rightarrow {}
\ctsdisplay leftrightarrow {}
\ctsdisplay Leftrightarrow {}
\ctsdisplay longleftarrow {}
\ctsdisplay Longleftarrow {}
\ctsdisplay longrightarrow {}
\ctsdisplay Longrightarrow {}
\ctsdisplay longleftrightarrow {}
\ctsdisplay Longleftrightarrow {}
\basicdisplay {$\Longleftrightarrow$}{\\iff}\pix\ctsidxref{iff}\xrdef{\iff}
\ctsdisplay hookleftarrow {}
\ctsdisplay hookrightarrow {}
\ctsdisplay leftharpoondown {}
\ctsdisplay rightharpoondown {}
\ctsdisplay leftharpoonup {}
\ctsdisplay rightharpoonup {}
\ctsdisplay rightleftharpoons {}
\ctsdisplay mapsto {}
\ctsdisplay longmapsto {}
\ctsdisplay downarrow {}
\ctsdisplay Downarrow {}
\ctsdisplay uparrow {}
\ctsdisplay Uparrow {}
\ctsdisplay updownarrow {}
\ctsdisplay Updownarrow {}
\ctsdisplay nearrow {}
\ctsdisplay searrow {}
\ctsdisplay nwarrow {}
\ctsdisplay swarrow {}
}
\explain
Ces commandes procurent des fl\`eches de diff\'erentes sortes.  Elles
sont class\'ees parmi les relations (\xref{relations}).
Les fl\`eches verticales de la liste sont aussi des  \minref{d\'elimiteur}s, donc vous pouvez les rendre
plus grandes en utilisant |\big| et al.\ (\xref \big).

La commande |\iff| diff\`ere de |\Longleftrightarrow| parce qu'elle
produit un espace suppl\'ementaire \`a gauche et \`a droite de la fl\`eche.

Vous pouvez placer des symboles ou d'autres l\'egendes au dessus des fl\`eche gauches et droite
avec |\buildrel| (\xref \buildrel).

\example
$$f(x)\mapsto f(y) \iff x \mapsto y$$
|
\dproduces
$$f(x)\mapsto f(y) \iff x \mapsto y$$

\eix^^{fl\`eches}
\endexample
\enddesc

%==========================================================================
\subsection {Fonctions math\'ematiques nomm\'ees}

\begindesc
\xrdef{namedfns}
\bix^^{fonctions, noms de}
{\symbolspace = 36pt
\threecolumns 32
\easy\ctsdisplay cos {}
\ctsdisplay sin {}
\ctsdisplay tan {}
\ctsdisplay cot {}
\ctsdisplay csc {}
\ctsdisplay sec {}
\ctsdisplay arccos {}
\ctsdisplay arcsin {}
\ctsdisplay arctan {}
\ctsdisplay cosh {}
\ctsdisplay coth {}
\ctsdisplay sinh {}
\ctsdisplay tanh {}
\ctsdisplay det {}
\ctsdisplay dim {}
\ctsdisplay exp {}
\ctsdisplay ln {}
\ctsdisplay log {}
\ctsdisplay lg {}
\ctsdisplay arg {}
\ctsdisplay deg {}
\ctsdisplay gcd {}
\ctsdisplay hom {}
\ctsdisplay ker {}
\ctsdisplay inf {}
\ctsdisplay sup {}
\ctsdisplay lim {}
\ctsdisplay liminf {}
\ctsdisplay limsup {}
\ctsdisplay max {}
\ctsdisplay min {}
\ctsdisplay Pr {}
\egroup}
\explain
Ces commandes mettent les noms de diff\'erentes fonctions math\'ematiques
en type romain, comme \`a l'accoutum\'ee.
Si vous appliquez un exposant ou un indice sur une de ces commandes,
\TeX\ le composera dans la plupart des cas \`a l'endroit usuel.
En style d'affichage, \TeX\ compose les exposants et les indices
de |\det|, |\gcd|, |\inf|, |\lim|, |\liminf|,
|\limsup|, |\max|, |\min|, |\Pr| et |\sup|
comme s'ils \'etaient des limites,
c'est-\`a-dire, directement au-dessus ou directement au-dessous du nom de la fonction.

\example
$\cos^2 x + \sin^2 x = 1\qquad\max_{a \in A} g(a) = 1$
|
\produces
$\cos^2 x + \sin^2 x = 1\qquad\max_{a \in A} g(a) = 1$
\endexample
\enddesc

\begindesc
\cts bmod {}
\explain
Cette commande produit une op\'eration binaire pour indiquer un ^{modulo}
dans une formule.
\example
$$x = (y+1) \bmod 2$$
|
\dproduces
$$x = (y+1) \bmod 2$$
\endexample
\enddesc

\begindesc
\cts pmod {}
\explain
Cette commande procure une notation pour indiquer un ^{modulo} entre parenth\`eses
\`a la fin d'une formule.
\example
$$x \equiv y+1 \pmod 2$$
|
\dproduces
$$x \equiv y+1 \pmod 2$$

\eix^^{fonctions, noms de}
\endexample
\enddesc

%==========================================================================
\subsection {Grands op\'erateurs}

\begindesc 
\bix^^{op\'erateurs//grands}
\threecolumns 15
\easy\ctsdoubledisplay bigcap {}
\ctsdoubledisplay bigcup {}
\ctsdoubledisplay bigodot {}
\ctsdoubledisplay bigoplus {}
\ctsdoubledisplay bigotimes {}
\ctsdoubledisplay bigsqcup {}
\ctsdoubledisplay biguplus {}
\ctsdoubledisplay bigvee {}
\ctsdoubledisplay bigwedge {}
\ctsdoubledisplay coprod {}
{\symbolspace = 42pt\basicdisplay {\hskip 26pt$\smallint$}%
   {\\smallint}\ddstrut}%
   \xrdef{\smallint} \pix\ctsidxref{smallint}
\ctsdoubledisplay int {}
\ctsdoubledisplay oint {}
\ctsdoubledisplay prod {}
\ctsdoubledisplay sum {}
}
\explain
Ces commandes produisent diff\'erents grands symboles d'op\'erateur.  
\TeX\ produit la plus petite taille quand il est en \minrefs{mode math\'ematique} 
^{style texte} et les plus grandes tailles quand il est en ^{style d'affichage}.
Les Op\'erateurs sont une des \minref{classe}s de symboles math\'ematiques de \TeX.
\TeX\ met diff\'erents montants d'espace
autour des diff\'erentes classes de symboles math\'ematiques.

Les grands symboles d'op\'erateur avec `|big|' dans leurs noms sont diff\'e\-rents
des op\'erations binaires correspondantes (voir \xref{binops}) comme 
|\cap| ($\cap$) puisqu'ils apparaissent habituellement au d\'ebut
d'une formule.  \TeX\ utilise des espacements diff\'erents pour un grand op\'erateur
et pour une op\'eration binaire.

Ne confondez pas `$\sum$' (|\sum|) avec `$\Sigma$'^^|\Sigma| (|\Sigma|)
ou `$\prod$' (|\prod|) avec `$\Pi$' ^^|\Pi| (|\Pi|).
|\Sigma| et |\Pi| produisent les lettres grecques capitales, qui sont plus petites et
ont une apparence diff\'erente.

Un grand op\'erateur peut avoir des ^{limites}.  La limite la plus basse est sp\'ecifi\'ee comme un indice
et la plus haute comme un exposant.

\example
$$\bigcap_{k=1}^r (a_k \cup b_k)$$
|
\dproduces
$$\bigcap_{k=1}^r (a_k \cup b_k)$$
\endexample
\interexampleskip
\example
$${\int_0^\pi \sin^2 ax\,dx} = {\pi \over 2}$$
|
\dproduces
$${\int_0^\pi \sin^2 ax\,dx} = {\pi \over 2}$$
\endexample
\enddesc

\begindesc
\cts limits {}
\explain
Quand il est en style texte, \TeX\ place normalement les limites apr\`es un grand op\'erateur.
Cette commande demande \`a \TeX\ de placer
les limites sur et sous un grand op\'erateur plut\^ot qu'apr\`es lui.

Si vous sp\'ecifiez plusieurs commandes |\limits|, |\nolimits|, 
et |\dis!-play!-limits|, seule la derni\`ere sera prise en compte.

\example
Suppose that $\bigcap\limits_{i=1}^Nq_i$ contains at least 
two elements.
|
\produces
Suppose that $\bigcap\limits_{i=1}^Nq_i$ contains at least 
two elements.
\endexample
\enddesc

\begindesc
\cts nolimits {}
\explain
Quand il est en style d'affichage, \TeX\ place normalement les limites sur et sous un grand op\'erateur.
(L'op\'erateur |\int| est une exception---\TeX\
place les limites pour |\int| apr\`es l'op\'erateur dans tous les cas.)
^^|\int//limites apr\`es|
Cette commande demande \`a \TeX\ de placer
les limites apr\`es un grand op\'erateur plut\^ot que sur et sous lui.

Si vous sp\'ecifiez plusieurs commandes |\limits|, |\nolimits|, 
et |\dis!-play!-limits|, seule la derni\`ere sera prise en compte.

\example
$$\bigcap\nolimits_{i=1}^Nq_i$$
|
\dproduces
$$\bigcap\nolimits_{i=1}^Nq_i$$
\endexample
\enddesc

\begindesc
\cts displaylimits {}
\explain
Cette commande demande \`a \TeX\ de
suivre ses r\`egles normales pour le placement des limites~:
\olist\compact
\li Les limites de ^|\int| sont plac\'ees apr\`es l'op\'erateur.
\li Les limites des autres grands op\'erateurs sont plac\'ees apr\`es 
l'op\'era\-teur en style texte.
\li Les limites des autres grands op\'erateurs sont plac\'ees sur et sous 
l'op\'erateur en style d'affichage.
\endolist
Il est habituellement plus simple d'utiliser |\limits| ou |\nolimits|
pour produire un effet sp\'ecifique, mais |\display!-limits| est parfois
utile dans les d\'efinitions de \minref{macro}.

Notez que \plainTeX\ d\'efinit ^|\int| comme une macro qui comprend |\no!-limits|,
donc |\int\displaylimits| en style de texte restaure la convention de |\limits|.

Si vous sp\'ecifiez plusieurs commandes |\limits|, |\nolimits|, 
et |\dis!-play!-limits|, seule la derni\`ere sera prise en compte.

\example
$$a(\lambda) = {1 \over {2\pi}} \int\displaylimits
_{-\infty}^{+\infty} f(x)e^{-i\lambda x}\,dx$$
|
\dproduces
$$a(\lambda) = {1 \over {2\pi}} \int\displaylimits
_{-\infty}^{+\infty} f(x)e^{-i\lambda x}\,dx$$

\eix^^{op\'erateurs//grands}
\endexample
\enddesc


%==========================================================================
\subsection {Ponctuation}

\begindesc
\bix^^{ponctuation et formules de math.}
\cts cdotp {}
\cts ldotp {}
\explain
Ces deux commandes produisent respectivement un point centr\'e et un point
positionn\'e sur la \minref{ligne de base}.  Elles ne sont valides qu'en
\minref{mode} math\'ematique.  \TeX\ les traite comme de la ponctuation, ne mettant aucun espace suppl\'ementaire devant
elles mais en ajoutant un petit en plus apr\`es elles.
En contrepartie, \TeX\ met un montant d'espace \'egal des deux cot\'es
d'un point centr\'e g\'en\'er\'e par la commande ^|\cdot| (\xref \cdot).
\example
$x \cdotp y \quad x \ldotp y \quad x \cdot y$
|
\produces
$x \cdotp y \quad x \ldotp y \quad x \cdot y$
\endexample
\enddesc

\begindesc
\cts colon {}
\explain
Cette commande produit une symbole de ponctuation deux-points.
Elle n'est valide qu'en mode math\'ematique.
La diff\'erence entre |\colon| et le caract\`ere deux-points (|:|) est que
`|:|' est un op\'erateur, donc \TeX\ met une espace suppl\'ementaire \`a sa gauche tandis qu'il
ne met pas d'espace suppl\'ementaire \`a la gauche de |\colon|.
\example
$f \colon t \quad f : t$
|
\produces
$f \colon t \quad f : t$

\eix^^{ponctuation et formules de math.}
\endexample
\enddesc


%==========================================================================
%\secondprinting{\vfill\eject\null\vglue-30pt\vskip0pt}
\section {Puissances et indices}

\begindesc
\margin{Two groups of commands have been combined here.}
\bix^^{exposants}
\bix^^{indices}
%\secondprinting{\vglue-12pt}
\makecolumns 4/2:
\easy\ctsact _ \xrdef{@underscore} {\<argument>}
\cts sb {\<argument>}
\ctsact ^ \xrdef{@hat} {\<argument>}
\cts sp {\<argument>}
%\secondprinting{\vglue-4pt}
\explain
Les commandes de chaque colonne sont \'equivalentes.  Les commandes de la premi\`ere
colonne composent \<argument> comme un indice et celle de la seconde
colonne composent \<argument> comme une puissance.  Les commandes |\sb| et |\sp|
sont principalement utiles si vous travaillez sur un terminal sur lequel manque un
signe soulign\'e ou de puissance ou si vous avez red\'efini `|_|' or `|^|' et avez besoin
d'acc\'eder \`a sa d\'efinition originale.  Ces commandes sont aussi utilis\'ees pour 
mettre des limites inf\'erieures ou sup\'erieures sur des sommes ou des int\'egrales.  ^^{limites
inf\'erieures} ^^{limites sup\'erieures}

Si une puissance ou un indice n'est pas un \minref{token} seul, vous devez
l'englober dans un \minref{groupe}.  \TeX\ ne met pas de priorit\'e aux puissances
ou aux indices, donc il rejettera des formules telles que |a_i_j|, |a^i^j| ou
|a^i_j|.

Les puissances et les indices sont normalement compos\'es en ^{style script} ou
en ^{style scriptscript} s'il sont de second ordre, c'est-\`a-dire, un indice d'un
indice ou une puissance d'une puissance.  Vous pouvez mettre \emph{tout}
texte d'une formule math\'ematique en  \minref{style}script ou scriptscript avec
les commandes ^|\scriptstyle| et ^|\scriptscriptstyle| (\xref
\scriptscriptstyle).

Vous pouvez appliquer une puissance ou un indice \`a toutes les commandes qui
produisent des fonctions math\'ematiques nomm\'ees en caract\`ere romain (voir
\xref{namedfns}).  Dans certain cas (voyez encore \xref{namedfns}) la
puissance ou l'indice appara\^\i t directement sur ou sous la fonction
nomm\'ee comme montr\'e dans les exemples de ^|\lim| et de ^|\det| ci-dessous.

\example
$x_3 \quad t_{\max} \quad a_{i_k} \quad \sum_{i=1}^n{q_i}
   \quad x^3\quad e^{t \cos\theta}\quad r^{x^2}\quad 
   \int_0^\infty{f(x)\,dx}$ 
$$\lim_{x\leftarrow0}f(x)\qquad\det^{z\in A}\qquad\sin^2t$$
|
\produces
%\secondprinting{\divide\abovedisplayskip by 2}
$x_3 \quad t_{\max} \quad a_{i_k} \quad \sum_{i=1}^n{q_i}
   \quad x^3\quad e^{t \cos\theta}\quad r^{x^2}\quad 
   \int_0^\infty{f(x)\,dx}$ 
$$\lim_{x \leftarrow 0} f(x)\qquad
   \det^{z \in A}\qquad \sin^2 t$$

\eix^^{exposants}
\eix^^{indices}
\endexample
\enddesc

%\secondprinting{\vfill\eject}

%==========================================================================
\subsection {Choisir et utiliser des mod\`eles}

\begindesc
\bix^^{styles}
\cts textstyle {}
\cts scriptstyle {}
\cts scriptscriptstyle {}
\cts displaystyle {}
\explain
^^{style texte} ^^{style script} ^^{style scriptscript} ^^{style d'affichage}
Ces commandes forcent le \minref{style} normal et par cons\'equent la
police que \TeX\ utilise pour composer une formule.  Comme les
commandes de choix de police telles que |\it|, elles sont effectives
jusqu'\`a la fin du groupe les contenant.
Elles sont utiles quand le choix de style de \TeX\  est inappropri\'e pour la formule
que vous essayez de composer.
\example
$t+{\scriptstyle t + {\scriptscriptstyle t}}$
|
\produces
$t+{\scriptstyle t + {\scriptscriptstyle t}}$
\endexample
\enddesc


\begindesc
\cts mathchoice {%
   \rqbraces{\<math$_1$>}
   \rqbraces{\<math$_2$>}
   \rqbraces{\<math$_3$>}
   \rqbraces{\<math$_4$>}}
\explain
Cette commande demande \`a \TeX\ de composer une des sous-formules
\<math$_1$>, \<math$_2$>, \<math$_3$> ou \<math$_4$>, en faisant son choix
selon le \minref{style} courant.
Ainsi, si \TeX\ est en 
style d'affichage il compose |\math!-choice| comme \<math$_1$>~; en style texte il la compose
comme \<math$_2$>~; en style script il la compose comme \<math$_3$>~;
et en style scriptscript il la compose comme \<math$_4$>.
\example
\def\mc{{\mathchoice{D}{T}{S}{SS}}}
The strange formula $\mc_{\mc_\mc}$ illustrates a 
mathchoice.
|
\produces
\def\mc{{\mathchoice{D}{T}{S}{SS}}}
The strange formula $\mc_{\mc_\mc}$ illustrates a 
mathchoice.
\endexample
\enddesc

\begindesc
\cts mathpalette {\<argument$_1$> \<argument$_2$>}
\explain
^^{symboles math\'ematiques}
Cette commande procure un moyen simple de 
produire une construction math\'ematique qui marche dans les quatre \minref{style}s.
Pour l'utiliser, vous devrez normalement d\'efinir une macro suppl\'ementaire,
que nous appellerons |\build|.
L'appel de |\math!-palette| pourra alors avoir la forme
|\mathpalette|\allowbreak|\build|\<argument>.

|\build| teste dans quel style est \TeX\ et compose \<argu\-ment> en cons\'e\-quence.
Il peut \^etre d\'efini pour avoir deux param\`etres.
Quand vous appelez |\math!-palette|, il appellera \`a son tout |\build|,
avec |#1| une
commande qui s\'electionne le style courant et |#2| un \<argument>.
Ainsi, dans la d\'efinition de |\build| vous pouvez composer quelque chose
dans le style courant en le faisant pr\'ec\'eder de `|#1|'.
Voir la \knuth{page~360}{417--418} pour des exemples d'utilisation de |\mathpalette|
et la \knuth{page~151}{177} pour plus d'explication sur son fonctionnement. 

\eix^^{styles}
\enddesc

%==========================================================================
\section {Symboles compos\'es}

%==========================================================================
\subsection {Accents math\'ematiques}

\begindesc
\xrdef{mathaccent}
^^{accents}
^^{math\'ematiques//accents}
%
\easy\ctsx acute {^{accent aigu} comme dans $\acute x$}
\ctsx b {^{accent barre dessous} comme dans $\b x$}
\ctsx bar {^{accent barre} comme dans $\bar x$}
\ctsx breve {^{accent bref} comme dans $\breve x$}
\ctsx check {^{accent tch\`eque} comme dans $\check x$}
\ctsx ddot {^{accent double point} comme dans $\ddot x$}
\ctsx dot {^{accent point} comme dans $\dot x$}
\ctsx grave {^{accent grave} comme dans $\grave x$}
\ctsx hat {^{accent circonflexe} comme dans $\hat x$}
\ctsx widehat {^{accent circonflexe large} comme dans $\widehat {x+y}$}
\ctsx tilde {^{accent tilde} comme dans $\tilde x$}
\ctsx widetilde {^{accent tilde large} comme dans $\widetilde {z+a}$}
\ctsx vec {^{accent vecteur} comme dans $\vec x$}
\explain
Ces commandes produisent des accents dans les formules math\'e\-matiques.  
Vous devrez normalement laisser un espace apr\`es chacune d'entre elles.
Un accent large peut s'appliquer \`a une sous-formule comportant plusieurs caract\`eres~;
\TeX\ centrera l'accent sur la sous-formule.
Les autres accents ne sont applicables qu'a des caract\`eres simples.

\example
$\dot t^n \qquad \widetilde{v_1 + v_2}$
|
\produces
$\dot t^n \qquad \widetilde{v_1 + v_2}$
\endexample

\begindesc
\cts mathaccent {\<mathcode>}
\explain
Cette commande demande \`a \TeX\ de composer un accent math\'ematique
dont la famille et le code de caract\`ere sont donn\'es par \<mathcode>.  (\TeX\ ignore
la classe du \minref{mathcode}.)
Voir l'\knuth{annexe~G}{} pour les d\'etails sur la mani\`ere dont \TeX\ positionne de tels accents.
La fa\c con usuelle d'utiliser |\mathaccent| est de le mettre dans une d\'efinition de macro
que donne un nom \`a l'accent math\'ematique.
\example
\def\acute{\mathaccent "7013}
|
\endexample
\enddesc

\see ``Accents'' (\xref {accents}).
\enddesc

%==========================================================================
\subsection {Fractions et autres op\'erations empil\'ees}

\begindesc
\bix^^{fractions}
\bix^^{sous-formules empil\'ees}
\easy\cts over {}
\cts atop {}
\cts above {\<dimension>}
\cts choose {}
\cts brace {}
\cts brack {}
\explain
{\def\fri{\<formula$_1$>}%
\def\frii{\<formula$_2$>}%
Ces commandes empilent une sous-formule sur une autre.  Nous expliquerons comment
marche |\over| et lui relierons ensuite les autres commandes.

|\over| est la commande que vous utilisez normalement pour produire une fraction.
^^{fractions//produites par \b\tt\\over\e} 
Si vous \'ecrivez quelque chose sous une des formes suivantes~:
\csdisplay
$$!fri\over!frii$$
$!fri\over!frii$
\left!<delim>!fri\over!frii\right!<delim>
{!fri\over!frii}
|
vous obtiendrez une fraction avec un num\'erateur \fri\  et un d\'eno\-minateur \<for\-mu\-la$_2$>,
c'est-\`a-dire, \fri\ sur \frii.
Dans la premi\`ere de ces trois
formes, le |\over| n'est pas implicitement contenu dans un groupe~;
il absorbe
tout ce qui est \`a sa gauche et \`a sa droite jusqu'a ce qu'il trouve une fronti\`ere,
\`a savoir, le d\'ebut ou la fin d'un groupe.

Vous ne pouvez pas utiliser |\over| ou une autre de ces commandes de ce groupe
plus d'une fois dans une formule.
Ainsi une formule telle que~:
\csdisplay
$$a \over n \choose k$$
|
n'est pas l\'egale.
Ce n'est pas une s\'ev\`ere restriction parce que
vous pouvez toujours englober une de ces commandes entre accolades.
La raison de cette restriction est que si vous avez deux de ces commandes
dans une seule formule, \TeX\ ne saura pas comment les grouper.

Les autres commandes sont similaires \`a |\over|, avec les exceptions suivantes~:
\ulist\compact
\li |\atop| omet la barre de fraction. 
\li |\above| procure une barre de fraction d'\'epaisseur \<dimension>.
\li |\choose|
omet la barre de fraction et englobe la construction entre parenth\`eses.
(Elle est appel\'ee ``choose'' parce que $n \choose k$ est la notation pour le
nombre de fa\c cons de choisir $k$ choses parmi $n$.)
\li |\brace| omet la barre de fraction et englobe la construction entre accolades.
\li |\brack|
omet la barre de fraction et englobe la construction entre crochets.
\endulist
}%
\example
$${n+1 \over n-1}      \qquad {n+1 \atop n-1}   \qquad
  {n+1 \above 2pt n-1} \qquad {n+1 \choose n-1} \qquad
  {n+1 \brace n-1}     \qquad {n+1 \brack n-1}$$
|
\dproduces
$${n+1 \over n-1}      \qquad {n+1 \atop n-1}   \qquad
  {n+1 \above 2pt n-1} \qquad {n+1 \choose n-1} \qquad
  {n+1 \brace n-1}     \qquad {n+1 \brack n-1}$$
\endexample
\enddesc

\begindesc
\cts overwithdelims {\<delim$_1$> \<delim$_2$>}
\cts atopwithdelims {\<delim$_1$> \<delim$_2$>}
\cts abovewithdelims {\<delim$_1$> \<delim$_2$> \<dimension>}
\explain
Chacune de ces commandes superpose une sous-formule sur une autre et
entoure le r\'esultat avec \<delim$_1$> \`a gauche et
\<delim$_2$> \`a droite.  Ces commandes suivent les m\^eme r\`egles que
|\over|, |\atop| et |\above|. Le \<dimension> dans |\abovewithdelims|
sp\'ecifie l'\'epaisseur de la barre de fraction.
\example
$${m \overwithdelims () n}\qquad
  {m \atopwithdelims !|!| n}\qquad
  {m \abovewithdelims \{\} 2pt n}$$
|
\dproduces
$${m \overwithdelims () n}\qquad
  {m \atopwithdelims || n}\qquad
  {m \abovewithdelims \{\} 2pt n}$$
\endexample
\enddesc

\begindesc
\cts cases {}
\explain
^^{combinaisons, notation pour des}
Cette commande produit la forme math\'ematique qui d\'ecrit un choix entre
plusieurs cas.
Chaque cas a deux parties, s\'epar\'ees par `|&|'.
\TeX\ traite la premi\`ere partie comme une formule math\'ematique
et la seconde partie comme du texte ordinaire.  Chaque
cas doit \^etre suivi par |\cr|.

\example
$$g(x,y) = \cases{f(x,y),&if $x<y$\cr
                  f(y,x),&if $x>y$\cr
                  0,&otherwise.\cr}$$
|
\dproduces
$$g(x,y) = \cases{f(x,y),&if $x<y$\cr
                  f(y,x),&if $x>y$\cr
                  0,&otherwise.\cr}$$
\endexample
\enddesc

\begindesc
\cts underbrace {\<argument>}
\cts overbrace {\<argument>}
\cts underline {\<argument>}
\cts overline {\<argument>}
\cts overleftarrow {\<argument>}
\cts overrightarrow {\<argument>}
\explain
Ces commandes placent des ^{accolades}, des lignes ou des ^{fl\`eches} extensibles 
sur ou sous la sous-formule donn\'ee par \<argument>.
\TeX\ fera cette construction aussi large que n\'ecessaire pour
le contexte.
Quand \TeX\ produit les accolades, lignes ou fl\`eches \'etendues, il ne consid\`ere
que les dimensions de la \minref{bo\^\i te} contenant \<argument>.
Si vous utilisez plus d'une de ces commandes dans une seule formule, les
accolades, lignes ou fl\`eches qu'elle produira
pourront ne pas s'aligner proprement les unes avec les autres.
Vous pouvez utiliser la commande |\mathstrut| (\xref \mathstrut)
pour contourner cette difficult\'e.
\example
$$\displaylines{
\underbrace{x \circ y}\qquad \overbrace{x \circ y}\qquad
\underline{x \circ y}\qquad \overline{x \circ y}\qquad
\overleftarrow{x \circ y}\qquad
\overrightarrow{x \circ y}\cr
{\overline r + \overline t}\qquad
{\overline {r \mathstrut} + \overline {t \mathstrut}}\cr
}$$
|
\dproduces
$$\displaylines{
\underbrace{x \circ y}\qquad \overbrace{x \circ y}\qquad
\underline{x \circ y}\qquad \overline{x \circ y}\qquad
\overleftarrow{x \circ y}\qquad
\overrightarrow{x \circ y}\cr
{\overline r + \overline t}\qquad
{\overline {r \mathstrut} + \overline {t \mathstrut}}\cr
}$$
\endexample
\enddesc

\begindesc%\secondprinting{\vglue-.5\baselineskip\vskip0pt}
\cts buildrel {\<formule> {\bt \\over} \<relation>}
\explain
^^{relations//mettre des formules sur des}
Cette commande produit une \minref{bo\^\i te} dans laquelle \<formula>
est plac\'e sur \<relation>. \TeX\ traite le r\'esultat comme une relation
en ce qui concerne l'espacement \seeconcept{classe}.
\example
$\buildrel \rm def \over \equiv$
|
\produces
$\buildrel \rm def \over \equiv$

\eix^^{fractions}
\eix^^{sous-formules empil\'ees}
\endexample
\enddesc

%\secondprinting{\vfill\eject}


%==========================================================================
\subsection {Points}

\begindesc
\bix^^{points}
\easy\cts ldots {}
\cts cdots {}
\explain
Ces commandes produisent trois ^{points} align\'es.  Pour |\ldots|, les points
sont sur la ligne de base~; pour |\cdots|, les points sont centr\'es en respectant 
les axes (voir l'explication de |\vcenter|, \xref\vcenter).

\example
$t_1 + t_2 + \cdots + t_n \qquad x_1,x_2, \ldots\,, x_r$
|
\produces
$t_1 + t_2 + \cdots + t_n \qquad x_1,x_2, \ldots\,, x_r$
\endexample
\enddesc

\begindesc
\easy\cts vdots {}
\explain
Cette commande produit trois points verticaux.
\example
$$\eqalign{f(\alpha_1)& = f(\beta_1)\cr
   \noalign{\kern -4pt}%
   &\phantom{a}\vdots\cr % moves the dots right a bit
   f(\alpha_k)& = f(\beta_k)\cr}$$
|
\dproduces
$$\eqalign{f(\alpha_1)& = f(\beta_1)\cr
   \noalign{\kern -4pt}%
   &\phantom{a}\vdots\cr
   f(\alpha_k)& = f(\beta_k)\cr}$$
\endexample
\enddesc

\begindesc
\cts ddots {}
\explain
Cette commande produit trois points sur une diagonale.
Son usage le plus commun est d'indiquer une r\'ep\'etition le long de la diagonale d'une matrice.
\example
$$\pmatrix{0&\ldots&0\cr
           \vdots&\ddots&\vdots\cr
           0&\ldots&0\cr}$$
|
\dproduces
$$\pmatrix{0&\ldots&0\cr
           \vdots&\ddots&\vdots\cr
           0&\ldots&0\cr}$$

\eix^^{points}
\endexample
\enddesc

\see |\dots| \ctsref\dots.

%==========================================================================
\subsection {D\'elimiteurs}

\begindesc
\bix^^{d\'elimiteurs}
%
\cts lgroup {}
\cts rgroup {}
\explain
Ces commandes produisent de grandes ^{parenth\`eses} gauches et droites
qui sont d\'efinies comme des \minref{d\'elimiteur}s ouvrants et fermants.
La plus grande taille disponible pour ces d\'elimiteurs est |\Big|.
Si vous utilisez des tailles plus petites, vous obtiendrez des caract\`eres \'etranges.
\example
$$\lgroup\dots\rgroup\qquad\bigg\lgroup\dots\bigg\rgroup$$
|
\dproduces
$$\lgroup\dots\rgroup\qquad\bigg\lgroup\dots\bigg\rgroup$$
\endexample
\enddesc

\begindesc
\margin{{\tt\\vert} and {\tt\\Vert} were explained elsewhere.}
\easy\cts left {}
\cts right {}
\explain
Ces commandes doivent \^etre employ\'ees ensemble selon le mod\`ele~:
\display
{{\bt \\left} \<delim$_1$> \<subformula> {\bt \\right} \<delim$_2$>}
Cette construction demande \`a \TeX\ de produire \<subformula>, 
entour\'ee des \minref{d\'elimiteur}s \<delim$_1$> et \<delim$_2$>.
La taille verticale des d\'elimiteurs est ajust\'ee pour correspondre \`a la 
taille verticale (hauteur plus profondeur) de \<subformula>.  \<delim$_1$> et
\<delim$_2$> n'ont pas besoin de correspondre.
Par exemple, vous pouvez utiliser `|]|' comme d\'elimiteur gauche
et `|(|' comme d\'elimiteur droite dans une utilisation simple de |\left|
et |\right|.

|\left| et |\right| ont la propri\'et\'e importante de d\'efinir un
groupe, c'est-\`a-dire, qu'ils agissent comme des accolades gauches et droite.  cette propri\'et\'e de
groupement est particuli\`erement utile quand vous mettez ^|\over| (\xref{\over}) ou
une commande similaire entre |\left| et |\right|, car vous n'avez pas besoin
de mettre d'accolades autour de la fraction construite par |\over|.

Si vous voulez un d\'elimiteur gauche mais pas le droit, vous pouvez utiliser `|.|' \`a
la place du d\'elimiteur que vous ne voulez pas et il se transformera en espace blanc
(de largeur ^|\nulldelimiterspace|).
\example
$$\left\Vert\matrix{a&b\cr c&d\cr}\right\Vert
  \qquad \left\uparrow q_1\atop q_2\right.$$
|
\dproduces
$$\left\Vert\matrix{a&b\cr c&d\cr}\right\Vert
  \qquad \left\uparrow q_1\atop q_2\right.$$
\endexample
\enddesc

\begindesc
\cts delimiter {\<nombre>}
\explain
Cette commande produit un d\'elimiteur dont les caract\'eristiques sont donn\'ees par
\<number>.  \<number> est normalement \'ecrit en notation hexa\-d\'ecimale.
Vous pouvez utiliser la commande |\delimiter| \`a la place d'un caract\`ere dans tout contexte
o\`u \TeX\ attend un d\'elimiteur (bien que la commande soit rarement 
employ\'ee en dehors d'une d\'efinition de macro).
Supposez que \<number> soit le nombre hexad\'ecimal $cs_1s_2s_3
l_1l_2l_3$.  Alors \TeX\ prendra le d\'elimiteur ayant 
la \minref{classs} $c$, la petite variante
$s_1s_2s_3$ et la grande variante $l_1l_2l_3$.  Ici $s_1s_2s_3$ indique
le caract\`ere math\'ematique trouv\'e en position $s_2s_3$ de la famille $s_1$ et
de m\^eme pour $l_1l_2l_3$.  C'est la m\^eme convention que celle
utilis\'ee pour ^|\mathcode| (\xref\mathcode).
\example
\def\vert{\delimiter "026A30C} % As in plain TeX.
|
\endexample
\enddesc


\begindesc 
\margin{{\tt\\delcode} was explained in two places.  The
combined explanation is now in `General operations'.}
\cts delimiterfactor {\param{nombre}}
\cts delimitershortfall {\param{nombre}}
\explain
^^{d\'elimiteurs//hauteur des}
Ces deux param\`etres signalent \`a \TeX\ comment la hauteur d'un \minref{d\'elimiteur}
doit \^etre reli\'ee \`a la taille verticale de la sous-formule
\`a laquelle le d\'elimi\-teur est associ\'e.
|\delimiterfactor| donne le ratio minimum
de la taille du d\'elimiteur par rapport \`a la taille verticale de la sous-formule et
|\deli!-mitershortfall| donne le maximum par lequel la hauteur du
d\'elimiteur sera r\'eduite pour correspondre \`a la taille verticale de la sous-formule.

Supposez que la \minref{bo\^\i te} contenant la sous-formule
ait une hauteur $h$ et une profondeur $d$ et $y=2\,\max(h,d)$.
soit la valeur de |\delimiterfactor| $f$ et la valeur de
|\delimitershortfall| $\delta$.
Alors \TeX\ prend la taille de d\'elimiteur minimum \'etant au moins $y \cdot
f/1000$ et au moins $y-\delta$.  En particulier, si |\delimiterfactor|
est exactement \`a $1000$ alors \TeX\ essayera de faire un d\'elimiteur au moins aussi grand
que la formule \`a laquelle il est attach\'e.
Voir les \knuth{pages~152 et~446 (R\`egle 19)}{pages~178 et~511 (R\`egle 19)}
pour les d\'etail exacts sur la fa\c con dont \TeX\ utilise ces param\`etres.
\PlainTeX\ met |\delimiter!-factor| \`a $901$ et 
|\delimiter!-shortfall| \`a |5pt|.
\enddesc

\see |\delcode| (\xref\delcode), |\vert|, |\Vert|
et |\backslash| (\xref\vert).
\eix^^{d\'elimiteurs}

%==========================================================================
\subsection {Matrices}

\begindesc
\cts matrix
   {{\bt \rqbraces{\<ligne> \\cr $\ldots$ \<ligne> \\cr}}}
\cts pmatrix
   {{\bt \rqbraces{\<ligne> \\cr $\ldots$ \<ligne> \\cr}}}
\cts bordermatrix
   {{\bt \rqbraces{\<ligne> \\cr $\ldots$ \<ligne> \\cr}}}
\explain
Chacune de ces trois commandes produit une ^{matrice}.  
Les \'el\'ements de chaque rang\'e de la matrice entr\'ee
sont s\'epar\'es par `|&|' et chaque rang\'e est \`a son tour termin\'ee
par |\cr|.
(C'est la m\^eme forme que celle utilis\'ee pour un
\minref{alignement}.)
Les commandes diff\`erent de la mani\`ere suivante~:
\ulist\compact
\li |\matrix| produit une matrice sans aucun entourage ou \minref{d\'elimiteur}s ins\'er\'es.
\li |\pmatrix| produit une matrice entour\'ee par des parenth\`eses.
\li |\bordermatrix| produit une matrice dans laquelle la premi\`ere rang\'ee et la premi\`ere
colonne sont trait\'ees comme des labels.  (Le premier \'el\'ement de la premi\`ere rang\'ee est
habituellement laiss\'e \`a blanc.)  Le reste de la matrice est englob\'ee dans
des parenth\`eses.
\endulist
\TeX\ peut rendre les parenth\`eses de |\pmatrix| et de |\bordermatrix| aussi larges que
n\'ecessaire en ins\'erant des extensions verticales.  Si vous voulez une matrice
entour\'ee par des d\'elimiteurs autres que des paren\-th\`eses, vous devez utiliser
|\matrix| en conjugaison avec |\left| et |\right| (\xref \left).

\example
$$\displaylines{
   \matrix{t_{11}&t_{12}&t_{13}\cr
           t_{21}&t_{22}&t_{23}\cr
           t_{31}&t_{32}&t_{33}\cr}\qquad
\left\{\matrix{t_{11}&t_{12}&t_{13}\cr
           t_{21}&t_{22}&t_{23}\cr
           t_{31}&t_{32}&t_{33}\cr}\right\}\cr
\pmatrix{t_{11}&t_{12}&t_{13}\cr
           t_{21}&t_{22}&t_{23}\cr
           t_{31}&t_{32}&t_{33}\cr}\qquad
\bordermatrix{&c_1&c_2&c_3\cr
           r_1&t_{11}&t_{12}&t_{13}\cr
           r_2&t_{21}&t_{22}&t_{23}\cr
           r_3&t_{31}&t_{32}&t_{33}\cr}\cr}$$
|
\dproduces
$$\displaylines{
   \matrix{t_{11}&t_{12}&t_{13}\cr
   t_{21}&t_{22}&t_{23}\cr
   t_{31}&t_{32}&t_{33}\cr}\qquad
\left\{\matrix{t_{11}&t_{12}&t_{13}\cr
   t_{21}&t_{22}&t_{23}\cr
   t_{31}&t_{32}&t_{33}\cr}\right\}\cr
\pmatrix{t_{11}&t_{12}&t_{13}\cr
   t_{21}&t_{22}&t_{23}\cr
   t_{31}&t_{32}&t_{33}\cr}\qquad
\bordermatrix{&c_1&c_2&c_3\cr
   r_1&t_{11}&t_{12}&t_{13}\cr
   r_2&t_{21}&t_{22}&t_{23}\cr
   r_3&t_{31}&t_{32}&t_{33}\cr}\cr}$$
\endexample
\enddesc

%==========================================================================
\subsection {Racines et radicaux}

\begindesc
\easy\cts sqrt {\<argument>}
\explain
Cette commande produit la notation pour la racine carr\'ee de \<argument>.
\example
$$x = {-b\pm\sqrt{b^2-4ac} \over 2a}$$
|
\dproduces
$$x = {-b\pm\sqrt{b^2-4ac} \over 2a}$$
\endexample
\enddesc

\begindesc
\easy\cts root {\<argument$_1$> {\bt \\of} \<argument$_2$>}
\explain
Cette commande produit la notation pour une racine de \<argument$_2$> o\`u l'ordre
est donn\'e par \<argument$_1$>.
\example
$\root \alpha \of {r \cos \theta}$
|
\produces
$\root \alpha \of {r \cos \theta}$
\endexample
\enddesc

\begindesc
\cts radical {\<nombre>}
\explain
Cette commande produit un signe radical 
dont les caract\'eristiques sont donn\'ees par
\<nombre>.  Elle utilise la m\^eme repr\'esentation que le code d\'elimiteur
^^{codes d\'elimiteurs}
dans la commande ^|\delcode| (\xref \delcode).

\example
\def\sqrt{\radical "270370} % as in plain TeX
|
\endexample
\enddesc

%==========================================================================
\section {Num\'eros d'\'equation}

\begindesc
\easy\cts eqno {}
\cts leqno {}
\explain
Ces commandes attachent un num\'ero d'\'equation \`a une formule affich\'ee.
|\eqno| met le num\'ero d'\'equation \`a droite et |\leqno| le met \`a gauche.
Les commandes doivent \^etre mises \`a la fin de la formule.
Si vous avez un affichage multi-ligne et voulez num\'eroter plus d'une ligne,
utilisez les commandes |\eq!-alignno| ou |\leq!-alignno|
(\xref \eqalignno).

Ces commandes ne sont valides qu'en mode math\'ematique affich\'e.

\example
$$e^{i\theta} = \cos \theta + i \sin \theta\eqno{(11)}$$
|
\produces
$$e^{i\theta} = \cos \theta + i \sin \theta\eqno{(11)}$$
\endexample
\example
$$\cos^2 \theta + \sin^2 \theta = 1\leqno{(12)}$$
|
\produces
\abovedisplayskip = -\baselineskip
$$\cos^2 \theta + \sin^2 \theta = 1\leqno{(12)}$$
\endexample
\enddesc


%==========================================================================
\section {Affichages multi-ligne}

\begindesc
\bix^^{affichages//multi-ligne}
\cts displaylines
   {{\bt \rqbraces{\<ligne>\ths\\cr$\ldots$\<ligne>\ths\\cr}}}
\explain
Cette commande produit un affichage math\'ematique multi-ligne dans lequel chaque ligne est
centr\'ee ind\'ependamment des autres lignes.
Vous pouvez utiliser la commande |\noalign| (\xref \noalign) pour changer le montant
d'espace entre deux lignes d'un affichage multi-ligne.

Si vous voulez attacher des num\'eros d'\'equation \`a certaines ou toutes les \'equations
dans un affichage math\'ematique multi-ligne, vous devez utiliser |\eqalignno| ou
|\leqalignno|.
\example
$$\displaylines{(x+a)^2 = x^2+2ax+a^2\cr
                (x+a)(x-a) = x^2-a^2\cr}$$
|
\dproduces\centereddisplays
$$\displaylines{
(x+a)^2 = x^2+2ax+a^2\cr
(x+a)(x-a) = x^2-a^2\cr
}$$
\endexample
\enddesc

\begindesc
\cts eqalign {}
   {{\bt \rqbraces{\<ligne> \\cr $\ldots$ \<ligne> \\cr}}}
\cts eqalignno {}
   {{\bt \rqbraces{\<ligne> \\cr $\ldots$ \<ligne> \\cr}}}
\cts leqalignno {}
   {{\bt \rqbraces{\<ligne> \\cr $\ldots$ \<ligne> \\cr}}}
\explain
^^{num\'eros d'\'equation}
Ces commandes produisent un affichage math\'ematique multi-ligne
dans lequel certaines parties correspondantes des lignes sont align\'ees verticalement.
Les commandes |\eqalignno| et |\leqalignno| vous laissent aussi
apporter des num\'eros d'\'equation pour certaines ou toutes les lignes.
|\eqalignno| met les num\'eros d'\'equation \`a droite et
|\leqalignno| les met \`a gauche.

Chaque ligne dans l'affichage est termin\'ee par |\cr|.  Chacune des parties devant \^etre align\'ee
(le plus souvent un signe \'egal) est pr\'ec\'ed\'ee par
`|&|'.  Un `|&|' pr\'ec\`ede aussi chaque num\'ero d'\'equation, qui est donn\'e \`a la fin d'une ligne.
Vous pouvez mettre plus d'une de ces commandes dans un seul affichage pour
produire plusieurs groupes d'\'equation.  Dans ce cas, seuls les groupes les plus \`a droite
ou les plus \`a gauche peuvent \^etre produits avec |\eqalignno| ou |\leqalignno|.

Vous pouvez utiliser la commande |\noalign| (\xref \noalign) pour changer le montant
d'espace entre deux lignes d'un affichage multi-ligne.
\example
$$\left\{\eqalign{f_1(t) &= 2t\cr f_2(t) &= t^3\cr
         f_3(t) &= t^2-1\cr}\right\}
  \left\{\eqalign{g_1(t) &= t\cr g_2(t) &= 1}\right\}$$
|
\dproduces
$$\left\{\eqalign{f_1(t) &= 2t\cr f_2(t) &= t^3\cr
   f_3(t) &= t^2-1\cr}\right\}
\left\{\eqalign{g_1(t) &= t\cr g_2(t) &= 1}\right\}$$
\nextexample
$$\eqalignno{
\sigma^2&=E(x-\mu)^2&(12)\cr
   &={1 \over n}\sum_{i=0}^n (x_i - \mu)^2&\cr
   &=E(x^2)-\mu^2\cr}$$
|
\produces
\abovedisplayskip = -\baselineskip
$$\eqalignno{
\sigma^2&=E(x-\mu)^2&(12)\cr
   &={1 \over n}\sum_{i=0}^n (x_i - \mu)^2&\cr
   &=E(x^2)-\mu^2\cr}$$
\nextexample
$$\leqalignno{
\sigma^2&=E(x-\mu)^2&(6)\cr
   &=E(x^2)-\mu^2&(7)\cr}$$
|
\produces
\abovedisplayskip = -\baselineskip
$$\leqalignno{
\sigma^2&=E(x-\mu)^2&(6)\cr
   &=E(x^2)-\mu^2&(7)\cr}$$
\nextexample
$$\eqalignno{
  &(x+a)^2 = x^2+2ax+a^2&(19)\cr
  &(x+a)(x-a) = x^2-a^2\cr}$$
% same effect as \displaylines but with an equation number
|
\dproduces
$$\eqalignno{
&(x+a)^2 = x^2+2ax+a^2&(19)\cr
&(x+a)(x-a) = x^2-a^2\cr
}$$
% same effect as \displaylines but with an equation number

\eix^^{affichages//multi-ligne}
\endexample
\enddesc

%==========================================================================
\section {Polices dans des formules math\'ematiques}

\begindesc
^^{polices}
\xrdef{mathfonts}
%
\easy\ctsx cal {utilise une police calligraphique majuscule}
\ctsx mit {utilise une police math\'ematique italique}
\ctsx oldstyle {utilise une police de chiffres elz\'eviriens}
\explain
Ces commandes demandent \`a \TeX\ de composer le texte suivant dans la police
sp\'ecifi\'ee.  Vous ne pouvez les utiliser qu'en \minref{mode math\'ematique}.
La commande |\mit| est utile pour produire des ^{lettres grecques} capitales pench\'ees.
Vous pouvez aussi utiliser les commandes donn\'ees dans
\headcit{S\'electionner des polices}{selfont} pour changer des polices en mode math\'ematique.
\example
${\cal XYZ} \quad
{\mit AaBb\Gamma \Delta \Sigma} \quad 
{\oldstyle 0123456789}$
|
\produces
${\cal XYZ} \quad
{\mit AaBb\Gamma \Delta \Sigma} \quad 
{\oldstyle 0123456789}$
\endexample
\enddesc

^^{styles de caract\`ere}
\begindesc
\ctsx itfam {famille de type italique}
\ctsx bffam {famille de type grasse}
\ctsx slfam {famille de type pench\'ee}
\ctsx ttfam {famille de type machine \`a \'ecrire}
\explain
Ces commandes d\'efinissent des familles de type \minrefs{family} devant \^etre utilis\'ees en
\minref{mode math\'ematique}.  Leur principale utilisation est dans la d\'efinition des commandes
|\it|, |\bf|, |\sl| et |\tt| pour qu'elles puissent fonctionner en mode math\'ematique.
\enddesc

\begindesc
\cts fam {\param{nombre}}
\explain
Quand \TeX\ est en \minref{mode math\'ematique}, il compose normalement un caract\`ere
en utilisant la ^^{classe} de famille de police donn\'ee dans son \minref{mathcode}.
^^{famille//donn\'ee par \b\tt\\fam\e}
Cependant, quand \TeX\ est en mode math\'ematique et rencontre un caract\`ere dont
la \minref{classe} est $7$ (Variable), il compose ce caract\`ere en utilisant
la \minref{famille} de police donn\'ee par la valeur de |\fam|, \`a condition que la 
valeur de |\fam| soit entre $0$ et $15$.
Si la valeur de |\fam| n'est pas dans cette intervalle, \TeX\ utilise la famille du
mathcode du caract\`ere comme dans le cas ordinaire.
\TeX\ met |\fam| \`a $-1$ \`a chaque fois qu'il entre en mode math\'ematique.
En dehors du mode math\'ematique, |\fam| n'a pas d'effet.

En assignant une valeur \`a
|\fam| vous pouvez changer la fa\c con dont \TeX\ compose des caract\`eres ordinaires
tels que des variables.    
Par exemple, en mettant |\fam| \`a |\ttfam|, vous demandez \`a \TeX\ de composer
des variables en utilisant une police de machine \`a \'ecrire.
\PlainTeX\ d\'efinit |\tt| comme une \minref{macro} qui, entre autres choses,
met |\fam| \`a |\ttfam|.
\example
\def\bf{\fam\bffam\tenbf} % As in plain TeX.
|
\endexample
\enddesc

\begindesc
\cts textfont {\<family>\param{fontname}}
\cts scriptfont {\<family>\param{fontname}}
\cts scriptscriptfont {\<family>\param{fontname}}
\explain
^^{style texte}
^^{style script}
^^{style scriptscript}
Chacun de ces param\`etres sp\'ecifie la police que \TeX\ utilise pour
composer le \minref{style} indiqu\'e dans la \minref{famille} indiqu\'ee.
Ces choix n'ont aucun effet en dehors du \minref{mode math\'ematique}.
\example
\scriptfont2 = \sevensy % As in plain TeX.
|
\endexample
\enddesc

\see ``Type styles'' (\xref{seltype}).
%==========================================================================
\section {Construire des symboles math\'ematiques}

%==========================================================================
\subsection {Rendre des d\'elimiteurs plus grands}

\begindesc
\makecolumns 16/4:
\easy\cts big {}
\cts bigl {}
\cts bigm {}
\cts bigr {}
\cts Big {}
\cts Bigl {}
\cts Bigm {}
\cts Bigr {}
\cts bigg {}
\cts biggl {}
\cts biggm {}
\cts biggr {}
\cts Bigg {}
\cts Biggl {}
\cts Biggm {}
\cts Biggr {}
\explain
^^{d\'elimiteurs//\'elargis}
Ces commandes rendent des \minref{d\'elimiteur}s plus grands que leur taille normale.
Les commandes dans les quatre colonnes
produisent successivement de plus grandes tailles.  La diff\'erence entre |\big|,
|\bigl|, |\bigr| et |\bigm| est en relation avec la \minref{classe} du
d\'elimiteur agrandi~:
\ulist\compact
\li |\big| produit un symbole ordinaire.
\li |\bigl| produit un symbole ouvrant.
\li |\bigr| produit un symbole fermant.
\li |\bigm| produit un symbole de relation.
\endulist
\noindent
\TeX\ utilise la classe d'un symbole pour d\'ecider combien d'espace mettre
autour de ce symbole.

Ces commandes, contrairement \`a |\left| et |\right|,
ne d\'efinissent \emph{pas} un groupe.

\example
$$(x) \quad \bigl(x\bigr) \quad \Bigl(x\Bigr) \quad
   \biggl(x\biggr) \quad \Biggl(x\Biggr)\qquad
[x] \quad \bigl[x\bigr] \quad \Bigl[x\Bigr] \quad
   \biggl[x\biggr] \quad \Biggl[x\Biggr]$$
|
\dproduces
$$(x) \quad \bigl(x\bigr) \quad \Bigl(x\Bigr) \quad
\biggl(x\biggr) \quad \Biggl(x\Biggr)\qquad
[x] \quad \bigl[x\bigr] \quad \Bigl[x\Bigr] \quad
\biggl[x\biggr] \quad \Biggl[x\Biggr]$$
\endexample
\enddesc

%==========================================================================
\subsection {Parties de grands symboles}

\begindesc
\cts downbracefill {}
\cts upbracefill {}
\explain
Ces commandes produisent respectivement des ^{accolades horizontales} 
et extensibles dirig\'ees vers le haut et vers le bas. ^^{accolades} 
\TeX \ rendra les accolades aussi larges que n\'ecessaire. 
Ces commandes 
sont utilis\'ees dans les d\'efinitions de ^|\overbrace| et de ^|\underbrace|
(\xref \overbrace).
\example
$$\hbox to 1in{\downbracefill} \quad
   \hbox to 1in{\upbracefill}$$
|
\dproduces
$$\hbox to 1in{\downbracefill} \quad
   \hbox to 1in{\upbracefill}$$
\endexample
\enddesc

\begindesc
\cts arrowvert {}
\cts Arrowvert {}
\cts lmoustache {}
\cts rmoustache {}
\cts bracevert {}
\explain
Ces commandes produisent des portions de certains grands
d\'elimiteurs
^^{d\'elimiteurs//parties de}
qui peuvent eux-m\^eme \^etre utilis\'es comme d\'elimiteurs.
Elles font r\'ef\'e\-rence aux caract\`eres de la police math\'ematique ^|cmex10|.
\example
$$\cdots \Big\arrowvert \cdots \Big\Arrowvert \cdots
  \Big\lmoustache \cdots \Big\rmoustache \cdots
  \Big\bracevert \cdots$$
|
\dproduces
$$\cdots \Big\arrowvert \cdots \Big\Arrowvert \cdots
  \Big\lmoustache \cdots \Big\rmoustache \cdots
  \Big\bracevert \cdots$$
\endexample
\enddesc


%==========================================================================
\section {Aligner des parties d'une formule}

%==========================================================================
\subsection {Aligner des accents}

\begindesc
\bix^^{accents//aligner des}
\cts skew {\<nombre> \<argument$_1$> \<argument$_2$>}
\explain
Cette commande d\'eplace l'accent \<argument$_1$> de
\<number> \minref{unit\'es math\'e\-matiques} vers la droite de sa position normale
en respectant \<argu\-ment$_2$>.
L'utilisation la plus commune de cette commande est pour 
modifier la position d'un accent qui se trouve plac\'e sur
un autre accent.
\example
$$\skew 2\bar{\bar z}\quad\skew 3\tilde{\tilde y}\quad
  \skew 4\tilde{\hat x}$$   
|
\dproduces
$$\skew 2\bar{\bar z}\quad\skew 3\tilde{\tilde y}\quad
  \skew 4\tilde{\hat x}$$   
\endexample
\enddesc

\begindesc
\cts skewchar {\<font>\param{number}}
\explain
Le |\skewchar| d'une police
est le caract\`ere dans la police dont les cr\'enages,
d\'efinis dans le fichier de m\'etrique de la police, d\'eterminent les positions
des accents math\'ematiques. Cela dit, supposez que \TeX\ applique un accent math\'ematique
au caract\`ere `|x|'.  \TeX\ regarde si la paire de caract\`eres
`|x\skewchar|' a un cr\'enage~; si oui, il d\'eplace l'accent d'un montant de
ce cr\'enage. L'algorithme complet que \TeX\ utilise pour positionner les accents
math\'ematiques (qui d\'eterminent beaucoup d'autres choses) se trouve dans l'\knuth{annexe~G}{}.

Si la valeur de |\skewchar| n'est pas dans l'intervalle $0$--$255$,
\TeX\ prend une valeur de cr\'enage \`a z\'ero.

Notez que \<font> est une s\'equence de contr\^ole
qui nomme une police, pas un \<nom de police> qui nomme des fichiers de police.
Attention~: 
un assignement de |\skewchar| n'est \emph{pas} enlev\'e \`a la fin d'un groupe.
Si vous voulez changer |\skewchar| localement, vous devrez sauvegarder et
restaurer sa valeur originale explicitement.
\enddesc

\begindesc
\cts defaultskewchar {\param{number}}
\explain
Quand \TeX\ lit le fichier de m\'etriques
^^{fichier de m\'etriques//caract\`ere oblique par d\'efaut dans le}
pour une police en r\'eponse \`a une
commande ^|\font|, il met le ^|\skewchar| de la police
dans |\default!-skewchar|.
Si la valeur de |\default!-skewchar| n'est 
pas dans l'intervalle $0$--$255$, \TeX\ n'assigne aucun caract\`ere oblique
par d\'efaut.
\PlainTeX\ met |\defaultskewchar| \`a $-1$, et il est normalement pr\'ef\'erable
de le laisser ainsi.
\margin{Misleading example deleted.}
\eix^^{accents//aligner des}
\enddesc

%==========================================================================
\subsection {Aligner du mat\'eriel verticalement}

\begindesc
\cts vcenter {\rqbraces{\<mat\'eriel en mode vertical>}}
\ctsbasic {\\vcenter to \<dimension> \rqbraces{\<mat\'eriel en mode vertical>}}{}
\ctsbasic {\\vcenter spread \<dimension> \rqbraces{\<mat\'eriel en mode vertical>}}{}
\explain
Toutes les formules math\'ematiques ont un
``^{axe}'' invisible que \TeX\ traite comme une sorte de
ligne de centrage horizontale pour cette formule.
Par exemple, l'axe d'une formule constitu\'ee d'une
fraction est au centre de la barre de fraction.
La commande |\vcenter| demande \`a \TeX\ de placer le \<mat\'eriel en mode vertical>
dans une \minref{vbox} et de centrer la vbox
en respectant l'axe de la formule qu'il est en train de construire.

La premi\`ere forme de la commande
centre le mat\'eriel comme donn\'e.  Les deuxi\`emes et troisi\`emes
formes \'elargissent ou r\'etr\'ecissent le mat\'eriel verticalement comme dans la 
commande |\vbox| (\xref \vbox).

\example
$${n \choose k} \buildrel \rm def \over \equiv \>
\vcenter{\hsize 1.5 in \noindent the number of 
combinations of $n$ things taken $k$ at a time}$$
|
\dproduces
$${n \choose k} \buildrel \rm def \over \equiv \>
\vcenter{\hsize 1.5 in \noindent the number of 
combinations of $n$ things taken $k$ at a time}$$
\endexample
\enddesc

%==========================================================================
\section {Produire des espaces}

%==========================================================================
\subsection {Espaces math\'ematiques de largeur fixe}

\begindesc
\bix^^{espace//dans des formules math\'ematiques}
\ctspecial ! \ctsxrdef{@shriek}
\ctspecial , \ctsxrdef{@comma}
\ctspecial > \ctsxrdef{@greater}
\ctspecial ; \ctsxrdef{@semi}
\explain
Ces commandes produisent des montants vari\'es d'^{espace suppl\'ementaire} dans des formules.  Elles
sont d\'efinies en termes d'\minref{unit\'es math\'ematiques}, donc \TeX\ ajuste
le montant d'espace en accord avec le \minref{style} courant.
\ulist
\li |\!!| produit un espace fin n\'egatif, c'est-\`a-dire, qu'il r\'eduit l'espace
entre ses sous-formules voisines du montant d'un espace fin.
\li |\,| produit un espace fin.
\li |\>| produit un espace moyen.
\li |\;| produit un espace large.
\endulist
\example
$$00\quad0\!!0\quad0\,0\quad0\>0\quad0\;0\quad
{\scriptstyle 00\quad0\!!0\quad0\,0\quad0\>0\quad0\;0}$$
|
\dproduces
$$00\quad0\!0\quad0\,0\quad0\>0\quad0\;0\quad
{\scriptstyle 00\quad0\!0\quad0\,0\quad0\>0\quad0\;0}$$
\endexample
\enddesc

\begindesc
\cts thinmuskip {\param{muglue}}
\cts medmuskip {\param{muglue}}
\cts thickmuskip {\param{muglue}}
\explain
Ces param\`etres d\'efinissent des espaces fins, moyens et larges en
mode math\'ematique.
\example
$00\quad0\mskip\thinmuskip0\quad0\mskip\medmuskip0
   \quad0\mskip\thickmuskip0$
|
\produces
$00\quad0\mskip\thinmuskip0\quad0\mskip\medmuskip0
   \quad0\mskip\thickmuskip0$
\endexample
\enddesc

\begindesc
\cts jot {\param{dimension}}
\explain
Ce param\`etre d\'efinit une distance \'egale \`a trois points (\`a moins que
vous le changiez).
Le |\jot| est une unit\'e de mesure pratique pour ouvrir des affichages math\'ematiques.
\enddesc

%==========================================================================
\subsection {Espaces math\'ematiques de largeur variable}

\begindesc
\cts mkern {\<mudimen>}
\explain
^^{cr\'enages//dans des formules math\'ematiques}
Cette commande
produit un \minref{cr\'enage}, c'est-\`a-dire, un espace blanc, de largeur \<mudimen>.
Le cr\'enage est mesur\'e
en \minref{unit\'es math\'ematiques}, qui varie en fonction du style.
Hormis son unit\'e de la mesure, cette commande se comporte comme 
|\kern| (\xref \kern) le fait en mode horizontal.

\example
$0\mkern13mu 0 \qquad {\scriptscriptstyle 0 \mkern13mu 0}$
|
\produces
$0\mkern13mu 0 \qquad {\scriptscriptstyle 0 \mkern13mu 0}$
\endexample
\enddesc

\begindesc
\cts mskip {\<mudimen$_1$> {\bt plus} \<mudimen$_2$> {\bt minus}
   \<mudimen$_3$>}
\explain
^^{ressort}
Cette commande produit un \minref{ressort} horizontal
qui a une largeur naturelle \<mu\-dimen$_1$>, s'\'etire de \<mudimen$_2$>
et se r\'etr\'ecit de \<mudimen$_3$>.
Le ressort est mesur\'e en \minref{unit\'es math\'ematiques}, qui varie en fonction
du style.  Hormis son unit\'e de mesure, cette commande se comporte 
comme |\hskip| (\xref \hskip).

\example
$0\mskip 13mu 0 \quad {\scriptscriptstyle 0 \mskip 13mu 0}$
|
\produces
$0\mskip 13mu 0 \quad {\scriptscriptstyle 0 \mskip 13mu 0}$
\endexample
\enddesc

\begindesc
\cts nonscript {}
\explain
Quand \TeX\ est en train de composer en \minref{style} script ou 
scriptscript et rencontre cette commande
imm\'ediatement devant d'un ressort ou d'un cr\'enage,
il supprime le ressort ou le cr\'enage.
|\nonscript| n'a pas d'effet dans les autres styles.

Cette commande fournit une mani\`ere de ``resserrer'' l'espacement dans 
les styles script et scriptscript, qui sont g\'en\'eralement faits dans une
police plus petite. Elle est peu utile en dehors des d\'efinitions de macro.
\example
\def\ab{a\nonscript\; b}
$\ab^{\ab}$
|
\produces
\def\ab{a\nonscript\; b}
$\ab^{\ab}$
\endexample
\enddesc

\see |\kern| (\xref\kern), |\hskip| (\xref\hskip).
\eix^^{espace//dans des formules math\'ematiques}


%==========================================================================
\subsection {param\`etres d'espacement pour les affichages}

\begindesc
\bix^^{affichages//param\`etres d'espacement pour}
\cts displaywidth {\param{dimension}}
\explain
Ce param\`etre sp\'ecifie la largeur maximum que
\TeX\ alloue pour un affichage math\'ematique.  Si \TeX\ ne peut pas mettre l'affichage
dans un espace de cette largeur, il fait un ``overfull \minref{hbox}''
et se plaint.
\TeX\ met la valeur de |\displaywidth| quand il rencontre le `|$$|'
qui d\'ebute l'affichage.  Cette valeur initiale est
|\hsize| (\xref \hsize) \`a moins qu'elle soit modifi\'ee par des 
changements de la forme du paragraphe.
Voir les \knuth{pages~188--189}{220--222} pour une explication plus d\'etaill\'ee de ce param\`etre.
\enddesc

\begindesc
\cts displayindent {\param{dimension}}
\explain
Ce param\`etre sp\'ecifie l'espace par lequel \TeX\ indente un
affichage math\'e\-matique.
\TeX\ met la valeur de |\displayindent| quand il rencontre le `|$$|'
qui d\'ebute l'affichage.  Normalement la valeur initiale est z\'ero,
mais si la forme du paragraphe indique que l'affichage doit \^etre
d\'ecal\'e d'un montant $s$,
\TeX\ mettra |\displayindent| \`a $s$.
Voir les \knuth{pages~188--189}{220--222} pour une explication plus d\'etaill\'ee de ce param\`etre.
\enddesc

\begindesc
\cts predisplaysize {\param{dimension}}
\explain
\TeX\ met ce param\`etre \`a la largeur de la ligne pr\'ec\'edant
un affichage math\'ematique.
\TeX\ utilise |\predisplaysize| pour d\'eterminer si l'affi\-chage
d\'ebute ou non \`a
gauche d'o\`u la ligne pr\'ec\'edente se termine, c'est-\`a-dire, s'il 
recouvre visuellement ou non la ligne pr\'ec\'edente.  
S'il la recouvre, il utilise les ressorts |\abovedisplayskip| et
|\belowdisplayskip| en mettant l'affichage~;
autrement il utilise les ressorts |\abovedisplay!-shortskip| et
|\belowdisplay!-shortskip|.
Voir les \knuth{pages~188--189}{220--222} pour une explication plus d\'etaill\'ee de ce param\`etre.
\enddesc

\begindesc
\cts abovedisplayskip {\param{ressort}}
\explain
Ce param\`etre sp\'ecifie le montant de ressort vertical que
\TeX\ ins\`ere avant un affichage quand l'affichage d\'ebute
\`a gauche d'o\`u la ligne pr\'ec\'edente se termine, c'est-\`a-dire, quand il recouvre 
visuellement la ligne pr\'ec\'edente.
\PlainTeX\ met |\abovedisplayskip| \`a |12pt plus3pt minus9pt|.
Voir les \knuth{pages~188--189}{220--222} pour une explication plus d\'etaill\'ee de ce param\`etre.
\enddesc

\begindesc
\cts belowdisplayskip {\param{ressort}}
\explain
Ce param\`etre sp\'ecifie le montant de ressort vertical que
\TeX\ ins\`ere apr\`es un affichage quand l'affichage d\'ebute
\`a gauche d'o\`u la ligne pr\'ec\'edente se termine, c'est-\`a-dire, quand il recouvre 
visuellement la ligne pr\'ec\'edente.
\PlainTeX\ met |\belowdisplay!-skip| \`a |12pt plus3pt minus9pt|.
Voir les \knuth{pages~188--189}{220--222} pour une explication plus d\'etaill\'ee de ce param\`etre.
\enddesc

\begindesc
\cts abovedisplayshortskip {\param{ressort}}
\explain
Ce param\`etre sp\'ecifie le montant de ressort vertical que
\TeX\ ins\`ere avant un affichage quand l'affichage d\'ebute
\`a droite d'o\`u la ligne pr\'ec\'edente se termine, c'est-\`a-dire, quand il ne recouvre 
pas visuellement la ligne pr\'ec\'edente.
\PlainTeX\ met |\abovedisplay!-shortskip| \`a |0pt plus3pt|.
Voir les \knuth{pages~188--189}{220--222} pour une explication plus d\'etail\-l\'ee de ce param\`etre.
\enddesc

\begindesc
\cts belowdisplayshortskip {\param{ressort}}
\explain
Ce param\`etre sp\'ecifie le montant de ressort vertical que
\TeX\ ins\`ere apr\`es un affichage quand l'affichage d\'ebute
\`a droite d'o\`u la ligne pr\'ec\'edente se termine, c'est-\`a-dire, quand il ne recouvre 
pas visuellement la ligne pr\'ec\'edente.
\PlainTeX\ met |\belowdisplay!-shortskip| \`a |7pt plus3pt minus4pt|.
Voir les \knuth{pages~188--189}{220--222} pour une explication plus d\'etaill\'ee de ce param\`etre.
\eix^^{affichages//param\`etres d'espacement pour}
\enddesc


%==========================================================================
\subsection {autres param\`etres d'espacement pour les math\'ematiques}

\begindesc
\cts mathsurround {\param{dimension}}
\explain
Ce param\`etre sp\'ecifie le montant d'espace que \TeX\
ins\`ere avant et apr\`es une formule math\'ematique en mode texte (c'est-\`a-dire, une formule
entour\'ee de |$| simples).  Voir la \knuth{page~162}{190} pour plus de d\'etails sur
son comportement.
\PlainTeX\ laisse |\mathsurround| \`a |0pt|.
\enddesc

\begindesc
\cts nulldelimiterspace {\param{dimension}}
\explain
^^{d\'elimiteurs//nuls, espace pour}
Ce param\`etre sp\'ecifie la largeur de
l'espace produit par un \minref{d\'elimiteur} nul.
\PlainTeX\ met |\nulldelimiterspace| \`a |1.2pt|.
\enddesc

\begindesc
\cts scriptspace {\param{dimension}}
\explain
Ce param\`etre sp\'ecifie le montant d'espace que \TeX\
ins\`ere avant et apr\`es un exposant ou un indice.
La commande |\nonscript| (\xref\nonscript) ^^|\nonscript|
apr\`es un exposant ou un indice annule cet espace.
\PlainTeX\ met |\script!-space| \`a |0.5pt|.
\enddesc

%==========================================================================
\section {Cat\'egoriser des constructions math\'ematiques}

\begindesc
\makecolumns 7/2:
\cts mathord {}
\cts mathop {}
\cts mathbin {}
\cts mathrel {}
\cts mathopen {}
\cts mathclose {}
\cts mathpunct {}
\explain
Ces commandes demandent \`a \TeX\ de traiter la construction qui suit comme appartenant
\`a une ^{classe} particuli\`ere (voir la \knuth{page~154}{179--180} pour la d\'efinition
des classes).  Elles sont list\'ees ici dans l'ordre des num\'eros de classe,
de $0$ \`a $6$.  
Leur effet primaire est d'ajuster l'espace\-ment autour de la construction 
comme \'etant celui de la classe indiqu\'ee.

\example
$\mathop{\rm minmax}\limits_{t \in A \cup B}\,t$
% By treating minmax as a math operator, we can get TeX to
% put something underneath it.
|
\produces
$\mathop{\rm minmax}\limits_{t \in A \cup B}\,t$
\endexample
\enddesc

\begindesc
\cts mathinner {}
\explain
Cette command demande \`a \TeX\ de traiter la construction qui suit
comme un ``formule interne'', c'est-\`a-dire, une fraction, pour l'espacement.
Elle ressemble aux commandes de classe donn\'ees ci-dessus.
\enddesc

%==========================================================================
\section {Actions sp\'eciales pour des formules math\'ematiques}

\begindesc
\cts everymath {\param{liste de token}}
\cts everydisplay {\param{liste de token}}
\explain
^^{affichages//actions pour chaque affichage}
Ces param\`etres sp\'ecifient des listes de \minref{token} que \TeX\ ins\`ere
au d\'ebut de toute formule math\'ematique d'affichage ou de texte, respectivement.
Vous pouvez prendre des mesures sp\'eciales au d\'ebut de chaque formule math\'ematique 
en assignant ces actions \`a |\everymath| ou \`a |\everydis!-play|. N'oubliez pas que 
si vous voulez que les deux genres de formules soient affect\'es, vous devez mettre 
les \emph{deux} param\`etres.
\example
\everydisplay={\heartsuit\quad}
\everymath = {\clubsuit}
$3$ is greater than $2$ for large values of $3$.
$$4>3$$
|
\produces
\everydisplay={\heartsuit\quad}
\everymath = {\clubsuit}
$3$ is greater than $2$ for large values of $3$.
$$4>3$$
\endexample
\enddesc

\enddescriptions
\eix^^{math\'ematiques}
\endchapter
\byebye

