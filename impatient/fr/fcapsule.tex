% This is part of the book TeX for the Impatient.
% Copyright (C) 2003 Paul W. Abrahams, Kathryn A. Hargreaves, Karl Berry.
% Copyright (C) 2004 Marc Chaudemanche pour la traduction fran�aise.
% See file fdl.tex for copying conditions.

\input fmacros
\chapter{Sommaire \linebreak des commandes}

\chapterdef{capsule}

Cette section contient des descriptions sur une ligne des commandes primitives de\TeX\
et des commandes \TeX\ d\'efinies dans \plainTeX. Cela inclut les s\'equences de
contr\^ole et les caract\`eres  
sp\'eciaux.
Nous avons omis les commandes qui ne sont  qu'\`a l'usage interne de la d\'efinition 
\plainTeX\ (\knuth{annexe~B}{}).
Notez que les caract\`eres ordinaires
comme `|a|' ou `|6|' sont aussi des commandes, et m\^eme les plus communes
\seeconcept{caract\`ere}.

Pour rendre la desciption la plus br\`eve possible, nous avons adopt\'e
certaines conventions~:

\ulist

\li Un ast\'erisque devant une commande indique que la commande est une primitive,
^^{primitive//commande} c'est-\`a-dire, construite dans le programme \TeX\
\seeconcept{primitive}. 

\li Les mots ``musique'', ``ponctuation'', ``fonction'',
``symbole'', ``relation'', ``d\'elimiteur'' ou ``op\'erateur'' dans une description de 
commande impliquent que la commande n'est l\'egale qu'en mode math\'ema\-tique.
 
\li Le verbe ``afficher'' s'applique \`a l'information que \TeX\ envoit au ^{\logfile}, 
sauf indication contraire. Si |\tracingonline| est positif, \TeX\ envoit \'egalement 
cette information sur le terminal. Nous utilisons le mot ``affichage'' pour nous 
r\'ef\'erer aux affichages math\'ema\-tiques (voir \xref{math\'ematique affich\'ee}), 
c'est-\`a-dire, \`a ce qui se trouve entre des |$$|. 

\li La phrase ``produit $x$'' indique que la commande composera
$x$ et mettra le r\'esultat dans une bo\^\i te.
Nous omettons parfois ``produit'' quand l'omission ne peut pas porter \`a
confusion. Par exemple, nous d\'ecrivons |\alpha| comme ``lettre Grecque
math\'ematique $\alpha$'' et non ``produit la lettre grecque math\'ematique $\alpha$''.
\margin{Remove explanations of ``space'' and ``glue''}

\endulist

\begincapsum

{\catcode `@ = \letter
\caplineout {\\\visiblespace} {espace inter mot}*{\@space}}%
   {\catcode `\ =\other\ctsidxref{ }}
\capcs ! {espace fin n\'egatif en math\'ematique}{}{\@shriek}
\capcs " {accent tr\'ema en texte, comme dans \"o}{}{\@quote}
\capactwo # {introduit un param\`etre de macro ou indique o\`u le texte    
source va dans un pr\'eambule d'alignement}{}{@msharp:@asharp}
\capcs # {produit le caract\`ere \# de la police courante}{}{\@pound}
\capac $ {commence ou finit une formule math\'ematique}{}{mathform}
\capcs $ {produit le caract\`ere \$ de la police courante}{}{\@bucks}
\capac % {commence un commentaire}*{commentaires}
\capcs % {produit le caract\`ere \% de la police courante}{}{\@percent}
\capac & {s\'epare les mod\`eles et les entr\'ees dans un alignement}{}{@and}
\capcs & {produit le caract\`ere \& de la police courante}{}{\@and}
\capac ' {symbole prime en math\'ematique, comme dans $p'$}{}{@prime}
\capcs ' {accent aigu en texte, comme dans \'e}{}{\@prime}
\capcs * {symbole multiplication qui autorise une coupure de ligne}{}{\@star}
\capcs + {d\'ebut d'une ligne tabul\'ee}{}{\@plus}
\capcs , {espace fin en math\'ematique}{}{\@comma}
\capcs - {sp\'ecifie un point de c\'esure l\'egal}*{\@minus}
\capcs . {accent point en texte, comme dans \.n}{}{\@dot}
\capcs / {correction italique pour le caract\`ere pr\'ec\'edent}*{\@slash}
\capcs ; {espace \'epaissi en math\'ematique}{}{\@semi}
\capcs = {accent macron en texte, comme dans \=r}{}{\@equal}
\capac \ {d\'ebute une s\'equence de contr\^ole}*{@backslash}
\capcs > {espace moyen en math\'ematique}{}{\@greater}
\capac ^ {produit une sous-formule sp\'ecifi\'ee en exposant}{}{@hat}
\capcs ^ {accent circonflexe en texte, comme dans \^o}{}{\@hat}
{\catcode `@ = \letter
\caplineout {\twocarets L}{\'equivalent \`a la primitive |\\par|}
            {}{\@par}\ttidxref{^^L}
\caplineout {\twocarets M}{une fin de ligne}*{@newline}\ttidxref{^^M}
}%
\capac _ {produit une sous-formule sp\'ecifi\'ee en indice}{}{@underscore}
\capcs _ {soulignement~: \_}{}{\@underscore}
%x \capac ` {in a \<number> context, \ascii\ code for character that follows}*{@lquote}
\capcs ` {accent grave en texte, comme dans \`e}{}{\@lquote}
\capac { {d\'ebute un groupe}{}{@lbrace}
\capcs { {d\'elimiteur accolade ouvrante en math\'ematique~ $\{$}{}{\@lbrace}
\capcs | {lignes parall\`eles en math\'ematique~: $\Vert$}{}{\@bar}
\capac } {fin d'un groupe}{}{@rbrace}
\capcs } {d\'elimiteur accolade fermante en math\'ematique~: $\}$}{}{\@rbrace}
\capac ~ {espace inter-mot sur lequel une ligne ne sera pas coup\'ee}{}{@not}
\capcs ~ {accent tilde en texte, comme dans \~a}{}{\@not}
\capcs aa {lettre scandinave~: \aa}{}{}
\capcs AA {lettre scandinave~: \AA}{}{}
\capcs above {produit une fraction avec une barre d'\'epaisseur sp\'ecifi\'ee}*{}
\capcs abovedisplayshortskip {ressort que \TeX\ ins\'ere avant un affichage quand la 
   ligne pr\'ec\'edente d\'epasse l'indentation d'affichage, par 
   d\'efaut 0\pt\ plus 3\pt}*{}
\capcs abovedisplayskip {ressort que \TeX\ ins\'ere avant un affichage quand la 
   ligne pr\'ec\'edente ne d\'epasse pas l'indentation d'affichage, par default 12\pt\
   plus 3\pt\ minus 9\pt}*{}
\capcs abovewithdelims {produit une fraction avec une barre d'\'epaisseur sp\'ecifi\'ee
   et entour\'ee des d\'elimiteurs sp\'ecifi\'es}*{}
\capcs accent {met l'accent sp\'ecifi\'e sur le caract\`ere suivant}*{}
\capcs active {code de cat\'egorie pour des caract\`eres actifs, c'est-\`a-dire, le nombre $13$}{}{}
\capcs acute {accent aigu en math\'ematique, comme dans $\acute x$}{}{}
\capcs adjdemerits {d\'em\'erites additionnels pour une coupure de ligne qui seront mises dans
   des lignes adjacentes avec un espacement de mot incompatible, par d\'efaut~10000}*{}
\capcs advance {ajoute un nombre \`a un registre |\\count|}*{}
\capcs advancepageno {si |\\pageno| est positif, ajoute un~;
   s'il est n\'egatif, soustrait un}{}{}
\capcs ae {ligature \ae}{}{}
\capcs AE {ligature \AE}{}{}
\capcs afterassignment {attend, pour d\'evelopper le token suivant, que 
   l'assignement d'apr\`es soit effectu\'e}*{}
\capcs aftergroup {attend, pour d\'evelopper le token suivant, la fin
   du groupe courant}*{}
\capcs aleph {lettre h\'ebra\^\i que seule en math\'ematique~: $\aleph$}{}{}
\capcstwo allowbreak {fait |\\penalty0|, c'est-\`a-dire, autorise une coupure de ligne ou
   de page l\`a o\`u elle ne devrait pas \^etre}{}{hallowbreak:vallowbreak}
\capcs alpha {lettre math\'ematique grecque $\alpha$}{}{}
\capcs amalg {op\'erateur d'amalgamation~: $\amalg$}{}{}
\capcs angle {symbole d'angle~: $\angle$}{}{}
\capcs approx {relation d'approximation~: $\approx$}{}{}
\capcs arccos {fonction arc cosinus~: $\arccos$}{}{}
\capcs arcsin {fonction arc sinus~: $\arcsin$}{}{}
\capcs arctan {fonction arc tangente~: $\arctan$}{}{}
\capcs arg {fonction argument (phase)~: $\arg$}{}{}
\capcs arrowvert {partie verticale d'une double fl\`eche extensible}{}{}
\capcs Arrowvert {partie verticale d'une fl\`eche simple extensible}{}{}
\capcs ast {op\'erateur ast\'erisque~: $\ast$}{}{}
\capcs asymp {relation asymptote~: $\asymp$}{}{}
\capcs atop {produit une fraction sans barre de fraction}*{}
\capcs atopwithdelims {produit une fraction sans barre de fraction et 
   entour\'ee des d\'elimiteurs sp\'ecifi\'es}*{}
\capcs b {accent barre du dessous en math\'ematique, comme dans $\b x$}{}{}
\capcs backslash {symbole antislash~: $\backslash$}{}{}
\capcs badness {la m\'ediocrit\'e du ressort mis dans la derni\`ere bo\^\i te
   fabriqu\'ee}*{}
\capcs bar {accent barre en math\'ematique, comme dans $\bar x$}{}{}
\capcs baselineskip {ressort pour la distance verticale normale entre une ligne de base 
   et la suivante, par d\'efaut 12\pt}*{}
\capcs batchmode {ne s'arr\^ete pas sur les erreurs et n'\'ecrit rien sur le terminal}*{}
\capcs begingroup {d\'ebute un groupe termin\'e par |\\endgroup|}*{}
\capcs beginsection {d\'ebute une subdivision majeure d'un document}{}%
   {\@beginsection}
\capcs belowdisplayshortskip {ressort que \TeX\ ins\`ere apr\`es un affichage quand la
   ligne pr\'ec\'edente d\'epasse l'indentation de l'affichage, 
      par d\'efaut 7\pt\ plus 0.3\pt\ minus 4\pt}*{}
\capcs belowdisplayskip {ressort que \TeX\ ins\`ere apr\`es un affichage quand la
   ligne pr\'ec\'edente ne d\'epasse pas l'indentation de l'affichage, 
      par d\'efaut 12\pt\ plus 3\pt\ minus 9\pt}*{}
\capcs beta {lettre grecque math\'ematique $\beta$}{}{}
\capcs bf {utilise le gras, c'est-\`a-dire, fait |\\tenbf\\fam=\\bffam|}{}{}
\capcs bffam {famille grasse en math\'ematique}{}{}
\capcs bgroup {caract\`ere de d\'ebut de groupe implicite}{}{}
\capcs big {rend le d\'elimiteur sp\'ecifi\'e plus grand qu'\`a l'ordinaire, mais
   encore assez petit pour du texte}{}{}
\capcs Big {rend le d\'elimiteur sp\'ecifi\'e haut de 11.5\pt}{}{}
\capcs bigbreak {indique une coupure de page d\'esirable avec |\\penalty-200|
   et produit un ressort |\\bigskipamount|}{}{}
\capcs bigcap {grand op\'erateur cap (non, cela ne produit pas une grande
   lettre capitale~!)~: $\bigcap$}{}{}
\capcs bigcirc {grand op\'erateur cercle~: $\bigcirc$}{}{}
\capcs bigcup {grand op\'erateur coupe~: $\bigcup$}{}{}
\capcs bigg {rend le d\'elimiteur sp\'ecifi\'e haut de 14.5\pt}{}{}
\capcs Bigg {rend le d\'elimiteur sp\'ecifi\'e haut de 17.5\pt}{}{}
\capcs biggl {taille comme |\\bigg|, mais espac\'e comme une ouverture}{}{}
\capcs Biggl {taille comme |\\Bigg|, mais espac\'e comme une ouverture}{}{}
\capcs biggm {taille comme |\\bigg|, mais espac\'e comme une relation}{}{}
\capcs Biggm {taille comme |\\Bigg|, mais espac\'e comme une relation}{}{}
\capcs biggr {taille comme |\\bigg|, mais espac\'e comme une fermeture}{}{}
\capcs Biggr {taille comme |\\Bigg|, mais espac\'e comme une fermeture}{}{}
\capcs bigl {taille comme |\\big|, mais espac\'e comme une ouverture}{}{}
\capcs Bigl {taille comme |\\Big|, mais espac\'e comme une ouverture}{}{}
\capcs bigm {taille comme |\\big|, mais espac\'e comme une relation}{}{}
\capcs Bigm {taille comme |\\Big|, mais espac\'e comme une relation}{}{}
\capcs bigodot {grand op\'erateur cercle point\'e~: $\bigodot$}{}{}
\capcs bigoplus {grand op\'erateur cercle plus~: $\bigoplus$}{}{}
\capcs bigotimes {grand op\'erateur cercle multipli\'e~: $\bigotimes$}{}{}
\capcs bigr {taille comme |\\big|, mais espac\'e comme une fermeture}{}{}
\capcs Bigr {taille comme |\\Big|, mais espac\'e comme une fermeture}{}{}
\capcs bigskip {produit un ressort |\\bigskipamount|}{}{}
\capcs bigskipamount {ressort pour une grand saut vertical, par d\'efaut 12\pt\
   plus 4\pt\ minus 4\pt}{}{}
\capcs bigsqcup {grand op\'erateur en coupe carr\'e~: $\bigsqcup$}{}{}
\capcs bigtriangledown {op\'erateur en triangle pointant vers le bas~:
   $\bigtriangledown$}{}{}
\capcs bigtriangleup {op\'erateur en triangle pointant vers le haut~: $\bigtriangleup$}{}{}
\capcs biguplus {grand op\'erateur en coupe avec plus~: $\biguplus$}{}{}
\capcs bigvee {grand op\'erateur logique ``ou''~: $\bigvee$}{}{}
\capcs bigwedge {grand op\'erateur logique ``et''~: $\bigwedge$}{}{}
\capcs binoppenalty {p\'enalit\'e suppl\'ementaire pour une coupure apr\`es un op\'erateur
   math\'ematique binaire, par d\'efaut~700}*{}
\capcs bmod {op\'erateur modulus, comme dans $n \bmod 2$}{}{}
\capcs bordermatrix {produit une matrice avec des labels de rang\'ees et de colonnes}{}{}
\capcs bot {symbole du bas d'un treillis ~: $\bot$}{}{}
\capcs botmark {le dernier \'el\'ement marqu\'e de la page qui vient d'\^etre mis en bo\^\i te}*{}
\capcs bowtie {relation en n\oe ud papillon~: $\bowtie$}{}{}
\capcs box {affiche la bo\^\i te d'un registre de bo\^\i te sp\'ecifi\'e
   dans la liste courante, et vide le registre}*{}
\capcs boxmaxdepth {profondeur maximale des vbox, par d\'efaut |\\maxdimen|}*{}
\capcs brace {|\char36 n\\brace k\char36| produit 
   une notation entre accolade~: $n \brace k$}{}{}
\capcs bracevert {morceau vertival d'une grande accolade extensible }{}{}
\capcs brack {|\char36 n\\brack k\char36| produit une notation entre crochets~: $n \brack k$}{}{}
\capcstwo break {fait |\\penalty-10000|, c'est-\`a-dire., force une coupure de ligne 
   ou de page}{}{hbreak:vbreak}
\capcs breve {accent bref en math\'ematique, comme dans $\breve x$}{}{}
\capcs brokenpenalty {p\'enalit\'e pour une coupure de ligne sur un \'el\'ement discr\'etionnaire, par
   d\'efaut~100}*{}
\capcs buildrel {produit une formule sp\'ecifi\'ee sur la relation sp\'ecifi\'e}{}{}
\capcs bullet {operateur bullet ~: $\bullet$}{}{}
\capcs bye {|\\vfill| la derni\`ere page avec de l'espace blanc, la |\\supereject|,
   et |\\end| l'ex\'ecution}{}{\@bye}
\capcs c {accent c\'edille en texte, comme dans \c c}{}{}
\capcs cal {utilise une police calligraphique pour des lettres capitales en math\'ematique,
   comme dans $\cal XYZ$}{}{}
\capcs cap {op\'erateur cap~: $\cap$}{}{}
\capcs cases {produit des cas en math\'ematique, comme dans $\bigl\{{\cdots\atop\cdots}$}{}{}
\capcs catcode {le code de cat\'egorie d'un caract\`ere sp\'ecifi\'e}*{}
\capcs cdot {op\'erateur point centr\'e~: $\cdot$}{}{}
\capcs cdotp {ponctuation point centr\'e~: $\cdotp$}{}{}
\capcs cdots {points centr\'es en math\'ematique~: $\cdots$}{}{}
\capcs centerline {produit une ligne avec son texte centr\'e}{}{}
\capcs char {produit le caract\`ere de la police courante avec le code 
   sp\'ecifi\'e}*{}
\capcs chardef {d\'efinit une s\'equence de contr\^ole sp\'ecifi\'ee comme \'etant un code de
   caract\`ere, un nombre entre 0 et $255$}*{}
\capcs check {accent tch\`eque en math\'ematique, comme dans $\check x$}{}{}
\capcs chi {lettre grecque math\'ematique $\chi$}{}{}
\capcs choose {|\char36 n\\choose k\char36| produit une notation combinatoire~:
   $n \choose k$}{}{}
\capcs circ {op\'eration cercle~: $\circ$}{}{}
\capcs cleaders {produit une r\`eglure avec la moiti\'e de l'espace restant avant
   la premi\`ere bo\^\i te et l'autre moiti\'e apr\`es la derni\`ere}*{}
\capcs cleartabs {efface toutes les tabulations pour des alignements tabul\'es}{}{}
\capcs closein {ferme un flot d'entr\'ee sp\'ecifi\'e}*{}
\capcs closeout {ferme un flot de sortie sp\'ecifi\'e}*{}
\capcs clubpenalty {p\'enalit\'e additionnelle pour une ligne seule restante avant
   une coupure de page, par d\'efaut~150}*{}
\capcs clubsuit {symbole tr\`efle~: $\clubsuit$}{}{}
\capcs colon {symbole :  en math\'ematique~: $:$}{}{}
\capcs cong {relation congruence~: $\cong$}{}{}
\capcs coprod {op\'erateur co-produit~: $\coprod$}{}{}
\capcs copy {comme |\\box|, mais n'efface pas le registre}*{}
\capcs copyright {marque de copyright~: \copyright}{}{}
\capcs cos {fonction cosinus~: $\cos$}{}{}
\capcs cosh {fonction cosinus hyperbolique~: $\cosh$}{}{}
\capcs cot {fonction cotangente~: $\cot$}{}{}
\capcs coth {fonction cotangente hyperbolique~: $\coth$}{}{}
\capcs count {le registre entier sp\'ecifi\'e}*{}
\capcs countdef {d\'efinit une s\'equence de contr\^ole sp\'ecifi\'ee comme \'etant un nombre
   correspondant \`a un registre |\\count|}*{}
\capcs cr {termine une rang\'ee (ou une colonne) dans un alignement}*{}
\capcs crcr {ne fait rien si la derni\`ere commande \'etait |\\cr| ou |\\noalign|~;
   autrement, \'equivalent \`a |\\cr|}*{}
\capcs csc {fonction co-s\'ecante~: $\csc$}{}{}
\capcs csname {d\'ebute un nom de s\'equence de contr\^ole devant \^etre termin\'e par |\\endcsname|}*{}
\capcs cup {op\'erateur cup~: $\cup$}{}{}
\capcs d {accent point en dessous en texte, comme dans \d r}{}{}
\capcs dag {symbole dague en texte~: \dag}{}{}
\capcs dagger {op\'erateur dague en math\'ematique~: $\dagger$}{}{}
\capcs dashv {relation touren \`a droite~: $\dashv$}{}{}
\capcs day {jour courant du mois, comme un nombre}*{}
\capcs ddag {symbole double dague en texte~: \ddag}{}{}
\capcs ddagger {op\'erateur double dague en math\'ematique~: $\ddagger$}{}{}
\capcs ddot  {accent deux points en math\'ematique~: $\ddot x$}{}{}
\capcs ddots {points en diagonale en math\'ematique~: \smash{$\ddots$}}{}{}
\capcs deadcycles {nombre d'initialisations d'|\\output| depuis le dernier
   |\\shipout|}*{}
\capcs def {d\'efini une s\'equence de contr\^ole comme \'etant une macro}*{}
\capcs defaulthyphenchar {code du caract\`ere de c\'esure par d\'efaut}*{}
\capcs defaultskewchar {code caract\`ere accent par d\'efaut }*{}
\capcs deg {fonction degr\'e~: $\deg$}{}{}
\capcs delcode {le code d\'elimiteur d'un caract\`ere sp\'ecifi\'e}*{}
\capcs delimiter {produit un d\'elimiteur sp\'ecifi\'e}*{}
\capcs delimiterfactor {1000 fois le ratio de la taille minimum d'un
   d\'elimiteur par rapport \`a la taille qui recouvrirait compl\`etement la formule, par
   d\'efaut~901}*{}
\capcs delimitershortfall {diff\'erence minimum entre une hauteur de formule et
   une hauteur de d\'elimiteur, par d\'efaut 5\pt}*{}
\capcs delta {lettre grecque math\'ematique $\delta$}{}{}
\capcs Delta {lettre grecque math\'ematique $\Delta$}{}{}
\capcs det {fonction d\'eterminant~: $\det$}{}{}
\capcs diamond {op\'erateur diamant~: $\diamond$}{}{}
\capcs diamondsuit {symbole carr\'e~: $\diamondsuit$}{}{}
\capcs dim {fonction dimension~: $\dim$}{}{}
\capcs dimen {le registre de dimension sp\'ecifi\'e}*{}
\capcs dimendef {d\'efinit une s\'equence de contr\^ole sp\'ecifi\'ee comme \'etant un nombre
   correspondant \`a un registre |\\dimen|}*{}
\capcs discretionary {sp\'ecifie trois textes, les deux premiers pour avant et
   apr\`es une coupure de ligne, le troisi\`eme quand il n'y a pas de coupure de ligne}*{}
\capcs displayindent {\TeX\ met ceci pour l'indentation d'un affichage}*{}
\capcs displaylimits {place des limites sur et sous des op\'erateurs seulement en style
   d'affichage}*{}
\capcs displaylines {produit un affichage multi-lignes sp\'ecifi\'e avec chaque 
   ligne centr\'ee}{}{}
\capcs displaystyle {utilise la taille displaystyle dans une formule}*{}
\capcs displaywidowpenalty {p\'enalit\'e pour une ligne seule d\'ebutant une page 
   juste avant un affichage, par d\'efaut~50}*{}
\capcs displaywidth {\TeX\ met ceci pour la largeur d'un affichage}*{}
\capcs div {op\'erateur division~: $\div$}{}{}
\capcs divide {divise un registre |\\count| sp\'ecifi\'e par un entier sp\'ecifi\'e}*{}
\capcs dot {accent point en math\'ematique, comme dans $\dot x$}{}{}
\capcs doteq {relation d'\'egalit\'e point\'ee~: $\doteq$}{}{}
\capcs dotfill {remplit un espace horizontal d\'elimit\'e avec des points}{}{}
\capcs dots {ellipse pour des suites~: $x_1$, \dots, $x_n$}{}{}
\capcs doublehyphendemerits {d\'em\'erites pour deux lignes cons\'ecutives se terminant
   par de c\'esures, par d\'efaut~10000}*{}
\capcs downarrow {relation~: $\downarrow$}{}{}
\capcs Downarrow {relation~: $\Downarrow$}{}{}
\capcs downbracefill {remplit la hbox inclue avec une accolade vers le bas ~:
   \hbox to 3.5em{\downbracefill}}{}{}
\capcs dp {profondeur de la bo\^\i te d'un registre de bo\^\i te sp\'ecifi\'e}*{}
\capcs dump {termine l'ex\'ecution et produit un fichier de format}*{}
\capcs edef {d\'efinit une s\'equence de contr\^ole comme \'etant une macro, 
   d\'eveloppant imm\'ediatement le texte de remplacement}*{}
\capcs egroup {caract\`ere de fin de groupe implicite}{}{}
\capcs eject {termine le paragraphe courant et force une coupure de page,
   en \'etirant la page courante}{}{}
\capcs ell {lettre ``l'' manuscrite en math\'ematique~: $\ell$}{}{}
\capcs else {cas faux ou par d\'efaut pour une condition}*{\@else}
\capcs emergencystretch {\'etirement suppl\'ementaire ajout\'e \`a toute ligne si
   |\\tol\-er\-ance| n'est pas satisfait}*{}
\capcs empty {macro dont le d\'eveloppement ne fait rien}{}{}
\capcs emptyset {symbole d'ensemble vide~: $\emptyset$}{}{}
\capcs end {|\\output| la derni\`ere page et termine l'ex\'ecution}*{}
\capcs endcsname {termine un nom de s\'equence de contr\^ole commenc\'e par
   |\\csname|}*{}
\capcs endgraf {\'equivalent \`a la primitive |\\par|}{}{}
\capcs endgroup {termine un groupe d\'ebut\'e par |\\begingroup|}*{}
\capcs endinput {termine l'entr\'ee du fichier courant}*{}
\capcs endinsert {fin d'insertion}{}{}
\capcs endline {\'equivalent \`a la primitive |\\cr|}{}{}
\capcs endlinechar {caract\`ere que \TeX\ ins\`ere \`a la fin de chaque ligne
   entr\'ee, par d\'efaut |\twocarets M|}*{}
\capcs enskip {ressort horizontal de largeur \frac1/2\em}{}{}
\capcs enspace {cr\'enage de \frac1/2\em}{}{}
\capcs epsilon {lettre grecque math\'ematique $\epsilon$}{}{}
\capcs eqalign {produit un affichage multi-lignes sp\'ecifi\'e dont les parties indiqu\'ees
   sont align\'ees verticalement}{}{}
\capcs eqalignno {produit un affichage multi-lignes sp\'ecifi\'e 
   avec des num\'eros d'\'equation dont les parties indiqu\'ees sont align\'ees verticalement}{}{}
\capcs eqno {met un num\'ero d'\'equation sp\'ecifi\'e \`a droite d'un affichage}*{}
\capcs equiv {relation d'\'equivalence~: $\equiv$}{}{}
\capcs errhelp {liste de token dont le d\'eveloppement \TeX\ s'affiche quand l'utilisateur
   demande de l'aide en r\'eponse \`a un |\\errmessage|}*{}
\capcs errmessage {donne un message d'erreur sp\'ecifi\'e}*{}
\capcs errorcontextlines {le nombre de lignes de contexte que \TeX\
   affiche pour une erreur, par d\'efaut~5}*{}
\capcs errorstopmode {stoppe pour une interaction sur des messages d'erreur}*{}
\capcs escapechar {caract\`ere avec lequel \TeX\ pr\'ec\`ede les noms de s\'equence de contr\^ole
   qui sont affich\'es}*{}
\capcs eta {lettre grecque math\'ematique $\eta$}{}{}
\capcs everycr {liste de token que \TeX\ d\'eveloppe apr\`es un |\\cr| ou un |\\crcr|
   non suivi de |\\cr| ou de |\\noalign|}*{}
\capcs everydisplay {liste de token que \TeX\ d\'eveloppe quand un affichage math\'ematique d\'ebute}*{}
\capcs everyhbox {liste de token que \TeX\ d\'eveloppe quand une hbox d\'ebute}*{}
\capcs everyjob {liste de token que \TeX\ d\'eveloppe quand une ex\'ecution d\'ebute}*{}
\capcs everymath {liste de token que \TeX\ d\'eveloppe quand un texte en mode math\'ematique
    d\'ebute}*{}
\capcs everypar {liste de token que \TeX\ d\'eveloppe quand un paragraphe d\'ebute}*{}
\capcs everyvbox {liste de token que \TeX\ d\'eveloppe quand une vbox d\'ebute}*{}
\capcs exhyphenpenalty {p\'enalit\'e suppl\'ementaire pour une coupure de ligne apr\`es 
   une c\'esure explicite, par d\'efaut~50}*{}
\capcs exists {symbole ``il existe''~: $\exists$}{}{}
\capcs exp {fonction exponentielle~: $\exp$}{}{}
\capcs expandafter {ne d\'eveloppe le token suivant qu'apr\`es d\'eveloppement du token
   le suivant}*{}
\capcs fam {famille de police que \TeX\ utilise pour des caract\`eres de classe 7
   (c'est-\`a-dire, des variables) en mathi\'ematiques}*{}
\capcs fi {termine une condition}*{\@fi}
\capcs filbreak {force une coupure de page \`a moins que le texte contenant un autre |\\filbreak|
   convienne aussi sur la page}{}{}
\capcs finalhyphendemerits {p\'enalit\'e pour l'avant-derni\`ere ligne coup\'ee sur une
   c\'esure, par d\'efaut~5000}*{}
\capcs firstmark {premier \'el\'ement marqu\'e sur la page qui vient d'\^etre mise en bo\^\i te}*{}
\capcs fivebf {utilise une police grasse en $5$ points, |cmbx5|}{}{}
\capcs fivei {utilise une police math\'ematique italique en $5$ points, |cmmi5|}{}{}
\capcs fiverm {utilise une police romaine en $5$ points, |cmr5|}{}{}
\capcs fivesy {utilise une police de symbole en $5$ points, |cmsy5|}{}{}
\capcs flat {symbole b\'emol en musique~: $\flat$}{}{}
\capcs floatingpenalty {p\'enalit\'e pour des insertions qui sont s\'epar\'ees sur des
   pages, par d\'efaut~0}*{}
\capcs fmtname {nom du format courant}{}{}
\capcs fmtversion {num\'ero de version du format courant}{}{}
\capcs folio {produit |\\pageno| comme des caract\`eres~; 
   en chiffres romains s'il est n\'egatif}{}{}
\capcs font {d\'efinit une s\'equence de contr\^ole sp\'ecifique pour s\'electionner une police}*{}
\capcs fontdimen {un param\`etre sp\'ecifi\'e d'une police sp\'ecifi\'ee}*{}
\capcs fontname {produit le nom de fichier d'une police sp\'ecifi\'ee en caract\`eres}*{}
\capcs footline {liste de token qui produit une ligne en bas de chaque page}{}{}
\capcs footnote {produit une note de pied de page sp\'ecifi\'ee 
   avec une marque de r\'ef\'erence sp\'ecifi\'ee}{}{}
\capcs forall {symbole ``pour tout''~: $\forall$}{}{}
\capcs frenchspacing {rend l'espacement inter-mot ind\'ependant de la ponctuation}{}{}
\capcs frown {relation frown~: $\frown$}{}{}
\capcs futurelet {assigne le troisi\`eme token suivant \`a une s\'equence de contr\^ole 
   sp\'ecifi\'ee, puis d\'eveloppe le deuxi\`eme token}*{}
\capcs gamma {lettre grecque math\'ematique $\gamma$}{}{}
\capcs Gamma {lettre grecque math\'ematique $\Gamma$}{}{}
\capcs gcd {fonction plus grand d\'enominateur commun~: $\gcd$}{}{}
\capcs gdef {\'equivalent \`a |\\global\\def|, c'est-\`a-dire, d\'efinit une macro globale}*{}
\capcs ge {relation plus grand ou \'egal~: $\ge$}{}{}
\capcs geq {\'equivalent \`a |\\ge|}{}{}
\capcs gets {relation ''donne''~: $\gets$}{}{}
\capcs gg {relation beaucoup plus grand que~: $\gg$}{}{}
\capcs global {rend la d\'efinition suivante globale}*{}
\capcs globaldefs {remplace le pr\'efixe|\\global| dans les assignements}*{}
\capcs goodbreak {indique une coupure de page souhaitable avec |\\penalty-500|}{}{}
\capcs grave {accent grave en math\'ematique, comme dans $\grave x$}{}{}
\capcs H {accent tr\'ema hongrois en texte, comme dans \H o}{}{}
\capcs halign {aligne du texte en colonnes}*{}
\capcs hang {indente le paragraphe courant de |\\parindent|}{}{}
\capcs hangafter {num\'ero de ligne d\'ebutant une indentation hanging}*{}
\capcs hangindent {espace pour indentation hanging}*{}
\capcs hat {accent circonflexe en math\'ematique, comme dans $\hat x$}{}{}
\capcs hbadness {seuil de badness pour reporter une erreur "underfull or overfull
   hboxes", par d\'efaut 1000}*{}
\capcs hbar {symbole math\'ematique~: $\hbar$}{}{}
\capcs hbox {produit une hbox sp\'ecifi\'ee}*{}
\capcs headline {liste de token qui produit la ligne en haut
   de chaque page}{}{}
\capcs heartsuit {symbole c\oe ur~: $\heartsuit$}{}{}
\capcs hfil {produit un ressort horizontal infiniment \'etirable}*{}
\capcs hfill {produit un ressort horizontal encore plus infiniment \'etirable
   que celui produit par |\\hfil|}*{}
\capcs hfilneg {produit un ressort horizontal n\'egatif infiniment \'etirable}*{}
\capcs hfuzz {seuil pour reporter overfull hboxes, par d\'efaut 
   0.1\pt}*{}
\capcs hglue {produit un ressort horizontal qui ne disparait pas sur des
   coupure de ligne}{}{}
\capcs hidewidth {ignore la largeur d'une entr\'ee d'un alignement, pour qu'il 
   l'\'etende au dehors de sa boite dans la direction de |\\hidewidth|}{}{}
\capcs hoffset {offset horizontal relatif d'un pouce par rapport au bord gauche du papier}*{}
\capcs holdinginserts {si positif, n'enl\`eve pas d'insertion de la
   page courante}*{}
\capcs hom {fontion homologie~: $\hom$}{}{}
\capcs hookleftarrow {relation~: $\hookleftarrow$}{}{}
\capcs hookrightarrow {relation~: $\hookrightarrow$}{}{}
\capcs hphantom {produit une formule invisible de hauteur et largeur z\'ero mais
   de largeur naturelle}{}{}
\capcs hrule {produit un filet horizontal~; l\'egale seulement en mode vertical}*{}
\capcs hrulefill {rempli un espace d\'elimit\'e d'un filet horizontal}{}{}
\capcs hsize {longueur de ligne, par d\'efaut 6.5\thinspace in}*{}
\capcs hskip {produit un ressort horizontal sp\'ecifi\'e}*{}
\capcs hss {produit un ressort horizontal qui est infiniment \'etirable et 
   infiniment r\'etr\'ecissable}*{}
\capcs ht {la hauteur de la bo\^\i te d'un registre de bo\^\i te sp\'ecifi\'e}*{}
\capcs hyphenation {ajoute des mots sp\'ecifi\'es au dictionnaire
   d'exception de c\'esure}*{}
\capcs hyphenchar {le caract\`ere de c\'esure d'une police sp\'ecifi\'ee}*{}
\capcs hyphenpenalty {p\'enalit\'e suppl\'ementaire pour une coupure de ligne sur une c\'esure, par
   d\'efaut~50}*{}
\capcs i {lettre `\i' sans point \`a utiliser avec des accents}{}{}
\capcs ialign {d\'ebute un |\\halign| avec le ressort |\\tabskip| \`a z\'ero et
   |\\everycr| vide}{}{}
\capcs if {teste si deux tokens sp\'ecifi\'es ont le m\^eme code de caract\`ere}*{\@if}
\capcs ifcase {d\'eveloppe un cas $n$ pour un valeur $n$ sp\'ecifi\'ee}*{\@ifcase}
\capcs ifcat {teste si deux tokens sp\'ecifi\'es ont le m\^eme code de
   cat\'egorie}*{\@ifcat}
\capcs ifdim {teste une relation sp\'ecifi\'ee entre deux dimensions }*{\@ifdim}
\capcs ifeof {teste la fin d'un fichier sp\'ecifi\'e}*{\@ifeof}
\capcs iff {relation si et seulement si~: $\iff$}{}{}
\capcs iffalse {test qui est toujours faux}*{\@iffalse}
\capcs ifhbox {teste si un registre de bo\^\i te sp\'ecifi\'e contient une hbox}*{\@ifhbox}
\capcs ifhmode {teste si \TeX\ est dans un mode horizontal}*{\@ifhmode}
\capcs ifinner {teste si \TeX\ est dans un mode interne}*{\@ifinner}
\capcs ifmmode {teste si \TeX\ est dans un mode math\'ematique}*{\@ifmmode}
\capcs ifnum {teste une relation sp\'ecifi\'ee entre deux nombres}*{\@ifnum}
\capcs ifodd {teste si un nombre sp\'ecifi\'e est impair}*{\@ifodd}
\capcs iftrue {test qui est toujours vrai}*{\@iftrue}
\capcs ifvbox {teste si un registre de bo\^\i te sp\'ecifi\'e contient une vbox}*{\@ifvbox}
\capcs ifvmode {teste si \TeX\ est dans un mode vertical}*{\@ifvmode}
\capcs ifvoid {teste si un registre de bo\^\i te sp\'ecifi\'e est vide}*{\@ifvoid}
\capcs ifx {teste si deux tokens sont les m\^emes, ou si
   deux macros ont les m\^emes d\'efinitions finales}*{\@ifx}
\capcs ignorespaces {ignore tous les tokens espaces suivants}*{}
\capcs Im {symbole d'une partie imaginaire d'un complexe~: $\Im$}{}{}
\capcs imath {lettre `$\imath$' sans point \`a utiliser avec des accents math\'ematiques}{}{}
\capcs immediate {effectue l'op\'eration de fichier sp\'ecifi\'ee sans d\'elai}*{}
\capcs in {relation ``appartient \`a''~: $\in$}{}{}
\capcs indent {produit une bo\^\i te vide de largeur |\\parindent| et entre
   en mode horizontal}*{}
\capcs inf {fonction inf\'erieur~: $\inf$}{}{}
\capcs infty {symbole infini~: $\infty$}{}{}
\capcs input {commence \`a lire \`a partir d'un fichier sp\'ecifi\'e}*{}
\capcs inputlineno {le num\'ero de ligne courante du fichier d'entr\'ee courant}*{}
\capcs insert {produit une insertion d'une classe sp\'ecifi\'ee}*{}
\capcs insertpenalties {somme des p\'enalit\'es des insertions}*{}
\capcs int {symbole int\'egrale~: $\int$}{}{}
\capcs interlinepenalty {p\'enalit\'e suppl\'ementaire pour une coupure de page
   entre des lignes d'un paragraphe, par d\'efaut~0}*{}
\capcs iota {lettre grecque math\'ematique $\iota$}{}{}
\capcs it {utilise des italiques, c'est-\`a-dire, fait |\\tenit\\fam=\\itfam|}{}{}
\capcs item {d\'ebute un paragraphe avec une indentation accroch\'ee de |\\parindent| 
   et pr\'ec\'ed\'ee d'un label sp\'ecifi\'e}{}{}
\capcs itemitem {comme |\\item|, mais avec une indentation de |2\\parindent|}{}{}
\capcs itfam {famille italique en math\'ematique}{}{}
\capcs j {lettre `\j' sans point, pour utiliser avec des accents}{}{}
\capcs jmath {lettre `$\jmath$' sans point, pour utiliser avec des accents math\'ematiques}{}{}
\capcs jobname {nom de base du fichier par lequel \TeX\ a \'et\'e appel\'e}*{}
\capcs jot {unit\'e de mesure for opening up displays}{}{}
\capcs kappa {lettre grecque math\'ematique $\kappa$}{}{}
\capcs ker {fonction kern~: $\ker$}{}{}
\capcs kern {produit un montant d'espace sp\'ecifi\'e sur lequel
   une coupure n'est pas autoris\'ee}*{}
\capcs l {lettre polonaise~: \l}{}{}
\capcs L {lettre polonaise~: \L}{}{}
\capcs lambda {lettre grecque math\'ematique $\lambda$}{}{}
\capcs Lambda {lettre grecque math\'ematique $\Lambda$}{}{}
\capcs land {op\'erateur logique ``et''~: $\land$}{}{}
\capcs langle {left angle delimiter~: $\langle$}{}{}
\capcs language {le jeu de patrons de c\'esure courant}*{}
\capcs lastbox {retrouve et enl\`eve le dernier \'el\'ement de la liste courante, si c'est une
   bo\^\i te}*{}
\capcs lastkern {retrouve le dernier \'el\'ement de la liste courante, si c'est un
   cr\'enage}*{}
\capcs lastpenalty {enl`eve le dernier item de la liste courante, si c'est une p/'enalit\'e }*{}
\capcs lastskip {enl\`eve le dernier item de la liste courante, si c'est une glue}*{}
\capcs lbrace {d\'elimiteur accolade gauche~: $\lbrace$}{}{}
\capcs lbrack {d\'elimiteur crochet gauche~: $\lbrack$}{}{}
\capcs lccode {le code de caract\`ere pour la forme minuscule d'une lettre}*{}
\capcs lceil {d\'elimiteur plafond gauche~: $\lceil$}{}{}
\capcs ldotp {point sur la ligne de base comme ponctuation~: $\ldotp$}{}{}
\capcs ldots {points sur la ligne de base en math\'ematique~: $\ldots$}{}{}
\capcs le {relation inf\'erieur ou \'egal~: $\le$}{}{}
\capcs leaders {remplit un espace horizontal ou vertical sp\'ecifi\'e en r\'ep\'etant un
   filet ou une bo\^\i te sp\'ecifi\'ee}*{}
\capcs left {produit le d\'elimiteur sp\'ecifi\'e, en l'agrandissant pour couvrir la
   sous-formule suivante se finissant par |\\right|}*{}
\capcs leftarrow {relation~: $\leftarrow$}{}{}
\capcs Leftarrow {relation~: $\Leftarrow$}{}{}
\capcs leftarrowfill {rempli la hbox contenante avec un |\\leftarrow|~:
   \hbox to 3.5em{\leftarrowfill}}{}{}
\capcs leftharpoondown {relation~: $\leftharpoondown$}{}{}
\capcs leftharpoonup {relation~: $\leftharpoonup$}{}{}
\capcs lefthyphenmin {taille du plus petit fragment de mot que \TeX\ autorise
   avant une c\'esure en d\'ebut d'un mot, par d\'efaut~2}*{}
\capcs leftline {produit une ligne avec son texte pouss\'e vers la marge de gauche}{}{}
\capcs leftrightarrow {relation~: $\leftrightarrow$}{}{}
\capcs Leftrightarrow {relation~: $\Leftrightarrow$}{}{}
\capcs leftskip {ressort que \TeX\ ins\`ere \`a gauche de chaque ligne}*{}
\capcs leq {\'equivalent \`a |\\le|}{}{}
\capcs leqalignno {produit un affichage multi-ligne sp\'ecifi\'e avec des num\'eros d'\'equation
   \`a gauche dont les parties indiqu\'ees sont align\'ees verticalement}{}{}
\capcs leqno {d\'epose un num\'ero d'\'equation sp\'ecifi\'e \`a gauche d'un affichage}*{}
\capcs let {d\'efinit une s\'equence de contr\^ole comme \'etant le token suivant}*{}
\capcs lfloor {d\'elimiteur plancher gauche~: $\lfloor$}{}{}
\capcs lg {fonction logarithme~: $\lg$}{}{}
\capcs lgroup {d\'elimiteur groupe gauche (la plus petite taille est montr\'ee ici)~:
   $\Big\lgroup$}{}{}
\capcs lim {fonction limite~: $\lim$}{}{}
\capcs liminf {fonction limite inf\'erieur~: $\liminf$}{}{}
\capcs limits {place un exposant sur et un indice sous un
   grand op\'erateur}*{}
\capcs limsup {fonction limite sup\'erieure~: $\limsup$}{}{}
\capcs line {produit une ligne de caract\`eres justifi\'ee}{}{}
\capcs linepenalty {p\'enalit\'e pour coupure de ligne ajout\'ee \`a chaque ligne, 
   par d\'efaut~10}*{}
\capcs lineskip {ressort vertical d'une ligne de base \`a la suivante si les
   lignes sont plus proche l'une de l'autre que |\\lineskiplimit|, par d\'efaut 1\pt}*{}
\capcs lineskiplimit {seuil pour utiliser |\\lineskip| au lieu de |\\base\-line\-skip|, par d\'efaut 0\pt}*{}
\capcs ll {relation beaucoup moins que~: $\ll$}{}{}
\capcs llap {produit du texte (sans largeur) \`a partir de la gauche
   de la position courante}{}{}
\capcs lmoustache {moiti\'e haute d'une grande accolade~: $\big\lmoustache$}{}{}
\capcs ln {fonction logarithme naturel~: $\ln$}{}{}
\capcs lnot {symbole logique ``non''~: $\lnot$}{}{}
\capcs log {fonction logarithme~: $\log$}{}{}
\capcs long {autorise des tokens |\\par| dans le(s) argument(s) de
   la d\'efinition suivante}*{}
\capcs longleftarrow {relation~: $\longleftarrow$}{}{}
\capcs Longleftarrow {relation~: $\Longleftarrow$}{}{}
\capcs longleftrightarrow {relation~: $\longleftrightarrow$}{}{}
\capcs Longleftrightarrow {relation~: $\Longleftrightarrow$}{}{}
\capcs longmapsto {relation~: $\longmapsto$}{}{}
\capcs longrightarrow {relation~: $\longrightarrow$}{}{}
\capcs Longrightarrow {relation~: $\Longrightarrow$}{}{}
\capcs loop {d\'ebute une boucle devant se finir par |\\repeat|}{}{}
\capcs looseness {diff\'erence entre le nombre de lignes d'un paragraphe que vous 
   voulez par rapport au nombre optimal}*{}
\capcs lor {op\'erateur logique ``ou''~: $\lor$}{}{}
\capcs lower {abaisse une bo\^\i te sp\'ecifi\'e d'un montant sp\'ecifi\'e}*{}
\capcs lowercase {convertit les lettres capitales du texte sp\'ecifi\'e
   en minuscules}*{}
\capcs lq {caract\`ere quote gauche pour du texte~: \lq}{}{}
\capcs mag {$1000$ fois le ratio pour agrandir toutes les dimensions}*{}
\capcs magnification {comme |\\mag|, mais n'agrandit pas la taille de la page}{}{}
\capcs magstep {$1000 \cdot 1.2^n$ pour un $n$ sp\'ecifi\'e}{}{}
\capcs magstephalf {$1000\cdot\sqrt{1.2}$}{}{}
\capcs mapsto {relation~: $\mapsto$}{}{}
\capcs mark {produit un \'el\'ement marqu\'e avec un texte sp\'ecifi\'e}*{}
\capcs mathaccent {met un accent math\'ematique sp\'ecifi\'e sur le caract\`ere suivant}*{}
\capcs mathbin {espace une sous-formule sp\'ecifi\'ee comme une op\'eration binaire}*{}
\capcs mathchar {produit le caract\`ere math\'ematique de mathcode sp\'ecifi\'e}*{}
\capcs mathchardef {d\'efinit une s\'equence de contr\^ole sp\'ecifi\'ee comme \'etant un mathcode,
   un nombre entre 0 et $2^{15}-1$}*{}
\capcs mathchoice {s\'electionne une des quatre sous-formules math\'ematiques sp\'ecifi\'ees
   selon le style courant}*{}
\capcs mathclose {espace une sous-formule sp\'ecifi\'ee comme un d\'elimiteur fermant}*{}
\capcs mathcode {le mathcode d'un caract\`ere sp\'ecifi\'e}*{}
\capcs mathinner {espace une sous-formule sp\'ecifi\'ee comme une formule interne, c'est-\`a-dire, 
   une fraction}*{}
\capcs mathop {espace une sous-formule sp\'ecifi\'ee comme un grand op\'erateur math\'ematique}*{}
\capcs mathopen {espace une sous-formule sp\'ecifi\'ee comme un d\'elimiteur ouvrant}*{}
\capcs mathord {espace une sous-formule sp\'ecifi\'ee comme un caract\`ere ordinaire}*{}
\capcs mathpalette {produit un |\\mathchoice| qui d\'eveloppe une s\'equence de contr\^ole sp\'ecifi\'ee d\'ependant du style courant}{}{}
\capcs mathpunct {espace une sous-formule sp\'ecifi\'ee comme une ponctuation}*{}
\capcs mathrel {espace une sous-formule sp\'ecifi\'ee comme une relation}*{}
\capcs mathstrut {produit une bo\^\i te invisible avec la hauteur et la largeur d'une
   parenth\`ese gauche et sans largeur}{}{}
\capcs mathsurround {espace que \TeX\ ins\`ere avant et apr\`es les maths dans le texte}*{}
\capcs matrix {produit une matrice sp\'ecifi\'ee}{}{}
\capcs max {fonction maximum~: $\max$}{}{}
\capcs maxdeadcycles {valeur de |\\deadcycles| sur laquelle \TeX\ se plaint,
   et utilise alors sa propre routine de sortie, par d\'efaut~25}*{}
\capcs maxdepth {profondeur maximum de la bo\^\i te en bas d'une page,
   par d\'efaut 4\pt}*{}
\capcs maxdimen {plus grande dimension acceptable par \TeX}{}{}
\capcs meaning {produit la signification compr\'ehensible humainement d'un token
   sp\'ecifi\'e comme des caract\`eres}*{}
\capcs medbreak {indique une coupure de page
   souhaitable avec |\\penalty-100| et produit un ressort |\\medskipamount|}{}{}
\capcs medmuskip {ressort pour un espace math\'ematique moyen, par d\'efaut 4\mud\ plus 2\mud\
   minus 4\mud}*{}
\capcs medskip {produit un ressort |\\medskipamount|}{}{}
\capcs medskipamount {ressort pour un saut vertical moyen, par d\'efaut 6\pt
   plus 2\pt\ minus 2\pt}{}{}
\capcs message {affiche le d\'eveloppement du texte sp\'ecifi\'e sur le terminal}*{}
\capcs mid {middle relation~: $\mid$}{}{}
\capcs midinsert {produit le texte sp\'ecifi\'e \`a la position courante si
   possible, sinon en haut de la page suivante}{}{}
\capcs min {fonction minimum~: $\min$}{}{}
\capcs mit {utilise des italiques math\'ematiques, c'est-\`a-dire, fait |\\fam=1|}{}{}
\capcs mkern {produit un cr\'enage sp\'ecifi\'e en unit\'es |mu| en math\'ematique}*{}
\capcs models {relation mod\`ele ~: $\models$}{}{}
\capcs month {mois courant, comme un nombre}*{}
\capcs moveleft {d\'eplace une bo\^\i te sp\'ecifi\'e \`a gauche d'un espace sp\'ecifi\'e~; l\'egal
   uniquement en mode vertical}*{}
\capcs moveright {d\'eplace une bo\^\i te sp\'ecifi\'e \`a droite d'un espace sp\'ecifi\'e~; l\'egal
   uniquement en mode vertical}*{}
\capcs mp {op\'erateur plus et moins~: $\mp$}{}{}
\capcs mskip {produit un ressort sp\'ecifi\'e en unit\'es |mu| en math\'ematique}*{}
\capcs mu {lettre grecque math\'ematique $\mu$}{}{}
\capcs multiply {multiplie un registre |\\count| sp\'ecifi\'e par un entier
   sp\'ecifi\'e}*{}
\capcs multispan {fait traverser l'entr\'ee d'alignement suivante un nombre sp\'ecifi\'e
   de colonnes (ou de rang\'ees)}{}{}
\capcs muskip {le registre muglue sp\'ecifi\'e}*{}
\capcs muskipdef {d\'efinit une s\'equence de contr\^ole sp\'ecifi\'e comme un nombre
   correspondant \`a un registre |\\muskip|}*{}
\capcs nabla {symbole diff\'erence \`a l'envers~: $\nabla$}{}{}
\capcs narrower {diminue les marges droite et gauche d'un |\\parindent|}{}{}
\capcs natural {symbole naturel en musique~: $\natural$}{}{}
\capcs nearrow {relation fl\`eche montante vers la droite~: $\nearrow$}{}{}
\capcs ne {relation non \'egal~: $\ne$}{}{}
\capcs neg {symbole logique ``non''~: $\neg$}{}{}
\capcs negthinspace {kern $-\frac1/6$\em}{}{}
\capcs neq {relation non \'egal~: $\neq$}{}{}
\capcs newbox {r\'eserve et nomme un registre |\\box|}{}{\@newbox}
\capcs newcount {r\'eserve et nomme un registre |\\count|}{}{\@newcount}
\capcs newdimen {r\'eserve et nomme un registre |\\dimen|}{}{\@newdimen}
\capcs newfam {r\'eserve et nomme une famille math\'ematique}{}{\@newfam}
\capcs newhelp {nomme un message d'erreur sp\'ecifi\'e}{}{\@newhelp}
\capcs newif {d\'efinit un nouveau test conditionel avec le nom sp\'ecifi\'e}{}{\@newif}
\capcs newinsert {nomme une classe d'insertion et r\'eserve des registres
   |\\box|, |\\count|, |\\dimen| et |\\skip| correspondants}
   {}{\@newinsert}
\capcs newlanguage {r\'eserve et nomme un |\\language|}{}{\@newlanguage}
\capcs newlinechar {caract\`ere fin de ligne pour |\\write|, etc.}*{}
\capcs newmuskip {r\'eserve et nomme un registre |\\muskip|}{}{\@newmuskip}
\capcs newread {r\'eserve et nomme un flot d'entr\'ee}{}{\@newread}
\capcs newskip {r\'eserve et nomme un registre |\\skip|}{}{\@newskip}
\capcs newtoks {r\'eserve et nomme un registre |\\toks|}{}{\@newtoks}
\capcs newwrite {r\'eserve et nomme un flot de sortie}{}{\@newwrite}
\capcs ni {relation ``dans invers\'e''~: $\ni$}{}{}
\capcs noalign {ins\`ere du mat\'eriel entre des rang\'ees (ou des colonnes) d'un
   alignement}*{}
\capcs noboundary {inhibe des ligatures ou cr\'enages dues au |boundarychar| de la police
   courante}*{}
\capcstwo nobreak {fait |\\penalty10000|, c'est-\`a-dire,
   inhibe une coupure de ligne ou de page}{}{hnobreak:vnobreak}
\capcs noexpand {supprime le d\'eveloppement du token suivant}*{}
\capcs noindent {entre en mode horizontal sans indenter le paragraphe}*{}
\capcs nointerlineskip {inhibe le ressort inter-ligne avant la ligne suivante}{}{}
\capcs nolimits {place un exposant et un indice apr\`es de grands op\'erateurs}*{}
\capcs nonfrenchspacing {rend l'espacement inter-mot d\'ependant de la ponctuation}{}{}
\capcs nonscript {inhibe tout ressort ou cr\'enage suivant 
   en styles script et scriptscript}*{}
\capcs nonstopmode {ne stoppe pas sur des erreurs, m\^eme celles relatives aux fichiers manquants}*{}
\capcs nopagenumbers {inhibe l'impression des num\'eros de page, c'est-\`a-dire, fait
   |\\footline = {\\hfil}|}{}{}
\capcs normalbaselines {met |\\baselineskip|, |\\line\-skip| et
   |\\line\-skip\-limit| aux valeurs normales pour la taille de la police courante}{}{}
\capcs normalbaselineskip {valeur de |\\baselineskip| pour la
   taille de la police courante}{}{}
\capcs normalbottom {rend la marge du bas identique de page en page}{}{}
\capcs normallineskip {valeur de |\\lineskip| pour la taille de caract\`ere
   courante}{}{}
\capcs normallineskiplimit {valeur de |\\lineskiplimit| pour la taille de caract\`ere
   courante}{}{}
\capcs not {un |\\/| sans largeur pour construire des n\'egations de relation
   mat\'ematiques, comme dans $\not=$}{}{}
\capcs notin {relation n'appartient pas~: $\notin$}{}{}
\capcs nu {lettre grecque math\'ematique $\nu$}{}{}
\capcs null {se d\'eveloppe en une bo\^\i te vide}{}{}
\capcs nulldelimiterspace {espace produit par un d\'elimiteur nul, par 
   d\'efaut 1.2\pt}*{}
\capcs nullfont {police primitive sans characters}*{}
\capcs number {produit un chiffre sp\'ecifi\'e en caract\`eres}*{}
\capcs nwarrow {relation fl\`eche en haut vers la gauche~: $\nwarrow$}{}{}
\capcs o {lettre danoise~: \o}{}{}
\capcs O {lettre danoise~: \O}{}{}
\capcs obeylines {rend chaque fin de ligne du fichier entr\'ee
   \'equivalente \`a |\\par|}{}{}
\capcs obeyspaces {produit un espace dans la sortie pour chaque caract\`ere espace dans
   l'entr\'ee}{}{}
\capcs odot {op\'eration point centr\'e~: $\odot$}{}{}
\capcs oe {ligature \oe}{}{}
\capcs OE {ligature \OE}{}{}
\capcs offinterlineskip {inhibe le ressort inter-ligne \`a partir de maintenant}{}{}
\capcs oint {op\'erateur contour int\'egral~: $\oint$}{}{}
\capcs oldstyle {utilise des chiffres elz\'eviriens~: {\oldstyle1234567890}}{}{}
\capcs omega {lettre grecque math\'ematique $\omega$}{}{}
\capcs Omega {lettre grecque math\'ematique $\Omega$}{}{}
\capcs ominus {op\'erateur moins dans un cercle~: $\ominus$}{}{}
\capcs omit {saute un patron de colonne (ou de rang\'ee) dans un alignement}*{}
\capcs openin {pr\'epare un flot d'entr\'ee sp\'ecifi\'e pour lire un fichier}*{}
\capcs openout {pr\'epare un flot de sortie sp\'ecifi\'e pour \'ecrire dans un fichier}*{}
\capcs openup {augmente |\\baselineskip|, |\\lineskip| et
   |\\lineskiplimit| d'un montant sp\'ecifi\'e}{}{}
\capcs oplus {op\'erateur plus dans un cercle~: $\oplus$}{}{}
\capcs or {s\'epare les cas d'un |\\ifcase|}*{\@or}
\capcs oslash {op\'erateur divise dans un cercle~: $\oslash$}{}{}
\capcs otimes {op\'erateur multiplie dans un cercle~: $\otimes$}{}{}
\capcs outer {rend la d\'efinition de macro suivante ill\'egale dans des contextes dans
   lesquels des tokens sont absorb\'e \`a haute vitesse}*{}
\capcs output {liste de token que \TeX\ d\'eveloppe quand il trouve une coupure de page}*{}
\capcs outputpenalty {si la coupure de page arrive sur une p\'enalit\'e, la valeur
   de cette p\'enalit\'e~; sinon z\'ero}*{}
\capcs over {produit une fraction avec une barre d'\'epaisseur par d\'efaut}*{}
\capcs overbrace {produit une accolade recouvrant le haut d'une formule,
   comme dans $\overbrace{h+w}{}$}{}{}
\capcs overfullrule {largeur de la r\'eglure d'un ``overfull box''}*{}
\capcs overleftarrow {produit une fl\`eche vers la gauche recouvrant le haut d'une
   formule, comme dans $\overleftarrow{r+a}$}{}{}
\capcs overline {produit une ligne recouvrant le haut d'une formule,
   comme dans~$\overline{2b}$}*{}
\capcs overrightarrow {produit une fl\`eche vers la droite recouvrant le haut d'une 
   formule, comme dans $\overrightarrow{i+t}$}{}{}
\capcs overwithdelims {produit une fraction avec une barre d'\'epaisseur par d\'efaut  
   et surmont\'ee des d\'elimiteurs sp\'ecifi\'es}*{}
\capcs owns {relation appartient~: $\owns$}{}{}
\capcs P {caract\`ere paragraphe en texte~: \P}{}{}
\capcs pagedepth {\TeX\ met ceci \`a la profondeur courante de la page 
   courante}*{}
\capcs pagefilllstretch {\TeX\ met ceci au montant d'\'etirement |filll| sur
   la page courante}*{}
\capcs pagefillstretch {\TeX\ met ceci au montant d'\'etirement |fill| sur
   la page courante}*{}
\capcs pagefilstretch {\TeX\ met ceci au montant d'\'etirement |fil| sur
   la page courante}*{}
\capcs pagegoal {\TeX\ met ceci \`a la hauteur d\'esir\'ee pour la page courante
   (c'est-\`a-dire, |\\vsize| quand la premi\`ere bo\^\i te est mise sur la page)}*{}
\capcs pageinsert {produit le texte sp\'ecifi\'e sur la page suivante et utilise 
    jusqu'a la page enti\`ere}{}{}
\capcs pageno {le registre |\\count0|, qui contient le num\'ero de page
   (\'eventuellement n\'egatif)}{}{}
\capcs pageshrink {\TeX\ met ceci au montant total de r\'etr\'ecissement
   sur la page courante}*{}
\capcs pagestretch {\TeX\ met ceci au montant total d'\'etirement
   sur la page courante}*{}
\capcs pagetotal {\TeX\ met ceci \`a la hauteur naturelle de la page
   courante}*{}
\capcs par {finit un paragraphe et termine le mode horizontal}*{\@par}
\capcs parallel {relation parall\`ele~: $\parallel$}{}{}
\capcs parfillskip {ressort horizontal que \TeX\ ins\`ere \`a la fin d'un
   paragraphe}*{}
\capcs parindent {espace horizontal que \TeX\ ins\`ere au d\'ebut d'un
   paragraphe}*{}
\capcs parshape {sp\'ecifie la largeur et la longueur de chaque ligne 
   dans le prochain paragraphe}*{}
\capcs parskip {ressort vertical que \TeX\ ins\`ere avant un paragraphe}*{}
\capcs partial {symbole d\'eriv\'ee partielle~: $\partial$}{}{}
\capcs pausing {si positif, stoppe apr\`es avoir lu chaque ligne d'entr\'ee pour un
   remplacement possible}*{}
\capcstwo penalty {produit une p\'enalit\'e (ou un bonus, si n\'egatif) pour couper
   une ligne ou une page ici}*{hpenalty:vpenalty}
\capcs perp {relation perpendiculaire~: $\perp$}{}{}
\capcs phantom {produit une formule invisible avec les 
   dimensions d'une sous formule sp\'ecifi\'e}{}{}
\capcs phi {lettre grecque math\'ematique $\phi$}{}{}
\capcs Phi {lettre grecque math\'ematique $\Phi$}{}{}
\capcs pi {lettre grecque math\'ematique $\pi$}{}{}
\capcs Pi {lettre grecque math\'ematique $\Pi$}{}{}
\capcs plainoutput {routine |\\output| de \plainTeX}{}{}
\capcs pm {op\'erateur plus et moins~: $\pm$}{}{}
\capcs pmatrix {produit une matrice entre parenth\`eses}{}{}
\capcs pmod {notation modulo entre parenth\`eses \'a la fin d'une formule comme dans $x \equiv y+1 \pmod 2$}{}{}
\capcs postdisplaypenalty {p\'enalit\'e suppl\'ementaire pour une coupure de ligne
   juste apr\`es un affichage, par d\'efaut~0}*{}
\capcs Pr {fonction probabilit\'e~: $\Pr$}{}{}
\capcs prec {relation pr\'ec\`ede~: $\prec$}{}{}
\capcs preceq {relation pr\'ec\`ede ou \'egale~: $\preceq$}{}{}
\capcs predisplaypenalty {p\'enalit\'e suppl\'ementaire pour une coupure de ligne juste 
   avant un affichage, par d\'efaut~0}*{}
\capcs predisplaysize {\TeX\ met ceci \`a la largeur de la
   ligne pr\'ecedant un affichage}*{}
\capcs pretolerance {tol\'erance de m\'ediocrit\'e pour une coupure de ligne sans
   c\'esure, par d\'efaut~100}*{}
\capcs prevdepth {profondeur de la derni\`ere bo\^\i te dans la liste verticale courante
   (sauf filet)}*{}
\capcs prevgraf {\TeX\ met ceci au nombre de lignes dans le paragraphe en plus
   (en mode horizontal) ou dans le paragraphe pr\'ec\'edent (en mode vertical)}*{}
\capcs prime {symbole prime math\'ematique, comme dans $r^\prime$}{}{}
\capcs proclaim {d\'ebute un th\'eor\`eme, un lemme, une hypoth\`ese, $\ldots$}{}{\@proclaim}
\capcs prod {grand op\'erateur produit~: $\prod$}{}{}
\capcs propto {relation proportionnel \`a~: $\propto$}{}{}
\capcs psi {lettre grecque math\'ematique $\psi$}{}{}
\capcs Psi {lettre grecque math\'ematique $\Psi$}{}{}
\capcs qquad {produit un ressort horizontal de largeur 2\em}{}{}
\capcs quad {produit un ressort horizontal de largeur 1\em}{}{}
\capcs radical {produit un symbole radical sp\'ecifi\'e}*{}
\capcs raggedbottom {autorise la marge du bas \`a varier d'une page \`a l'autre}{}{}
\capcs raggedright {autorise la marge droite \`a varier d'une ligne \`a l'autre}{}{}
\capcs raise {abaisse une bo\^\i te sp\'ecifi\'ee d'un montant sp\'ecifi\'e}*{}
\capcs rangle {d\'elimiteur angle fermant~: $\rangle$}{}{}
\capcs rbrace {d\'elimiteur accolade fermante~: $\rbrace$}{}{}
\capcs rbrack {d\'elimiteur crochet fermant~: $\rbrack$}{}{}
\capcs rceil {d\'elimiteur plancher fermant~: $\rceil$}{}{}
\capcs Re {symbole partie r\'eelle de complexe~: $\Re$}{}{}
\capcs read {lit une ligne d'un flot d'entr\'ee sp\'ecifi\'e}*{}
\capcs relax {ne fait rien}*{}
\capcs relpenalty {p\'enalit\'e suppl\'ementaire pour une coupure apr\`es une relation,
   par d\'efaut~500}*{}
\capcs repeat {termine une boucle d\'ebut\'ee avec |\\loop|}{}{\@repeat}
\capcs rfloor {d\'elimiteur plafond fermant~: $\rfloor$}{}{}
\capcs rgroup {d\'elimiteur de groupe fermant (la plus petite taille est montr\'ee ici)~:
   $\Big\rgroup$}{}{}
\capcs rho {lettre grecque math\'ematique $\rho$}{}{}
\capcs right {produit le d\'elimiteur sp\'ecifi\'e \`a l'extr\'emit\'e d'une
   sous-formule d\'ebut\'ee par |\\left|}*{}
\capcs rightarrow {relation~: $\rightarrow$}{}{}
\capcs Rightarrow {relation~: $\Rightarrow$}{}{}
\capcs rightarrowfill {remplit une hbox avec un |\\rightarrow|~:
   \hbox to 3.5em{\rightarrowfill}}{}{}
\capcs rightharpoondown {relation~: $\rightharpoondown$}{}{}
\capcs rightharpoonup {relation~: $\rightharpoonup$}{}{}
\capcs rightleftharpoons {relation~: $\rightleftharpoons$}{}{}
\capcs rightline {produit une ligne avec son texte pouss\'e vers la marge droite}{}{}
\capcs rightskip {ressort que \TeX\ ins\`ere \`a droite de chaque ligne}*{}
\capcs righthyphenmin {taille du plus petit fragment de mot que \TeX\ autorise
   apr\`es une c\'esure \`a la fin d'un mot, par d\'efaut~3}*{}
\capcs rlap {produit du texte (sans largeur) \`a partir de la droite
   de la position courante}{}{}
\capcs rm {utilise des caract\`eres romains, c'est-\`a-dire, fait |\\tenrm\\fam=0|}{}{}
\capcs rmoustache {moiti\'e basse d'une grande accolade~: $\big\rmoustache$}{}{}
\capcs romannumeral {produit la repr\'esentation en chiffre romain minuscule d'un
   nombre sp\'ecifi\'e, en caract\`eres}{}{}
\capcs root {produit une racine sp\'ecifi\'ee d'une sous-formule sp\'ecifi\'ee, comme dans
   $\root 3 \of 2$}{}{}
\capcs rq {caract\`ere quote fermante en texte~: \rq}{}{}
\capcs S {caract\`ere section en texte~: \S}{}{}
\capcs sb {caract\`ere indice implicite}{}{}
\capcs scriptfont {la police de style script dans une famille math\'ematique sp\'ecifi\'ee}*{}
\capcs scriptscriptfont {la police de style scriptscript dans une famille math\'ematique 
   sp\'ecifi\'ee}*{}
\capcs scriptscriptstyle {utilise la taille de style scriptscript dans une formule}*{}
\capcs scriptspace {espace suppl\'ementaire que \TeX\ cr\`ee apr\`es un exposant ou un
   indice, par d\'efaut 0.5\pt}*{}
\capcs scriptstyle {utilise la taille de style script dans une formule}*{}
\capcs scrollmode {ne stoppe pas sur les erreurs, mais stoppe sur les erreurs
   de fichiers manquants}*{}
\capcs searrow {relation fl\`eche en bas \`a gauche~: $\searrow$}{}{}
\capcs sec {fonction s\'ecante~: $\sec$}{}{}
\capcs setbox {d\'efinit un registre de bo\^\i te sp\'ecifi\'e comme \'etant une bo\^\i te}*{}
\capcs setlanguage {change un jeu de r\`egles de c\'esure sp\'ecifi\'e, mais
   ne change pas |\\language|}{*}{}
\capcs setminus {operateur difference ~: $\setminus$}{}{}
\capcs settabs {d\'efinit les tabulations pour un alignement tabul\'e}{}{}
\capcs sevenbf {utilise une police grasse de $7$ points, |cmbx7|}{}{}
\capcs seveni {utilise une police math\'ematique italique de $7$ points, |cmmi5|}{}{}
\capcs sevenrm {utilise une police romaine de $7$ points, |cmr7|}{}{}
\capcs sevensy {utilise une police  symbole de $7$ points, |cmsy7|}{}{}
\capcs sfcode {le code du facteur d'espace  d'un caract\`ere sp\'ecifi\'e}*{}
\capcs sharp {symbole di\`ese en musique~: $\sharp$}{}{}
\capcs shipout {envoit une bo\^\i te vers le fichier |.dvi|}*{}
\capcs show {montre, dans la log et 
   sur le terminal, la signification d'un token sp\'ecifi\'e}*{}
\capcs showbox {affiche le contenu d'un registre de bo\^\i te sp\'ecifi\'e}*{}
\capcs showboxbreadth {nombre maximum d'\'el\'ements montr\'e sur chaque niveau imbriqu\'e,
   par d\'efaut~5}*{}
\capcs showboxdepth {maximum de niveaux imbriqu\'es montr\'es, par d\'efaut~3}*{}
\capcs showhyphens {montre, dans la log
   et sur le terminal, les c\'esures dans un texte sp\'ecifi\'e}{}{}
\capcs showlists {affiche toutes les listes sur lequel il travaille}*{}
\capcs showthe {montre, dans la log 
   et sur le terminal, ce que |\\the| produira}*{}
\capcs sigma {lettre grecque math\'ematique $\sigma$}{}{}
\capcs Sigma {lettre grecque math\'ematique $\Sigma$}{}{}
\capcs sim {relation similarit\'e~: $\sim$}{}{}
\capcs simeq {relation similaire ou \'egal~: $\simeq$}{}{}
\capcs sin {fonction sinus~: $\sin$}{}{}
\capcs sinh {fonction sinus hyperbolique~: $\sinh$}{}{}
\capcs skew {d\'eplace un accent sp\'ecifi\'e d'un montant sp\'ecifi\'e
   sur un caract\`ere accentu\'e sp\'ecifi\'e}{}{}
\capcs skewchar {caract\`ere d'une police sp\'ecifi\'ee utilis\'e pour positionner des accents}*{}
\capcs skip {le registre de ressort sp\'ecifi\'e}*{}
\capcs skipdef {d\'efinit une s\'equence de contr\^ole sp\'ecifi\'ee  comme \'etant un nombre
   correspondant \`a un registre |\\skip|}*{}
\capcs sl {utilise des caract\`eres pench\'es, c'est-\`a-dire, fait |\\tensl\\fam=\\slfam|}{}{}
\capcs slash {caract\`ere \slash\ autorisant une coupure de ligne}{}{}
\capcs slfam {famille pench\'ee en math\'ematique}{}{}
\capcs smallbreak {indique une coupure de page assez d\'esirable 
   avec |\\penalty-50| et produit un ressort |\\smallskipamount|}{}{}
\capcs smallint {petit symbole d'int\'egrale~: $\smallint$}{}{}
\capcs smallskip {produit un ressort |\\smallskipamount|}{}{}
\capcs smallskipamount {ressort pour un petit saut vertical, par d\'efaut 3\pt\
   plus 1\pt\ minus 1\pt}{}{}
\capcs smash {produit une formule sans hauteur, ni profondeur}{}{}
\capcs smile {relation smile~: $\smile$}{}{}
\capcs sp {caract\`ere d'exposant implicite}{}{}
\capcs space {produit un ressort inter-mot normal}{}{}
\capcs spacefactor {modifie l'\'etirement et le r\'etr\'ecissement des ressorts inter-mot 
   s'il diff\'erent de 1000}*{}
\capcs spaceskip {si diff\'erent de z\'ero et |\\spacefactor|${}<2000$, surcharge
    le ressort inter-mot normal}*{}
\capcs spadesuit {symbole pique~: $\spadesuit$}{}{}
\capcs span {soit combine des entr\'ees dans un corps d'alignement body soit 
   d\'eveloppe des tokens dans un pr\'eambule}*{}
\capcs special {\'ecrit des tokens dans le fichier |.dvi| devant \^etre interpr\'et\'es par un
   programme de lecture de DVI}*{}
\capcs splitbotmark {dernier \'el\'ement de marque dans une bo\^\i te r\'esultant de |\\vsplit|}*{}
\capcs splitfirstmark {premier \'el\'ement de marque dans une bo\^\i te r\'esultant de 
   |\\vsplit|}*{}
\capcs splitmaxdepth {profondeur maximum d'une bo\^\i te r\'esultant de |\\vsplit|}*{}
\capcs splittopskip {ressort que \TeX\ ins\`ere en haut d'une bo\^\i te r\'esultant de
   |\\vsplit|}*{}
\capcs sqcap {op\'erateur square cap~: $\sqcap$}{}{}
\capcs sqcup {op\'erateur square cup~: $\sqcup$}{}{}
\capcs sqrt {produit une racine carr\'ee d'une sous-formule, comme dans $\sqrt 2$}{}{}
\capcs sqsubseteq {relation square subset ou \'egal~: $\sqsubseteq$}{}{}
\capcs sqsupseteq {relation square superset ou \'egal~: $\sqsupseteq$}{}{}
\capcs ss {lettre allemande~: \ss}{}{}
\capcs star {op\'erateur \'etoile~: $\star$}{}{}
\capcs string {produit un token sp\'ecifi\'e, le plus souvent une s\'equence de 
   contr\^ole, comme des caract\`eres}*{}
\capcs strut {bo\^\i te sans largeur, mais de hauteur et profondeur d'une ligne standard,
   de ligne de base \`a ligne de base, dans la police courante}{}{}
\capcs subset {relation subset~: $\subset$}{}{}
\capcs subseteq {relation subset ou \'egal~: $\subseteq$}{}{}
\capcs succ {relation successeur~: $\succ$}{}{}
\capcs succeq {relation successeur ou \'egal~: $\succeq$}{}{}
\capcs sum {grand op\'erateur de somme~: $\sum$}{}{}
\capcs sup {fonction sup\'erieur~: $\sup$}{}{}
\capcs supereject {force une coupure de page, et d\'echarge toutes les insertions}{}{}
\capcs supset {relation superset~: $\supset$}{}{}
\capcs supseteq {relation superset ou \'egal~: $\supseteq$}{}{}
\capcs surd {symbole surd~: $\surd$}{}{}
\capcs swarrow {relation fl\`eche en bas \`a gauche~: $\swarrow$}{}{}
\capcs t {accent tie-after en texte, comme dans \t uu}{}{}
\capcs tabalign {\'equivalent \`a |\\+|, sauf s'il n'est pas |\\outer|}{}{}
\capcs tabskip {ressort entre colonnes (ou rang\'ees) d'un alignement}*{}
\capcs tan {fonction tangente~: $\tan$}{}{}
\capcs tanh {fonction tangente hyperbolique~: $\tanh$}{}{}
\capcs tau {lettre grecque math\'ematique $\tau$}{}{}
\capcs tenbf {utilise une police grasse de $10$ points, |cmbx10|}{}{}
\capcs tenex {utilise une police d'extension math\'ematique de $10$ points, |cmex10|}{}{}
\capcs teni {utilise une police italique math\'ematique de $10$ points, |cmmi10|}{}{}
\capcs tenit {utilise une police italique de texte de $10$ points, |cmti10|}{}{}
\capcs tenrm {utilise une police romaine de texte de $10$ points, |cmr10|}{}{}
\capcs tensl {utilise une police romaine pench\'ee de $10$ points, |cmsl10|}{}{}
\capcs tensy {utilise une police de symbole math\'ematique de $10$ points, |cmsy10|}{}{}
\capcs tentt {utilise une police de machine \`a \'ecrire de $10$ points, |cmtt10|}{}{}
\capcs TeX {produit le logo \TeX}{}{}
\capcs textfont {la police de style dans une famille math\'ematique sp\'ecifi\'ee}*{}
\capcs textindent {comme |\\item|, mais ne fait pas d'indentation associ\'ee}{}{}
\capcs textstyle {utilise une taille textstyle dans une formule}*{}
\capcs the {donne la valeur d'un token sp\'ecifi\'e}*{}
\capcs theta {lettre grecque math\'ematique $\theta$}{}{}
\capcs Theta {lettre grecque math\'ematique $\Theta$}{}{}
\capcs thickmuskip {ressort pour une espace math\'ematique gras, par d\'efaut 5\mud\ plus 5\mud}*{}
\capcs thinmuskip {ressort pour un espace math\'ematique fine, par d\'efaut 3\mud}*{}
\capcs thinspace {cr\'enage de \frac1/6\em}{}{}
\capcs tilde {accent tilde en math\'ematique, comme dans $\tilde x$}{}{}
\capcs time {l'heure d'aujourd'hui, en minutes depuis minuit}*{}
\capcs times {op\'erateur multipli\'e~: $\times$}{}{}
\capcs toks {le registre de token sp\'ecifi\'e}*{}
\capcs toksdef {d\'efinit une s\'equence de contr\^ole sp\'ecifi\'e comme \'etant un nombre
   correspondant \`a un registre |\\toks|}*{}
\capcs tolerance {tol\'erance de m\'ediocrit\'e pour des coupures de ligne avec c\'esure}*{}
\margin{{\tt\\topglue} command added~; recent addition to \TeX}
\capcs to {relation mapping~: $\to$}{}{}
\capcs top {symbole lattice top~: $\top$}{}{}
\capcs topglue {produit un ressort vertical sp\'ecifi\'e en haut 
   d'une page}{}{}
\capcs topinsert {produit le texte sp\'ecifi\'e en haut d'une page}{}{}
\capcs topmark {|\\botmark| avant que la page courante soit en bo\^\i te}*{}
\capcs topskip {ressort entre la ligne de t\^ete et la premi\`ere ligne de texte
   sur une page, par d\'efaut 10\pt}*{}
\capcs tracingall {d\'eclanche des traces maximal}{}{}
\capcs tracingcommands {affiche l'ex\'ecution des commandes}*{}
\capcs tracinglostchars {affiche les caract\`eres demand\'es, mais non
   d\'efinis}*{}
\capcs tracingmacros {affiche les d\'eveloppements de macros}*{}
\capcs tracingonline {montre les sorties de diagnostic sur le terminal aussi bien que
   dans le fichier log}*{}
\capcs tracingoutput {affiche les contenus des bo\^\i tes sorties}*{}
\capcs tracingpages {affiche les calculs de coupure de page}*{}
\capcs tracingparagraphs {affiche les calculs de coupure de ligne}*{}
\capcs tracingrestores {affiche les valeurs restaur\'ees \`a la fin
   d'un groupe}*{}
\capcs tracingstats {affiche les statistiques d'utilisation de la m\'emoire}*{}
\capcs triangle {symbole triangle~: $\triangle$}{}{}
\capcs triangleleft {op\'erateur triangle gauche~: $\triangleleft$}{}{}
\capcs triangleright {op\'erateur triangle droite~: $\triangleright$}{}{}
\capcs tt {utilise des caract\`eres de machine \`a \'ecrire, c'est-\`a-dire, fait |\\tentt\\fam=\\ttfam|}{}{}
\capcs ttfam {famille machine \`a \'ecrire en math\'ematique}{}{}
\capcs ttraggedright {utilise des caract\`eres de machine \`a \'ecrire et autorise la marge
   droite des paragraphes \`a varier d'une ligne \`a l'autre}{}{}
\capcs u {accent bref en texte, comme dans \u r}{}{}
\capcs uccode {le code de caract\`ere de la forme capitale d'une lettre}*{}
\capcs uchyph {si positif, laisse c\'esurer des mots qui commencent avec une
   lettre capitale}*{}
\capcs underbar {souligne le texte sp\'ecifi\'e sans toucher les descendantes , comme dans \underbar{fog}}{}{}
\capcs underbrace {produit une accolade couvrant le bas d'une formule, comme dans
   $\underbrace{x+x}{}$}{}{}
\capcs underline {souligne une formule math\'ematique sous les descendantes, comme dans
   $\underline{x+y}$}*{}
\capcs unhbox {d\'echarge le contenu de la bo\^\i te
   d'un registre de bo\^\i te sp\'ecifi\'e
   de la liste courante et vide le registre~; l\'egal en
   mode horizontal seulement}*{}
\capcs unhcopy {comme |\\unhbox|, mais ne vide pas le registre}*{}
\capcs unkern {si le dernier \'el\'ement de la liste courante est un cr\'enage, l'enl\`eve}*{}
\capcs unpenalty {si le dernier \'el\'ement de la liste courante est une p\'enalit\'e, l'enl\`eve}*{}
\capcs unskip {si le dernier \'el\'ement de la liste courante est un ressort, l'enl\`eve}*{}
\capcs unvbox {d\'echarge le contenu de la bo\^\i te
   d'un registre de bo\^\i te sp\'ecifi\'e
   de la liste courante et vide le registre~; l\'egal en
   modes verticaux seulement}*{}
\capcs unvcopy {comme |\\unvbox|, mais ne vide pas le registre}*{}
\capcs uparrow {relation~: $\uparrow$}{}{}
\capcs Uparrow {relation~: $\Uparrow$}{}{}
\capcs upbracefill {fill enclosing hbox with an upwards facing brace~:
   \hbox to 3.5em{\upbracefill}}{}{}
\capcs updownarrow {relation~: $\updownarrow$}{}{}
\capcs Updownarrow {relation~: $\Updownarrow$}{}{}
\capcs uplus {op\'erateur plus cupped~: $\uplus$}{}{}
\capcs uppercase {convertit les lettres minuscules d'un texte sp\'ecifi\'e
   en capitales}*{}
\capcs upsilon {lettre grecque math\'ematique $\upsilon$}{}{}
\capcs Upsilon {lettre grecque math\'ematique $\Upsilon$}{}{}
\capcs v {accent tch\`eque en texte, comme dans \v o}{}{}
\capcs vadjust {produit du mat\'eriel en mode vertical apr\`es la ligne courante}*{}
\capcs valign {aligne du texte dans des rang\'ees}*{}
\capcs varepsilon {variante de la lettre grecque math\'ematique $\varepsilon$}{}{}
\capcs varphi {variante de la lettre grecque math\'ematique $\varphi$}{}{}
\capcs varpi {variante de la lettre grecque math\'ematique $\varpi$}{}{}
\capcs varrho {variante de la lettre grecque math\'ematique $\varrho$}{}{}
\capcs varsigma {variante de la lettre grecque $\varsigma$}{}{}
\capcs vartheta {variante de la lettre grecque math\'ematique $\vartheta$}{}{}
\capcs vbadness {badness threshold for reporting underfull or overfull
   vboxes, par d\'efaut~1000}*{}
\capcs vbox {produit une vbox dont la ligne de base est celle de la bo\^\i te du haut
   incluse}*{}
\capcs vcenter {centre le texte sp\'ecifi\'e sur l'axe math\'ematique}*{}
\capcs vdash {symbole left turnstile~: $\vdash$}{}{}
\capcs vdots {points verticaux en math\'ematique~: \smash{$\vdots$}}{}{}
\capcs vec {accent vecteur en math\'ematique, comme dans $\vec x$}{}{}
\capcs vee {op\'erateur ``ou'' logique~: $\vee$}{}{}
\capcs vert {relation barre~: $\vert$}{}{}
\capcs Vert {relation double barre~: $\Vert$}{}{}
\capcs vfil {produit un ressort vertical infiniment \'etirable}*{}
\capcs vfill {produit un ressort vertical encore plus infiniment \'etirable
   que celui produit par |\\vfil|}*{}
\capcs vfilneg {produit un ressort vertical n\'egatif infiniment \'etirable}*{}
\capcs vfootnote {produit une note de bas de page sp\'ecifi\'ee avec une marque 
   de r\'ef\'erence sp\'ecifi\'ee, mais ne produit la marque de r\'ef\'erence dans le texte}{}{}
\capcs vfuzz {seuil de d\'epassement pour dire overfull vboxes, par d\'efaut 
   0.1\pt}*{}
\capcs vglue {produit un ressort vertical sp\'ecifi\'e
   qui ne disparait pas sur des coupures de page}{}{}
\capcs voffset {offset vertical relatif d'un pouce par rapport au coin
   haut du papier}*{}
\capcs vphantom {produit un formule invisible sans largeur mais de hauteur 
   et profondeur naturelle}{}{}
\capcs vrule {produit un filet vertical~; l\'egal seulement en modes horizontaux}*{}
\capcs vsize {hauteur de page, par d\'efaut 8.9\thinspace in}*{}
\capcs vskip {produit un ressort vertical sp\'ecifi\'e}*{}
\capcs vsplit {coupe le contenu d'un registre de  bo\^\i te sp\'ecifi\'ee
   de la hauteur sp\'ecifi\'e}*{}
\capcs vss {produit un ressort vertical qui est infiniment \'etirable et
   infiniment r\'etr\'ecissable}*{}
\capcs vtop {produit une vbox dont la ligne de base est celle de la bo\^\i te du haut englob\'ee}*{}
\capcs wd {la largeur d'une bo\^\i te dans un registre de bo\^\i te sp\'ecifi\'e}*{}
\capcs wedge {op\'erateur logique ``et''~: $\wedge$}{}{}
\capcs widehat {accent math\'ematique, comme dans $\widehat {y+z+a}$}{}{}
\capcs widetilde {accent math\'ematique $\widetilde {b+c+d}$}{}{}
\capcs widowpenalty {p\'enalit\'e pour une ligne seule d\'ebutant une page, 
   par d\'efaut~150}*{}
\capcs wlog {|\\write| la liste de token sp\'ecifi\'ee dans le fichier log}{}{}
\capcs wp {symbole Weierstra\ss\ `p'~: $\wp$}{}{}
\capcs wr {op\'erateur produisant un ruban~: $\wr$}{}{}
\capcs write {\'ecrit une ligne vers un flot de sortie sp\'ecifi\'e}*{}
\capcs xdef {\'equivalent \`a |\\global\\edef|, c'est-\`a-dire, d\'efini globalement une
   macro, d\'eveloppant imm\'ediatement le texte de remplacement}*{}
\capcs xi {lettre grecque math\'ematique $\xi$}{}{}
\capcs Xi {lettre grecque math\'ematique $\Xi$}{}{}
\capcs xleaders {produit des r\'eglures avec l'espace en trop \`a gauche redistribu\'e egalement}*{}
\capcs xspaceskip {si diff\'erent de z\'ero et |\\spacefactor|${}\ge2000$, 
   surcharge le ressort inter-mot normal}*{}
\capcs year {l'ann\'ee courante, comme un nombre}*{}
\capcs zeta {lettre grecque math\'ematique $\zeta$}{}{}

\endcapsum
\endchapter
\byebye

