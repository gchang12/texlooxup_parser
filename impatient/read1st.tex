% This is part of the book TeX for the Impatient.
% Copyright (C) 2003 Paul W. Abrahams, Kathryn A. Hargreaves, Karl Berry.
% See file fdl.tex for copying conditions.

\input macros
\frontchapter{Read this first}

% We don't need anything but \rm here.
{\font\rm = cmr10 scaled \magstephalf \baselineskip = 1.1\baselineskip
If you're new to \TeX:
\ulist
\li Read Sections \chapternum{usebook}--\chapternum{usingtex} first.
\li Look at the examples in \chapterref{examples}
for things that resemble what you want to do.
Look up any related commands in ``Capsule summary of commands'',
\chapterref{capsule}.
Use the page references there to find the more complete descriptions of
those commands and others that are similar.
\li Look up unfamiliar words in ``Concepts'', \chapterref{concepts},
using the list on the back cover of the book to find the
explanation quickly.
\li Experiment and explore.
\endulist
\bigskip
\noindent
If you're already familiar with \TeX, or if you're editing or
otherwise modifying a \TeX\ document that someone else has created:
\ulist
\li For a quick reminder of what a command does, look in
\chapterref{capsule}, ``Capsule summary of commands''.  It's
alphabetized and has page references for more complete descriptions of
the commands.
\li Use the functional groupings of command descriptions
to find those related to a
particular command that you already know, or to find a command that serves a
particular purpose.
\li Use \chapterref{concepts}, ``Concepts'', to get an explanation of
any concept that you don't understand, or need to understand more
precisely, or have forgotten.
Use the list on the inside back cover of the book to find a concept quickly.
\endulist
}
\pagebreak

\byebye
