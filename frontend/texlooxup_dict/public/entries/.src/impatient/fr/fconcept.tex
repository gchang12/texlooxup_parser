% This is part of the book TeX for the Impatient.
% Copyright (C) 2003 Paul W. Abrahams, Kathryn A. Hargreaves, Karl Berry.
% Copyright (C) 2004 Marc Chaudemanche pour la traduction fran�aise.
% See file fdl.tex for copying conditions.

\input fmacros
\chapter{Concepts}

\chapterdef{concepts}

Cette partie du livre contient les d\'efinitions et explications des 
concepts que nous utilisons pour d\'ecrire \TeX. Les concepts comprennent 
les termes techniques que nous utilisons pour expliquer les commandes et 
des points importants qui ne sont vus nulle part ailleurs dans le livre

Les concepts sont class\'es dans l'ordre alphab\'etique. La page de garde int\'erieure 
contient une liste compl\`ete des concepts et les pages o\`u ils sont expliqu\'es. 
Nous vous sugg\'erons de faire une copie de cette page int\'erieure et de la 
garder de cot\'e pour que vous puissiez identifier et regarder un concept non 
familier imm\'ediatement. Autant que possible, nous gardons notre 
terminologie coh\'erente avec celle de \texbook.\idxref{\texbook}

\beginconcepts

\conceptindex{alignements}
\concept alignement

\bix^^{tables}
Un \defterm{alignement} est une construction pour aligner du mat\'e\-riel tel
qu'un tableau, en colonne ou en rang\'ee. Pour former un alignement vous devez
(a)~d\'ecrire le format des colonnes ou des rang\'ees et (b)~ dire \`a \TeX\ quel
mat\'eriel va dans les colonnes ou les rang\'ees. Un alignement tabul\'e ou un
alignement horizontal est organis\'e comme une suite de rang\'ees~; un alignement
vertical est organis\'e comme une suite de colonnes. Nous d\'ecrirons d'abord
les alignements tabul\'es et horizontaux et ensuite, plus bri\`evement, les
alignements verticaux.

Les alignements tabul\'es sont d\'efinis par \plainTeX. Ils sont plus simples
mais moins flexibles que les alignements horizontaux. Ils diff\`erent 
principalement sur la fa\c con de d\'ecrire leurs formats.

\bix^^|\settabs|
\bix\ctsidxref{+}
\bix\ctsidxref{cr}

Pour construire un alignement tabul\'e vous devez tout d'abord former une 
commande |\settabs| \ctsref{\settabs} qui sp\'ecifie comment \TeX\ doit
diviser l'es\-pace horizontal disponible en colonne, ensuite vous procurer
une suite de rang\'ees \`a tabuler. Chaque rang\'ee consiste en une s\'equence de 
contr\^ole |\+| \ctsref{\@plus} suivi par une liste d'``entr\'ees'', c'est-\`a-dire
des intersections rang\'ees\slash colonnes.
^^{entr\'ee (colonne ou rang\'ee)}
Des entr\'ees qui se suivent sur une rang\'ee sont s\'epar\'ees par une esperluette
(|&|).
\xrdef{@and}
\ttidxref{&}
La fin d'une rang\'ee est indiqu\'ee par ^|\cr| apr\`es sa derni\`ere entr\'ee. 
Si une rang\'ee a moins d'entr\'ees qu'il y a de colonnes dans l'alignement,
\TeX\ rempli jusqu'\`a la fin la rang\'ee de blanc.

Tant qu'il est pr\'ec\'ed\'e d'une commande |\settabs|, vous pouvez mettre une 
rang\'ee d'un alignement tabul\'e n'importe o\`u dans votre document. En particulier,
vous pouvez mettre autres choses entre les rang\'ees d'aligne\-ments tabul\'es ou
d\'ecrire plusieurs alignements tabul\'es avec un seul |\set!-tabs|. Voici un exemple
d'alignement tabul\'e~:

\xrdef{tabbedexample}\csdisplay
{\hsize = 1.7 in \settabs 2 \columns
\+cattle&herd\cr
\+fish&school\cr
\+lions&pride\cr}
|
La commande |\settabs 2 \columns| de cet exemple \ctsref{\settabs}
demande \`a \TeX\ de produire deux colonnes de largeurs \'egales.
La longueur de la ligne est de $1.7$ pouces.
L'alignement compos\'e ressemble \`a ceci~:

{\def\+{\tabalign}% so it isn't \outer.
\vdisplay{%
\hsize 1.7 in \settabs 2 \columns
\+cattle&herd\cr
\+fish&school\cr
\+lions&pride\cr}
}%

\margin{Missing explanation added here.}
Il y a une autre forme d'alignement tabul\'e dans lequel vous sp\'ecifiez les
largeurs de colonne avec un exemple. La largeur de colonne de l'exemple
d\'etermine les largeurs de colonne du reste de l'alignement~:
\csdisplay
{\settabs\+cattle\quad&school\cr
\+cattle&herd\cr
\+fish&school\cr
\+lions&pride\cr}
|
Voici le r\'esultat~:
{\def\+{\tabalign}% so it isn't \outer.
\vdisplay{%
\settabs\+cattle\quad&school\cr
\+cattle&herd\cr
\+fish&school\cr
\+lions&pride\cr}
}%

\eix^^|\settabs|
\eix\ctsidxref{+}
\bix^^|\halign|
Les alignements horizontaux sont construits avec |\halign| \ctsref\halign.
\TeX\ ajuste la largeur des colonnes d'un alignement horizontal en fonction
du contenu des colonnes. Quand \TeX\ rencontre la commande |\halign| qui
d\'ebute un alignement horizontal, il examine d'abord toutes les rang\'ees de
l'alignement pour voir quelle est la largeur des entr\'ees. Il fixe alors
la largeur des colonnes pour accommoder les entr\'ees les plus larges dans ces
colonnes.

Un alignement horizontal g\'er\'e par |\halign| est constitu\'e d'un 
``\pix^{pr\'eam\-bule}'' qui indique le sch\'ema des rang\'ees suivi par les
rang\'ees elles-m\^emes.
\ulist
\li Le pr\'eambule est constitu\'e d'une suite de \pix^{patron}s, un pour 
chaque colonne. Le patron d'une colonne sp\'ecifie comment le texte de cette 
colonne doit \^etre compos\'e. Chaque patron doit inclure un seul carac\-t\`ere |#| 
\ttidxref{#}\xrdef{@asharp}
pour indiquer o\`u \TeX\ doit substituer le texte d'une entr\'ee dans le patron.
Les patrons sont s\'epar\'es par une esperluette (|&|) \ttidxref{&} et la fin
du pr\'eambule est indiqu\'ee par |\cr|. En procurant un patron appro\-pri\'e, vous 
pouvez obtenir des effets comme centrer une colonne, justifier une colonne 
\`a gauche ou \`a droite ou composer une colonne dans une \refterm{police}
particuli\`ere.

\li Les rang\'ees ont la m\^eme forme que dans l'alignement  tabul\'e,  sauf que
vous ne mettez  pas de |\+| au d\'ebut de chaque rang\'ee.
Comme avant, des entr\'ees sont s\'epar\'ees par |&| et la fin d'une rang\'ee est
indiqu\'ee par |\cr|.
\TeX\ traite chaque entr\'ee comme un \refterm{groupe}, donc, chaque commande
de changement de police ou autre \refterm{assignement} dans le patron de 
colonne ne prend effet que pour les entr\'ees de cette colonne.
\endulist
\noindent Le pr\'eambule et les  rang\'ees doivent \^etre inclus entre les 
accolades qui suivent |\halign|. Chaque alignement |\halign| doit inclure 
son propre pr\'eambule.

Par exemple, l'alignement horizontal~:
\csdisplay
\tabskip=2pc
\halign{\hfil#\hfil &\hfil#\hfil &\hfil#\hfil \cr
  &&\it Table\cr
\noalign{\kern -2pt}
  \it Creature&\it Victual&\it Position\cr
\noalign{\kern 2pt}
  Alice&crumpet&left\cr
  Dormouse&muffin&middle\cr
  Hatter&tea&right\cr}
|

\noindent produit le r\'esultat~:

\xrdef{halignexample}
\vdisplay{%
\tabskip=2pc \halign{\hfil#\hfil &\hfil#\hfil &\hfil#\hfil \cr
  &&\it Table\cr
\noalign{\kern -2pt}
  \it Creature&\it Victual&\it Position\cr
\noalign{\kern 2pt}
  Alice&crumpet&left\cr
  Dormouse&muffin&middle\cr
  Hatter&tea&right\cr}
}%
\noindent Le ^|\tabskip| \ctsref{\tabskip} de cet exemple 
demande \`a \TeX\ d'ins\'erer |2pc| de 
\refterm{ressort} entre les colonnes.
La commande ^|\noalign| \ctsref{\noalign} d'ins\'erer
du mat\'eriel de \refterm{mode vertical} entre deux rang\'ees.
Dans cet exemple nous avons
utilis\'e |\noalign| pour produire de l'espace suppl\'ementaire entre les 
rang\'ees de titre et les rang\'ees de donn\'ees, et aussi rapprocher ``Table'' 
et ``Position''.
(Vous pouvez aussi utiliser |\noalign| avant la premi\`ere rang\'ee ou apr\`es la
derni\`ere.)
\eix^^|\halign|

Vous pouvez aussi construire un alignement vertical avec la commande ^|\valign|
\ctsref{\valign}. Un alignement vertical est organis\'e comme une s\'erie de
colonnes plut\^ot que comme une s\'erie de rang\'ees. Un alignement vertical suit
les m\^eme r\`egles qu'un alignement horizontal sauf que le r\^ole des rang\'ees et
des colonnes est invers\'e. Par exemple, l'alignement vertical~:

\csdisplay
{\hsize=0.6in \parindent=0pt
\valign{#\strut&#\strut&#\strut\cr
  one&two&three\cr
  four&five&six\cr
  seven&eight&nine\cr
  ten&eleven\cr}}
|

\noindent produit~:
\vdisplay{%
{\hsize=0.6in \parindent=\listleftindent % Because lists and displays
                                         % are not indented just by \parindent.
\valign{#\strut&#\strut&#\strut\cr
  one&two&three\cr
  four&five&six\cr
  seven&eight&nine\cr
  ten&eleven\cr}}
}
La commande ^|\strut| \ctsref{\strut} dans le patron est n\'ecessaire pour
que les entr\'ees de chaque rang\'ee s'alignent proprement, c'est-\`a-dire,
pour avoir une \refterm{ligne de base} commune et garder une distance 
uniforme entre les lignes de base.
\eix\ctsidxref{cr}
\eix^^{tables}

\endconcept


\concept {\anatomy}

\texbook\ d\'ecrit la fa\c con dont {\TeX} compile ses sources en termes de
``t\^aches digestives'' de \TeX---ses ``^{yeux}'', ``^{bouche}'',
``^{\oesophage}'', ``^{estomac}'' et ``^{intestin}''.  Savoir comment ces
organes travaillent peut \^etre utile quand vous essayer de comprendre de subtils
aspects du raisonnement de \TeX\ quand il dig\`ere un document.

\ulist

\li En utilisant ses ``\pix^{yeux}'', \TeX\ lit les \refterm{caract\`eres:caract\`ere} du
^{fichier source} et les passe \`a sa bouche.  Puisqu'un fichier source
peut contenir des commandes ^|\input| \ctsref{\input},
\TeX\ peut en effet d\'etourner son regard d'un fichier \`a l'autre.

\li En utilisant sa ``\pix^{bouche}'', {\TeX} assemble les caract\`eres en
\refterm{tokens:token} et les passe \`a son \oesophage.
Chaque token est soit une \refterm{s\'equence de contr\^ole} ou un simple
caract\`ere.  Une s\'equence de contr\^ole commence toujours par un \refterm{caract\`ere
d'\'echappement}.  Notez que les espaces et les fins de ligne sont des caract\`eres 
de plein droit, de plus \TeX\ compresse une suite d'espaces en un seul token
d'espace.  Voir \knuth{pages~46--47}{56--58} pour les r\`egles avec lesquelles \TeX\ 
assemble les caract\`eres en tokens.
^^{tokens//assembl\'ee \`a partir de caract\`eres}

\li En utilisant son ``\pix^{\oesophage}'', {\TeX} d\'eveloppe chaque macro, conditions, et
^^{macros//d\'evelopp\'ees dans l'estomac de \TeX}
^^{tokens//pass\'ee \`a l'estomac de \TeX}
constructions similaires qu'il trouve (voir \knuth{pages~212--216}{249--253}) et passe
la suite de \refterm{tokens:token} r\'esultante 
\`a l'estomac de \TeX.  D\'evelopper un token
peu faire appara\^\i tre d'autres tokens qui \`a leur tour ont besoin d'\^etre 
d\'evelopp\'es.  {\TeX} parcourt ce d\'eveloppement de gauche \`a droite \`a moins 
que l'ordre soit modi\-fi\'e par
une commande telle que |\expandafter| \ctsref{\expandafter}.
En d'autres mots, l'\oe sophage de \TeX\ d\'eveloppe toujours le token le plus \`a
gauche non-d\'evelopp\'e qu'il n'a pas encore envoy\'e vers l'estomac de \TeX.

\li En utilisant son ``\pix^{estomac}'', {\TeX} ex\'ecute tous les groupes de token.
Chaque groupe contient une commande primitive suivie par ses 
arguments, s'il y en a.
La plupart des commandes sont de type ``compose ce caract\`ere'',
dont leur groupe n'est constitu\'e que d'un token.
Suivant les instructions donn\'ees par les commandes, l'estomac de 
\TeX\ assemble de plus en plus grandes unit\'es, d\'ebutant par des
caract\`eres et finissant par des pages,
et passe les pages \`a l'intestin de \TeX.
^^{pages//assembl\'ees dans l'estomac de \TeX}
L'estomac de \TeX\ ex\'ecute les taches de \refterm{coupure de ligne}---%
^^{coupures de ligne}
c'est-\`a-dire, couper chaque paragraphe en suite de lignes---%
et de \refterm{coupure de page}---c'est-\`a-dire, 
couper une suite continue de lignes
et autres mat\'eriels de mode vertical en pages. 

\li En utilisant ses ``\pix^{intestin}'', \TeX\ transforme les pages produites par son
estomac en une forme destin\'ee \`a \^etre lue
par d'autres program\-mes.  Il envoit alors la sortie transform\'ee dans le 
\dvifile.
^^{\dvifile//cr\'e\'e par l'intestin de \TeX}

\endulist

La plupart du temps, vous pouvez penser que les processus qui prennent place dans
les yeux, bouches, \oesophage, estomac et intestin de \TeX\ ont lieu les uns apr\`es
les autres. La v\'erit\'e en la mati\`ere est que des commandes ex\'ecut\'ees dans l'estomac 
de \TeX\ peuvent influencer les \'etapes suivantes de la digestion. Par exemple,
quand l'estomac de \TeX\ rencontre la commande |\input| \ctsref{\input}, ses
yeux commencent \`a lire un fichier diff\'erent~; quand l'estomac de \TeX\ 
rencontre une commande |\catcode| \ctsref{\catcode} sp\'ecifiant un code de
cat\'egorie pour le caract\`ere $c$, l'interpr\'etation de $c$ par la bouche de
\TeX\ s'en trouve affect\'e. Et quand l'estomac de \TeX\ rencontre une d\'efinition
de \refterm{macro}, les d\'eveloppements charg\'es dans l'\oe sophage de \TeX\ 
sont eux aussi affect\'es.

Vous pouvez comprendre comment les processus agissent les uns sur les autres 
en imaginant que chaque processus avale goul\^ument le 
rendu de son pr\'ed\'ecesseur d\`es qu'il devient disponible. Par 
exemple, une fois que l'estomac de \TeX\ a vu le dernier caract\`ere du 
nom de fichier dans une commande |\input|, \TeX\  fixe son regard 
imm\'ediatement sur le premier caract\`ere du fichier d'entr\'ee indiqu\'e.

\endconcept

%\pagebreak
\conceptindex{arguments}
\concept argument

^^{commandes//arguments de}
Un \defterm{argument} contient le texte pass\'e \`a une
\refterm{commande}.
Les arguments d'une commande compl\`etent la description de ce que 
la commande est suppos\'ee faire.
La commande peut \^etre soit une commande \refterm{primitive}, soit une 
\refterm{macro}.

Chaque commande primitive ^^{primitive//commande}
a sa propre convention sur la forme de ses
arguments.  Par exemple, la suite de \refterm{tokens:token}~:

\csdisplay
\hskip 3pc plus 1em
|
consiste en la commande `|\hskip|' et les arguments
`|3pc plus 1em|'.  Mais si vous \'ecrivez~:

\csdisplay
\count11 3pc plus 1em
|
vous obtiendrez un effet enti\`erement diff\'erent.
\TeX\ traitera `|\count11|' comme une commande avec un argument `|3|',
suivi par les tokens de texte ordinaire `|pc plus 1em|'
(parce que les registres de compteur attendent l'affectation d'un nombre)%
---probablement pas ce dont vous aviez 
intention.  L'effet de la commande, donc, sera d'assigner 
$3$ au registre du compteur $11$ (voir la discussion sur ^|\count|,
\xref\count).

Les macros, d'un autre cot\'e, suivent toutes la m\^eme convention
pour leurs arguments.
^^{macros//arguments de}
Chaque argument pass\'e \`a une macro
correspond \`a un \refterm{param\`etre}
^^{param\`etres//et arguments}
dans la d\'efinition de cette
macro. ^^{macros//param\`etres de}
Un param\`etre de macro est soit ``d\'elimit\'e'', soit ``non d\'elimite''.
La d\'efinition de la macro d\'etermine le nombre et la nature des param\`etres de la macro
et donc le nombre et la nature des arguments de la macro.

La diff\'erence entre un argument d\'elimit\'e et un argument non d\'elimit\'e tient
\`a la fa\c con qu'a \TeX\ de d\'ecider o\`u se termine l'argument.
^^{arguments d\'elimit\'es}
^^{arguments non d\'elimit\'es}
\ulist
\li Un argument d\'elimit\'e est constitu\'e des tokens \`a partir du d\'ebut de 
l'argument jusqu'\`a, mais sans l'inclure, la suite de tokens particuliers 
qui servent comme d\'elimiteur pour cet argument. Le d\'elimiteur est sp\'ecifi\'e 
dans la d\'efinition de la macro. Ainsi, vous ajoutez un argument d\'elimit\'e \`a 
une macro en \'ecrivant l'argument lui-m\^eme, suivi par le d\'elimiteur. Un 
argument d\'elimit\'e peut \^etre vide, c'est-\`a-dire, ne pas avoir de texte du 
tout. Chaque accolade d'un argument d\'elimit\'e doit \^etre appair\'e 
correctement, c'est-\`a-dire, chaque accolade gauche doit avoir une accolade 
droite correspondante et vice versa.

\li Un argument non d\'elimit\'e consiste en un token simple ou une suite de
tokens entour\'e d'accolades, comme ceci~:
`|{Here is {the} text.}|'.  Malgr\'e les apparences, les accolades ext\'erieures 
ne forment pas un \refterm{groupe}---\TeX\ ne les utilise que pour
d\'eterminer quel est l'argument. Chaque accolade interne, comme celles autour
de `|the|', doivent \^etre appair\'ees correctement. Si vous faites une erreur 
et mettez trop d'accolades fermantes, \TeX\ se plaindra d'un `unexpected right brace'
\TeX\ se plaindra aussi si vous mettez trop d'accolades ouvrantes, mais vous
obtiendrez \emph{cette} plainte longtemps apr\`es l'endroit o\`u vous pensiez
finir l'argument (voir \xref{mismatched}).
\endulist
\noindent 
Voir \conceptcit{macro} pour plus d'information sur les param\`etres et les
arguments. Vous trouverez les r\`egles pr\'ecises concernant les arguments 
d\'elimit\'es et non d\'elimit\'es dans \knuth{pages~203--204}{237--239}.
\endconcept


\concept {ASCII}

\defterm{\ascii} est l'abr\'eviation de ``American Standard Code for
Information Interchange''.  Il y a $256$ ^{caract\`eres} \ascii, chacun avec
son propre num\'ero de code, mais seuls les~$128$ premiers
ont \'et\'e standardis\'es. Vous pouvez trouver leur signification dans une ``table
de codes'' \ascii\ comme celle  de la \knuth{page~367}{425}.
Les caract\`eres $32$--$126$ sont des ``caract\`eres imprimables'',
^^{caract\`eres imprimables} comme les lettres, les chiffres et les signes de 
ponctuation. Les caract\`eres restants sont des ``^{caract\`eres de contr\^ole}''
qui sont utilis\'es sp\'ecialement (dans l'industrie informatique, pas dans \TeX)
pour contr\^oler les entr\'ees\slash sorties et les instruments de communication
de donn\'ees.
Par exemple, le code \ascii\ $84$ correspond \`a la lettre `T', tandis que 
le code \ascii\ $12$ correspond \`a la fonction ``form feed'' 
(interpr\'et\'e par la plupart des imprimantes par ``commence une nouvelle page'').  
Bien que le standard \ascii\ sp\'ecifie des significations pour les caract\`eres de
contr\^ole, beaucoup de fabricants d'\'equipements tels que modems et imprimantes
ont utilis\'e les caract\`eres de contr\^ole pour des usages autres que le standard.

La signification d'un caract\`ere dans \TeX\ est normalement li\'ee \`a sa signification
dans le standard \ascii, et les \refterm{polices:police} qui contiennent
des caract\`eres imprimables \ascii\ ont normalement ces caract\`eres aux m\^emes
positions que leurs contreparties \ascii.
Mais certaines polices, notamment celles utilis\'ees en math\'ematique, remplacent 
les caract\`eres imprimables \ascii\ par d'autres caract\`eres sans aucune relation
avec les caract\`eres \ascii.
Par exemple, La police Computer Modern math\'ematique
^^{police Computer Modern}
|cmsy10| a le symbole math\'ematique
`{$\forall$}' \`a la place du chiffre \ascii\ `8'.


\endconcept

\conceptindex{assignements}
\concept assignement

Un \defterm{assignement} est une construction qui dit \`a \TeX\ d'assigner une
valeur \`a un registre,
^^{registres//assignement \`a des}
\`a un de ces
\refterm{param\`etres:param\`etre} internes,
^^{param\`etres//assignements \`a des}
\`a une entr\'ee dans une de ces tables internes
ou \`a une \refterm{s\'equence de contr\^ole}. Quelques exemples d'assignements sont~:

\csdisplay
\tolerance = 2000
\advance\count12 by 17
\lineskip = 4pt plus 2pt
\everycr = {\hskip 3pt \relax}
\catcode\`@ = 11
\let\graf = \par
\font\myfont = cmbx12
|

Le premier assignement indique que \TeX\ doit assigner la valeur num\'e\-rique
|2000| au param\`etre num\'erique |\tolerance|, c'est-\`a-dire, met la valeur de
|\tolerance| \`a $2000$.  Les autres assignements sont similaires.  Le `|=|'
et les espaces sont optionnels, donc vous pouvez aussi \'ecrire le premier
assignement plus bri\`evement~:

\csdisplay
\tolerance2000
|

Voir \knuth{pages~276--277}{321--322} pour la syntaxe d\'etaill\'ee des assignements.
\endconcept


\conceptindex{\boites}
\concept {\boites}

Une \defterm{bo\^\i te}  est un rectangle de mati\`ere \`a composer.  Un simple 
\refterm{caract\`ere} est une bo\^\i te par lui-m\^eme, et une page enti\`ere est 
aussi une bo\^\i te. \TeX\ forme une page comme une suite de bo\^\i tes contenant 
des bo\^\i tes contenant des bo\^\i tes. La bo\^\i te sup\'erieure est la page elle-m\^eme, 
les bo\^\i tes inf\'erieures sont souvent de simples caract\`eres et de simples 
lignes sont des bo\^\i tes qui sont quelque part au milieu.

\TeX\ effectue la plupart de ses activit\'es de construction de bo\^\i te implicitement quand il
construit des paragraphes et des pages.
Vous pouvez construire des bo\^\i tes explicitement
en utilisant nombre de \refterm{commandes} de \TeX, notamment 
^|\hbox| \ctsref{\hbox},
^|\vbox| \ctsref{\vbox} et
^|\vtop| \ctsref{\vtop}.
La commande ^|\hbox| 
construit une bo\^\i te en assemblant des plus petites bo\^\i tes horizontalement de gauche \`a droite~;
il op\`ere sur une \refterm{liste horizontale} et donne
une \refterm{hbox} ^^{hbox} (bo\^\i te horizontale).
^^{listes horizontales}
Les commandes ^|\vbox| et |\vtop| 
construisent une bo\^\i te en assemblant de plus petites bo\^\i tes verticalement de haut en bas~;
^^{vbox}
elles op\`erent sur une \refterm{liste verticale}
et donnent une \refterm{vbox} ^^{vbox} (bo\^\i te verticale).
^^{listes verticales}
Ces listes horizontales et verticales peuvent ne pas inclure que des bo\^\i tes plus petites mais
plusieurs autres sortes d'entit\'es, par exemple,  un \refterm{ressort} et
des cr\'enages.
^^{cr\'enages//comme \'el\'ements de liste}

Une bo\^\i te a une \refterm{hauteur}, une \refterm{profondeur} et une \refterm{largeur},
^^{hauteur} ^^{profondeur} ^^{largeur}
comme ceci~:
\vdisplay{\offinterlineskip\sevenrm
   \halign{#&#&\kern3pt \hfil#\hfil\cr
      &\hrulefill\cr
      &\vrule
         \vtop to .7in{\vss \hbox to .9in{\hss ligne de base\hss}\vskip4pt}%
       \vrule
      &\labelledheight{.7in}{hauteur}\cr
      %
       \vbox to 0pt{
          \vss
          \hbox{point de r\'ef\'erence\hbox to 15pt{\rightarrowfill}%
          \hskip3pt}%
       \kern-4.5pt}&{\box\refpoint}\hrulefill\cr
      %
      \omit
      &\vrule\hfil\vrule
      &\labelledheight{.4in}{profondeur}\cr
      %
      &\hrulefill\cr
      %
   \noalign{\vskip3pt}%
      &\leftarrowfill { largeur }\rightarrowfill\cr
}}

^^{lignes de base}
La \refterm{ligne de base} est comme une des
lignes-guide claire d'un bloc de papier lign\'e.  
Les bo\^\i tes de lettres telles qu'un `g'
s'\'etendent sous la ligne de base~; Les bo\^\i tes de lettres telles qu'un `h' ne le font pas. 
La hauteur d'une bo\^\i te est la distance par laquelle une bo\^\i te s'\'etend au-dessus de sa
ligne de base, tandis que sa profondeur est la distance par laquelle elle s'\'etend sous sa
ligne de base. \bix^^{point de r\'ef\'erence}
Le \minref{point de r\'ef\'erence}
d'une bo\^\i te est l'endroit o\`u sa ligne de base rejoint son cot\'e gauche.

{\tighten
\TeX\ construit une hbox $H$ d'une liste horizontale en d\'esignant un point 
de r\'ef\'erence pour $H$ et en enfilant les items dans la liste $H$ un par un 
de gauche \`a droite. chaque bo\^\i te de la liste est plac\'ee de telle fa\c con que 
sa ligne de base co\"\i ncide avec la ligne de base de $H$\kern-2pt, c'est-\`a-%
dire que les bo\^\i tes la composant sont align\'ees horizontalement% 
\footnote{Si une bo\^\i te est hauss\'ee ou baiss\'ee avec ^|\raise| ou ^|\lower|, 
\TeX\ utilise son point de r\'ef\'erence avant le d\'eplacement quand il la 
place.}. La hauteur de $H$ est la hauteur de la bo\^\i te la plus grande de la 
liste, et la profondeur de $H$ est la profondeur de la bo\^\i te la plus 
profonde de la liste. La largeur de $H$ est la somme des largeurs de tous 
les items dans la liste. Si quelques-uns de ces items sont des 
\refterm{ressort}s et que \TeX\ a besoin de r\'etr\'ecir ou d'\'etirer le ressort, la 
largeur de $H$ sera plus large ou plus \'etroite en cons\'equence. Voir la 
\knuth{page~77}{90} pour les~d\'etails.
\par}

De m\^eme, \TeX\ construit une vbox $V$ d'une liste verticale en d\'esignant un
point de r\'ef\'erence temporaire pour $V$ et ensuite en enfilant les items dans la
liste $V$ un par un de haut en bas.  Chaque bo\^\i te dans la liste est
plac\'ee de telle fa\c con que leur point de r\'ef\'erence est align\'e verticalement avec le
point de r\'ef\'erence de $V$\footnote{Si une bo\^\i te est d\'eplac\'ee \`a gauche
ou \`a droite avec
^|\moveleft| ou ^|\moveright|, \TeX\ utilise son point de r\'ef\'erence avant
le d\'eplacement quand il la place.}. Quand chaque bo\^\i te autre que la premi\`ere est ajout\'ee \`a
\Vcomma, \TeX\ met un \minref{ressort interligne} juste au-dessus d'elle.  (Ce
^{ressort interligne} n'a pas d'\'equivalent pour les hbox.)  La largeur de $V$ est la
largeur de la bo\^\i te la plus large de la liste, et les extensions verticales (hauteur
plus profondeur) de $V$ est la somme des \'el\'ements verticaux \'etendus de tous les
items de la liste.

\bix^^|\vbox|
\bix^^|\vtop|
La diff\'erence entre |\vbox| et |\vtop| est sur leur fa\c con de partitionner
l'extension verticale de $V$ en hauteur et en largeur.
Le choix du point de r\'ef\'erence de $V$ d\'etermine cette partition.
\ulist
\li Pour |\vbox|, \TeX\ place le point de r\'ef\'erence sur une ligne 
horizontale avec le point de r\'ef\'erence du dernier composant bo\^\i te ou trait 
de \Vcomma, sauf que si la derni\`ere bo\^\i te (ou le dernier trait) est 
suivie d'un ressort ou d'un cr\'enage, \TeX\ place le point de r\'ef\'erence tout en 
bas de $V$% 
\footnote{La profondeur est limit\'ee par le param\`etre 
^|\boxmaxdepth| \ctsref{\boxmaxdepth}.}.

\li Pour |\vtop|, \TeX\ place le point de r\'ef\'erence sur une ligne horizontale
avec le point de r\'ef\'erence du premier composant bo\^\i te ou trait de \Vcomma,
sauf que si la premi\`ere bo\^\i te (ou le premier trait)
est pr\'ec\'ed\'ee par un ressort ou un cr\'enage, \TeX\ place
le point de r\'ef\'erence tout en haut de \Vperiod.

\endulist
\noindent
A proprement parler, alors, |\vbox| met le point de r\'ef\'erence pr\`es du haut
du vbox et |\vtop| le met pr\`es du haut.
Quand vous voulez aligner une rang\'ee de vbox pour que leur sommet
s'aligne horizontalement,
vous utiliserez normalement |\vtop| plut\^ot que |\vbox|. 
Voir les \knuth{pages~78 et 80--81}{91 et 94--95} pour les
d\'etails sur comment \TeX\ construit des vbox.
\eix^^|\vbox|
\eix^^|\vtop|
\eix^^{point de r\'ef\'erence}

Vous avez une certaine libert\'e dans la construction de bo\^\i tes.  Le mat\'eriel
\`a composer
dans une bo\^\i te peut s'\'etendre au-del\`a des bords de la bo\^\i te comme il le fait
pour certaines lettres (pour la plupart italiques ou pench\'ees). Les composants 
bo\^\i tes des grandes bo\^\i tes peuvent se chevaucher.  
Une bo\^\i te peut avoir une largeur, une profondeur ou une hauteur n\'egative,
des bo\^\i tes comme celles-ci ne sont pas souvent utiles.

Vous pouvez sauvegarder une bo\^\i te dans un \refterm{registre} de bo\^\i te et la rappeler plus tard.
Avant d'utiliser un registre de bo\^\i te,
^^{registres de \boite}
Vous devez la r\'eserver et lui donner un nom
avec la commande ^|\newbox| \ctsref{\@newbox}.  Voir  
\conceptcit{registre} pour plus d'information sur les registres de bo\^\i tes.
\endconcept


\conceptindex{caract\`eres}
\concept caract\`ere

{\tighten
\TeX\ travaille avec des \defterm{caract\`eres} provenant de deux contex\-tes~:
des caract\`eres du source, quand il lit et des caract\`eres de sortie,
quand il compose. \TeX\ transforme la plupart des caract\`eres saisis dans
le caract\`ere de sortie qui lui correspond. Par exemple, il transforme
normalement la lettre saisie `|h|' en `h' compos\'e dans la police courante.
Ce n'est pas le cas, n\'eanmoins, pour un caract\`ere saisi tel que `|$|' qui 
a une signification sp\'eciale.
\par}

\TeX\ obtient ses caract\`eres saisis en les lisant \`a partir du fichier source 
(ou de votre terminal) et en d\'eveloppant les \refterm{macros:macro}. Ce
sont les seuls moyens par lesquels \TeX\  puisse acqu\'erir un caract\`ere saisi.
Chaque caract\`ere d'entr\'ee a un num\'ero de code correspondant \`a sa position 
dans la table de code \refterm{\ascii}. ^^{\ascii}
Par exemple, la lettre `|T|' a le code \ascii~$84$.

Quand \TeX\ lit un caract\`ere, il lui attache un \refterm{code de cat\'egorie}.
^^{codes de cat\'egorie//attach\'e pendant l'entr\'ee}
Le code de cat\'egorie affecte la fa\c con dont \TeX\ interpr\`ete le caract\`ere
une fois qu'il a \'et\'e lu. \TeX\ d\'etermine (et se souvient) des codes de
cat\'egorie des caract\`eres d'une macro quand il lit sa d\'efinition. Comme \TeX\
lit les caract\`eres avec ses yeux \seeconcept{\anatomy} il fait quelques
``filtrages'' comme condenser des suites de caract\`eres espace en un espace
seul. Voir les \knuth{pages~46--48}{56--59} pour les d\'etails de ce filtrage.

Les ``^{caract\`eres de contr\^ole}'' \ascii\ ont les codes $0$--$31$ et $127$--$255$.
Ils ne sont pas visualisables ou provoquent des choses \'etranges si vous essayez
de les afficher. 
N\'eanmoins, on en a parfois besoin dans les sources \TeX, donc \TeX\ a une
fa\c con sp\'eciale de les noter.
\xrdef{twocarets}
Si vous saisissez `|^^|$c$', o\`u $c$ est n'importe quel caract\`ere, vous obtenez 
le caract\`ere dont le code \ascii\ est soit plus grand soit plus petit de $64$
que le code \ascii\ de $c$.
La plus grande valeur de code acceptable en utilisant cette notation est $127$
donc la notation n'est pas ambigu\"e.
Trois instances particu\-li\`erement communes de cette notation sont 
`|^^M|' (le caract\`ere \ascii\ \asciichar{retour}),
`|^^J|' (le caract\`ere \ascii\ \asciichar{saut\ de\ page}) et 
`|^^I|' (le caract\`ere \ascii\ \asciichar{tabulation\ horizontale}).
\ttidxref{^^M}\ttidxref{^^J}\ttidxref{^^I}

{\tighten
\TeX\ a aussi une autre notation pour indiquer la valeur des codes \ascii\ 
qui fonctionne pour tous les codes de caract\`ere de $0$ \`a $255$.
\xrdef{hexchars}
Si vous saisissez `|^^|$xy$', o\`u $x$ et $y$ sont des ^{chiffres
hexad\'ecimaux} `|0123456789abcdef|', vous obtenez le caract\`ere dont le 
code est sp\'ecifi\'e. Des lettres minuscules sont n\'ecessaires ici.
\TeX\ opte pour l'interpr\'etation des ``chiffres hexad\'ecimaux'' quand il
a le choix, donc vous ne devez pas faire suivre un caract\`ere comme `|^^a|' 
par un chiffre hexad\'ecimal minuscule---si vous le faites, vous aurez la
mauvaise interpr\'etation.
Si vous avez besoin d'utiliser cette notation, vous trouverez pratique
d'avoir une table des codes \ascii.
\par}

Un caract\`ere de sortie est un caract\`ere de composition.
Une commande qui produit un caract\`ere de sortie signifie
``produit une \refterm{bo\^\i te} contenant un caract\`ere num\'ero $n$ dans
la \refterm{police} courante'', o\`u $n$ est d\'etermin\'e par la commande.
\TeX\ produit votre document compos\'e en combinant de telles bo\^\i tes avec 
d'autres \'el\'ements typographiques et les arrange sur la page.

Un caract\`ere d'entr\'ee dont le code de cat\'egorie est $11$ (^{lettre}) ou
$12$ (autre)
^^{autres caract\`eres}
agit comme une commande pour produire le caract\`ere de sortie correspondant.
En plus, vous pouvez demander \`a \TeX\ de produire un carac\-t\`ere $n$ en
entrant la commande `|\char |$n$' \ctsref{\char}, ^^|\char| o\`u $n$ est un
\refterm{nombre} entre $0$ et $255$.  Les commandes `|h|',
|\char`h| et |\char104| ont toutes le m\^eme effet.  ($104$ est le
code \ascii\ de `h'.)

\endconcept


\conceptindex{caract\`eres actifs}
\concept {caract\`ere actif}

Un \defterm{caract\`ere actif} est un \refterm{caract\`ere} poss\'edant une 
d\'efinition, c'est-\`a-dire, une d\'efinition de macro, qui lui est associ\'ee.  
^^{macros//nomm\'ees par des caract\`eres actifs} 
Vous pouvez penser \`a un caract\`ere actif comme \`a une s\'equence de contr\^ole 
d'un type sp\'ecial. Quand \TeX\ rencontre un caract\`ere actif, il ex\'ecute la 
d\'efinition associ\'ee avec le caract\`ere. Si \TeX\ rencontre un caract\`ere actif 
qui n'a pas de d\'efinition associ\'ee, il se plaindra d'un ``undefined control 
sequence''.

Un caract\`ere actif a un \refterm{code de cat\'egorie} \`a $13$ (la valeur
d'^|\active|).
Pour d\'efinir un caract\`ere actif, vous devez d'abord
utiliser la commande ^|\catcode|
\ctsref{\catcode} pour le rendre actif
et ensuite donner la d\'efini\-tion du caract\`ere, en utilisant
une commande comme |\def|, |\let| ou |\chardef|.
La d\'efinition d'un caract\`ere actif a la m\^eme forme que
la d\'efinition d'une \refterm{s\'equence de contr\^ole}.
^^{codes de cat\'egorie//de caract\`eres actifs}
Si vous essayez de d\'efinir la macro pour un caract\`ere actif
avant d'avoir rendu le caract\`ere actif, \TeX\ se plaindra d'un
"missing control sequence".  

Par exemple, le caract\`ere tilde (|~|) est d\'efini comme un  caract\`ere
actif dans \plainTeX. Il
produit un espace entre deux mots mais relie ces mots pour que 
\TeX\ ne transforme pas cet espace en \refterm{coupure de ligne}.
\refterm{\PlainTeX:\plainTeX} d\'efinit `|~|' par les commandes~:

\csdisplay
\catcode `~ = \active \def~{\penalty10000\!visiblespace}
|
(|\penalty| inhibe une coupure de ligne et `|\!visiblespace|'
ins\`ere un espace.)
\endconcept


\concept {caract\`ere d'\'echappement}

Un \defterm{caract\`ere d'\'echappement} introduit une s\'equence de contr\^ole.  
Le caract\`ere d'\'echappement dans \refterm{\plainTeX} est l'antislash (|\|).
\indexchar \
Vous pouvez changer le caract\`ere d'\'echappement de $c_1$ vers $c_2$
en r\'eassignant les codes de cat\'egorie de $c_1$ et $c_2$
avec la commande ^|\catcode| \ctsref{\catcode}.
Vous pouvez aussi d\'efinir des caract\`eres d'\'echap\-pement additionnels de la m\^eme fa\c con.
Si vous voulez composer du mat\'e\-riel contenant des caract\`eres d'\'echappement litt\'eraux, 
vous devez
soit 
(a) d\'efinir une s\'equence de  contr\^ole  qui transforme le caract\`ere d'\'echap\-pe\-ment imprim\'e soit
(b) changer temporairement son code de cat\'egorie, en utilisant la
m\'ethode montr\'ee en \xrefpg{verbatim}.  La d\'efinition~:

\csdisplay
\def\\{$\backslash$}
|
est un moyen de cr\'eer une s\'equence de  contr\^ole  qui le transforme en `$\backslash$'
(un antislash compos\'e dans une police math\'ematique).

Vous pouvez utiliser le param\`etre ^|\escapechar| \ctsref{\escapechar} pour sp\'ecifier
comment le caract\`ere de  contr\^ole  est repr\'esent\'e dans des s\'equences de contr\^ole
synth\'etis\'ees,
c'est-\`a-dire, celles cr\'e\'ees par |\string| et |\message|.


\endconcept


\concept {c\'esure}

\TeX\ c\'esure les mots automatiquement en compilant votre document.
\TeX\ n'est pas un acharn\'e de l'insertion de c\'esure, pr\'ef\'erant trouver \`a la
place une bonne coupure de ligne en ajustant l'espace entre les mots et en
d\'epla\c cant les mots d'une ligne \`a l'autre.
\TeX\ est assez intelligent pour comprendre les c\'esures qui sont d\'ej\`a dans
les mots

Vous pouvez contr\^oler les c\'esures de \TeX\ de plusieurs fa\c cons~:
\ulist
\li Vous pouvez dire \`a \TeX\ d'autoriser une c\'esure \`a un endroit particulier
en ins\'erant une c\'esure optionnelle
^^{c\'esures optionnelles}
avec la commande ^|\-| \ctsref{\@minus}.
\li Vous pouvez dire \`a \TeX\ comment couper des mots particuliers dans votre
document avec la commande ^|\hyphenation| \ctsref{\hyphenation}.  
\li Vous pouvez englober un mot dans une \refterm{hbox} pour emp\^echer \TeX\
de le couper.
\li Vous pouvez fixer la valeur des p\'enalit\'es comme avec |\hyphen!-pe!-nalty|
\ctsref\hyphenpenalty.
\endulist
\noindent Si un mot contient une c\'esure explicite ou optionnelle,
\TeX\ ne le coupera jamais ailleurs.
\endconcept


\concept classe

La \defterm{classe} d'un \refterm{caract\`ere} sp\'ecifie le r\^ole de ce caract\`ere
dans les formules math\'ematiques.  La classe d'un caract\`ere est encod\'ee dans son
\refterm{mathcode}.  ^^{mathcodes//classe encod\'ee en} Par exemple, le
signe \'egal `|=|' est dans la classe $3$ (Relation).  \TeX\ utilise sa
connaissance de classe de caract\`ere pour d\'ecider combien d'espace mettre
entre diff\'erents composants d'une formule math\'ematique.  \margin{clarifying 
material added} Par exemple, voici une formule math\'ematique affich\'ee comme
\TeX\ l'im\-prime normalement et avec la classe de chaque caract\`ere modifi\'e
al\'eatoi\-rement~:
$$\strut a+(b-a)=a \qquad
   \mathopen{a}\mathord{+}\mathrel{(}\mathclose{b}\mathclose{-}
   \mathop{a}\mathopen{)}\mathord{=}\mathopen{a}$$

Voir \xrefpg\mathord\ de ce livre pour une liste des classes et la
\knuth{page~154}{180} pour leurs significations.

\endconcept


\conceptindex{codes de cat\'egorie}
\concept {codes de cat\'egorie}

Le \defterm{code de cat\'egorie} d'un \refterm{caract\`ere} d\'etermine le r\^ole
de ce caract\`ere dans \TeX.
^^{caract\`eres//code de cat\'egorie}
Par exemple, \TeX\ assigne un certain r\^ole aux
lettres, un autre au caract\`ere espace, et ainsi de suite.  \TeX\ attache un
code \`a chaque caract\`ere qu'il lit.  Quand \TeX\ lit la
lettre `|r|', par exemple, il lui attache normalement
le code de cat\'egorie $11$ (lettre). 
Pour de simples applications \TeX\ vous n'avez pas \`a vous pr\'eoccuper des
codes de cat\'egorie, mais ils deviennent importants quand vous essayez 
des effets sp\'eciaux.

Les codes de cat\'egorie ne s'appliquent qu'aux caract\`eres que \TeX\ lit dans les
fichiers source.  Une fois qu'un caract\`ere a pass\'e l'^{\oesophage} de \TeX\ 
\seeconcept{\anatomy} et a \'et\'e interpr\'et\'e, son code de cat\'egorie ne
sert plus.  Un caract\`ere que vous produisez avec la commande ^|\char|
\ctsref{\char} n'a pas de code de cat\'egorie parce que |\char|
est une instruction pour que \TeX\ produise un certain caract\`ere d'une 
certaine \refterm{police}.  Par exemple, le code ^{\ascii} pour `|\|' 
(le caract\`ere d'\'echappement usuel) est $92$.  Si
vous saisissez `|\char92 grok|', Ce n'est \emph{pas} \'equivalent \`a |\grok|.
\`a la place il demande \`a \TeX\ de
composer `$c$grok', o\`u $c$ est le caract\`ere en position $92$
de la table de code de la police courante.

Vous pouvez utiliser la commande ^|\catcode| \ctsref{\catcode} pour r\'eassigner le
code de cat\'egorie de tout caract\`ere.  En changeant les code de cat\'egorie vous
pouvez changer les r\^oles de caract\`eres vari\'es.  Par exemple, si vous saisissez 
`|\catcode`\@ = 11|', le code de cat\'egorie du signe arrose (|@|) sera celui 
d'une ``lettre''.  Vous pourrez alors avoir `|@|' dans le nom d'une s\'equence de 
contr\^ole.

Voici une liste des codes de cat\'egorie tels qu'ils sont d\'efinis dans
\refterm{\plainTeX} (voir \xref{twocarets} pour une explication de la
notation |^^|),
ainsi que les caract\`eres de chaque cat\'egorie~:

\xrdef{catcodes}
\vskip\abovedisplayskip
%k \vskip 0pt plus 2pt % to fix bad page break
{%k \interlinepenalty = 10000
\halign{\indent\hfil\strut#&\qquad#\hfil\cr
\it Code&\it Signification\cr
\noalign{\vskip\tinyskipamount}
0&Caract\`ere d'\'echappement \quad |\| ^^{caract\`ere d'\'echappement//code de cat\'egorie}
   {\recat!ttidxref[\//code de cat\'egorie]]
   \cr
1&D\'ebut d'un groupe \quad |{| ^^{groupes}
   {\recat!ttidxref[{//code de cat\'egorie]]
   \cr
2&Fin d'un groupe   \quad |}|
   {\recat!ttidxref[}//code de cat\'egorie]]
   \cr
3&basculement en math\'ematique   \quad |$| ^^{basculement en math\'ematique}
   {\recat!ttidxref[$//code de cat\'egorie]]
   \cr
4&Alignement tab   \quad |&| ^^{tabulations} ^^{alignements//caract\`ere tabulation pour}
   \ttidxref{&//code de cat\'egorie} \cr
5&Fin d'une ligne   \quad |^^M| \tequiv \ascii\ \asciichar{retour chariot}
   ^^{fin de ligne} \ttidxref{^^M//code de cat\'egorie}\cr
6&Param\`etre de macro   \quad |#|
   ^^{macros//param\`etres de}
   ^^{param\`etres//indiqu\'es par \b\tt\#\e}
   \ttidxref{#//code de cat\'egorie} \cr
7&Exposant   \quad |^| et |^^K| ^^{exposants}
   \ttidxref{^^K}
   \ttidxref{^//code de cat\'egorie} \cr
8&Indice   \quad |_| et |^^A| ^^{indices}
   \ttidxref{^^A}
   \ttidxref{_//code de cat\'egorie} \cr
9&caract\`ere ignor\'e  \quad |^^@| \tequiv \ascii\ \asciichar{nul}
  ^^{caract\`eres ignor\'es} \indexchar ^^@ \cr
10&Espace  \quad \visiblespace\ et |^^I| \tequiv \ascii\ 
  \asciichar{tabulation}
  ^^{tabulation horizontale}
  ^^{caract\`eres espaces//code de cat\'egorie} \indexchar ^^I
  {\recat!ttidxref[ ]]
  \cr
11&Lettre  \quad |A| \dots |Z| et |a| \dots |z| ^^{lettre}\cr
12&Autres caract\`eres \quad (tout ce qui n'est pas ci-dessus ou ci-dessous) 
  ^^{autres caract\`eres}\cr
13&Caract\`ere actif  \quad |~| et |^^L| \tequiv\ascii\ \asciichar{saut\ de\ page} 
   ^^{caract\`eres actifs} ^^{retour chariot} \indexchar ~ \indexchar ^^L \cr
14&Caract\`ere de commentaire  \quad |%| ^^{commentaires}
   {\recat!ttidxref[%//code de cat\'egorie]]
   \cr
15&Caract\`ere invalide  \quad |^^?| \tequiv \ascii\ \asciichar{suppression}
  ^^{caract\`eres invalides} \indexchar ^^? \cr
}}
\vskip\belowdisplayskip
%k \vskip 0pt plus 2pt % to fix bad page break

\noindent Mise \`a part ceux des cat\'egories $11$--$13$,
tous les caract\`eres d'une cat\'egorie particuli\`ere produisent le m\^eme effet.
\margin{Misleading material removed.}
Par exemple, supposez
que vous saisissiez~:
\csdisplay
\catcode`\[ = 1 \catcode`\] = 2
|
Alors les caract\`eres crochets droits et gauches deviennent \'equivalents aux caract\`eres de
d\'ebut et de fin d'un groupe, les caract\`eres accolades droites et gauches.
Avec ces d\'efinitions `|[a b]|'
est un groupe valide, ainsi que \hbox{`|[a b}|'} et~\hbox{`|{a b]|'}.

Les caract\`eres des cat\'egories $11$ (lettre) et $12$ 
(autres caract\`eres) agis\-sent comme des \refterm{commandes:commande}
qui signifient
``produit une \refterm{bo\^\i te} contenant ce caract\`ere 
compos\'e dans la police courante''.
La seule distinction entre les lettres et les ``autres'' caract\`eres est
que les lettres peuvent appara\^\i tre dans des \refterm{mots de contr\^ole} mais
pas les ``autres'' caract\`eres.

Un caract\`ere de cat\'egorie $13$ (actif) agit comme une s\'equence de contr\^o\-le
\`a lui tout seul. \TeX\ se plaint s'il rencontre un caract\`ere actif qui n'est 
pas associ\'e avec une d\'efinition.
^^{caract\`eres actifs}

Si \TeX\ rencontre un caract\`ere invalide ^^{caract\`eres invalides} (cat\'egorie $15$) 
dans votre source, il s'en plaindra. 

Les caract\`eres `|^^K|' et `|^^A|' ont \'et\'e inclus dans les cat\'egories 
$8$ (indice) et $9$ (exposant), m\^eme si ces significations
ne suivent pas l'interpr\'etation du standard \refterm{\ascii}.
C'est parce que certains claviers, tout du moins certains \`a 
l'universit\'e de Stanford d'o\`u \TeX\ est originaire, n'ont pas
de touche fl\`eche vers le bas et fl\`eche vers le haut qui g\'en\`erent ces caract\`eres.
\ttidxref{^^A}
\ttidxref{^^K}

Il y a une subtilit\'e sur la fa\c con dont \TeX\ assigne les codes de cat\'egories
qui peut vous d\'eboussoler si vous ne la connaissez pas. \TeX\ a parfois
besoin de rechercher un caract\`ere deux fois quand il fait un survol initial~:
la premi\`ere fois pour la fin d'une pr\'ec\'edente construction, c'est-\`a-dire, 
une s\'equence de contr\^ole et ensuite pour transformer ce caract\`ere en token.
\TeX\ n'a pas besoin d'assigner le code de cat\'egorie avant d'avoir vu le
\emph{second} caract\`ere. Par exemple~:

\csdisplay
\def\foo{\catcode`\$ = 11 }% Make $ be a letter.
\foo$ % Produces a `$'.
\foo$ % Undefined control sequence `foo$'.
|
\noindent
Ce bout de code \TeX\ produit `\$' dans la sortie. Quand \TeX\ voit le 
`|$|' sur la seconde ligne, il recherche la fin du nom d'une s\'equence de 
contr\^ole. Puisque le `|$|' n'est pas encore une lettre, il marque la fin de 
`|\foo|'. Ensuite, \TeX\ d\'eveloppe la macro `|\foo|' et change le code de 
cat\'egorie de `|$|' en $11$ (lettre). Alors, \TeX\ lit le `|$|' ``pour de 
vrai''. Puisque `|$|' est maintenant une lettre, \TeX\ produit une bo\^\i te 
contenant le caract\`ere `|$|' de la police courante. Quand \TeX\ voit la 
troisi\`eme ligne, il traite `|$|' comme une lettre et donc consid\`ere qu'elle 
fait partie du nom de la s\'equence de contr\^ole. Avec comme r\'esultat qu'il se 
plaint d'un ``undefined control sequence |\foo$|''.

\TeX\ se comporte de cette fa\c con m\^eme lorsque le caract\`ere de 
terminaison est une fin de ligne. Par exemple, supposez que la macro 
|\fum| active le caract\`ere de fin de ligne. Alors si |\fum| 
appara\^\i t sur une ligne $\ell$ elle-m\^eme, \TeX\ interpr\'etera 
d'abord la fin de la ligne $\ell$ comme la fin de la s\'equence de 
contr\^ole |\fum| et ensuite \emph{r\'einterpr\'etera} la fin de la 
ligne de $\ell$ comme un caract\`ere actif.

\endconcept


\conceptindex{commandes}
\concept commande

Une \defterm{commande} demande \`a \TeX\ de faire une certaine action.
Chaque \refterm{token} qui atteint l'estomac de \TeX\seeconcept{\anatomy}
agit comme une commande, sauf celui qui fait partie des arguments d'une
autre commande (voir dessous).
^^{tokens//comme commandes}
Une commande peut \^etre appel\'ee par une
\refterm{s\'equence de contr\^ole}, par un \refterm{caract\`ere actif} ou par un
caract\`ere ordinaire.  Il peut sembler bizarre que \TeX\ traite un carac\-t\`ere
ordinaire comme une commande, mais en fait c'est ce qu'il fait~:
Quand \TeX\ voit 
un caract\`ere ordinaire
il construit une \refterm{bo\^\i te} contenant ce caract\`ere compos\'e dans
la police courante.

Une commande peut avoir des arguments.
Les arguments d'une commande sont de simples tokens ou
groupes de tokens qui compl\`etent la description de ce que
la commande est suppos\'ee faire.
Par exemple, la commande `|\vskip 1in|' demande \`a \TeX\ de sauter
$1$ pouce verticalement.  Elle a un argument `|1in|',
qui est constitu\'e de trois tokens.
La description de ce que |\vskip| est suppos\'ee faire serait incompl\`ete
sans sp\'ecifier de combien elle est suppos\'ee sauter.
Les tokens dans les arguments d'une commande ne sont pas consid\'er\'es 
eux-m\^emes comme des commandes.

Quelques exemples de diff\'erentes sortes de commandes de \TeX\ sont~: 
\ulist\compact
\li Des caract\`eres ordinaires, comme `|W|', qui demande \`a \TeX\
de produire une bo\^\i te contenant un `W' compos\'e.
\li Des commandes de modification de police,
telles que |\bf|, qui commence une composition en gras
\li Des accents, tels que |\`|, qui produit un accent grave comme dans `\`e'
\li Des symboles sp\'eciaux et des ligatures, comme |\P| (\P) et |\ae| (\ae)
\li Des param\`etres, tels que |\parskip|, le montant du ressort que
\TeX\ met entre les paragraphes
\li des symboles math\'ematiques, tels que |\alpha| ($\alpha$) et |\in| ($\in$)
\li Des op\'erateurs math\'ematiques, tels que |\over|, qui produit une fraction
\endulist
\endconcept


\concept {constantes d\'ecimales}

See \conceptcit{nombre}.
\endconcept

\concept {construction de page}

Voir \conceptcit{page}.
\endconcept


\conceptindex{coupures de ligne}
\concept {coupure de ligne}

Une \defterm{coupure de ligne} est un endroit dans votre document o\`u \TeX\ 
termine une ligne quand il compose un paragraphe. Quand \TeX\ compile votre 
document, il collecte le contenu de chaque paragraphe dans une \refterm{liste
horizontale}. Quand il a collect\'e un paragraphe entier, il analyse la liste 
pour trouver ce qu'il consid\`ere comme \'etant les meilleures coupures de ligne
possibles. \TeX\ associe des ``^{d\'em\'erites}'' avec diff\'erents sympt\^omes de
coupure de ligne non attractive---des lignes ayant trop ou trop peu d'espace
entre les mots, des lignes cons\'ecutives se terminant par des c\'esures, et 
ainsi de suite. Il choisit alors les coupures de lignes de mani\`ere \`a minimiser
le nombre total de d\'em\'erites. Voir les \knuth{pages~96--101}{112--118} pour une 
description compl\`ete des r\`egles de coupure de ligne de \TeX.

Vous pouvez contr\^oler le choix des coupures de ligne de \TeX\ de plusieurs mani\`eres~:
\ulist

\li vous pouvez ins\'erer une \refterm{p\'enalit\'e} (\xref{hpenalty}) quelque part
dans la liste horizontale que \TeX\ construit quand il forme un paragraphe.
^^{p\'enalit\'es//en listes horizontales}
Une p\'enalit\'e
positive d\'ecourage \TeX\ de couper la ligne \`a cet endroit, tandis qu'une
p\'enalit\'e n\'egative---un bonus, en quelque sorte---encourage \TeX\ \`a couper
la ligne \`a cet endroit.  Une p\'enalit\'e de $10000$ ou plus emp\^eche une coupure de
ligne, tandis qu'une p\'enalit\'e de $-10000$ ou moins force une coupure de ligne.
Vous pouvez obtenir les m\^emes effets avec les commandes ^|\break| et
^|\nobreak| (voir \pp\xrefn{hbreak} et~\xrefn{hnobreak}).

\li Vous pouvez dire \`a \TeX\ d'autoriser une c\'esure \`a un endroit particulier 
en ins\'erant une c\'esure optionnelle
^^{c\'esures optionnelles}
avec la commande |\-| \ctsref{\@minus}, ou
sinon en contr\^olant comment \TeX\ met des c\'esures dans votre document \seeconcept
{c\'esure}.
^^|\-//sur une coupure de ligne|

\li Vous pouvez dire \`a \TeX\ d'autoriser une coupure de ligne apr\`es un 
(/) ^{slash} entre
deux mots en ins\'erant un ^|\slash| \ctsref{\slash}
entre eux, par exemple, `|fur!-longs\slash fortnight|'.

\li Vous pouvez dire \`a \TeX\ de ne pas couper une ligne entre deux mots particuliers en
ins\'erant une ^{punaise} (|~|) entre ces mots.
^^|~//sur une coupure de ligne|

\li Vous pouvez ajuster les p\'enalit\'es associ\'ees avec les coupures de ligne en
assignant des valeurs diff\'erentes aux \refterm{param\`etres:param\`etre} de coupures de ligne de \TeX. 

\li Vous pouvez englober un mot ou une s\'equence de mots dans une \refterm{hbox}, 
emp\^echant ainsi \TeX\ de couper la ligne o\`u que cela soit dans l'hbox.
^^{hbox//contr\^oler des coupures de lignes}
\endulist

Il est pratique de savoir o\`u \TeX\ peut couper une ligne~:
\ulist
\li sur un ressort, sous r\'eserve que~:
\olist
\li l'objet pr\'ec\'edant le ressort est un des suivants~:
une bo\^\i te, un objet optionnel (par exemple, une c\'esure optionnelle),
la fin d'une formule math\'ematique,
un \'el\'ement extraordinaire, 
ou du mat\'eriel vertical produit par |\mark|, |\vadjust|
ou |\insert|
\li Le ressort ne fait pas partie d'une formule math\'ematique
\endolist
\noindent
Quand \TeX\ coupe une ligne sur un ressort, il fait la coupure du cot\'e gauche
de l'espace du ressort et oublie le reste du ressort.
\li sur un cr\'enage qui est imm\'ediatement suivi par un ressort,
\`a condition que ce cr\'enage ne soit pas dans une formule math\'ematique
\li \`a la fin d'une formule math\'ematique qui est imm\'ediatement suivie par un ressort
\li sur une p\'enalit\'e, m\^eme dans une formule math\'ematique
\li sur une c\'esure optionnelle
\endulist
Quand \TeX\ coupe une ligne, il efface toute s\'equence de ressort, cr\'enage et 
p\'enalit\'e qui suivent le point de coupure.
Si une telle s\'equence est suivie par le d\'ebut d'une formule math\'ematique, il
efface aussi tout cr\'enage produit au d\'ebut de la formule.
\endconcept


\conceptindex{coupures de page}
\concept {coupure de page}

Une \defterm{coupure de page} est un endroit dans votre document o\`u \TeX\ 
termine une page et (sauf \`a la fin du document) en commence une nouvelle.  
Voir \conceptcit{page} pour le processus que \TeX\ traverse pour choisir
une coupure de page.

Vous pouvez contr\^oler le choix de coupure de page de \TeX\ de plusieurs fa\c cons~:
\ulist
\li Vous pouvez ins\'erer une p\'enalit\'e (\xref{vpenalty}) 
^^{p\'enalit\'es//en listes verticales}
entre deux \'el\'ements de la liste verticale principale.  Une p\'enalit\'e positive
d\'ecourage \TeX\ de couper la page \`a cet endroit, tandis qu'une p\'enalit\'e 
n\'egative---un bonus, autrement dit---%
encourage \TeX\ de couper la page \`a cet endroit. Une p\'enalit\'e de $10000$
ou plus pr\'evient une coupure de page, tandis qu'une p\'enalit\'e de $-10000$ ou 
moins force une coupure de page. Vous pouvez obtenir le m\^eme effet avec les
commandes |\break| et |\nobreak| \ctsref{vbreak}.

\li Vous pouvez ajuster les p\'enalit\'es associ\'ees aux coupures de page
en assignant diff\'erentes valeurs aux \refterm{param\`etres:param\`etre} de 
coupure de page de \TeX.

\li Vous pouvez englober une suite de paragraphes ou autres \'el\'ements dans
la liste verticale principale avec une \refterm{vbox}, ceci emp\^eche \TeX\ de
couper la page n'importe o\`u dans la suite.
\endulist

Une fois que \TeX\ a choisi une coupure de page, il place la portion 
de la liste verticale principale qui pr\'ec\`ede la coupure en |\box255|.
Il appelle alors la \refterm{routine de sortie} courante pour ex\'ecuter |\box255| 
et \'eventuellement envoyer son contenu dans le \dvifile.
^^{\dvifile//mat\'eriel de la routine de sortie}
La routine de sortie doit aussi inclure des \refterm{insertions}, telles 
que des notes de pied de page, que \TeX\ a accumul\'e en ex\'ecutant la page.

Il est utile de conna\^\i tre les endroits o\`u \TeX\ peut couper une page~:
\ulist
\li Sur le ressort, \`a condition que l'\'el\'ement pr\'ec\'edent le ressort soit
une bo\^\i te, un \'el\'ement extraordinaire, une marque ou une insertion.
Quand \TeX\ coupe une page sur un ressort, il fait la coupure devant
l'espace du ressort et oublie le reste du ressort.
\li Sur un cr\'enage qui est imm\'ediatement suivi par un ressort.
\li Sur une p\'enalit\'e, vraisemblablement entre les lignes d'un paragraphe.
\endulist
Quand \TeX\ coupe une page, il efface toute s\'equence de ressort, cr\'enages et \'el\'ements
de p\'enalit\'e qui suivent le point de coupure.


\endconcept


\conceptindex{cr\'enages}
\concept cr\'enage

^^{espacement//ajuster avec des cr\'enages}
Un \defterm{cr\'enage} indique un changement de l'espacement normal entre 
les \'el\'ements  d'une liste horizontale ou verticale.
Un cr\'enage peut \^etre soit positif, soit n\'egatif.  En 
mettant un cr\'enage positif entre deux \'el\'ements, vous les \'eloignez du
montant du cr\'enage. en mettant un cr\'enage positif entre deux \'el\'ements,
vous les rapprochez du montant du cr\'enage. Par exemple, ce texte~:
\csdisplay
11\quad 1\kern1pt 1\quad 1\kern-.75pt 1
|
produit des paires de lettres qui ressemblent \`a ceci~:
\display{11\quad 1\kern1pt 1\quad 1\kern-.75pt 1}
Vous pouvez utiliser des cr\'enages en mode vertical pour ajuster l'espace
entre des paires de lignes particuli\`eres.

Un cr\'enage de taille $d$ est tr\`es similaire \`a un \'el\'ement \refterm{ressort}
qui a une taille $d$ sans \'etirement ni r\'etr\'ecissement. Le cr\'enage et le ressort
ins\`erent ou enl\`event de l'espace entre deux \'el\'ements voisins. La diff\'erence
essentielle est que \TeX\ consid\`ere deux bo\^\i tes n'ayant qu'un cr\'enage entre 
elles comme \'etant li\'ees. Cela fait que, \TeX\ ne coupe pas une ligne ou une
page sur un cr\'enage sauf si le cr\'enage est imm\'ediatement suivi d'un ressort.
Garder cette diff\'erence \`a l'esprit quand vous choisissez d'utiliser un
cr\'enage ou un \'el\'ement ressort pour un usage particulier.

\TeX\ ins\`ere automatiquement des cr\'enages entre des paires particuli\`eres de
lettres adjacentes, ajustant ainsi l'espace entre ces lettres et mettant en
valeur l'apparence de votre document compos\'e.
Par exemple, la police Computer Modern romaine de $10$ points contient un
cr\'enage pour la paire `To' qui remet le cot\'e gauche du `o' sous le `T'.
Sans le cr\'enage, vous obtiendriez \hbox{``T{o}p''} au lieu de ``Top''---
la diff\'erence est l\'eg\`ere mais visible.
Le fichier de m\'etrique
(^{\tfmfile})
de chaque \refterm{police} sp\'ecifie le placement et la taille des cr\'enages
que \TeX\ ins\`ere automatiquement quand il met du texte dans cette~police.
\margin{paragraph deleted to save space; most of the material was
already in this section.}

\endconcept


\conceptindex{dimensions}
\concept dimension

Une \defterm{dimension} sp\'ecifie une distance, c'est, une mesure lin\'eaire
d'espace.  Vous utilisez des dimensions pour sp\'ecifier des tailles de choses, comme les grandeurs
d'une ligne.  Les imprimeurs dans les pays de langue Anglaise mesurent traditionnellement
les distances en points et en picas, tandis que les imprimeurs d'Europe continentale
mesurent traditionnellement les distances en points Didot et en ciceros.  
Vous pouvez utiliser ces mesures ou d'autres, comme les pouces, qui peuvent
vous \^etre plus familier. Les ^{unit\'es de mesure} ind\'ependantes des polices
que \TeX\ comprend sont~:

\xrdef{dimdefs}
\nobreak\vskip\abovedisplayskip
\halign{\indent\hfil\tt #\qquad&#\hfil\cr
pt&^{point} (72.27 points = 1 pouce)\cr
pc&^{pica} (1 pica = 12 points)\cr
bp&big point (72 gros points = 1 pouce)\cr
in&^{pouce}\cr
cm&^{centim\`etre} (2.54 centim\`etres = 1 pouce)\cr
mm&^{millim\`etre} (10 millim\`etres = 1 centim\`etre)\cr
dd&^{\didotpt} (1157 {\didotpt}s = 1238 points)\cr
cc&^{cic\'ero} (1 cicero = 12 points Didot)\cr
sp&^{point d'\'echelle} (65536 points d'\'echelle = 1 point)\cr
}
\vskip\belowdisplayskip

Deux unit\'es de mesure suppl\'ementaires sont associ\'ees avec chaque police~: `^|ex|',
une mesure verticale correspondant habituellement \`a la hauteur de la lettre `x'
de la police et `^|em|', une mesure
horizontale habituellement \'egale \`a la taille du point de la police et
environ la largeur de la lettre `M' de la police. Finalement,
\TeX\ procure trois unit\'e de mesure ``infinies''~: `^|fil|', `^|fill|' et
`^|filll|', dans l'ordre de force croissant.

Une dimension est \'ecrite comme un ^{facteur}, c'est-\`a-dire, un multiplieur, 
suivi par une unit\'e de mesure.
^^{unit\'es de mesure}
Le facteur peut \^etre soit un \refterm{nombre} entier, soit
une \refterm{constante d\'ecimale} contenant un point d\'ecimal
ou une virgule d\'ecimale.  
Le facteur peut \^etre pr\'ec\'ed\'e par un signe plus ou moins, donc une dimension
peut \^etre positive ou n\'egative.
^^{dimensions//n\'egatives}
L'unit\'e de mesure doit \^etre l\`a, m\^eme si le nombre est
z\'ero.  Des espaces entre le nombre et l'unit\'e de mesure sont permis
mais facultatifs.  Vous trouverez une d\'efinition pr\'ecise d'une
dimension sur la \knuth{page~270}{315}.  Voici quelques exemples de dimensions~:

\csdisplay
5.9in    0pt    -2,5 pc    2fil
|
La derni\`ere repr\'esente un ordre de distance infinie.

Une distance infinie \'ecrase toute distance finie ou toute distance infinie
plus l\'eg\`ere.  Si vous additionnez |10in| \`a |.001fil|, vous obtiendrez |.001fil|~; si vous additionnez
|2fil| \`a |-1fill| vous aurez |-1fill|~; et ainsi de suite. 
\TeX\ n'accepte des distances infinies que quand
vous avez sp\'ecifi\'e l'\refterm{\'etirement} et le \refterm{r\'etr\'ecissement}
du \refterm{ressort}.

\TeX\ multiplie toutes les dimensions de votre document par un
facteur de \refterm{magnification} $f/1000$,
o\`u $f$ est la valeur du param\`etre ^|\mag|.
^^{magnification}
Puisque la valeur par d\'efaut de
|\mag| est $1000$, le cas normal est que votre document est
compos\'e comme sp\'ecifi\'e.  Vous pouvez sp\'ecifier une dimension comme elle sera 
mesur\'ee dans le document final ind\'ependamment d'une magnification en mettant
`|true|' devant l'unit\'e.  Par exemple, `|\kern 8 true pt|'
produit un cr\'enage de $8$ points quelle que soit la magnification.
\endconcept


\concept {disposition de page}

\bix^^{marges}
\bix^^{ent\^etes}
\bix^^{pieds de page}
Quand vous concevez un document, vous devez d\'ecider de son 
\defterm{ajustement de page}~: la taille de la page, les marges des
quatre cot\'es, les ent\^etes et pieds de page, s'il y en a, qui
apparaissent en haut et en bas de la page et la taille de l'espace
entre le corps du texte et l'ent\^ete ou le pied de page. \TeX\ en 
d\'efinit par d\'efaut. Il d\'efinit une page de $8 \frac1/2$
par $11$ pouces avec des marges d'approximativement un pouce de
chaque cot\'e, sans ent\^ete et avec un pied de page constitu\'e d'un
num\'ero de page centr\'e.

Les marges sont d\'etermin\'ees conjointement par les quatre param\`etres
^|\hoffset|, ^|\voffset|, ^|\hsize| et ^|\vsize| (voir
``marges'', \xrefpg{marges},
pour des conseils sur la fa\c con de les ajuster).
\eix^^{marges}
L'ent\^ete consiste normalement en une simple ligne qui appara\^\i t en haut de chaque
page, dans l'aire de la marge du haut.  Vous pouvez la d\'efinir en assignant
une liste de \refterm{token} dans le param\`etre ^|\headline| (\xref{\headline}).
Similairement,
le pied de page consiste normalement en une simple ligne qui appara\^\i t en bas de chaque
page, dans l'aire de la marge du bas.  Vous pouvez la d\'efinir en assignant
une liste de \refterm{token} dans le param\`etre ^|\footline| (\xref{\footline}).
Par exemple, l'entr\'ee~:
\csdisplay
\headline = {Baby's First Document\dotfill Page\folio}
\footline = {\hfil}
|
produit une ligne d'ent\^ete comme celle-ci sur chaque page~:
\vdisplay{
\dimen0 = \hsize
\advance \dimen0 by -\parindent
\hbox to \dimen0{Baby's First Document\dotfill Page 19}}
\noindent
et aucune ligne de pied de page.

Vous pouvez utiliser des marques pour placer le sujet d'une section de texte
courante dans l'ent\^ete ou le pied de page.
^^{marques//avec ent\^etes ou pieds de page}
Voir \conceptcit{marque} pour une explication sur comment faire cela.
\eix^^{ent\^etes}
\eix^^{pieds de page}
\endconcept

\conceptindex{d\'elimiteurs}
\concept d\'elimiteur

Un \defterm{d\'elimiteur} est un caract\`ere mis comme fronti\`ere visible d'une 
formule math\'ematique. La propri\'et\'e essentielle d'un d\'elimiteur est que \TeX\
peut ajuster sa taille par rapport \`a la taille verticale  (\refterm{hauteur} 
plus \refterm{profondeur}) de la sous-formule.
De plus, \TeX\ n'ex\'ecute d'ajustement que si le d\'elimiteur
appara\^\i t dans un ``contexte de d\'elimiteur'', plus pr\'ecis\'ement, comme argument
d'une des commandes ^|\left|,
^|\right|,
|\overwith!-delims|,
|\atop!-with!-delims|,
ou |\above!-with!-delims|
^^|\overwithdelims|
^^|\atopwithdelims|
^^|abovewithdelims|
\margin{Footnote deleted}
(voir les \pp\xrefn{\overwithdelims},~\xrefn{\left}).
Les contextes de d\'elimiteur incluent aussi tout \refterm{argument}
\`a une \refterm{macro} qui utilise l'argument dans un contexte de d\'elimiteur.

Par exemple, les parenth\`eses gauches et droites sont des d\'elimiteurs.
Si vous utilisez des ^{parenth\`eses} dans un contexte de d\'elimiteur autour
d'une formule, \TeX\ rend les parenth\`eses assez  grandes pour entourer la
\refterm{bo\^\i te} qui contient la formule (\`a condition que les
\refterm{polices:police} que vous utilisez aient des parenth\`eses assez grandes).
Par exemple~:
\csdisplay
$$ \left( a \over b \right) $$
|
donne~:
\centereddisplays $$\left (a \over b \right ) $$
Ici \TeX\ a rendu les parenth\`eses assez grandes pour s'accorder \`a la fraction.
Mais si vous \'ecrivez, \`a la place~:
\csdisplay
$$({a \over b})$$
|
vous aurez~:
$$({a \over b})$$
Puisque les parenth\`eses ne sont pas dans un contexte de d\'elimiteur, elles ne
sont \emph{pas} \'elargies.  

Les d\'elimiteurs vont par paires~:
un d\'elimiteur ouvrant \`a la gauche de la sous-formule
et un d\'elimiteur fermant \`a sa droite.
Vous pouvez choisir explicitement une hauteur plus grande pour un
d\'elimiteur avec les commandes |\bigl|, ^|\bigr|, et leurs
pendantes \ctsref{\bigl}\footnote
{\PlainTeX\ d\'efini les diff\'erentes commandes |\big| en utilisant |\left| 
et |\right| pour obtenir un contexte de d\'elimiteur. Il r\`egle la taille en
construisant une formule vide de la hauteur d\'esir\'ee.}.
Par exemple, pour avoir la 
formule affich\'ee~:
$$\bigl(f(x) - x \bigr) \bigl(f(y) - y \bigr)$$

\noindent dans laquelle les parenth\`eses externes sont un peu plus grandes 
que leurs internes, vous devez \'ecrire~:

\csdisplay
$$\bigl( f(x) - x \bigr) \bigl( f(y) - y \bigr)$$
|

Les $22$ d\'elimiteurs \plainTeX, montr\'es \`a leur taille normale, sont~:
\display{%
$( \>) \>[ \>] \>\{ \>\}
\>\lfloor \>\rfloor \>\lceil \>\rceil
\>\langle \>\rangle \>/ \>\backslash
\>\vert \>\Vert
\>\uparrow \>\downarrow \>\updownarrow
\>\Uparrow \>\Downarrow \>\Updownarrow$}
^^|)| ^^|)| ^^|[| ^^|]| ^^|\lbrack| ^^|\rbrack|
^^|\{| ^^|\}| ^^|\lbrace| ^^|\rbrace|
^^|\lfloor| ^^|\rfloor|  ^^|\lceil|  ^^|\rceil|
^^|\langle|  ^^|\rangle|  ^^|/|  ^^|\backslash|
^^|\vert|  ^^|\Vert|
^^|\uparrow|  ^^|\downarrow|  ^^|\updownarrow|
^^|\Uparrow|  ^^|\Downarrow|  ^^|\Updownarrow|
\noindent
Voici les plus grandes tailles fournies explicitement par \plainTeX\
 (les versions |\Biggl|, |\Biggr|, etc.)~:
\nobreak\vskip .5\abovedisplayskip
\display{%
$\Biggl( \>\Biggr) \>\Biggl[ \>\Biggr]
\>\Biggl\{ \>\Biggr\} \>\Biggl\lfloor \>\Biggr\rfloor
\>\Biggl\lceil \>\Biggr\rceil
\>\Biggl\langle \>\Biggr\rangle
\>\Biggm/ \>\Biggm\backslash
\>\Biggm\vert \>\Biggm\Vert
\>\Biggm\uparrow \>\Biggm\downarrow \>\Biggm\updownarrow
\>\Biggm\Uparrow \>\Biggm\Downarrow \>\Biggm\Updownarrow$}
\vskip .5\belowdisplayskip
\noindent
Les d\'elimiteurs (sauf pour `|(|', `|)|', et
`|/|')
sont les symboles list\'es sur les
pages~\xrefn{\lbrace}--\xrefn{\Uparrow}.
Ils sont list\'es \`a un seul endroit sur la \knuth{page~146}{170--171}.

Un d\'elimiteur peut appartenir \`a n'importe quelle classe.
^^{classe//d'un d\'elimiteur}
Pour un d\'elimiteur que vous agrandissez avec
|\bigl|, |\bigr|, etc., la classe est d\'etermin\'ee par la commande~:
``opener'' pour les commandes-|l|, ``closer'' pour les commandes-|r|, 
``relation'' pour les commandes-|m|, et ``ordinary symbol'' pour les 
commandes-|g|,
c'est-\`a-dire, |\Big|.

Vous  pouvez obtenir un d\'elimiteur de deux diff\'erentes mani\`eres~:
\olist
\li Vous pouvez rendre un caract\`ere d\'elimiteur en lui assignant un
codes de d\'elimiteur non n\'egatif
\bix^^{codes d\'elimiteurs}
(voir plus bas) avec la commande ^|\delcode| (\xref\delcode).
N\'eanmoins le caract\`ere n'agit comme d\'elimiteur que si vous l'utilisez dans un
contexte de d\'elimiteur\footnote{%
Il est possible d'utiliser un caract\`ere avec un code de d\'elimiteur non n\'egatif 
dans un contexte o\`u il n'est pas un d\'elimiteur. Dans ce cas \TeX\ ne fait pas
la recherche~; \`a la place, il utilise simplement le caract\`ere de fa\c con ordinaire
(voir la \knuth{page~156}{181--282}).}.
\li Vous pouvez produire un d\'elimiteur explicitement avec la commande 
^|\delimiter| 
(\xref\delimiter), en analogie avec la mani\`ere par laquelle vous pouvez produire un 
caract\`ere ordinaire avec la commande |\char| ou un caract\`ere math\'ematique 
avec la commande |\mathchar|.
La commande |\delimiter| utilise les m\^emes code de d\'elimiteur que ceux utilis\'es
dans une entr\'ee de table
|\delcode|, mais avec un chiffre suppl\'ementaire devant pour indiquer une 
classe.
Il est rare d'utiliser |\delimiter| en dehors d'une d\'efinition de macro.
\endolist

Un code de d\'elimiteur dit \`a
\TeX\ comment chercher un caract\`ere de sortie appropri\'e pour repr\'esenter
un d\'elimiteur.
Les r\`egles de cette recherche sont assez compliqu\'ees
(voir les \knuth{pages~156 et 442}{181--282 et 506--507}).
Une compr\'ehension compl\`ete de ces r\`egles requiert des connaissances
sur l'organisation des ^{fichiers de m\'etriques} des polices, un sujet qui n'est pas seulement en dehors
de l'objectif de ce livre mais \'egalement en dehors de l'objectif de \texbook.

En r\'esum\'e la recherche travaille comme cela.  Le code de d\'elimiteur sp\'ecifie un
``petit'' caract\`ere de sortie et un ``grand'' caract\`ere de sortie en 
fournissant une position dans la \refterm{police} et une \refterm{famille} de 
police pour chacun
(see \xref\delcode).
En utilisant cette information, \TeX\ peut trouver (ou construire)
de plus en plus grandes versions du d\'elimiteur.  \TeX\ essaye d'abord
diff\'erentes tailles (de la petite \`a la grande) 
du ``petit'' caract\`ere dans la ``petite'' police
et ensuite
diff\'erentes tailles (l\`a aussi de la petite \`a la grande)
du ``grand'' caract\`ere dans la ``grande'' police,
en cherchant un dont la hauteur plus la profondeur soit suffisamment grande.
Si aucun de ces caract\`eres trouv\'e n'est assez
grand, il utilise le plus grand.
Il est possible que
le petit caract\`ere, le grand caract\`ere, ou les deux ait \'et\'e laiss\'e non 
sp\'ecifi\'e
(indiqu\'e par un z\'ero dans la partie appropri\'ee du code de d\'elimiteur).
Si seulement un caract\`ere
a \'et\'e sp\'ecifi\'e, \TeX\ l'utilise. Si aucun ne l'a \'et\'e, 
il remplace le d\'elimiteur par un espace de largeur ^|\nulldelimiterspace|.
\eix^^{codes d\'elimiteurs}

\endconcept


\concept d\'em\'erites

\TeX\ utilise des \refterm{d\'em\'erites} comme une mesure de combien une ligne n'est pas d\'esirable
quand il coupe un paragraphe en lignes \seeconcept{coupure de ligne}.
^^{coupures de ligne//d\'em\'erites pour}
Les d\'em\'erites d'une ligne sont affect\'es par la \refterm{m\'ediocrit\'e} de la ligne
et par les \refterm{p\'enalit\'es:p\'enalit\'e} associ\'ees avec la ligne.
^^{m\'ediocrit\'e}
Le but de \TeX\ dans le choix d'un arrangement particulier de lignes est de 
minimiser le total des
d\'em\'erites pour le paragraphe, qu'il calcule en additionnant les d\'em\'erites
des lignes individuelles.
voir les \knuth{pages~97--98}{114--115} pour les d\'etails sur comment \TeX\
coupe un paragraphe en lignes.  
\TeX\ n'utilise pas de d\'em\'erites quand il choisit des coupures de page~; 
\`a la place, il utilise
une mesure similaire appel\'ee le ``co\^ut'' d'une coupure de page particuli\`ere.
\endconcept

\conceptindex{\elements}
\concept \'el\'ement

Le terme \defterm{\'el\'ement} est souvent utilis\'e pour faire r\'ef\'erence \`a un composant 
d'une liste horizontale, verticale ou math\'ematique, c'est-\`a-dire, une liste d'\'el\'ements que
\TeX\ construit quand il est dans un mode horizontal, vertical ou math\'ematique.
\endconcept


\concept {\elementextra}

Un \defterm{\'el\'ement extraordinaire} est un \'el\'ement d'information qui 
dit \`a \TeX\ d'effectuer une action
qui n'entre pas dans le sch\'ema ordinaire des choses.
un \'el\'ement extraordinaire peut appara\^\i tre dans une liste horizontale ou 
verticale, comme une bo\^\i te ou un \'el\'ement ressort.
\TeX\ compose un \'el\'ement extraordinaire
comme une \refterm{bo\^\i te} de largeur, hauteur et profondeur \`a z\'ero---en d'autres
termes, une bo\^\i te qui ne contient rien et n'occupe pas d'espace.

Trois sortes d'\'el\'ements extraordinaires sont construits dans \TeX~: 
\ulist
\li Les commandes |\openout|, |\closeout| et |\write|
(\p\xrefn{\openout})
% (2nd) removed \xref to \write, since it's on the same page
produisent un \'el\'ement extraordinaire pour agir sur un fichier de sortie.
^^|\openout//\elementextra\ produit par|
^^|\write//\elementextra\ produit par|
^^|\closeout//\elementextra\ produit par|
\TeX\ diff\`ere l'op\'eration jusqu'a ce qu'il envoit la prochaine page
dans le {\dvifile}
^^{\dvifile//\elementextra\ en}
(\`a moins que l'op\'eration soit pr\'ec\'ed\'ee par ^|\immediate|).
\TeX\ utilise un \'el\'ement extraordinaire pour ces commandes parce qu'elles
 n'ont rien \`a faire avec ce qu'il compose quand il les rencontre.
\li La commande ^|\special| \ctsref{\special} demande \`a \TeX\ 
d'ins\'erer un certain texte directement dans le \dvifile.
comme avec la commande |\write|, \TeX\ diff\`ere l'op\'eration jusqu'a ce 
qu'il envoit la prochaine page dans le {\dvifile}
^^{\dvifile//mat\'eriel ins\'er\'e par \b\tt\\special\e}
Une utilisation typique de |\special| serait de 
nommer un fichier graphique que le driver d'affichage pourrait incorporer dans
votre sortie finale.
\li Quand vous changez de langage avec les commandes ^|\language| ou ^|\setlanguage|
\ctsref{\language},
\TeX\ ins\`ere un \'el\'ement extraordinaire qui l'instruit de l'usage d'un
certain jeu de r\`egle de c\'esure \`a suivre quand il coupe un paragraphe
en lignes.
\endulist
\noindent
Une impl\'ementation particuli\`ere de \TeX\ peut procurer des \'el\'ements extraordinaires additionnels.
\endconcept


\conceptindex{ent\^etes}
\concept ent\^ete

Un \defterm{ent\^ete} est un mat\'eriel que \TeX\ met en haut de chaque page,
au-dessus du texte de cette page. L'ent\^ete pour un simple rapport
peut \^etre constitu\'e du titre du cot\'e gauche de la page et du 
texte ``Page $n$'' du cot\'e droite de la page.
D'habitude, un ent\^ete est constitu\'e d'une seule ligne, que vous pouvez mettre en
assignant une liste de tokens \`a ^|\headline| \ctsref\headline.
L'ent\^ete par d\'efaut de \plainTeX\  est \`a blanc.
Il est aussi possible de produire des ent\^ete multiligne~ (voir \xrefpg{bighead}) 
\endconcept


\concept espace

Vous pouvez demander \`a \TeX\ de mettre un \defterm{espace} entre deux \'el\'ements de 
plusieurs fa\c cons~: 

\olist
^^{fin de ligne}
\li Vous pouvez \'ecrire quelque chose que \TeX\ traite comme un \refterm{token}
espace~: un ou plusieurs caract\`eres blancs, La fin d'une ligne (le
caract\`ere fin de ligne agit comme un espace) ou toute \refterm{commande} qui
se d\'eveloppe en token espace.  \TeX\ g\'en\'eralement traite plusieurs 
espaces cons\'ecutifs comme \'equivalent \`a un seul, en incluant le cas o\`u
les espaces comprennent une seule fin de ligne.  (Une ligne vide
indique la fin d'un paragraphe~; il fait que \TeX\ g\'en\`ere un token |\par|.)
^^|\par//d'une ligne vide|
\TeX\ ajuste la taille de cette sorte d'espace en fonction de la longueur
requise par le contexte.

^^{ressort//cr\'eer de l'espace avec}
\li Vous pouvez \'ecrire une commande de saut qui produit le ressort
que vous sp\'ecifiez dans la commande.  Le ressort peut
s'\refterm{\'etirer} ou se \refterm{r\'etr\'ecir},
produisant plus ou moins d'espace.  Vous pouvez avoir un ressort vertical 
comme un ressort horizontal.  Le ressort dispara\^\i t \`a chaque fois qu'il
est apr\`es une coupure de ligne ou de page.

^^{cr\'enages//cr\'eer de l'espace avec}
\li Vous pouvez \'ecrire un \refterm{cr\'enage}.  Un cr\'enage produit un montant
 d'espace fixe qui ne s'\'etire ni se r\'etr\'ecit et ne dispara\^\i t pas sur un
une coupure de ligne ou de page (\`a moins d'\^etre suivie imm\'ediatement par 
ressort). L'usage le plus courant du cr\'enage est d'\'etablir une relation
spatiale fixe entre deux \refterm{bo\^\i te}s adjacentes.
\endolist

Le ressort et les cr\'enages peuvent avoir des valeurs n\'egatives.  
Un ressort n\'egatif ou un cr\'enage n\'egatif
entre des \'el\'ements adjacents rapproche ces \'el\'ements l'un de l'autre.
\endconcept


\concept {\etirement}

Voir \conceptcit{ressort}.
\endconcept


\concept famille

Une \defterm{famille} est un groupe de trois \refterm{polices:police} reli\'ees utilis\'ees
quand \TeX\ est en \refterm{mode math\'ematique}.
^^{polices//familles de}
En dehors du mode math\'ematique, les familles
n'ont aucun effet.  Les trois polices d'une famille sont utilis\'ees pour les symboles normaux
(\refterm{taille de texte}), indices et exposants (\refterm{taille script}) 
et sous-indices, sur-exposants, etc.\
(\refterm{taille scriptscript}).
^^{taille texte}
^^{taille script}
^^{taille scriptscript}
Par exemple, le nombre `|2|' compos\'e dans ces trois polices vous donnera 
`$2$', `$\scriptstyle 2$' et `$\scriptscriptstyle 2$' (en \plainTeX). 
Normalement vous fixerez les trois polices d'une famille \`a diff\'erentes 
tailles du m\^eme type de police, mais rien ne vous emp\^eche d'utiliser des 
types diff\'erents pour les trois polices ni d'utiliser la m\^eme police 
deux fois dans une famille.

{\tighten
TeX\ procure jusqu'a seize familles, num\'erot\'ees de $0$ \`a $15$. Par
exemple, la famille $0$ en \refterm{\plainTeX} consiste en roman $10$ points
pour le texte, roman $7$ points pour le script et roman $5$ points pour le 
scriptscript.
^^{\plainTeX//familles de police en}
\PlainTeX\ d\'efini aussi la famille $1$ qui est constitu\'ee des polices 
math\'ematiques italiques et r\'eserve les familles $2$ et $3$ pour respectivement
les ^{symboles sp\'eciaux} et les ^{extensions math\'ematiques}\footnote{Les 
familles $2$ et $3$ sont sp\'eciales en ce que leurs fichiers de m\'etriques de 
police doivent inclure des param\`etres pour l'espacement math\'ematique.}. Si 
vous devez d\'efinir une famille vous m\^eme, vous devrez utiliser la commande
^|\newfam| \ctsref{\@newfam} pour avoir le num\'ero d'une famille qui n'est
pas utilis\'ee et les commandes ^|\textfont|, ^|\scriptfont| and 
^|\scriptscriptfont| commands \ctsref{\scriptscriptfont} pour assigner des
polices \`a cette famille.
\par}

\endconcept


\conceptindex{fichiers}
\concept fichier

Un \defterm{fichier} est un flot d'information que \TeX\ interpr\`ete ou
cr\'ee.  Des fichiers sont g\'er\'es par le ^{syst\`eme d'exploitation} qui supervise votre
ex\'ecution de \TeX.  \TeX\ manipule des fichiers dans quatre contextes diff\'erents~:
\olist
\li Un ``^{fichier source}'' est celui que \TeX\ lit avec ses ``yeux''
\seeconcept{\anatomy} et interpr\`ete en accord avec ses r\`egles ordinaires.
Votre fichier d'entr\'ee primaire---celui que vous sp\'ecifiez apr\`es `|**|' ou
sur la ligne de commande quand vous
invoquez \TeX---est un fichier source, et ainsi pour tout fichier que vous
appelez avec une commande ^|\input| \ctsref \input.

\li Un ``^{fichier r\'esultat}'' est celui qui contient les r\'esultats de 
l'ex\'ecution de \TeX.  \TeX\ cr\'ee deux fichiers r\'esultat~: Le
\dvifile\ et le fichier .log.
^^{\dvifile//comme un fichier r\'esultat}
^^{\logfile//comme un fichier r\'esultat}
Le \dvifile\ contient l'information n\'ecessaire pour imprimer votre document.
Le fichier
.log contient un enregistrement de ce qui s'est pass\'e pendant l'ex\'ecution, 
incluant tout message d'erreur que \TeX\ a g\'en\'er\'e.
Si votre fichier source primaire s'appelle
|screed.tex|, vos \dvifiles\ et log se nommeront |screed.dvi|
et |screed.log|\footnote{C'est la convention usuelle, mais 
des impl\'ementations particuli\`eres de \TeX\ sont libres de changer cela.}.

\li Pour lire un fichier avec la commande ^|\read|
\ctsref{\read} vous devez associer le fichier avec un flot d'entr\'ee.
^^{flots d'entr\'ee//lire avec \b\tt\\read\e}
Vous pouvez avoir jusqu'a $16$ flots d'entr\'ee actifs 
\`a la fois, num\'erot\'es de $0$ \`a $15$. 
La commande |\read| lit un seul fichier et le fait avec la valeur d'une
\refterm{s\'equence de contr\^ole} d\'esign\'ee, donc lire avec 
|\read| est tr\`es diff\'erent de lire avec ^|\input| (qui apporte un
fichier entier).
\TeX\ prend tout flot d'entre qui n'est pas num\'erot\'e entre
$0$ et $15$ comme r\'ef\'erence au terminal,
donc `|\read16|', disons, lit la prochaine ligne que vous saisissez sur le 
terminal.

\li Pour \'ecrire dans un fichier avec la commande |\write|
\ctsref \write\ vous devez associer le fichier
avec un flot de sortie.
^^|\write//flot d'entr\'ee pour|
^^{flots de sortie}
Vous pouvez avoir jusqu'a $16$ flots de sortie actifs
\`a la fois, num\'erot\'es de $0$ \`a $15$.  
Les flots d'entr\'ee et de sortie sont ind\'ependants.
Tout ce qui est envoy\'e vers un flot de sortie avec un num\'ero n\'egatif va vers le
fichier log~; tout ce qui est envoy\'e vers un flot de sortie avec un num\'ero 
sup\'erieur \`a $15$
va \`a la fois vers le fichier log et le terminal.
Ainsi `|\write16|', disons, \'ecrit une ligne sur le terminal et aussi envoit
cette ligne vers le fichier~log.

\endolist

Vous devez ouvrir un fichier de flot avant de pouvoir l'utiliser.
Un fichier de flot d'entr\'ee est ouvert avec une commande ^|\openin|
\ctsref \openin\ et un fichier de flot de sortie est ouvert avec une
commande ^|\openout| \ctsref\openout.
Par propret\'e,
vous devez fermer un fichier de flot quand vous en avez fini avec lui, n\'eanmoins
\TeX\ le fera \`a la fin du traitement si vous ne l'avez pas fait.
Les deux commandes pour fermer un fichier de flot sont ^|\closein| \ctsref\closein\
et ^|\closeout| \ctsref\closeout.
Un avantage de fermer un flot quand vous en avez fini avec lui
est que vous pouvez r\'eutiliser le flot pour un autre fichier.
Ceci peut \^etre essentiel quand vous lisez un longue suite de fichiers.

De plus vous pouvez assigner des num\'eros vous-m\^eme aux flots d'entr\'ee et de sortie,
Il est pr\'ef\'erable de le faire avec les commandes ^|\newread| et
^|\newwrite| \ctsref{\@newwrite}.
Vous pouvez avoir plus d'un flot associ\'e avec un seul fichier,
mais vous aurez des salet\'es (probablement non diagnostiqu\'e) \`a moins que tous les flots
soit des flots d'entr\'ee.  Associer plus d'un flot avec un fichier d'entr\'ee
peut-\^etre utile quand vous voulez utiliser le m\^eme  fichier d'entr\'ee pour deux 
usages diff\'erents.

D'habitude, \TeX\ diff\`ere les actions d'ouverture, d'\'ecriture et de fermeture
d'un flot de sortie jusqu'a ce qu'il enregistre une page avec ^|\shipout|
(voir la \knuth{page~227}{266}
pour les d\'etails).  Cette propri\'et\'e s'applique m\^eme aux messages \'ecrits sur le
terminal avec |\write|.  Mais vous pouvez demander \`a \TeX\ de faire une action
sur un flot de sortie imm\'ediatement en faisant pr\'ec\'eder la commande d'action de
^|\immediate| \ctsref\immediate.  Par exemple~:
\csdisplay
\immediate\write16{Do not pass GO!! Do not collect $200!!}
|
\endconcept


\concept {fichier format}

{\tighten
Un \defterm{fichier format} est un fichier qui contient une image de
la m\'emoire de \TeX\ sous la forme dans laquelle il peut \^etre recharg\'e rapidement.
Un fichier format peut \^etre cr\'e\'e avec la commande ^|\dump| \ctsref\dump.
L'image contient un enregistrement complet
des d\'efinitions (de \refterm{polices:police}, \refterm{macros:macro}, etc.)
qui \'etaient pr\'esentes quand le dump a \'et\'e effectu\'e.
En utilisant ^|virtex|, un forme sp\'eciale ``vierge'' de \TeX,
vous pouvez recharger le fichier format \`a haute vitesse et 
continuer avec le m\^eme \'etat dans lequel \'etait \TeX\ au moment du dump.
L'avantage du fichier format sur un fichier d' entr\'ee ordinaire
contenant la m\^eme information est que \TeX\ peut la charger beaucoup plus
vite.
\par}

Des fichiers de format doivent \^etre cr\'e\'es par ^|initex|, une  
forme sp\'eciale de \TeX\ \'ecrite \`a cette intention.
Ni |virtex|, ni |initex| n'ont d'outils autres que les primitives 
construites dans le programme \TeX\ lui-m\^eme.

Une forme ^{pr\'echarg\'ee} de \TeX\ est celle qui a un fichier format conforme
charg\'e et est pr\^et \`a accepter l'entr\'ee de l'utilisateur.
La forme de \TeX\ qui se nomme |tex|
a souvent les d\'efinitions \plainTeX\ pr\'echarg\'ees.
(\PlainTeX\ est normalement fourni sous deux autres formes~:
comme une fichier de format et comme une fichier source de \TeX.
Dans certains environnements, |tex| est \'equivalent \`a appeler |virtex|
et charge alors |plain|.)
Cr\'eer des formes pr\'echarg\'ees de \TeX\ n\'ecessite un programme sp\'ecial~;
il ne peut pas \^etre fait en n'utilisant que les outils de \TeX\ lui-m\^eme.

\endconcept


\concept {fichier log}

Voir \conceptcit{fichier}.
\endconcept


\conceptindex{filets}
\concept filets

Un \defterm{filet} est un rectangle plein noir.
Un filet, comme une \refterm{bo\^\i te},
a une \refterm{largeur}, une \refterm{hauteur} et une \refterm{profondeur}.
La dimension verticale du rectangle
est la somme de sa hauteur et de sa profondeur.
Une ligne droite horizontale ou verticale est un cas sp\'ecial de filet.
 
\bix^^{filets horizontaux}
\bix^^{filets verticaux}
\bix^^|\hrule|
\bix^^|\vrule|
Un filet peut \^etre soit horizontal, soit vertical.  La distinction entre un
filet horizontal et un vertical diff\`ere selon la fa\c con de produire le
filet, un filet vertical peut \^etre court et \'epais (et donc ressembler
\`a une ligne horizontale), tandis qu'un filet horizontal peut \^etre grand et maigre
(et donc ressembler \`a une ligne verticale).  La notion de filet de \TeX\ est
plus g\'en\'erale que celle des typographes, qui pensent \`a un filet comme \`a une ligne
et ne veulent habituellement pas appeler filet une bo\^\i te noire carr\'ee.
 
Vous pouvez produire un filet horizontal en utilisant la commande
|\hrule| et un filet vertical en utilisant
la commande |\vrule| \ctsref{\vrule}.
Par exemple, la s\'equence de contr\^ole |\hrule| par elle-m\^eme
produit un filet fin qui traverse la page, comme ceci~:

{\offinterlineskip
\nobreak\medskip
\hrule
\medskip}

La commande `|\vrule height .25in|' produit un filet vertical
qui s'allonge de $.25$~pouces sur la page comme ceci~:
\nobreak\vskip \abovedisplayskip
\leftline{\vrule height .25in}
\vskip \belowdisplayskip

Il y a deux diff\'erences entre les filets horizontaux et verticaux~:
\olist
\li Pour un filet horizontal, \TeX\ prend comme largeur par d\'efaut la largeur 
de la plus petite \refterm{bo\^\i te} ou \refterm{alignement} qui l'englobe.  
Pour un filet vertical, \TeX\ prend comme hauteur et profondeur par d\'efaut 
de la m\^eme fa\c con.  (Le
d\'efaut est la taille que vous obtenez si vous ne donnez pas de taille explicitement 
pour cette dimension.)

^^{listes horizontales//filet en}
^^{listes verticales//filet en}
\li 
{\tighten
Un filet horizontal est un \'el\'ement fondamentalement vertical qui ne peut participer 
\`a une \refterm{liste horizontale}, tandis qu'un filet vertical est un \'el\'ement
horizontal qui ne peut participer \`a une \refterm{liste verticale}.  
Cette propri\'et\'e peut sembler \'etrange d'un premier abord, mais il y a une
bonne raison \`a cela~:
Un filet horizontal s'\'etend visuellement de gauche \`a droite et donc s\'epare des
\'el\'ements d'une liste verticale, tandis qu'un filet vertical s'\'etend visuellement 
de haut en bas et ainsi s\'epare des \'el\'ements d'une liste horizontale.
%(Look at the rules that are shown above.)
\par}
\endolist

{\tighten
Si vous construisez un filet avec trois dimensions explicites, 
cela donnera la m\^eme chose que vous fassiez un filet horizontal ou vertical.
Par exemple, la commande `|\vrule height1pt depth2pt width3in|' produit ce
filet au look horizontal~:
\par}

{\offinterlineskip
\nobreak\medskip\nobreak\vskip3pt
\leftline{\vrule height1pt depth2pt width3in}
\medskip}

Vous trouverez un \'etat pr\'ecis du traitement des filets de \TeX\  dans les
\knuth{pages~221--222}{259--260}.
\eix^^{filets horizontaux}
\eix^^{filets verticaux}
\eix^^|\hrule|
\eix^^|\vrule|
\endconcept


\conceptindex{flots d'entr\'ee}
\concept {flots d'entr\'ee}

Voir \conceptcit{fichier}.
\endconcept


\conceptindex{flots de sortie}
\concept {flots de sortie}

\margin{This concept was out of order.}
Voir \conceptcit{fichier}.
\endconcept


\concept global

Une d\'efinition \defterm{globale} est effective 
jusqu'a la fin du
document ou jusqu'a ce qu'elle soit \'ecras\'ee par une autre d\'efinition,
m\^eme quand elle appara\^\i t dans un \refterm{groupe}.  
Ainsi une d\'efinition globale n'est pas affect\'ee par les fronti\`eres de groupe.
Vous pouvez rendre n'importe quelle d\'efinition globale en la pr\'efixant avec la
commande |\global| \ctsref{\global} a moins que ^|\globaldefs| \ctsref\globaldefs{}
soit n\'egatif.
^^|\global|

Il y a un moyen sp\'ecial de rendre une d\'efinition de \refterm{macro} globale.
^^{macros//globale}
Normalement vous d\'efinissez une macro en utilisant soit la commande |\def| 
soit la commande |\edef| \ctsref{\edef}.
^^|\edef//rendre global|
^^|\def//rendre global|
Si vous utilisez |\gdef| ou |\xdef|
au lieu de |\def| et |\edef| respectivement, la d\'efinition de macro sera
globale.  Parce que, `^|\gdef|' est \'equivalent \`a `|\global\def|' et
`^|\xdef|' est \'equivalent \`a `|\global\edef|'.
\endconcept


\conceptindex{groupes}
\concept groupe

Un \defterm{groupe} est une partie de votre manuscrit que \TeX\ traite comme
une unit\'e.
Vous indiquez un groupe en l'entourant avec les accolades
`|{|' et `|}|' (ou tout autre caract\`ere avec le 
\refterm{code de cat\'egorie} appropri\'e).
^^|{//d\'ebuter un groupe|
^^|}//terminer un groupe|

La plus importante propri\'et\'e d'un groupe est que toute d\'efinition
ou assignation non globale que vous faites dans un groupe dispara\^\i t quand
le groupe se termine.  Par exemple, si vous \'ecrivez~:

\csdisplay
Please don't pour {\it any} more tea into my hat.
|
La \refterm{s\'equence de contr\^ole} |\it| demande \`a \TeX\ de mettre le mot
`|any|' en police italique mais n'affecte pas le reste du texte.
Comme autre exemple, si vous utilisez le param\`etre |\hsize| 
\ctsref{\hsize} pour changer la longueur de la ligne dans un groupe, la longueur de ligne
retourne \`a sa valeur pr\'ec\'edente un fois que \TeX\ est sorti du groupe.

Les groupes sont aussi pratiques comme moyen de contr\^oler l'espace\-ment.  
Par exemple, si vous saisissez~:

\csdisplay
\TeX for the Impatient and the Outpatient too.
|
\noindent
vous aurez~:
\display{%
\TeX for the Impatient and the Outpatient too.
}
\noindent
puisque la s\'equence de  contr\^ole  |\TeX| (qui produit le logo \TeX)
absorbe l'espace qui le suit. 
vous voulez probablement ceci~:
\display{%
{\TeX} for the Impatient and the Outpatient too.
}
\noindent
Une mani\`ere d'obtenir ceci est d'englober `|\TeX|' dans un groupe~:
\csdisplay
{\TeX} for the Impatient and the Outpatient too.
|
L'accolade droite emp\^eche la s\'equence de  contr\^ole  d'absorber l'espace.
\endconcept


\concept hauteur

La \defterm{hauteur} d'une \refterm{bo\^\i te} est la distance \`a laquelle la bo\^\i te
s'\'etend au dessus de sa \refterm{ligne de base}.
^^{\boites//hauteur de}
\endconcept


\conceptindex{hbox}
\concept hbox

Une \defterm{hbox} (bo\^\i te horizontale) est une \refterm{bo\^\i te} que \TeX\
construit en plaquant les articles d'une \refterm{liste horizontale} l'un 
apr\`es l'autre, de la gauche vers la droite.
^^{listes horizontales//hbox form\'es \`a partir de}
Une hbox, prise comme unit\'e, n'est ni
fondamentalement horizontale ni fondamentalement verticale, c'est-\`a-dire, 
qu'elle peut appara\^\i tre comme un
article soit d'une liste horizontale soit d'une \refterm{liste verticale}. 
Vous pouvez construire une hbox avec la commande |\hbox| \ctsref{\hbox}.
\endconcept


\conceptindex{insertions}
\concept insertion

\looseness = -1 
Une \defterm{insertion} est une liste verticale contenant du mat\'eriel devant
\^etre ins\'er\'e sur une page quand \TeX\ a fini de construire cette page\footnote
{\tighten 
\TeX\ lui-m\^eme n'ins\`ere pas le mat\'eriel---il rend juste le mat\'eriel accessible
\`a la routine de sortie, qui est alors responsable de la transf\'erer sur la
page compos\'ee.
^^{routine de sortie//insertions, traitement de}
Le seul effet imm\'ediat de la commande ^|\insert| 
\ctsref{\insert} est de changer les calculs de coupure de page de \TeX\ pour
qu'il laisse de la place sur la page pour le mat\'eriel \`a ins\'erer. Plus tard,
quand \TeX\ coupera r\'eellement la page, il divisera le mat\'eriel \`a ins\'erer en
deux groupes~: le mat\'eriel qui passe sur la page courante et le mat\'eriel qui
ne passe pas.
^^{coupures de page//insertions sur des}
Le mat\'eriel qui passe sur la page est plac\'e dans des registres de bo\^\i te, un
par insertion et le mat\'eriel qui ne passe pas est transf\'er\'e sur la page
suivante.
Cette proc\'edure autorise \TeX\ \`a faire des choses comme distribuer des parties 
d'une longue note de pied de page sur plusieurs pages cons\'ecutives.}. Des
exemples de telles insertions sont les notes de pied de page et les figures. 
Les commandes \refterm{\plainTeX} pour cr\'eer des insertions sont 
^|\footnote|, ^|\topinsert|, |\mid!-insert|,
^^|\midinsert|
et ^|\pageinsert|, aussi bien que la commande primitive ^|\insert|
elle-m\^eme (\pp\xrefn\footnote--\xrefn{endofinsert}).
Le m\'ecanisme de \TeX\ pour manipuler des insertions est plus compliqu\'e~;
voir les \knuth{pages~122--125}{142--146} pour les d\'etails.
\endconcept


\concept largeur

^^{\boites//largeur de}
La \defterm{largeur} d'une \refterm{bo\^\i te} est le montant d'espace horizontal
qu'il occupe, c'est-\`a-dire, la distance de son cot\'e gauche \`a son cot\'e droit.
Le mat\'eriel compos\'e dans une bo\^\i te peut \^etre plus large que la bo\^\i te elle-m\^eme.
\endconcept



\conceptindex{ligatures}
\concept ligature

Une \defterm{ligature} est un simple caract\`ere qui remplace une
s\'e\-quence particuli\`ere de caract\`eres adjacents dans un document compos\'e.
Par exemple, le mot `|office|' est compos\'e \hbox{``office''} et non 
\hbox{``of{f}ice''}, par les syst\`emes de composition de haute qualit\'e.
La connaissance des ligatures est construite dans les \refterm{polices:police}
que vous utilisez, donc vous n'avez rien \`a faire pour que \TeX\
les produise. (Vous pouvez emp\^echer la ligature d'``office'', comme nous
l'avons dit plus t\^ot, en saisissant `|of{f}ice|' dans votre entr\'ee.)
\TeX\ est aussi capable d'utiliser son m\'ecanisme de ligature pour composer
la premi\`ere ou la derni\`ere lettre d'un mot diff\'eremment de la m\^eme lettre
qui appara\^\i t au milieu d'un mot
Vous pouvez emp\^echer cet effet (si vous le rencontrez) en utilisant la
commande ^|\noboundary| (\xref\noboundary).

Parfois, vous aurez besoin d'une ligature d'une langue europ\'eenne.
^^{langues europ\'eennes}
\TeX\ ne les produit pas automatiquement sauf si vous utilisez une police
dessin\'ee pour cette langue. Un certain nombre de ces ligatures, par exemple,
\AE, sont accessible par une commande
(voir ``Lettres et ligatures pour alphabets europ\'eens'',
\xref{fornlets}).
\endconcept


\conceptindex{lignes de base}
\concept ligne de base

La \defterm{ligne de base} d'une \refterm{bo\^\i te} est une ligne imaginaire qui traverse
la bo\^\i te.
^^{\boites//lignes de base de}
Quand \TeX\ 
assemble les bo\^\i tes d'une \refterm{liste horizontale} en  une bo\^\i te plus grande, 
il aligne les bo\^\i tes de la liste pour que leurs lignes de base co\"\i ncident.
Par analogie, pensez que vous \'ecrivez sur un bloc de papier lign\'e.  Chaque lettre 
que vous \'ecrivez a
une ligne de base implicite.
Pour aligner les lettres horizontalement,
vous les placez sur le bloc pour que leurs lignes de base
s'ajustent avec les lignes-guides claires imprim\'ees sur le bloc.

Une bo\^\i te peut et souvent doit s'\'etendre sous sa ligne de base.
Par exemple, la lettre `g' s'\'etend sous la ligne de base de sa bo\^\i te parce
qu'elle a une descendante (la boucle du dessous du `g').
\endconcept


\conceptindex{listes}
\concept liste

Une \defterm{liste} est une s\'equence d'\refterm{\'el\'ements:\'el\'ement}
(\refterm{bo\^\i tes:bo\^\i te}, \refterm{ressorts:ressort}, \refterm{cr\'enages:cr\'enage}, etc.)
cela comprend tout le contenu d'une hbox, d'une vbox
ou d'une formule math\'ematique.
Voir \conceptcit{liste horizontale}, \conceptcit{liste verticale}.

\endconcept


\conceptindex{listes horizontales}
\concept {liste horizontale}

Une \defterm{liste horizontale} est une liste d'articles 
que \TeX\ a produit quand il \'etait dans un de ses
\refterm{modes horizontaux:mode horizontal}, c'est-\`a-dire, en assemblant
soit un paragraphe soit une hbox.  Voir ``mode horizontal'' ci-dessous.
\endconcept


\conceptindex{listes verticales}
\concept {liste verticale}

Une \defterm{liste verticale} est une liste d'\'el\'ements que \TeX\ a produit quand il
\'etait dans une de ses modes vertical, c'est-\`a-dire, en assemblant soit une
\refterm{vbox} soit une page.  Vois ``mode vertical'' ci-dessous.

\endconcept


\conceptindex{macros}
\concept macro

{% Use a brace here so that definitions of explanatory macros remain local.
% The closing brace is at the end of the concept.
Une \defterm{macro} est une d\'efinition qui donne un nom \`a un mod\`ele de texte 
d'entr\'ee de \TeX\footnote{Plus pr\'ecis\'ement, la d\'efinition donne un nom
\`a une s\'equence de tokens.}. Le nom peut \^etre soit une \refterm{s\'equence de
contr\^ole} soit un \refterm{caract\`ere actif}.  Le mod\`ele est appel\'e le
``texte de remplacement''.  La commande primaire pour d\'efinir des macros est 
la s\'equence de contr\^ole |\def|.

\def\arctheta{\cos \theta + i \sin \theta}
Comme exemple simple, supposez que vous ayez un document dans lequel
la s\'equence `$\cos \theta + i \sin \theta$' appara\^\i t plusieurs fois.
Au lieu de l'\'ecrire \`a chaque fois, vous pouvez d\'efinir une macro pour cela~:
\csdisplay
\def\arctheta{\cos \theta + i \sin \theta}
|
Maintenant \`a chaque fois que vous avez besoin de cette s\'equence, vous pouvez 
simplement ``appeler'' la macro en \'ecrivant `|\arctheta|' et vous l'aurez.  
Par exemple, `|$e^{\arctheta}$|' vous donnera `$e^{\arctheta}$'.

\bix^^{macros//param\`etres de}
Mais la r\'eelle puissance des macros tient au fait qu'une macro peut avoir des
param\`etres.  Quand vous appelez une macro qui a des param\`etres, vous fournissez
des arguments qui se substituent \`a ces param\`etres.  Par exemple, supposez
que vous \'ecrivez~:
\pix\indexchar #
\def\arc#1{\cos #1 + i \sin #1}
\csdisplay
\def\arc#1{\cos #1 + i \sin #1}
|

La notation |#1| \xrdef{@msharp} d\'esigne le premier param\`etre
de la macro, qui dans ce cas n'a qu'un param\`etre.  Vous pouvez maintenant
produire une forme similaire, telle que `$\arc{2t}$', avec l'appel de macro 
`|\arc{2t}|'.

Plus g\'en\'eralement, une macro peut avoir jusqu'a neuf param\`etres, que vous
d\'esignez par `|#1|', `|#2|', etc\null. dans la d\'efinition de la macro. \TeX\
fourni deux types de param\`etres~: les param\`etres d\'elimit\'es et les param\`etres
non d\'elimit\'es. Bri\`evement, un param\`etre d\'elimit\'e a un \refterm{argument} qui
est d\'elimit\'e ou termin\'e par une s\'equence sp\'ecifique de tokens (le d\'elimiteur),
tandis qu'un param\`etre non d\'elimit\'e a un argument qui n'a pas besoin de
d\'elimiteur pour le terminer.
Premi\`erement, nous expliquerons comment fonctionnent les macros quand elles
n'ont que des param\`etres non d\'elimit\'es, et ensuite, nous expliquerons comment
elles fonctionnent quand elles ont des param\`etres d\'elimit\'es.

^^{param\`etres//non d\'elimit\'es}
Si une macro n'a que des param\`etres non d\'elimit\'es, ces param\`etres doivent appara\^\i tre
l'un apr\`es l'autre dans la d\'efinition de macro \emph{sans rien entre
eux ou entre le dernier param\`etre et l'accolade gauche au d\'ebut du 
texte de remplacement}.
Un appel d'une telle macro consiste en le nom de la macro suivi par
les arguments de l'appel, un pour chaque param\`etre.  Chaque argument est
soit~:

\ulist \compact
\li un \refterm{token} simple autre qu'un accolade gauche ou droite ou

\li une s\'equence de tokens englob\'e entre une accolade gauche et
une accolade droite correspondante\footnote{L'argument peut avoir des paires
d'accolades englobantes et chacune de ces paires peut d\'esigner soit 
un \refterm{groupe} soit un autre argument de macro.}.
\endulist

Quand \TeX\ rencontre une macro, il d\'eveloppe la macro dans son \oe so\-phage
\seeconcept{\anatomy}
en substituant chaque argument par le param\`etre correspondant
dans le texte de remplacement.  Le texte r\'esultant peut contenir d'autres 
appels de macro. Quand \TeX\ rencontre un tel appel de macro imbriqu\'e, il
d\'eveloppe cet appel imm\'ediatement sans regarder ce qui suit l'appel
\footnote{En terminologie informatique, le d\'eveloppement est ``profondeur
d'abord'' plut\^ot que ``largeur d'abord''.  Notez que vous pouvez modifier l'ordre
du d\'eveloppement avec des commandes telles que |\expandafter|.}. Quand 
l'\oesophage de \TeX\ tombe sur une \refterm{commande} \refterm{primitive} 
qui ne peut \^etre d\'evelopp\'ee plus loin, \TeX\ passe cette commande \`a l'estomac 
de \TeX. L'ordre de d\'eveloppement est parfois critique, donc de fa\c con \`a vous 
aider \`a le comprendre, nous vous donnons un exemple de \TeX\ au travail.

Supposez que vous procurez \`a \TeX\ l'entr\'ee suivante~:
\csdisplay
\def\a#1#2{\b#2#1\kern 2pt #1}
\def\b{bb}
\def\c{\char49 cc}
\def\d{dd}
\a\c{e\d} % Call on \a.
|
Alors l'argument correspondant \`a |#1| est |\c|,
et l'argument correspondant \`a |#2| est |e\d|.
\TeX\ d\'eveloppe l'appel de macro d'apr\`es les \'etapes suivantes~:

{\vskip\abovedisplayskip\obeylines % ugly
|\b e\d\c\kern 2pt \c|
|bbe\d\c\kern 2pt \c|
|\d\c\kern 2pt \c|\quad(`|b|', `|b|', `|e|' envoy\'es vers l'estomac)
|dd\c\kern 2pt \c|
|\c\kern 2pt \c|\quad(`|d|', `|d|' envoy\'es vers l'estomac)
|\char49 cc\kern 2pt \c|
|\c|\quad(`|\char|', `|4|', `|9|', `|c|', `|c|', %
`|\kern|', `|2|', `|p|', `|t|' envoy\'es vers l'estomac)
|\char49 cc|
(`|\char49|', `|c|', `|c|' envoy\'es vers l'estomac)
\vskip\belowdisplayskip}

\noindent Notez que les lettres `|b|', `|c|', `|d|' et `|e|' et les
s\'equences de contr\^ole `|\kern|' et `|\char|' sont toutes des commandes
primitives qui ne peuvent \^etre d\'evelopp\'ees.

\bix^^{param\`etres//d\'elimit\'es}
Une macro peut aussi avoir des ``param\`etres d\'elimit\'es'', qui peuvent \^etre m\'elang\'es avec
des non d\'elimit\'es dans toutes les combinaisons.  L'id\'ee d'un param\`etre 
d\'elimit\'e est que \TeX\ trouve l'argument correspondant en recherchant
une certaine s\'equence de tokens qui marquent la fin de l'argument---le
d\'elimiteur.  Cela fait, quand \TeX\ recherche un tel argument, il
prend comme argument tout les tokens \`a partir de la position courante de \TeX\ 
mais sans inclure le d\'elimiteur.

Vous indiquez un param\`etre d\'elimit\'e en \'ecrivant `|#|$n$' ($n$ 
doit \^etre entre $0$
et $9$) suivi par un ou plusieurs tokens qui agissent comme le d\'elimiteur.  
Le d\'elimiteur s'\'etend jusqu'au prochain `|#|' ou `|{|'---qui ont un sens
car `|#|' d\'ebute un autre param\`etre et `|{|' le texte de remplacement.

Le d\'elimiteur ne peut \^etre `|#|' ou `|{|', donc vous pouvez appeler un param\`etre
d\'elimit\'e \`a partir d'un non d\'elimit\'e en recherchant ce qui le suit. 

Si le caract\`ere apr\`es le param\`etre est `|#|' ou `|{|', vous avez un
param\`etre non d\'elimit\'e~; autrement il sera  d\'elimit\'e.  Notez
la diff\'erence dans les arguments pour  les deux sortes de param\`etres---un
param\`etre non d\'elimit\'e est d\'esign\'e soit par un simple token, soit par une
s\'equence de tokens englob\'es entre des accolades, tandis qu'un param\`etre
d\'elimit\'e est d\'esign\'e par n'importe quel nombre de tokens, m\^eme aucun.

Un exemple de macro utilisant deux param\`etres d\'elimit\'es est~:
\def\diet#1 #2.{On #1 we eat #2!}
\csdisplay
\def\diet#1 #2.{On #1 we eat #2!!}
|
Ici le premier param\`etre est d\'elimit\'e par un simple espace et le
second param\`etre est d\'elimit\'e par un point.  Si vous saisissez~:
\csdisplay
\diet Tuesday turnips.
|
vous obtiendrez le texte ``\diet Tuesday turnips.''.
Mais si les tokens d\'elimitants sont englob\'es dans un groupe, \TeX\ 
ne les consid\`ere pas d\'elimit\'es.  Donc si vous \'ecrivez~:
\csdisplay
\diet {Sunday mornings} pancakes.
|
vous obtiendrez le texte `\diet {Sunday mornings} pancakes.'
m\^eme s'il y a un espace entre `|Sunday|' et `|morning|'.
Quand vous utilisez un espace comme d\'elimiteur,
un caract\`ere fin de ligne d\'elimite aussi normalement l'argument
car \TeX\ convertit la fin de ligne en espace avant que le m\'ecanisme de
macro ne le voit.
\eix^^{param\`etres//d\'elimit\'es}
\eix^^{macros//param\`etres de}

Il arrivera que vous ayez \`a d\'efinir une macro qui ait `|#|' comme caract\`ere
significatif. Vous aurez plus de chance d'avoir besoin de faire cela quand 
vous d\'efinirez une macro qui \`a son tour d\'efinit une seconde macro. Comment
faire \`a propos des param\`etres de la seconde macro sans mettre \TeX\
dans la confusion~? La r\'eponse est que vous saisissez deux `|#|' pour chacun
de ceux que vous voulez quand la premi\`ere macro est d\'evelopp\'ee. Par exemple,
supposons que vous \'ecriviez la d\'efinition de macro~:
\def\first#1{\def\second##1{#1/##1}}
\csdisplay
\def\first#1{\def\second##1{#1/##1}}
|
Alors l'appel `|\first{One}|' d\'efini `|\second|' comme~:
\csdisplay
\def\second#1{One/#1}
|
et l'appel cons\'ecutif `|\second{Two}|' produit le texte
\def\second#1{One/#1}%
`\second {Two}'.

Un certain nombre de commandes fournissent des moyens suppl\'emen\-taires pour d\'efinir des macros
(voir pp.~\xrefn{mac1}--\xrefn{mac2}).
Pour les r\`egles com\-pl\`etes concernant les macros, voir le \knuth{chapitre~20}{}
}% close brace at the start of the `macro' concept.
\endconcept


\concept magnification

Quand \TeX\ compose un document, il multiplie toutes les dimensions 
par un facteur de
\refterm{magnification} $f/1000$,
o\`u $f$ est la valeur du param\`etre ^|\mag| \ctsref\mag.
Puisque la valeur par d\'efaut de |\mag| est $1000$, le cas normal est que
votre document est compos\'e comme il est sp\'ecifi\'e.
Augmenter la magnification est souvent utile quand vous composez un document
qui sera r\'eduit photographiquement.

Vous pouvez aussi appliquer de la magnification sur une seule \refterm{police} 
pour avoir une version plus petite ou plus grande de cette police que sa
``^{taille d'origine}''.  Vous
devez procurer au driver d'impression un ^{fichier de forme}
\seeconcept{police} pour 
chaque magnification de police que vous utilisez---%
a moins que la police ne soit construite dans votre imprimante et que votre driver
d'impression la connaisse.
Quand vous d\'efinissez une police avec la commande
|\font| \ctsref{\font}, vous pouvez sp\'ecifier une magnification avec
le mot `|scaled|'.  Par exemple~:

\csdisplay
\font\largerbold = cmbx10 scaled 2000
|
d\'efinit `|\largerbold|' comme une police qui est
deux fois plus grande que |cmbx10| (Computer Modern
Grasse Etendue en $10$ points) et a les formes de caract\`ere
uniform\'ement agrandies d'un facteur de~$2$.

De nombreux centres informatiques trouvent pratique de fournir des polices 
agrandies par un  ratio
de $1.2$, correspondant \`a des valeurs de magnification de $1200$, $1440$, etc.  
\TeX\ a des noms sp\'eciaux pour ces valeurs~:
^^|\magstep|
`|\magstep1|' pour $1200$,
`|\magstep2|' pour $1440$, et ainsi de suite jusqu'a `|\magstep5|'.  
La valeur sp\'eciale `^|\magstephalf|' correspond \`a une magnification de $\sqrt{1.2}$, qui
est visuellement \`a mi-chemin entre `|\magstep0|' (pas de magnification) et
`|\magstep1|'.  Par exemple~:

\csdisplay
\font\bigbold = cmbx10 scaled \magstephalf
|

Vous pouvez sp\'ecifier une \refterm{dimension} comme elle sera 
mesur\'ee dans le document final ind\'ependamment des magnifications en mettant
`^|true|' devant l'unit\'e.  Par exemple, `|\kern 8 true pt|' 
produit un cr\'enage de $8$ points quelque soit la magnification. 

\endconcept

\concept marges

Les \refterm{marges}
d'une page d\'efinissent un rectangle qui normalement
contient la mati\`ere imprim\'ee sur la page.
Vous pouvez demander \`a \TeX\ d'imprimer du mat\'eriel en dehors de ce rectangle,
mais uniquement en demandant des actions explicites qui d\'eplacent le mat\'eriel
\`a cet endroit. \TeX\ consid\`ere les ent\^etes et les pieds de page comme \'etant
en dehors des marges.

Le rectangle est d\'efini en fonction de son coin en haut \`a gauche, de sa largeur, et
de sa profondeur.  L'emplacement du coin en haut \`a gauche est d\'efini par
les param\`etres ^|\hoffset|
et ^|\voffset| 
\ctsref\voffset.  L'usage est de placer ce coin \`a un pouce du haut
et \`a un pouce du bord gauche de la page, correspondant \`a une valeur de
z\'ero pour |\hoffset| et |\voffset|%
\footnote{Il nous semble qu'il s'agisse d'une convention bizarre.
Il aurait \'et\'e plus naturel d'avoir le point $(0,0)$
pour |\hoffset| et |\voffset| dans le coin en haut \`a gauche du
papier et d'avoir leur valeur par d\'efaut \`a un pouce.}.
La largeur du rectangle est donn\'e par ^|\hsize| et la profondeur par ^|\vsize|.

Les implications de ces conventions sont~:
\ulist\compact
\li La marge gauche est donn\'ee par |\hoffset|\tplus|1in|.
\li La marge droite est donn\'ee par la largeur du papier moins
    |\hoffset|\tplus|1in|\tplus|\hsize|.
\li La marge du haut est donn\'ee par |\voffset|\tplus|1in|.
\li La marge du bas est donn\'ee par la longueur du papier moins
    |\voff!-set|\tplus|1in|\tplus|\vsize|.
\endulist
A partir de ces informations vous pouvez voir quels param\`etres vous devez 
changer pour modifier les marges.

Chaque changement que vous faites \`a |\hoffset|, |\voffset| ou |\vsize| prendra
effet la prochaine fois que \TeX\ d\'ebutera une page.  En d'autres termes, si vous 
les changez dans une page, le changement n'affectera que la page \emph{suivante}.
Si vous changez |\hsize|, le changement deviendra effectif imm\'ediatement.
\endconcept


\conceptindex{marques}
\concept marque

Une \defterm{marque} est un \'el\'ement que vous pouvez ins\'erer dans une
liste horizontale, verticale ou math\'ematique et plus tard la retrouver dans 
votre routine de sortie.  Les marques sont utiles pour des usages tels que
garder une trace des sujets apparents dans des ent\^etes de page.
Chaque marque a une liste de tokens---le ``^{texte de marque}''---associ\'e avec elle.
La commande ^|\mark| \ctsref{\mark} attend une telle liste de token 
comme son argument, et attend un \'el\'ement contenant cette liste de token (apr\`es
expansion) pour toute liste  que \TeX\ est en train de construire.
Les commandes ^|\topmark|, ^|\firstmark| et ^|\botmark| 
\ctsref{\topmark} peuvent \^etre utilis\'ees pour retrouver diverses marques sur une page.
Ces commandes sont le plus souvent utilis\'ees dans les ent\^etes et les pieds de page.
^^{pieds de page//marques utilis\'es en}
^^{ent\^etes//marques utilis\'es en}

\margin{This example of {\tt\\mark} replaces the previous explanatory
paragraph.}
Voici un exemple simplifi\'e.
Supposez que vous d\'efinissiez une macro d'ent\^ete de section comme suit~:
\csdisplay
\def\section#1{\medskip{\bf#1}\smallskip\mark{#1}}
% #1 is the name of the section
|
^^|\mark|
Cette macro, quand elle est appel\'ee, produira un ent\^ete de section en caract\`eres gras et
enregistrera aussi le nom de la section comme marque.
Vous pouvez maintenant d\'efinir l'ent\^ete de chaque page imprim\'ee
comme suit~:
\csdisplay
\headline = {\ifodd\pageno \hfil\botmark\quad\folio
   \else \folio\quad\firstmark\hfil \fi}
|
Chaque page paire (celle de gauche) aura maintenant le num\'ero de page suivi par 
le nom de la premi\`ere section de cette page, tandis que chaque page impaire (celle de droite)
aura le num\'ero de page suivi du nom de la derni\`ere section de cette page.
Les cas sp\'eciaux, par exemple, pas de section commen\c cant sur une page, 
s'afficheront g\'en\'eralement correctement gr\^ace \`a la mani\`ere dont travaillent
^|\firstmark| et ^|\botmark|.

Quand vous partagez une page en utilisant la commande |\vsplit| \ctsref{\vsplit} vous
pouvez retrouver les textes de la premi\`ere et de la derni\`ere marque de la 
portion non partag\'ee avec les commandes ^|\splitfirstmark| et ^|\splitbotmark| 
\ctsref{\splitfirstmark}.

Voir les \knuth{pages~258--260}{302--305} pour une explication plus pr\'ecise sur la
fa\c con de cr\'eer et retrouver des marques.
\endconcept


\conceptindex{mathcodes}
\concept mathcode

Un \defterm{mathcode} est un nombre que \TeX\ utilise pour identifier et
d\'ecrire un caract\`ere math\'ematique,
^^{caract\`eres math\'ematiques//d\'ecrits par mathcodes}
c'est-\`a-dire, un caract\`ere qui a un
r\^ole particulier dans une formule math\'ematique.  Un mathcode 
transmet trois sortes 
d'informations sur un caract\`ere~: sa position dans la \refterm{police}, sa
\refterm{famille} et sa \refterm{classe}.
Chacun des $256$ caract\`eres entr\'es possibles
a un mathcode, qui est d\'efini par le programme \TeX\
mais peut \^etre chang\'e.

^^{famille//comme partie de mathcode}
\TeX\ a seize familles de police, num\'erot\'ees $0$--$15$.  Chaque
famille con\-tient trois polices~: une pour la \refterm{taille texte}, une pour
la \refterm{taille script} et une pour la \refterm{taille scriptscript}.  \TeX\
choisi la taille d'un caract\`ere particulier et donc sa police,
en fonction du contexte.  La classe d'un caract\`ere sp\'ecifie son r\^ole
dans une formule (voir la \knuth{page~154}{180}).  Par exemple, le signe \'egal `|=|'
est dans la classe $3$ (relation).  \TeX\ utilise sa connaissance des classes 
de caract\`eres quand il d\'ecide combien d'espace mettre entre diff\'erents
composants d'une formule math\'ematique.

La meilleure fa\c con de comprendre ce que sont les mathcodes est de voir comment
\TeX\ les utilise. Nous montrons ce que \TeX\ fait avec un
token de caract\`ere $t$ de \refterm{code de cat\'egorie}~11 ou~12 dans une
formule math\'ematique~:

\olist\compact
\li Il recherche le mathcode du caract\`ere.
\li Il d\'etermine une famille $f$ \`a partir du mathcode.
\li Il d\'etermine la taille $s$ \`a partir du contexte.
\li Il s\'electionne une police $F$ en prenant la police de taille $s$ dans la famille $f$.
\li Il d\'etermine un num\'ero de caract\`ere $n$ \`a partir du mathcode.
\li Il s\'electionne comme caract\`ere $c$ devant \^etre compos\'e le caract\`ere
\`a la position $n$ de la police $F$.
\li Il ajuste l'espacement autour de $c$ en fonction de la classe de $t$ et
du contexte ambiant.
\li Il compose le caract\`ere $c$.
\endolist

La d\'ependance du contexte dans les \'etapes
(3) et (7) implique que \TeX\ ne peut pas composer un caract\`ere math\'ematique
tant qu'il n'a pas vu toute la formule contenant le
caract\`ere math\'ematique.  Par exemple, dans la formule
`|$a\over b$|', \TeX\ ne sait pas quelle taille aura le `|a|' tant qu'il 
n'aura pas vu le |\over|.

{\tighten
Le mathcode d'un caract\`ere est encod\'e en fonction de la formule $4096c
+ 256f + n$, o\`u $c$ est la classe du caract\`ere, $f$ sa
\refterm{famille} et $n$ son code de \refterm{caract\`ere \ascii} dans
la famille.  Vous pouvez changer l'interpr\'etation de \TeX\ pour un caract\`ere entr\'e
dans un mode math\'ematique en assignant une valeur \`a la table d'entr\'ee des ^|\mathcode|
\ctsref{\mathcode} pour ce
caract\`ere.  Le caract\`ere doit avoir un
\refterm{code de cat\'egorie} \`a $11$ (lettre) ou $12$ (autre) pour que \TeX\ 
recherche son |\mathcode|.
}\par

^^{famille//variable}
Vous pouvez d\'efinir un caract\`ere math\'ematique pour avoir une famille ``variable'' en
lui donnant une classe de $7$.  A chaque fois que \TeX\ rencontre ce caract\`ere dans une
formule math\'ematique, il prend la famille du caract\`ere comme \'etant la valeur 
courante du param\`etre |\fam| \ctsref{\fam}.  Une famille variable vous autorise
\`a sp\'ecifier la police de texte ordinaire dans une formule math\'ematique.  Par
exemple, si le caract\`ere romain est dans la famille $0$, l'assignement
|\fam = 0|
fera que le texte ordinaire d'une formule math\'ematique sera mis en type romain
au lieu d'autre chose comme du type math\'ematique italique. Si la valeur de
|\fam| n'est pas dans la fourchette de $0$ \`a $15$, \TeX\ prend la valeur \`a
$0$, ainsi le rendu des classes $0$ et $7$ est \'equivalent.
\TeX\ fixe |\fam| \`a $-1$ a chaque fois qu'il entre en mode math\'ematique.
\endconcept


\conceptindex {math\'ematiques affich\'ees}
\concept {math\'ematique affich\'ee}

Le terme \defterm{math\'ematique affich\'ee} fait r\'ef\'e\-rence aux formules math\'ematiques que \TeX\
place sur une ligne seule avec de l'espace suppl\'ementaire dessus et dessous
pour le s\'eparer du texte l'environnant.
Une formule math\'ematique affich\'ee est entour\'ee de `|$$|'.
\ttidxref{$$}
\TeX\ lit les math\'ematiques affich\'ees dans le \refterm{mode} math\'ematiques affich\'ees.
\endconcept


\concept m\'ediocrit\'e

La \defterm{m\'ediocrit\'e} d'une ligne mesure combien les tailles des espaces 
inter-mot ^^{espacement inter-mots} de la ligne d\'evient de leurs valeurs 
naturelles, c'est-\`a-dire, des valeurs sp\'ecifi\'ees dans les 
\refterm{polices:police} utilis\'ees dans la ligne. Plus grande est la 
d\'eviation, plus grande est la m\'ediocrit\'e.  Similairement, la m\'ediocrit\'e d'une 
page est une mesure de combien les espaces entre les bo\^\i tes qui dessinent 
la page d\'evient de leurs valeurs id\'eales.  (Ordinairement, la plupart de ces 
bo\^\i tes sont des  lignes simples de paragraphes.)

Plus pr\'ecis\'ement, la m\'ediocrit\'e mesure de combien le \refterm{ressort} 
associ\'e avec ces espaces doit s'\'etirer ou se r\'etr\'ecir pour remplir la 
ligne ou la page exactement. \TeX\ calcule la m\'ediocrit\'e comme 
approximativement $100$ fois le cube du ratio par lequel il doit \'etirer ou 
r\'etr\'ecir le ressort pour composer une ligne ou une page de la taille d\'esir\'ee. 
^^{coupures de ligne//m\'ediocrit\'e pour}^^{coupures de page//m\'ediocrit\'e pour} Par exemple, 
\'etirer le ressort par deux fois son \'etat d'\'etirement  signifie un ratio de 
$2$ et une m\'ediocrit\'e de $800$~; la r\'etr\'ecir de moiti\'e son \'etat de r\'etr\'ecissement 
signifie un ratio de $.5$ et une m\'ediocrit\'e de $13$. \TeX\ traite une m\'ediocrit\'e 
sup\'erieure \`a $10000$ comme \'etant \'egale \`a $10000$.

\TeX\ utilise la m\'ediocrit\'e d'une ligne quand il coupe un paragraphe en lignes
\seeconcept{coupure de ligne}.  Il utilise cette information en deux \'etapes~:

\olist
\li Quand \TeX\ choisi des coupures de ligne,
il acceptera \'eventuellement des lignes dont la m\'ediocrit\'e est inf\'erieure ou \'egale \`a
la valeur de |\tole!-rance| (\xref \tolerance).  Si \TeX\ ne peut pas diminuer la conception
d'une ligne dont la m\'ediocrit\'e d\'epasse cette valeur, 
il lui mettra  un ``underfull ou overfull \refterm{hbox}''.
\TeX\ ne mettra
un ``overfull ou underfull hbox'' qu'en dernier ressort, c'est-\`a-dire, seulement s'il n'y a
pas d'autre moyen de d\'ecouper le paragraphe en lignes.
\li En assumant que toutes les lignes sont tol\'erablement mauvaises, \TeX\ utilise la m\'ediocrit\'e des lignes
pour \'evaluer les diff\'erents moyens de couper le paragraphe en lignes.
Durant cette \'evaluation il associe des ``d\'em\'erites'' avec chaque ligne potentielle.
La m\'ediocrit\'e augmente le nombre de \refterm{d\'em\'erites}.
\TeX\ alors
coupe le paragraphe en lignes d'une fa\c con qui minimise 
le total des d\'em\'erites pour le paragraphe.
Plus
souvent \TeX\ arrange le paragraphe pour minimiser la m\'ediocrit\'e de la
ligne la pire.  Voir \knuth{pages~97--98}{114--115} pour les d\'etails comment \TeX\
coupe un paragraphe en lignes.  
\endolist

La proc\'edure de \TeX\ pour assembler une s\'equence de lignes et autres mati\`eres en modes verticaux
en pages est similaire \`a sa proc\'edure de coupure de ligne.
De toutes fa\c cons, assembler des pages n'est pas aussi compliqu\'e parce que
\TeX\ ne consid\`ere qu'une page \`a la fois
quand il cherche des coupures de page.
Donc la seule d\'ecision qu'il doit prendre est o\`u finir la page courante.
De plus, quand \TeX\ choisit des coupures de ligne il en
consid\`ere plusieurs simultan\'ement.
(La plupart des traitements de texte choisissent des coupures de ligne une \`a la fois,
et donc ne font pas un aussi bon boulot que celui que fait \TeX\.)
Voir \knuth{pages~111--113}{129--131} pour les d\'etails sur comment \TeX\ choisit ses
coupures de page.
\endconcept


\conceptindex{modes}
\concept mode

Quand \TeX\ dig\`ere votre entr\'ee dans son estomac \seeconcept{\anatomy},
il est dans un des six \defterm{modes}~:
\ulist\compact
\li ^{mode horizontal ordinaire} (assemblage d'un paragraphe)
\li ^{mode horizontal restreint} (assemblage d'une \refterm{hbox})
\li ^{mode vertical ordinaire} (assemblage d'une page)
\li ^{mode vertical restreint} (assemblage d'une \refterm{vbox})
\li ^{mode math\'ematique de texte} (assemblage d'une formule apparaissant dans du texte)
\li ^{mode math\'ematique hors-texte}
(assemblage d'une formule apparaissant sur une ligne seule)
^^{mode horizontal}^^{mode vertical}^^{mode math\'ematique}
\endulist
Le mode d\'ecrit une sorte d'entit\'e que \TeX\ rassemble.

Parce que vous pouvez encastrer une sorte d'entit\'e avec une autre, c'est-\`a-dire, une vbox
dans une formule, \TeX\ garde trace non seulement d'un mode mais d'une
liste enti\`ere de modes (ce qu'un informaticien appelle une ``pile'').
Supposez que \TeX\ soit en mode $M$ et rencontre quelque chose qui le
met dans une nouveau mode \Mprimeperiod.  Quand il termine son travail dans
le mode \Mprimecomma, il reprend ce qu'il faisait dans le mode \Mperiod.

\endconcept


\concept {mode horizontal}

^^{hbox//mode horizontal pour}
Quand \TeX\ assemble un paragraphe ou une \refterm{hbox}, il est dans un
des deux \defterm{modes horizontaux}~: le ^{mode horizontal ordinaire} 
pour assembler des paragraphes et le ^{mode horizontal restreint} pour
assembler des bo\^\i tes horizontales.  A chaque fois que \TeX\ est dans un mode horizontal
son estomac \seeconcept{\anatomy} construit une \refterm{liste horizontale} 
d'articles (bo\^\i tes, ressort, p\'enalit\'es, etc.).
\TeX\ compose les articles dans la liste
l'un apr\`es l'autre, de gauche \`a droite.

Une liste horizontale ne peut contenir aucun
article produit par des commandes verticales internes, par exemple, |\vskip|.
^^{listes horizontales//ne peuvent contenir de commandes verticales}

\ulist
\li Si \TeX\ assemble une liste horizontale dans le mode horizontal ordinaire et
rencontre une commande verticale interne, \TeX\ termine le paragraphe et
entre dans le \refterm{mode vertical}.

\li Si \TeX\ assemble une liste horizontale dans le mode horizontal
interne et rencontre une commande verticale interne, il rousp\`ete.
\endulist 

Deux commandes que vous pensez d'abord horizontales internes
sont en fait verticales internes~: |\halign| \ctsref{\halign}
et |\hrule| \ctsref{\hrule}.
^^|\hrule//vertical par nature|
^^|\halign//vertical par nature|
voir la \knuth{page~286}{332} pour une liste 
des commandes verticales internes.

{\tighten
Vous devez faire attention \`a une subtile mais importante propri\'et\'e du mode
horizontal interne~: \emph{vous ne pouvez pas entrer en mode horizontal interne
quand vous \^etes en mode horizontal normal}. Ce que cela signifie en pratique
est que quand \TeX\ assemble une hbox il n'appr\'ehendera pas un texte en
paragraphe, c'est-\`a-dire, du texte pour lequel il ferait une \refterm{coupure
de ligne}. Vous pouvez contourner cette restriction en englobant le texte en
paragraphe dans une \refterm{vbox} \`a l'int\'erieur de l'hbox. la m\^eme m\'ethode
marche si vous voulez mettre, disons un \refterm{alignement} horizontal dans
un~hbox.
}% end scope of tighten

\endconcept


\concept {mode math\'ematique}

{\tighten
Un \defterm{mode math\'ematique} est un \refterm{mode} dans lequel \TeX\ 
construit une formule math\'ematique.  \TeX\ a deux modes math\'ematiques diff\'erents~: Le 
^{mode math\'ematique de texte} construit une formule devant \^etre englob\'ee dans une ligne de texte,
tanfis que le ^{mode math\'ematique d'affichage} construit une formule devant 
appara\^\i tre seul sur une ligne.  Vous indiquez le mode math\'ematique de texte
en englobant la formule entre |$| et dans le mode math\'ematique d'affichage en
englobant la formule entre |$$|.
%\TeX\ will accept most \refterm{commands:command} in
%math mode. If it encounters a command in math mode that doesn't make
%sense in a formula, it will complain.
Une propri\'et\'e importante des deux modes math\'ematiques est que les 
\emph{espace saisis ne comptent pas}.  Voir les 
\knuth{pages~290--293}{338--342} pour des d\'etails sur la fa\c con dont \TeX\ r\'epond aux 
diff\'erentes commandes en mode math\'ematique.
\par}

\endconcept


\concept {mode ordinaire}

Le \defterm{mode ordinaire} est le \refterm{mode} dans lequel est \TeX\ quand
il assemble un paragraphe en lignes ou assemble des lignes
en une page.  Voir \conceptcit{mode horizontal}, \conceptcit{mode vertical}.
\endconcept


\concept {mode restreint}

Le \defterm{mode restreint} est le \refterm{mode} dans lequel est \TeX\ quand
il assemble une \refterm{hbox} ou une \refterm{vbox}.
Nous suivons \texbook\ dans l'utilisation du terme ``mode vertical interne''
pour ce que vous vous attendez \^etre un ``mode vertical restreint''.
Voir \conceptcit{mode horizontal} et \conceptcit{mode vertical}.
\endconcept


\concept {mode vertical}

^^{vbox//mode vertical pour} Quand \TeX\ assemble soit une
\refterm{vbox} soit la liste verticale principale dont les pages sont d\'eriv\'ees,
il est dans un des deux \defterm{modes verticaux}~: le ^{mode vertical ordinaire} 
pour assembler la liste verticale principale et le ^{mode vertical restreint} 
pour assembler les vbox.  A chaque fois que \TeX\ est en mode vertical 
son estomac \seeconcept{\anatomy} construit une \refterm{liste verticale} 
d'\'el\'ements (bo\^\i tes, ressort, p\'enalit\'es, etc.).
\TeX\ compose les \'el\'ements de la liste
l'un apr\`es l'autre, de haut en bas.

Une liste verticale ne peut pas contenir tous les \'el\'ements produits par 
des commandes fondamentalement horizontales, c'est-\`a-dire,
^^{listes verticales//ne peuvent contenir de commandes horizontales}
|\hskip| ou un caract\`ere ordinaire (pas un espace)
\footnote{\TeX\ \emph{ignore} tout caract\`ere espace
qu'il rencontre quand il est dans un mode vertical.}.

\ulist
\li Si \TeX\ assemble une liste verticale dans le mode vertical ordinaire et
rencontre une commande fondamentalement horizontale, il bascule vers le
\refterm{mode horizontal} ordinaire. 
\li Si \TeX\ assemble une liste verticale dans le mode vertical interne et
rencontre une commande fondamentalement horizontale, il rousp\`ete.
\endulist

Deux commandes auxquelles vous devez penser en premier qu'elles sont fondamentalement verticales sont
en fait fondamentalement horizontales~: |\valign| \ctsref{\valign} et |\vrule|
\ctsref{\vrule}.
^^|\valign//horizontal par nature|
^^|\vrule//horizontal par nature|
Voir la \knuth{page~283}{329} pour une liste des commandes
fondamentalement horizontales.

Il est particuli\`erement important d'avoir conscience que \TeX\ consid\`ere un
caract\`ere ordinaire autre qu'un espace comme \'etant fondamentalement horizontal.
Si \TeX\ d\'ebute soudainement un nouveau paragraphe quand vous ne vous y
attendez pas une cause vraisemblable est un caract\`ere que \TeX\ a rencontr\'e
en \'etant en mode vertical. Vous pouvez convaincre \TeX\ de ne pas traiter ce
caract\`ere comme fondamentalement horizontal en l'englobant dans une
\refterm{hbox} puisque la commande |\hbox|, malgr\'e son nom, n'est ni
fondamentalement horizontale ni fondamentalement verticale.
\endconcept


\conceptindex{mots de contr\^ole} 
\concept {mot de contr\^ole} 

Un \defterm{mot de contr\^ole} est une \refterm{s\'equence de contr\^ole} qui
est constitu\'e d'un \refterm{caract\`ere d'\'echappement} suivi d'une ou plusieurs
lettres\footnote{Une ``lettre'' ici a la strict signification d'un 
caract\`ere de code de cat\'egorie $11$.}.
\TeX\
ignore tout espace ou fin de ligne qui suit un mot de contr\^ole, sauf pour
noter qu'ils terminent le mot de contr\^ole.
^^{caract\`ere d'\'echappement}
\endconcept


\concept muglue

Une \defterm{muglue} est une sorte de \refterm{ressort} 
que vous ne pouvez utiliser que dans les formules math\'ematiques.  
Elle est mesur\'ee en ^|mu| (\refterm{unit\'es math\'ematiques:unit\'e
math\'ematique}).
^^{unit\'es math\'ematiques}^^{ressort//math\'ematique}
Une |mu| est \'egale \`a \frac1/{18} em, o\`u
la taille d'un em est prise de la \refterm{famille} 2 des polices math\'ematiques.
\TeX\ ajuste automatiquement la taille de la muglue en fonction du contexte.
Par exemple, une taille de ressort de |2mu| est normalement plus petite dans un
sous-script que dans un texte ordinaire.  
Vous devez utiliser la commande ^|\mskip| pour produire une muglue.
Par exemple, `|\mskip 4mu plus 5mu|' produit un ressort math\'ematique avec un
espace naturel de quatre |mu| et un \refterm{\'etirement} de cinq |mu|.

\endconcept


\conceptindex{noms de fichier}
\concept {nom de fichier}

Un \defterm{nom de fichier} nomme un 
\refterm{fichier} qui est connu du ^{syst\`eme d'exploitation}
qui 
supervise votre ex\'ecution de \TeX.  La syntaxe d'un nom de fichier ne doit \emph{pas}
suivre les r\`egles usuelles de la syntaxe de \TeX\ et en fait est diff\'erente
selon les diff\'erentes impl\'emen\-ta\-tions de \TeX.
En particulier, la plupart des impl\'ementations de \TeX\ consid\`erent un nom de fichier 
comme \'etant
termin\'e par un blanc ou une fin de ligne.  Ainsi \TeX\ est susceptible de
mal interpr\'eter `|{\input chapter2}|' 
en prenant l'accolade droite comme partie du nom de fichier.
Comme r\`egle g\'en\'erale, vous devez faire suivre un nom de fichier d'un blanc ou 
d'une fin de ligne comme dans `|{\input chapter2!visiblespace}|'.

\endconcept


\conceptindex{nombres}
\concept nombre

En \TeX, un \defterm{nombre} est un entier positif ou n\'egatif.
Vous pouvez \'ecrire un nombre en \TeX\ de quatre fa\c cons diff\'erentes~:
\olist\compact
\li comme un entier d\'ecimal ordinaire, par exemple, |52|
\li comme un nombre octal, par exemple, |'14| ^^{nombres octaux}
\li comme un nombre hexad\'ecimal, par exemple, |"FF0| ^^{nombres hexad\'ecimaux}
\li comme le code d'un \refterm{caract\`ere \ascii}, par exemple, |`)|
ou |`\)|
\endolist
\noindent 
Chacune de ces formes peut \^etre pr\'ec\'ed\'ee par `|+|' ou `|-|'.

Un nombre octal ne peut avoir que les chiffres |0|--|7|.
^^{nombres octaux}
Un nombre hexa\-d\'ecimal peut avoir les chiffres |0|--|9| et
|A|--|F|, repr\'esentant
les valeurs de $0$ \`a $15$.
^^{nombres hexad\'ecimaux}
Vous ne pouvez, h\'elas, utiliser de lettres minuscules quand vous \'ecrivez un nombre hexad\'ecimal.
Si vous voulez des explications sur les nombres octaux et hexad\'ecimaux,
vous en trouverez dans les \knuth{pages~43--44}{53--54}.

Un nombre d\'ecimal, octal ou hexad\'ecimal 
se termine au premier caract\`ere qui ne peut faire partie du nombre.
Ainsi un nombre d\'ecimal se termine quand \TeX\ voit, disons, une lettre,
tandis qu'une lettre entre `|A|' et `|F|' ne terminera pas un nombre hexad\'ecimal.
Vous pouvez terminer un nombre avec un ou plusieurs espaces et normalement,
\TeX\ les ignorera\footnote{
Quand vous d\'efinissez une macro qui se termine par un nombre, vous devez
toujours mettre un espace apr\`es ce nombre~; autrement, \TeX\ pourrait combiner
ce nombre avec autre chose.}.

La quatri\`eme forme ci-dessus sp\'ecifie un nombre comme le code
\minref{\ascii} d'un caract\`ere.
^^{caract\`eres//codes \ascii\ pour}
\TeX\ ignore les espaces apr\`es cette forme de nombre aussi.
Vous pouvez \'ecrire un nombre de cette forme soit comme |`|$c$ soit comme |`\|$c$.
La seconde forme, bien que plus longue, a l'avantage que vous pouvez l'utiliser
avec \emph{tout} caract\`ere, m\^eme `|\|', `|%|', ou `|^^M|'.
En revanche, il y a un inconv\'enient plut\^ot technique~: quand \TeX\ d\'eveloppe
une s\'equence de tokens pour une commande telle que |\edef| ou |\write|,
^^|\edef//expansion de {\tt\\'\it c} en|
^^|\write//expansion de {\tt\\'\it c} en|
des occurrences de `|\|$c$' repr\'esentant un nombre sera aussi d\'evelopp\'ee
 si elle peut l'\^etre.
C'est rarement l'effet que vous d\'esirez.

Ce qui suit sont des repr\'esentations valides du nombre d\'ecimal
$78$~:
\csdisplay
78   +078   "4E   '116   `N   `\N
|


Vous ne pouvez utiliser un nombre dans du texte par lui-m\^eme car un nombre n'est pas
une commande.
N\'eanmoins, vous pouvez ins\'erer la forme d\'ecimale d'un nombre dans du texte
en mettant une commande ^|\number| (\xref\number) devant lui
ou la forme num\'erique romaine en mettant une commande ^|\romannumeral|
devant lui.

Vous pouvez aussi utiliser des ^{constantes d\'ecimales}, 
par exemple, des nombres avec une partie fractionnaire,
pour sp\'ecifier des dimensions \seeconcept{dimension}.
Une constante d\'ecimale a un ^{point d\'ecimal}, qui peut \^etre
le premier caract\`ere de la constante.  Vous pouvez utiliser une 
virgule au lieu d'un point pour repr\'esenter le point d\'ecimal.
Une constante d\'ecimale peut \^etre pr\'ec\'ed\'ee par un signe plus ou moins.
Ainsi `|.5in|', 
`|-3.22pt|' et `|+1,5\baselineskip|' sont des dimensions valides.
Vous ne pouvez, par contre, utiliser de constantes d\'ecimales
dans tous les contextes \emph{autres} comme la partie ``facteur'' d'une dimension,
c'est-\`a-dire, son multiplicateur.


\endconcept


\concept outer

\bix^^{macros//outer}
Une macro \defterm{outer} est une macro que vous ne pouvez utiliser dans 
certains contextes o\`u \TeX\ assemble des tokens \`a grande vitesse.
Le but de rendre une commande outer est d'obliger \TeX\ \`a g\'en\'erer
des erreurs avant d'aller trop loin.
Quand vous d\'efinissez une macro, vous pouvez la rendre outer avec la commande
^|\outer| \ctsref\outer.

Vous ne pouvez utiliser de macro outer dans aucun des contextes suivants~:
\ulist\compact
\li comme argument d'une macro
\li dans le texte param\`etre ou le texte de remplacement d'une d\'efinition
\li dans le pr\'eambule d'un alignement
\li dans la partie non ex\'ecut\'ee d'un test conditionnel
\endulist
\noindent
Un contexte outer est un contexte dans lequel vous pouvez utiliser une 
macro outer, c'est-\`a-dire c'est tout contexte autre que ceux list\'es ci-dessus.

Par exemple, l'entr\'ee suivante sera une utilisation interdite d'une macro outer
\csdisplay
\leftline{\proclaim Assertion 2. That which is not inner
   is outer.}
|
La macro |\proclaim| (\xref{\@proclaim}) est d\'efinie en \plainTeX\
comme \'etant outer, mais elle est utilis\'ee ici comme un macro argument \`a
 |\leftline|.
\eix^^{macros//outer}

\endconcept


\conceptindex{pages}
\concept page

\TeX\ compile un document en assemblant des \defterm{pages} une par une
et en les passant \`a la routine de sortie.
Quand il lit votre document, \TeX\ maintient une liste des lignes
et d'autres \'el\'ements \`a placer sur la page. (Les lignes sont normalement des 
hbox.) Cette liste est appel\'ee la ``^{liste verticale principale}''.
P\'eriodiquement \TeX\ rentre dans un processus appel\'e ``devoir du
^{constructeur de page}''.
Les \'el\'ements ajout\'es \`a la liste verticale principale entre les devoirs
du constructeur de page sont appel\'e ``^{contributions r\'ecentes}''.

Le constructeur de page examine d'abord la liste verticale principale pour voir si elle
a besoin de d\'eposer une page maintenant, soit parce que les \'el\'ements sur la liste
verticale principale ne tiennent pas sur la page, soit parce que un \'el\'ement
sp\'ecifique, tel que |\eject| \ctsref\eject, demande \`a \TeX\ de finir la page.
S'il n'est pas n\'ecessaire de d\'eposer une page, alors le constructeur de page 
est fait pour la prochaine fois.

Autrement, le constructeur de page analyse la liste verticale principale pour 
trouver ce qu'il consid\`ere comme \'etant la meilleure coupure de page possible.
Il associe des p\'enalit\'es avec plusieurs types de coupure de page peu attirantes---une 
coupure qui laisse une ligne isol\'ee en haut ou en bas de la page, une coupure 
juste avant un affichage math\'ematique, et ainsi de suite.  Il choisit alors
la coupure de page la moins ch\`ere
o\`u le co\^ut d'une coupure est augment\'e  de toute p\'enalit\'e associ\'ee avec cette
coupure et par la m\'ediocrit\'e de la page qui en r\'esulte
(voir la \knuth{page~111}{130} pour la formule du co\^ut).  S'il trouve plusieurs
coupures de page de co\^ut identique, il choisit la derni\`ere.

{\tighten
Une fois que le constructeur de page a choisi une coupure de page,
il place les \'el\'ements de la liste qui sont avant cette coupure
dans la ^|\box255| et garde le reste pour la page suivante.
Il appelle alors la routine de sortie. |\box255| agit comme une bo\^\i te
aux lettres, avec le constructeur de page comme exp\'editeur et la routine de
sortie comme destinataire.
Ordinairement la routine de sortie traite la |\box255|, ajoute
d'autres \'el\'ements, comme des insertions, ent\^ete et pieds de page \`a la page et
envoit la page vers le \dvifile\
^^{\dvifile//mat\'eriel de la routine de sortie}
avec une commande |\shipout|.
(Des routines de sortie sp\'ecialis\'ees peuvent agir diff\'eremment.)
Du point de vue de \TeX, il importe peu que la routine de sortie envoie
la page ou non~;
La seule
responsabilit\'e de la routine de sortie est de traiter la |\box255| d'une fa\c con 
ou d'une autre.
\par}

{\tighten
Il est important de r\'ealiser que le meilleur endroit pour couper une page
n'est pas n\'ecessairement le dernier endroit possible pour couper la page.
Des p\'enalit\'es et autres consid\'erations peuvent faire que la coupure de page
ait lieu plus t\^ot. De plus, \TeX\ appr\'ehende les \'el\'ements de la liste verticale
principale en batch, pas seulement un par un. Les lignes d'un paragraphe sont
un exemple d'un tel batch. Pour ces raisons, le constructeur de page garde
des \'el\'ements sous le coude quand il coupe une page. Ces \'el\'ements mis en r\'eserve
forment alors le d\'ebut de la liste verticale principale de la page suivante
(vraisemblablement au milieu d'un batch). Parce que des \'el\'ements sont d\'eplac\'es
d'une page \`a une autre, vous pouvez \^etre s\^ur que quand \TeX\ ex\'ecute l'entr\'ee, 
le num\'ero de la page courante refl\`ete la page sur laquelle la sortie correspondante
appara\^\i tra. Voir les \knuth{pages~110--114}{128--133} pour une description compl\`ete
des r\`egles de coupure de page de \TeX.
\par}

\endconcept


\conceptindex{paragraphes}
\concept paragraphe

Intuitivement, un \defterm{paragraphe} est une suite de lignes de saisie 
se terminant par une ligne blanche, une commande |\par| \ctsref{\@par}
^^|\par//terminer un paragraphe avec|
ou par une commande intrins\`equement verticale, telle que |\vskip|.
Plus pr\'ecis\'ement, un paragraphe est une s\'equence de commandes que \TeX\ ex\'ecute
en mode horizontal restreint.
Quand \TeX\ a collect\'e un paragraphe entier, il le transforme en une s\'equence de
lignes en choisissant les coupures de ligne \seeconcept{coupure de ligne}.
Le r\'esultat est une liste de hbox avec ressort, p\'enalit\'e interligne
et mat\'eriel vertical imbriqu\'e entre.
Chaque hbox est une simple ligne et le ressort est le ressort inter-ligne.

%\eject
\TeX\ d\'ebute un paragraphe quand il est dans un mode vertical et 
rencontre une commande fondamentalement horizontale.
En particulier, il est en mode vertical juste apr\`es avoir termin\'e un 
paragraphe, donc le mat\'eriel horizontal sur la ligne suivant une ligne 
blanche d\'ebute le paragraphe suivant de mani\`ere naturelle.
Il y a plusieurs types de commandes fondamentalement horizontales, mais le
type le plus commun est un caract\`ere ordinaire, c'est-\`a-dire, une lettre.

\looseness = -1
Les commandes ^|\indent| et ^|\noindent|
(\pp\xrefn{\indent},~\xrefn{\noindent})
sont aussi des commandes fondamentalement horizontales qui demandent \`a
\TeX\ d'in\-denter ou non le d\'ebut d'un paragraphe.
Toute autre commande horizontale en mode vertical
fait que \TeX\ fait un |\indent| implicite.
Une fois que \TeX\ a commenc\'e un paragraphe, il est en mode horizontal ordinaire.
Il commence par suivre toutes commandes situ\'ees dans ^|\everypar|.
Il commence alors \`a collecter des \'el\'ements pour le paragraphe jusqu'a 
ce qu'il re\c coive un signal de la fin du paragraphe.
A la fin du paragraphe il r\'einitialise les param\`etres de formation de
paragraphe ^|\parshape|, |\hang!-indent|,
^^|\hangindent| 
et ^|\looseness|.

\TeX\ traduit normalement une ligne blanche par |\par|.  
Il ins\`ere aussi
un |\par| dans l'entr\'ee chaque fois, en mode horizontal, qu'il 
rencontre une commande intrins\`equement verticale.
Donc finalement la chose qui termine un paragraphe est toujours une commande |\par|.

Quand \TeX\ re\c coit une commande |\par|, il compl\`ete d'abord\footnote{%
Plus pr\'ecis\'ement, il ex\'ecute les commandes~:
\csdisplay
\unskip \penalty10000 \hskip\parfillskip
|
apportant ainsi les \'el\'ements de ces commandes 
\`a la fin de la liste horizontale courante.}
le paragraphe sur lequel il travaille.
Il coupe alors le paragraphe en lignes,
ajoute la liste r\'esultante d'\'el\'ements dans la liste verticale englobante
et ex\'ecute le constructeur de page 
(dans le cas o\`u la liste verticale englobante est la liste verticale principale).
Si le paragraphe se termine par une commande intrins\`equement verticale,
\TeX\ ex\'ecute alors cette commande.

\endconcept

\conceptindex{param\`etres}
\concept param\`etre

Le terme \defterm{param\`etre} a deux significations diff\'erentes---il peut faire
r\'ef\'erence soit \`a un param\`etre de \TeX, soit \`a un param\`etre de macro.

Un param\`etre de \TeX\ est une \refterm{s\'equence de contr\^ole} qui nomme
une valeur.
La valeur d'un param\`etre peut \^etre un \refterm{nombre}, une \refterm{dimension},
un montant de \refterm{ressort},  de muglue ou une \refterm{liste de tokens}.
Par exemple, le param\`etre ^|\parindent|
sp\'ecifie la distance que \TeX\ saute au d\'ebut d'un paragraphe indent\'e.

Vous pouvez utiliser la s\'equence de contr\^ole d'un param\`etre soit pour r\'etablir la valeur
du param\`etre soit pour fixer cette valeur.  \TeX\ interpr\`ete la s\'equence de contr\^ole 
comme une requ\^ete pour une valeur si elle appara\^\i t dans un contexte o\`u une valeur est attendue,
et comme un \refterm{assignement} autrement.
^^{assignements}
Par exemple~:
\csdisplay
\hskip\parindent
|
produit un \refterm{ressort} horizontal dont la taille naturelle est donn\'ee
par |\par!-indent|,
tandis que~:
\csdisplay
\parindent = 2pc  % (ou \parindent 2pc)
|
fixe |\parindent| \`a une longueur de deux picas.  L'assignement~:
\csdisplay
\parindent = 1.5\parindent
|
utilise |\parindent| des deux fa\c cons.  Son effet est de multiplier la valeur de
|\parindent| par $1.5$.

Vous pouvez penser \`a un param\`etre comme \`a un \refterm{registre} construit.
^^{registres//param\`etres comme}
Vous trouverez la liste de tous les param\`etres de \TeX\ dans les \knuth{pages~272--275}{317--318}.

Un param\`etre de \refterm{macro} est un endroit qui garde la place du texte 
qui est branch\'e dans la d\'efinition d'une macro.  Voir \conceptcit{macro}
pour plus d'informations sur ce type de param\`etre.

\endconcept


\conceptindex{penalties}
\concept p\'enalit\'e

Une \defterm{p\'enalit\'e} est un \'el\'ement que vous pouvez inclure dans une 
liste horizontale, verticale ou math\'ematique
pour d\'ecourager \TeX\ de couper la liste
\`a cet endroit ou encourager \TeX\ de couper la liste ici.  
^^{listes horizontales//p\'enalit\'es dans}
^^{listes verticales//p\'enalit\'es dans}
Une p\'enalit\'e positive indique un mauvais point de coupure, tandis qu'une
p\'enalit\'e n\'egative indique un bon point de coupure.
Couper une liste horizontale ordinaire
produit une \refterm{coupure de ligne}, tandis que couper une liste
verticale ordinaire produit
une \refterm{coupure de page}.  
(Une p\'enalit\'e n'a aucun effet en \refterm{mode} horizontal restreint ou 
vertical interne.)

Vous pouvez utiliser la commande
|\penalty| (\pp\xrefn{hpenalty},~\xrefn{vpenalty})
pour ins\'erer une p\'enalit\'e explicitement.
Une p\'enalit\'e de $10000$ ou plus emp\^eche une coupure, tandis qu'une p\'enalit\'e de
$-10000$  ou moins force une coupure.
\endconcept


\conceptindex{pieds de page}
\concept {pied de page}

Un \defterm{pied de page} est du mat\'eriel que \TeX\ met en bas de chaque page,
sous le texte de cette page.
Le pied de page par d\'efaut en \plainTeX\ est un num\'ero de page centr\'e.
D'habitude un pied de page consiste en une ligne simple, que vous pouvez
d\'efinir en assignant une liste de tokens \`a ^|\footline| \ctsref\footline.
Voir \xrefpg{bighead} pour une m\'ethode de production de pied de page multi-ligne.

\endconcept

%k \vskip 0pt plus 2pt % to solve page break problem

%\eject
\concept {\plainTeX}

\defterm{\PlainTeX} est le format de \TeX\ d\'ecrit dans ce
livre et dans \texbook.  \PlainTeX\ est une partie du syst\`eme  \TeX\ standard,
donc des documents qui n'utilisent que les am\'enagements de \plainTeX\ peuvent
facilement \^etre transf\'er\'es d'une installation vers une autre sans
difficult\'e.

\PlainTeX\ est constitu\'e de commandes \refterm{primitive}s auxquelles se joignent
une large collection de macros et
autres d\'efinitions.  Ces d\'efinitions additionnelles sont d\'ecrites dans 
l'\knuth{annexe~B}{}.  Vous pourrez aussi les trouver dans le fichier |plain.tex|
quelque part sur votre syst\`eme informatique.
\endconcept


\concept {point de r\'ef\'erence}

^^{lignes de base}
^^{\boites//point de r\'ef\'erence de}
Le \defterm{point de r\'ef\'erence} d'une \refterm{bo\^\i te} est le point o\`u le
cot\'e gauche de la bo\^\i te est en intersection avec sa \refterm{ligne de base}.  
Quand \TeX\ ex\'ecute une \refterm{liste horizontale} ou 
\refterm{verticale:liste verticale}, il utilise les points de r\'ef\'erence 
des bo\^\i tes dans la liste pour aligner ces bo\^\i tes horizontalement ou 
verticalement \seeconcept{\boites}.
\endconcept

\conceptindex{polices}
\concept police

Une \defterm{police} en \TeX\ est une collection de jusqu'a $256$ caract\`eres
de sortie, ayant normalement les m\^emes design, style (romain,
italique, gras, condens\'e, etc.),
et taille\footnote{\PlainTeX\ utilise une police sp\'eciale
pour construire les ^{symboles math\'ematiques} pour lesquelles les caract\`eres 
ont diff\'erentes tailles.  D'autres polices sp\'eciales sont souvent utiles pour 
des applications comme le composition de ^{logos}.}.
La ^{police Computer Modern} qui est fournie 
g\'en\'eralement avec \TeX\ n'a que $128$ caract\`eres. Le colophon sur la
derni\`ere page de ce livre d\'ecrit les polices que nous avons utilis\'ees pour composer
ce livre.

Par exemple, voici l'alphabet en police romaine Palatino de $10$ points~:
^^{police Palatino}
\vskip\abovedisplayskip{\narrower\tenpal
\noindent ABCDEFGHIJKLMNOPQRSTUVWXYZ\hfil\break
abcdefghijklmnopqrstuvwxyz\par
}\vskip\belowdisplayskip
\noindent
et le voici dans la police Computer Modern Grasse \'etendue de $12$
point~:
^^{police Computer Modern}
\vskip\abovedisplayskip{\narrower\font\twelvebf=cmbx12\twelvebf
\noindent ABCDEFGHIJKLMNOPQRSTUVWXYZ\hfil\break
abcdefghijklmnopqrstuvwxyz\par
}\vskip\belowdisplayskip
Les caract\`eres d'une police sont num\'erot\'es.
La num\'erotation s'accorde normalement avec la num\'erotation ^{\ascii}
pour les caract\`eres existants dans le jeu de caract\`eres \ascii.
La table de code de chaque police indique \`a quoi ressemble le caract\`ere
avec le code $n$ dans cette police.
Certaines polices, comme celles utilis\'ees pour les symboles math\'ematiques, n'ont pas
de lettres.  Vous pouvez produire une \refterm{bo\^\i te} contenant le
caract\`ere num\'erot\'e $n$, compos\'e dans la police courante, en \'ecrivant `|\char |$n$'
 \ctsref{\char}.

Pour utiliser une police dans votre document,
vous devez d'abord la nommer avec une s\'equence de  contr\^ole  et la charger.  Alors vous
pourrez la s\'electionner en saisissant
cette s\'equence de  contr\^ole  quand vous voulez l'utiliser.
\PlainTeX\ procure un certain nombre de polices d\'ej\`a nomm\'ees et~charg\'ees.

Vous nommez et chargez une police comme une simple op\'eration, avec une
commande comme `|\font\twelvebf=cmbx12|'. Ici, `|\twelvebf|' est la s\'equence 
de  contr\^ole  que vous utilisez pour nommer la police et `|cmbx12|' identifie
le fichier de m\'etrique de la police |cmbx12.tfm|  sur votre syst\`eme de 
fichier. Vous pouvez alors commencer \`a utiliser la police en saisissant 
`|\twelvebf|'.  Apr\`es cela, la police sera effective jusqu'a ce que soit
(a)~vous s\'electionnez une autre police, (b)~vous terminez le \refterm{groupe}
 dans lequel vous avez commenc\'e avec la police. Par exemple,
la saisie~:

\csdisplay
{\twelvebf white rabbits like carrots}
|
fera que la police |cmbx12| ne sera effective que pour le texte
`|white rabbits like carrots|'.

Vous pouvez utiliser \TeX\ avec d'autres polices que Computer Modern 
(regardez l'exemple sur \xrefpg{palatino} et les ent\^etes de page). 
Les fichiers de telles polices doivent \^etre install\'es sur votre syst\`eme
de fichier \`a une place o\`u \TeX\ puisse les trouver. \TeX\ et ses programmes 
compagnons ont g\'en\'eralement besoin de deux fichiers pour chaque police~:
un pour les m\'etriques (|cmbx12.tfm|,
^^{\tfmfile}
par exemple) et un autre pour la forme 
des caract\`eres (|cmbx12.pk|, par exemple).
\TeX\ lui-m\^eme n'utilise que les fichiers de m\'etriques.
Un autre programme, le pilote de p\'eriph\'erique,
^^{drivers de p\'eriph\'erique}
convertit le \dvifile\
^^{\dvifile//convertit par driver}
produit par \TeX\ en une forme que votre imprimante
ou autre p\'eriph\'erique de sortie puisse interpr\'eter.  Le pilote
utilise le fichier de forme (s'il existe).

Le fichier de m\'etrique de la police contient l'information dont \TeX\ \`a
besoin pour allouer de l'espace pour chaque caract\`ere compos\'e.
Ainsi il inclut la taille de chaque caract\`ere, les ligatures et cr\'enages
qui modifient les suites de caract\`eres adjacents, et ainsi de suite.
Ce que le fichier de m\'etrique n'inclut \emph{pas} sont les informations
sur les formes des caract\`eres de la police.

{\tighten
Le fichier de forme (pixel) \xrdef{shape}
^^{fichier de pixel}^^{fichier de forme}
peut \^etre de diff\'erents
formats. La partie d'extension du nom (la partie apr\`es le point)
informe le pilote sur le format dans lequel est le fichier de forme.  
Par exemple,
|cmbx12.pk| ^^{\pkfile} est le fichier de forme de la police |cmbx12| en
format compress\'e, tandis que |cmbx12.gf| ^^{\gffile} est le fichier de forme
pour la police |cmbx12| en format de police g\'en\'erique.
Un fichier de forme peut ne pas \^etre n\'ecessaire pour une police r\'esidente dans votre
p\'eriph\'erique de sortie.
\par}

\endconcept


\concept {primitive}

Une \refterm{commande} \defterm{primitive} est une commande dont la d\'efini\-tion 
est construite dans le programme informatique \TeX.  
Au contraire, une commande qui n'est pas une
primitive est d\'efinie par une \refterm{macro} ou un autre format de 
d\'efinition \'ecrit en \TeX.  Les commandes dans \refterm{\plainTeX}
sont constitu\'ees de commandes primitives auxquelles se joignent
d'autres commandes d\'efinies en termes de
primitives.
\endconcept


\concept profondeur

^^{\boites//profondeur de}
La \defterm{profondeur} d'une \refterm{bo\^\i te} est la distance par laquelle 
la bo\^\i te s'\'etend sous sa \refterm{ligne de base}. 
\endconcept


\conceptindex{registres}
\concept registre

Un \defterm{registre} est une place nomm\'ee pour stocker une valeur.
C'est plus comme une variable dans un langage de programmation.
\TeX\ a cinq sortes de registres, comme montr\'e dans le tableau suivant~:

\vdisplay{\tabskip 10pt\halign{\tt #\hfil &#\hfil\cr
{\it Type de registre}&{\it Contenus}\cr
box&une \refterm{bo\^\i te} \idxref{registres de \boite}\cr
count&une \refterm{nombre} \idxref{registres de compteur}\cr
dimen&une \refterm{dimension} \idxref{registres de dimension}\cr
muskip&\refterm{muglue} \idxref{registres de muglue}\cr
skip&\refterm{ressort} \idxref{registres de ressort}\cr
toks&un \refterm{token} list\idxref{registres de token}\cr}}

Les registres de chaque type sont num\'erot\'es de $0$ \`a $255$.
Vous pouvez acc\'eder au registre $n$ de cat\'egorie $c$ en utilisant la forme `|\|$cn$',
c'est-\`a-dire, |\muskip192|.
Vous pouvez utiliser un registre
n'importe o\`u cette information de type appropri\'e est appel\'ee.  Par
exemple, vous pouvez utiliser |\count12|
dans tout contexte appelant un nombre ou |\skip0|
dans tout contexte appelant un ressort.

Vous mettez de l'information dans un registre en y \refterm{assignant:assignment}
quelque chose~:

\csdisplay
\setbox3 = \hbox{lagomorphs are not mesomorphs}
\count255 = -1
|
Le premier assignement construit une hbox et l'assigne au
registre de bo\^\i te~$3$.
Vous pouvez
ensuite utiliser `|\box3|' partout o\`u une bo\^\i te est appel\'ee et vous obtiendrez
cette hbox\footnote{Mais faites attention~: utiliser un registre de bo\^\i te
la vide aussi donc son contenu devient vide.  Les autres sortes de
registres n'ont pas cet mani\`ere de r\'eagir. Vous pouvez utiliser la commande |\copy| 
\ctsref{\copy} pour conserver le contenu d'un registre de bo\^\i te sans
le vider.}.
Le second assignement assigne $-1$ au registre de compteur~$255$.

Un registre d'un type donn\'e, par exemple, un registre de ressort, r\'eagit comme
un param\`etre de ce type.
^^{param\`etres//comme registres}
Vous retrouvez sa valeur ou la lui assignez
comme vous le feriez avec un \refterm{param\`etre}.
Quelques param\`etres de \TeX, par exemple, |\pageno|,
sont impl\'ement\'es comme des registres.

\PlainTeX\
utilise plein de registres pour son propre usage, donc vous ne devez pas
simplement prendre un num\'ero de registre arbitraire
quand vous avez besoin d'un registre.  \`A la place, vous devez demander
\`a \TeX\ de r\'eserver un registre en utilisant une des commandes
^|\newbox|, ^|\newcount|, ^|\newdimen|, ^|\newmuskip|, ^|\newskip|
ou ^|\newtoks|
\ctsref{\@newbox}.  Ces com\-man\-des sont outer, donc vous ne pouvez
pas les utiliser dans une d\'efinition de macro.
Si vous le pouviez,
vous utiliseriez un registre \`a chaque fois que la macro serait appel\'ee et 
probablement d\'epasseriez le nombre de
registres rapidement.

Quoiqu'il en soit, vous pouvez, avec pr\'ecaution utiliser tout registre temporairement
dans un \refterm{groupe}, m\^eme un que \TeX\ utilise pour autre chose.
Quand \TeX\ a fini d'ex\'ecuter les commandes dans un groupe,
il retrouve le contenu de tous les registres
dans l'\'etat o\`u ils \'etaient avant de commencer \`a ex\'ecuter le groupe.
Quand vous utilisez un registre num\'erot\'e explicitement dans un groupe,
vous devez \^etre sur que le registre n'est modifi\'e par aucune
\refterm{macro}
que vous pourriez appeler dans le groupe.
Soyez sp\'ecialement attentif
quand vous utilisez des registres arbitraires dans un groupe qui appelle des macros 
que vous n'avez pas \'ecrit vous-m\^eme.

{\tighten
\TeX\ r\'eserve certains registres \`a des t\^aches sp\'eciales~: de |\count0| \`a
|\count9| pour des informations de num\'erotation de page et
^^{num\'erotation de page}
^|\box255| pour le contenu
d'une page juste avant qu'elle soit offerte \`a la \refterm{routine de sortie}.
Les registres |\dimen0|--|\dimen9|, |\skip0|--|\skip9|,
|\muskip0|--|\mu!-skip9| et |\box0|--|\box9| ainsi que
les |255| registres de bo\^\i te autres que |\box255|
sont g\'en\'eralement utilisables comme registre ``brouillon''.
Aussi \plainTeX\ ne procure qu'un registre brouillon, |\count255|, pour
des compteurs.
Voir les \knuth{pages~122 et 346}{141--142 et 402} pour des conventions \`a suivre
pour choisir des num\'eros de registre.
\par}

Vous pouvez examiner le contenu des registres lors d'une ex\'ecution 
avec la commande ^|\showthe| \ctsref\showthe, par exemple, `|\showthe\dimen0|'.
\endconcept


\concept r\'eglures

Vous pouvez utiliser des \refterm{r\'eglures} pour remplir un espace avec des
copies d'un mod\`ele, c'est-\`a-dire, mettre des points r\'ep\'et\'es entre un titre et
un num\'ero de page dans une table des mati\`eres.
Une r\'eglure est une copie simple d'un mod\`ele.
La sp\'ecification des r\'eglures contient trois parties d'information.
\olist\compact
\li ce qu'est une r\'eglure seule
\li combien d'espace doit \^etre rempli
\li comment les copies du mod\`ele doivent \^etre arrang\'ees dans l'espace
\endolist

\bix^^|\leaders|
\bix^^|\cleaders|
\bix^^|\xleaders|
{\tighten
\TeX\ poss\`ede trois commandes pour sp\'ecifier des r\'eglures~:
|\leaders|, |\cleaders| et |\xleaders| (\xref\leaders). 
L'\refterm{argument} de chaque commande sp\'e\-ci\-fie la r\'eglure.
La commande doit \^etre suivie d'un \refterm{ressort}. La taille du ressort
sp\'ecifie combien d'espace doit \^etre rempli.  Le choix de la commande d\'etermine
comment les r\'eglures sont arrang\'ees dans l'espace.}

Voici un exemple montrant comment les |\leaders| travaillent~:
\csdisplay
\def\dotting{\leaders\hbox to 1em{\hfil.\hfil}\hfil}
\line{The Political Process\dotting 18}
\line{Bail Bonds\dotting 26}
|
Ici, nous avons plac\'e les r\'eglures et leur ressort associ\'e dans une d\'efinition
de \refterm{macro} pour pouvoir les utiliser facilement \`a deux endroits. 
Cette entr\'ee produit~:

\vdisplay{\advance\hsize by -\parindent
\def\dotting{\leaders\hbox to 1em{\hfil{.}\hfil}\hfill}%
\line{The Political Process{\dotting}18}
\line{Bail Bonds{\dotting}26}
}

La \refterm{hbox} suivant |\leaders| sp\'ecifie 
la r\'eglure, soit, une hbox d'1\em\ de large contenant un point en son
centre.
L'espace est rempli avec des copies de cette bo\^\i te,
effectivement remplies
de points dont les centres sont s\'epar\'es de 1\em.
Le |\hfil| suivant (celui \`a la fin de la d\'efinition de macro) est 
un ressort qui sp\'ecifie l'espace \`a remplir.
Dans ce cas, c'est l'espace n\'ecessaire pour remplir la ligne.
en choisissant |\leaders| plut\^ot que |\cleaders| ou |\xleaders|, nous nous
assurons que les points des diff\'erentes lignes soient align\'es entre eux.

En g\'en\'eral, l'espace \`a remplir agit comme une fen\^etre pour les copies 
r\'ep\'et\'ees de r\'eglures. \TeX\ ins\`ere autant de copies que possible, mais de 
l'espace est habituellement laiss\'e---soit parce que les r\'eglures s'arr\^etent
dans la fen\^etre, soit parce que la largeur de la fen\^etre n'est pas un multiple 
exact de la largeur de la r\'eglure.
La diff\'erence entre les trois commandes est dans leur fa\c con d'arranger les
r\'eglures dans la fen\^etre et comment elles distribuent l'espace perdu.

\ulist
\li Pour |\leaders|, \TeX\ produit d'abord une rang\'ee de copies de la r\'eglure.
Il aligne alors le d\'ebut de cette rang\'ee avec l'extr\'emit\'e gauche de la bo\^\i te 
la plus interne $B$ qui contient le r\'esultat de la commande |\leaders|.
Dans les deux lignes d'exemple ci-dessous, $B$ est une bo\^\i te produite par |\line|.
Ces r\'eglures qui rentrent enti\`erement dans la fen\^etre sont plac\'ees dans $B$,
et l'espace restant aux extr\`e\-mit\'ees gauche et droite est laiss\'e vide.
L'image est comme ceci~:
\vdisplay{%
\def\dotting{\leaders\hbox to 1em{\hfil{.}\hfil}\hfill}%
\def\pp{The Political Process}
\line{\dotting}
\line{\hphantom\pp\hfil$\Downarrow$\hfil\hphantom{18}}
\vskip 4pt
\setbox0 = \hbox{\pp}
\setbox1 = \hbox{18}
\dimen0 = \hsize \advance\dimen0 by -\wd0 \advance \dimen0 by -\wd1
\advance\dimen0 by -0.8pt
\hbadness=10000
\line{\pp
   \vrule\vbox{\hrule width \dimen0\vskip 2pt
   \hbox to \dimen0{\hfil window\strut\hfil}
   \vskip 2pt\hrule width \dimen0}%
   \vrule 18}
\line{\hphantom\pp\hfil$\Downarrow$\hfil\hphantom{18}}
\vskip 2pt
\line{\pp{\dotting}18}
}
\vskip\medskipamount
{\tighten
\noindent
Cette proc\'edure assure que dans les deux lignes d'exemple de la page pr\'ec\'edente,
les points des deux lignes soient align\'es verticalement (puisque les 
\refterm{points de r\'ef\'erence:point de r\'ef\'erence} des hbox produites par
|\line| sont align\'es verticalement).
\par}

\li Pour |\cleaders|, \TeX\ centre les r\'eglures dans la fen\^etre
en divisant l'espace perdu entre les deux extr\'emit\'es de la fen\^etre.
L'espace perdu est toujours inf\'erieur \`a la largeur d'une seule r\'eglure.

\li Pour |\xleaders|, \TeX\ distribue l'espace perdu de chaque cot\'e dans la 
fen\^etre. En d'autres mots, si l'espace perdu est $w$ et que la r\'eglure se 
r\'ep\`ete $n$ fois, \TeX\ met un espace de largeur $w/(n+1)$ entre les r\'eglures
adjacente et \`a deux extr\'emit\'es des r\'eglures.
L'effet est habituellement d'\'etaler les r\'eglures un petit peu. L'espace
perdu pour |\xleaders|, comme celui de |\cleaders|, est toujours inf\'erieur
\`a la largeur d'une r\'eglure seule.
\endulist

Jusqu'ici, nous supposions que les r\'eglures consistaient en hbox arrang\'ees
horizontalement. Deux variations sont possibles~:
\olist
\li Vous pouvez utiliser une r\`egle \`a la place d'une hbox pour la r\'eglure.
\TeX\ rend la r\`egle aussi large que possible pour s'\'etendre au del\`a du 
ressort (et les trois commandes sont \'equivalentes).
\li Vous pouvez produire des r\'eglures verticales qui traversent la page en
les incluant dans une \refterm{liste verticale} au lieu d'une \refterm{liste
horizontale}. Dans ce cas vous avez besoin de ressort vertical apr\`es les
r\'eglures.
\endolist
\noindent
Voir les \knuth{pages~223--225}{261--263} pour les r\`egles pr\'ecises que \TeX\ utilise
en composant des r\'eglures.
\eix^^|\leaders|
\eix^^|\cleaders|
\eix^^|\xleaders|
\endconcept


\concept ressort

\bix^^{\etirement}
\bix^^{r\'etr\'ecissement}
Le \defterm{ressort} est de l'espace blanc qui peut se r\'etr\'ecir ou s'\'etirer. 
Le ressort donne \`a \TeX\ la flexibilit\'e dont il a besoin pour produire
des documents impeccables.  
Le ressort existe en deux types~: le ressort horizontal et le ressort vertical.  
Le ressort horizontal appara\^\i t dans des \refterm{listes horizontales: liste horizontale},
tandis que le ressort vertical appara\^\i t dans des \refterm{listes verticales:liste verticale}.
^^{listes horizontales}
^^{listes verticales}
Vous
pouvez produire un ressort soit implicitement, c'est-\`a-dire, avec un espace inter-mot, soit
explicitement, c'est-\`a-dire, avec la commande ^|\hskip|.
^^{espace//inter-mots}
\TeX\ lui-m\^eme produit beaucoup de ressorts
en composant votre document.
Nous ne d\'ecrirons que le ressort horizontal---le ressort vertical \'etant analogue.

Quand \TeX\ assemble une liste de bo\^\i tes et de ressorts en grande
quantit\'e,
^^{\boites//ressort avec}
il ajuste la taille des ressorts pour garder les espaces demand\'ees d'unit\'e
de largeur.  Par exemple, \TeX\ s'assure que la ^{marge de droite} d'une page
est uniforme en ajustant les ressorts horizontaux des lignes.
Similairement, il s'assure que les diff\'erentes pages ont la
m\^eme ^{marge du bas}
en ajustant les ressorts entre les blocs de texte comme des paragraphes et
des affichages math\'ematiques
(o\`u le changement est probablement moins remarquable).

Un ressort a un espacement naturel---la taille qu'il ``veut avoir''.  
Le ressort a aussi deux autres attributs~: son \refterm{\'etirement} et son
\refterm{r\'etr\'ecissement}.  Vous pouvez produire un montant sp\'ecifique de ressort 
horizontal avec la \refterm{commande} ^|\hskip| \ctsref{\hskip}.  La commande 
|\hskip 6pt plus 2pt minus 3pt|
produit un ressort horizontal dont la taille naturelle
est de $6$ points, l'\'etirement de $2$ points et le r\'etr\'ecissement
$3$ points.  Similairement, vous pouvez produire un montant sp\'ecifique de ressort vertical
avec la commande ^|\vskip| commande \ctsref{\vskip}.

Le meilleur moyen de comprendre ce qui s'\'etire et ce qui se r\'etr\'ecit
et de voir un exemple de ressort au travail.
Supposez que vous construisez un \refterm{hbox} de trois bo\^\i tes et deux ressorts, 
comme dans l'image~:
\gluepicture 
   29 {\picbox 4 \gluebox 6 4 1 6 \picbox 5 \gluebox 10 8 3 10 \picbox 4 }
\noindent
L'unit\'e de mesure ici peut \^etre en points, en millim\`etres, etc.
Si le hbox est autoris\'e \`a assumer sa largeur naturelle, alors chaque ressort 
de la bo\^\i te assume aussi sa largeur naturelle.  La largeur totale du hbox est 
alors la somme des largeurs de ses parties, soit, $29$ unit\'es.

Ensuite, supposez que le hbox doit \^etre plus large que $29$ unit\'es, disons
$35$ unit\'es.  Cela
peut arriver, par exemple, si le hbox doit occuper une ligne enti\`ere
et que la largeur de la ligne soit de $35$ unit\'es.
Puisque les bo\^\i tes ne peuvent changer leurs largeurs, 
\TeX\ produit l'espace suppl\'ementaire n\'ecessaire en rendant les ressorts plus larges.
L'image maintenant ressemble \`a ceci~:
\gluepicture 
   35 {\picbox 4 \gluebox 6 4 2 8 \picbox 5 \gluebox 10 8 6 14 \picbox 4 }
Les ressorts ne deviennent pas plus larges de mani\`ere \'egale~; 
ils deviennent plus large en proportion de leur \'etirement.  
Puisque le second ressort 
a deux fois plus d'\'etirement que le premier, 
il devient plus large de quatre unit\'es tandis que le premier devient plus large de 
seulement deux unit\'es.
Le ressort peut s'\'etirer autant que n\'ecessaire, n\'eanmoins, \TeX\ rechigne
\`a l'\'etirer au del\`a du montant d'\'etirement donn\'e dans sa d\'efinition.

Finalement, supposez que le hbox doive se rapprocher de $29$ unit\'es \`a, disons
$25$ unit\'es.  Dans ce cas \TeX\ r\'etr\'ecit les ressorts.
L'image ressemble \`a ceci~:
\gluepicture 
   25 {\picbox 4 \gluebox 6 4 2 5 \picbox 5 \gluebox 10 8 6 7 \picbox 4 }
Les ressorts deviennent plus proches en proportion de leur r\'etr\'e\-cis\-sement.
Le premier ressort devient plus \'etroit d'une unit\'e, tandis que le second 
r\'etr\'ecit de trois unit\'es. Le ressort ne peut pas se r\'etr\'ecir
d'une distance inf\'erieure au montant de r\'etr\'ecissement donn\'e dans sa d\'efinition
m\^eme si la distance \`a laquelle il peut se r\'etr\'ecir est sans
limite.  Sur ce point important l'\'etirement et le r\'etr\'ecissement r\'eagissent
diff\'eremment.

Une bonne r\`egle \`a suivre pour le ressort est de fixer sa taille normale au
montant d'espace qui va le mieux, l'\'etirement au montant d'espace le plus grand
que \TeX\ puisse ajouter avant que le document devienne mauvais et le
r\'etr\'ecissement au montant d'espace le plus grand que \TeX\ puisse enlever
avant que le document commence \`a devenir mauvais. Vous pouvez devoir fixer
les valeurs par exp\'erimentation.

Vous pouvez produire un ressort \'etirable \`a l'infini en sp\'ecifiant son
\'etirement en unit\'es de `^|fil|', `^|fill|' ou `^|filll|'. Le ressort mesur\'e en
`|fill|' est infiniment plus \'etirable que le ressort mesur\'e en `|fil|' et
le ressort mesur\'e en `|filll|' est infiniment plus \'etirable que le ressort mesur\'e
en `|fill|'.  Vous n'aurez que tr\`es rarement besoin de ressort `|filll|'. Un ressort
s'\'etirant de |2fil| s'\'etire deux  fois plus qu'un ressort s'\'etirant de |1fil|,
et de m\^eme pour les autres sortes de ressorts \'etirables infiniment.

Quand \TeX\ apporte un espace suppl\'ementaire en plus d'un ressort, 
celui infiniment \'etirable, s'il y en a, en prend la totalit\'e. 
Le ressort infiniment \'etirable est particuli\`erement pratique pour
cadrer du texte \`a gauche, \`a droite ou centr\'e.

\ulist\compact
\li Pour faire un texte ^{cadr\'e \`a gauche}, mettez un ressort horizontal
infiniment \'etirable \`a sa droite.
Ce ressort consommera tout l'espace suppl\'ementaire possible sur la ligne.
Vous pouvez utiliser la commande ^|\leftline| \ctsref{\leftline} 
ou la commande |\raggedright| \ctsref{\raggedright} pour faire cela.
\li Pour faire un texte ^{cadr\'e \`a droite}, mettez un ressort horizontal
infiniment \'etirable \`a sa gauche. Comme ci-dessus
ce ressort consommera tout l'espace suppl\'ementaire possible sur la ligne.
Vous pouvez utiliser la commande ^|\rightline| \ctsref{\rightline} 
pour faire cela.
\li Pour faire du ^{texte centr\'e}, mettez un ressort horizontal
infiniment \'etirable de chaque cot\'e.
Ces deux ressorts diviseront tout l'espace suppl\'ementaire de la ligne
\'egalement entre eux.
Vous pouvez utiliser la commande ^|\centerline| \ctsref{\centerline} pour faire cela.
\endulist

Vous pouvez aussi sp\'ecifier un ressort r\'etr\'ecissable infiniment
^^{ressort//infiniment \'etirable}
de la m\^eme mani\`ere. Un ressort r\'etr\'ecissable infiniment peut agir comme espace n\'egatif.
Notez que |fil|, etc., ne peuvent \^etre utilis\'es que pour sp\'ecifier l'\'etirement 
et le r\'etr\'ecissement du ressort---Ils ne peuvent \^etre utilis\'es pour sp\'ecifier 
sa taille normale.
\eix^^{\etirement}
\eix^^{r\'etr\'ecissement}
\endconcept


\concept {ressort inter-ligne}

Le \defterm{ressort inter-ligne} est le ressort que \TeX\ ins\`ere au d\'ebut de chaque
\refterm{bo\^\i te} dans une \refterm{liste verticale} sauf pour la premi\`ere.
Le ressort inter-ligne est normalement sp\'ecifi\'ee de fa\c con \`a maintenir une 
distance constante entre les lignes de base des bo\^\i tes.
Sa valeur est d\'etermin\'ee conjointement par les param\`etres ^|\baselineskip|,
^|\lineskip| et ^|\lineskiplimit| \ctsref{\baselineskip}.
\endconcept


\conceptindex {r\'etr\'ecissements}
\concept {r\'etr\'ecissement}

Voir \conceptcit{ressort}.
\endconcept


\concept {routine de sortie}

Quand \TeX\ a accumul\'e
au moins assez de mat\'e\-riel pour remplir une page, il choisi un point d'arr\^et
et place le mat\'eriel situ\'e avant le point d'arr\^et en |\box255|. Il appelle
alors la \defterm{routine de sortie} courante, qui compile le mat\'eriel et 
l'envoie \'eventuellement dans le \dvifile.
^^{\dvifile//mat\'eriel de la routine de sortie}
La routine de sortie peut faire d'autres actions, comme ins\'erer des ent\^etes,
des pieds de page et des notes de pieds de page.  
\refterm{\PlainTeX:\plainTeX} procure
une routine de sortie par d\'efaut qui ins\`ere un num\'ero de page centr\'e
en bas de chaque page.  
En procurant une routine de sortie diff\'erente vous pouvez cr\'eer des
effets tels que des sortie en double colonnes.
Vous pouvez imaginer les routines de sortie comme ayant une seule responsabilit\'e~:
disposer le mat\'eriel en |\box255| d'une fa\c con ou d'une autre.

La routine de sortie courante est d\'efinie par la valeur de ^|\output|
\ctsref{\output}, qui est une liste de \refterm{tokens:token}.  Quand \TeX\
est pr\^et \`a produire une page, il d\'eveloppe simplement la liste de tokens.

Vous pouvez faire quelque simples changements aux actions de la routine de
sortie de \plainTeX\ sans la modifier r\'eellement.  Par exemple, en assignant
une liste de \refterm{tokens:token} \`a |\headline| ou
|\footline| \ctsref{\footline} vous pouvez faire que \TeX\ produise un ent\^ete
ou un pied de page diff\'erent de l'ordinaire.

La routine de sortie est aussi responsable pour collecter 
toutes \refterm{insertions:insertion}~; en combinant ces insertions et 
toutes ``d\'ecorations'' telles que ent\^etes et pieds de page avec 
les contenus principaux de la page et empaqu\`ete tout ce mat\'eriel 
dans une bo\^\i te~; et \'eventuellement envoyer cette bo\^\i te dans 
le \dvifile\ 
^^{\dvifile//mat\'eriel de la routine de sortie}
avec la commande ^|\shipout|\ctsref{\shipout}.
Bien que cela soit ce que fait une routine de sortie le plus souvent,
une routine de sortie d'usage sp\'ecial peut agir diff\'eremment.
\endconcept


\conceptindex{s\'equences de contr\^ole}
\concept {s\'equence de contr\^ole}

Une \defterm{S\'equence de contr\^ole} est un nom pour une \refterm{command}e \TeX.
Une s\'equence de contr\^ole commence par un ^{caract\`ere d'\'echappement}, 
habituellement un antislash (|\|).
\indexchar \
Une s\'equence de contr\^ole prend une des deux formes~:

\ulist

\li Un \refterm{mot de contr\^ole} est une S\'equence de contr\^ole constitu\'ee 
d'un \refterm{caract\`ere d'\'echappement} suivi par une ou plusieurs lettres.
^^{mots de contr\^ole}
Le mot de contr\^ole se termine quand \TeX\ voit une non-lettre. Par exemple,
quand \TeX\ lit
`\hbox{|\hfill!visiblespace,!visiblespace!.the|}', il voit six 
\refterm{tokens:token}~:
Les s\'equences de  contr\^ole  `|\hfill|', virgule, espace, `|t|', `|h|', `|e|'. 
L'espace apr\`es `|\hfill|' termine la s\'equence de  contr\^ole  et
est absorb\'ee par \TeX\ quand il lit la s\'equence de contr\^ole.
(Pour le texte `|\hfill,!visiblespace!.the|', d'un autre cot\'e,
La virgule termine aussi la s\'equence de   contr\^ole  et compte comme un caract\`ere 
de bon aloi.)

\li Un \refterm{symbole de contr\^ole}
^^{symboles de contr\^ole}
est une s\'equence de   contr\^ole  constitu\'ee d'un
^{caract\`ere d'\'echappement} suivi par n'importe caract\`ere autre qu'une lettre---%
m\^eme un espace ou une fin de ligne.
Un symbole de  contr\^ole  est auto-d\'elimit\'e, c'est-\`a-dire, \TeX\ sait o\`u il termine sans
devoir regarder le caract\`ere qui le suit.
Le caract\`ere apr\`es un symbole de  contr\^ole  n'est jamais absorb\'e par le
symbole de contr\^ole.
\endulist
\noindent Voir \xrefpg{espace} pour plus d'information sur les espaces apr\`es 
des s\'equences de contr\^ole. 

\TeX\ procure une grand nombre de s\'equences de  contr\^ole  pr\'ed\'efinies.  Les
s\'equences de  contr\^ole  \refterm{primitives} sont construites dans le programme 
informatique \TeX\ et donc, sont accessibles sous toutes les formes de \TeX.
^^{primitive//s\'equence de contr\^ole}
D'autres s\'equences de  contr\^ole  sont fournies par \refterm{\plainTeX}, la
forme de \TeX\ d\'ecrite dans ce livre

Vous pouvez augmenter des s\'equences de contr\^ole pr\'ed\'efinies avec celles de 
votre cru, en utilisant des commandes telles que ^|\def| et ^|\let| pour les
d\'efinir. La \chapterref{eplain} de ce livre contient une collection de
d\'efinitions de s\'equence de contr\^ole que vous pouvez trouver utile. De plus, 
votre syst\`eme informatique peut vous rendre capable de cr\'eer une collection
de macros \TeX\ d\'evelopp\'ees localement.
\endconcept


\conceptindex{struts}
\concept strut

{\tighten
Un \defterm{strut} est une \refterm{bo\^\i te} invisible 
^^{\boites//invisible}
dont la largeur est \`a z\'ero et dont la hauteur et la profondeur sont l\'eg\`erement
sup\'erieure \`a celle d'une ligne de texte ``normale'' dans le contexte.  
Les struts sont pratiques pour obtenir un espacement vertical uniforme quand 
l'espacement de ligne usuel de \TeX, par exemple, dans une formule math\'ematique
ou dans un alignement horizontal o\`u vous sp\'ecifiez ^|\offinterlineskip|.
Parce qu'un strut est plus haut et plus profond que tout autre chose sur sa ligne,
il d\'etermine la hauteur et la profondeur de la ligne.
Vous pouvez produire un strut avec la commande ^|\strut| \ctsref{\strut} ou 
la commande ^|\mathstrut| \ctsref\mathstrut.
Vous pouvez utiliser |\strut| n'importe o\`u, mais vous ne pouvez utiliser 
|\mathstrut| que quand \TeX\ est en \refterm{mode} math\'ematique.
Un strut dans \plainTeX\ a une hauteur de 8.5\pt\ et une profondeur de 3.5\pt, 
tandis qu'un strut math\'ematique \`a la hauteur et la profondeur d'une
parenth\`ese ouvrante du \refterm{style} courant (donc il est plus petit pour
des indices ou des exposants).
\par}

Voici un exemple montrant comment vous devriez utiliser un strut~:
\csdisplay
\vbox{\hsize = 3in \raggedright
   \strut Here is the first of two paragraphs that we're
   setting in a much narrower line length.\strut}
\vbox{\hsize = 3in \raggedright
   \strut Here is the second of two paragraphs that we're
   setting in a much narrower line length.\strut}
|
Cette entr\'ee donne~:
\display{\vbox{
\vbox{\hsize = 3in \raggedright
\strut Here is the first of two paragraphs that we're setting
in a much narrower line length.\strut}
\vbox{\hsize = 3in \raggedright
\strut Here is the second of two paragraphs that we're setting
in a much narrower line length.\strut}
}}
\noindent
Sans les struts la \refterm{vbox} serait trop rapproch\'ee.  
Similairement, dans la formule~:
\csdisplay
$\overline{x\mathstrut} \otimes \overline{t\mathstrut}$
|
Les struts math\'ematiques font que les deux barres sont mises \`a la m\^eme hauteur m\^eme
si le `$x$' et le `$t$' ont des hauteurs diff\'erentes~:
\display{
$\overline{x\mathstrut} \otimes \overline{t\mathstrut}$
}
\vskip -\belowdisplayskip
\endconcept
%\nobreak

\conceptindex{styles}
\concept {style} 

Le mat\'eriel d'une formule math\'ematique est mis dans un des huit \defterm{styles},
d\'ependant du contexte.  Conna\^\i tre les styles peut \^etre utile si vous voulez 
mettre une partie d'une formule dans une taille de caract\`ere diff\'erente de
celle que \TeX\ a choisi en fonction de ses r\`egles usuelles.

%\eject
Les quatre styles primaires sont~:

\vdisplay{%
\halign{\refterm{# style}\hfil&\hskip .25in(#)\hfil\cr
display&formules affich\'ees sur une ligne seule\cr
text&formules englob\'ees dans du texte ordinaire\cr
script&exposants et indices\cr
scriptscript&indices d'indices, etc.\cr
}}

Les quatre autres styles sont dits ^{variantes \'etroites}.  Dans ces variantes 
les exposants ne sont pas mont\'es aussi haut que normalement et donc la formule
n\'ecessite moins d'espace vertical qu'ils le devraient.  Voir les \knuth{pages~140--141}{164--165} 
pour les d\'etails sur la fa\c con de \TeX\ de s\'electionner le style.

\TeX\ choisit une taille de caract\`ere en fonction du style~:

\ulist\compact
^^{style d'affichage}^^{style texte}
\li Le style d'affichage et le style texte sont mis en \refterm{taille texte}, comme
`$\rm ceci$'.

^^{style script}
\li le style script est mis en \refterm{taille script}, comme `$\scriptstyle
\rm ceci$'.

^^{style scriptscript}
\li le style scriptscript est mis en \refterm{taille scriptscript}, comme
`$\scriptscriptstyle \rm ceci$'.
\endulist

Voir \conceptcit{famille} pour plus d'informations \`a propos de ces trois tailles.

\TeX\ n'a pas de style ``scriptscriptscript'' parce qu'un tel style
serait souvent mis dans une taille de caract\`ere trop petite \`a lire.  \TeX\
par cons\'equent met les indices, exposants, etc. de troisi\`eme ordre en utilisant le
style scriptscript.

Parfois vous pourrez trouver que \TeX\ a mis une formule dans un style diff\'erent
de celui que vous pr\'ef\'erez.  Vous pouvez passer outre le choix de \TeX\ avec les
commandes ^|\textstyle|, ^|\displaystyle|, ^|\scriptstyle| et ^|\scriptscriptstyle|
\ctsref{\textstyle}.
\endconcept


\conceptindex{symboles de contr\^ole}
\concept {symbole de contr\^ole}

Un \defterm{symbole de contr\^ole} est une \refterm{s\'equence de contr\^ole} qui 
est constitu\'ee d'un \refterm{caract\`ere d'\'echappement} suivi d'un caract\`ere 
autre qu'une lettre---m\^eme un espace et une fin de ligne.
^^{caract\`ere d'\'echappement}
\endconcept


\concept {taille script} 

Une \defterm{taille script} d\'ecrit une des trois 
\refterm{polices:police} reli\'ees dans une famille.
^^{famille//taille script en}
Une taille script est plus petite que la \refterm{taille texte} mais plus 
grande que la 
\refterm{taille scriptscript}.  \TeX\ utilise la taille script pour les 
indices et les exposants, aussi bien que pour les num\'erateurs et les 
d\'enominateurs de fractions dans du texte.
\endconcept


\concept {taille scriptscript}

Une \defterm{taille scriptscript} d\'ecrit la plus petite taille des trois 
\refterm{polices:police} reli\'ees dans une famille.
^^{famille//taille scriptscript en}
\TeX\ utilise la \refterm{taille scriptscript} pour des indices, exposants,
num\'erateurs et d\'enominateurs de second rang.  Par exemple, \TeX\ utilisera
la taille scriptscript pour un indice d'indice ou pour un exposant d'exposant.
\endconcept


\concept {taille texte}

La \defterm{taille texte} d\'ecrit la plus grande des trois 
\refterm{polices:police} li\'ees dans une \refterm{famille}.
^^{famille//taille texte en}
\TeX\ utilise la taille texte pour des symboles ordinaires apparaissant en 
\refterm{mode math\'ematique}.
\endconcept


\conceptindex{tests conditionnels}
\concept {test conditionnel}

Un \defterm{test conditionnel} est une commande qui teste si une condition est
vraie ou non et 
demande \`a \TeX\ soit de d\'evelopper soit de sauter du texte, selon le cas.
La forme g\'en\'erale d'une test conditionnel est soit~:
\display{
{\tt \\if}$\alpha$\<true text>{\tt \\else}\<false text>{\tt \\fi}}
^^|\else|^^|\fi|
\noindent soit~:\hfil\
\display{
{\tt \\if}$\alpha$\<true text>{\tt \\fi}}
\noindent o\`u $\alpha$ sp\'ecifie le test particulier.
Par exemple, |\ifvmode| teste la condition que \TeX\
soit actuellement dans un \refterm{mode vertical}.
Si la condition est vraie, \TeX\ d\'eveloppe \<true text>.
Si la condition est fausse, \TeX\ d\'eveloppe \<false text> (s'il est pr\'esent).
Les tests conditionnels sont interpr\'et\'es dans l'\oesophage\ de \TeX
\seeconcept{\anatomy}, donc tous les \minref{token}s d\'eveloppables dans
le texte interpr\'et\'e ne sont d\'evelopp\'es qu'apr\`es que le test n'ait \'et\'e r\'esolu.
Les diff\'erents tests conditionnels sont expliqu\'es dans 
\headcit{Tests conditionnels}{conds}.

\endconcept


\concept {\TeXMeX}

(a)~Une vari\'et\'e de \TeX\  utilis\'ee pour composer des math\'e\-matiques dans les
pays d'Am\'erique Centrale.
(b)~Une cuisine tr\`es \'epic\'ee appr\'e\-ci\'ee par les \TeX\-ni\-ciens d'^{El Paso}. 
\endconcept


\conceptindex{justification}
\concept {texte justifi\'e} 

Du \defterm{texte justifi\'e} est du texte qui a \'et\'e compos\'e pour que les
deux marges soit align\'ees.  Du texte non justifi\'e, d'un autre cot\'e, a \'et\'e compos\'e
avec des marges ``d\'echir\'ees'' d'un ou des deux cot\'es.
Les documents saisis sur les vieilles machines \`a \'ecrire ont la plupart du temps
des marges droites d\'echir\'ees.
Bien que les documents produits par \TeX\ soient justifi\'es par d\'efaut,  vous
pouvez si vous voulez produire des documents (ou des suites de lignes) qui
ont la marge ^{droite}---ou ^{gauche d\'echir\'ee}.
Vous pouvez aussi demander \`a \TeX\ de centrer un suite de lignes, ce qui rend
les deux marges d\'echir\'ees.
^^{texte centr\'e}
Vous pouvez utiliser les commandes 
^|\leftskip|, ^|\rightskip| et ^|\raggedright|
(\pp \xrefn{\leftskip},~\xrefn{\raggedright}) pour cela.

Quand \TeX\ produit du texte justifi\'e, il a normalement besoin de r\'e\-tr\'ecir ou 
d'\'etirer les ressorts de chaque ligne pour que les marges soient align\'ees.
Quand \TeX\ produit du texte non justifi\'e, d'un autre cot\'e, il laisse les ressorts
de chaque ligne \`a leur taille naturelle.
Beaucoup de typographes pr\'ef\`erent le texte non justifi\'e parce que son espacement
inter-mot est plus uniforme.
\endconcept


\concept {texte math\'ematique}

Nous utilisons le terme \defterm{texte math\'ematique} pour faire r\'ef\'erence
\`a une formule math\'ematique mise dans une ligne de texte, c'est-\`a-dire, 
entour\'ee de |$|.
\ttidxref{$}
\TeX\ met du texte math\'ematique dans le \refterm{mode} texte math\'ematique .
\endconcept


\conceptindex{tokens}
\concept token

Un \defterm{token} est soit un simple caract\`ere attach\'e \`a un
\refterm{code  de cat\'egorie} ou une \refterm{s\'equence de contr\^ole}.  
\TeX\ lit les caract\`eres d'un fichier en utilisant ses yeux \seeconcept{\anatomy}
et groupe les caract\`eres en tokens en utilisant sa bouche.  Quand un token
atteint l'estomac de \TeX, \TeX\ l'interpr\`ete comme une \refterm{commande}
\`a moins qu'il fasse partie d'un argument d'une commande pr\'ec\'edente.
\endconcept


\conceptindex{unit\'es de mesure}
\concept {unit\'e de mesure}

Voir \conceptcit{dimension}.
\endconcept


\conceptindex{unit\'es math\'ematiques}
\concept {unit\'e math\'ematique}

Une \defterm{unit\'e math\'ematique}, cod\'ee `|mu|', est une unit\'e de distance
utilis\'ee pour sp\'ecifier un \refterm{ressort} dans les formules math\'e\-matiques.  Voir
le \conceptcit{muglue}.
\endconcept


\conceptindex{vbox}
\concept vbox

^^{listes verticales//vbox form\'ees \`a partir de}
Une \defterm{vbox} (bo\^\i te verticale) est une \refterm{bo\^\i te} que \TeX\ construit
en pla\c cant les \'el\'ements d'une \refterm{liste verticale} l'une apr\`es l'autre, de haut
en bas.  Une vbox, prise en tant qu'unit\'e, n'est ni fondamentalement horizontale
ni fondamentalement verticale, c'est-\`a-dire, elle peut appara\^\i tre comme un \'el\'ement
soit d'une liste verticale soit d'une \refterm{liste horizontale}.  Vous 
pouvez construire une vbox avec les commandes ^|\vbox| ou ^|\vtop| \ctsref{\vtop}.  
La diff\'erence est que pour |\vbox|, le \refterm{point de r\'ef\'erence} de la
vbox construite est d\'eriv\'e de celui du dernier (et souvent plus bas)
\'el\'ement constituant la liste, mais pour |\vtop|, c'est celui du premier
(et normalement plus haut) \'el\'ement constituant la liste.
\endconcept


\endconcepts

\endchapter
\byebye

