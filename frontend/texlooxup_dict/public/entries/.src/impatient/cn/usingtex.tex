% -*- coding: utf-8 -*-
% This is part of the book TeX for the Impatient.
% Copyright (C) 2003 Paul W. Abrahams, Kathryn A. Hargreaves, Karl Berry.
% See file fdl.tex for copying conditions.

\input macros
\chapter{使用 \TeX}

\chapterdef{usingtex}

% Avoid underfull box complaint about the empty paragraph
% that precedes the first section heading.
%
\def\par{{\parfillskip = 0pt plus 1fil\endgraf}\let\par=\endgraf}
\vglue-\abovesectionskip % we've skipped enough already
\vskip0pt % Make \combineskips work.

%\section Turning input into ink
\section 从键盘输入变到油墨

%\subsection Programs and files you need
\subsection 所需的程序和文件

%In order to produce a \TeX\ document, you'll need to run the \TeX\
%program and several related programs as well.  You'll also need
%supporting files for \TeX\ and possibly for these other programs.  In
%this book we can tell you about \TeX, but we can't tell you about the
%other programs and the supporting files except in very general terms
%because they depend on your local \TeX\ environment.  The people who
%provide you with \TeX\ should be able to supply you with what we call
%\emph{local information}.
%\pix^^{local information}
%The local information tells you how to
%start up \TeX, how to use the related programs, and how to gain access
%to the supporting files.
为了制作一个 \TeX\ 文档, 你必需运行 \TeX\ 及其相关软件.
你同时也需要一些 \TeX\ 及其相关软件所用到的辅助文件.
在这本书中, 我们仅仅谈论 \TeX\ 和一些通用的软件及辅助文件,
不过我们不会介绍其它的, 因为它们仅仅和你自己的 \TeX\ 环境相关.
为你提供 \TeX\ 的人可以为你提供我们所说的\emph{本地信息}.
\pix^^{本地信息}
这些本地信息可以告诉你如何去启动 \TeX, 如何去使用相关的程序,
以及如何去访问这些所用到的辅助文件.

%Input to \TeX\ consists of a file of ordinary text that you can prepare
%with a ^{text editor}.  A \TeX\ input file, unlike an input file for a
%typical word processor, doesn't ordinarily contain any invisible
%^{control characters}.
%Everything that \TeX\ sees is visible to you too if you
%look at a listing of the file.
你可以使用一个 ^{文本编辑器} 来编写一个可以输入到 \TeX\ 的普通文本文件.
一个 \TeX\ 输入文件和一个文字处理软件的输入文件不同, 它通常并不存在任何不可见的^{控制字符}.
任何 \TeX\ 读到的字符对你来说, 都是你可以看到的.

%Your input file may turn out to be little more than a skeleton that
%calls for other input files.  \TeX\ users often organize large documents
%such as books this way.  You can use the ^|\input| command (\xref\input)
%to embed one input file within another.  In particular, you can use
%|\input| to incorporate files containing \emph{macro definitions}---%
%^^{macros//in auxiliary files}
%auxiliary definitions that enhance \TeX's capabilities.
%If any macro files are available at your \TeX\ installation, the local
%information about \TeX\ should tell you how to get at the macro files
%and what they can do for you.  The standard form of \TeX,
%the one described in this book, incorporates a
%collection of macros and other definitions known as ^{\plainTeX}
%(\xref{\plainTeX}).
你的输入文件, 可以仅仅是一个引用其它输入文件的框架.
\TeX\ 用户往往把大的文档, 比如说这本书, 组织成这种形式.
你可以使用 ^|\input| 命令 (\xref\input) 来把一个输入文件篏入到另一个中.
特别地, 你可以使用 |\input|来调入包含\emph{宏定义}的文件,
^^{宏//在辅助文件中}
宏定义是一些辅助的定义, 它可以来增强 \TeX\ 的功能.
如果你的 \TeX\ 系统中包含任何的宏文件,
和 \TeX\ 相关的本地信息会告诉你如何得到这些宏文件,
以及它们的功能.
\TeX\ 的标准安装形式, 也就是在这本书中所描述的, 
包括了一个被称为 ^{\plainTeX} (\xref{\plainTeX}) 的宏集.

%When \TeX\ processes your document, it produces a file called the
%^{\dvifile}.  The abbreviation ``|dvi|'' stands for ``device
%independent''.  The abbreviation was chosen because the information in
%the \dvifile\ is independent of the device that you use to print or
%display your document.
当 \TeX\ 处理你的文档时, 它会产生一种叫做  ^{\dvifile} 的文件.
其中, ``|dvi|'' 的全称为 ``device independent (设备无关)''.
之所以选用这个缩写, 是因为 \dvifile\ 中所包含的信息和你的打印或显示设备无关.

%To print your document or view it with a \emph{previewer},
%^^{previewer}
%you need to process the ^{\dvifile} with
%a \emph{device driver\/} program.
%^^{device drivers}
%(A previewer is a program  that
%enables you to see on a screen some approximation of what the typeset
%output will look like.)
%Different output devices usually
%require different device drivers.
%After running the device driver,
%you may also need to transfer the output of the device driver to the
%printer or other output device.
%^^{printers} ^^{output devices}
%The local information about \TeX\ should tell
%you how to get the correct device driver and use it.
当你需要打印或使用\emph{阅览器}查看你的文档时, ^^{阅览器}
你需要使用相应的\emph{设备驱动}来处理之. ^^{设备驱动}
(阅览器是一个程序, 它能把和打印出来的结果差不多的排版内容显示在屏幕上.)
不同的输出设备往往需要不同的设备驱动. 在运行设备驱动以后,
你还需要把设备驱动输出的内容传输到打印机或者其它输出设备上.
^^{打印机} ^^{输出设备}
和 \TeX\ 相关的本地信息会告诉你如何得到正确的设备驱动和如何使用它.

%Since \TeX\ has no built-in knowledge of particular fonts, it uses
%\emph{font files}
%^^{font files}
%to obtain information about the fonts used in your
%document.  The font files should also be part of your local \TeX\
%environment.  Each font normally requires two files: one containing the
%dimensions of the characters in the font (the \emph{metrics file})
%^^{metrics file}
%and one containing the shapes of the characters (the \emph{shape file}).
%^^{shape file}
%Magnified versions of a font share the metrics file but have
%different shape files. ^^{magnification} Metrics files are sometimes
%referred to as ^{\tfmfile}s, and the different varieties of shape files
%are sometimes referred to as ^{\pkfile}s, ^{\pxlfile}s, and ^{\gffile}s.
%These names correspond to the names of the files that \TeX\ and its
%companion programs use.  For example, |cmr10.tfm| is the metrics file
%for the |cmr10| font (10-point Computer Modern Roman).
因为 \TeX\ 内部并没有任何关于一个特定字体的任何信息,
它使用\emph{字体文件}
^^{字体文件}
来得到你文档中所使用的字体的信息.
字体文件也是你本地 \TeX\ 环境的一个组成部分.
一个字体一般对应两个文件:
一个文件 (\emph{字体度量文件}) ^^{度量文件} 包括了字体中每个字符的大小信息,
另一个文件 (\emph{字体轮廓文件}) ^^{轮廓文件} 则描述了字体中字符的形状.
一个字体的缩放版本使用同样的字体度量文件, 不过并不使用相同的字体轮廓文件.
^^{放大率}
字体度量文件往往用来指称 ^{\tfmfile},
而字体轮廓文件则往往用来指称 ^{\pkfile}, ^{\pxlfile} 和 ^{\gffile} 等不同的种类.
这些名称和 \TeX\ 以及相关软件所使用的文件的文件名所对应.
比如, |cmr10.tfm| 为 |cmr10| 字体的字体度量文件 (10-点的计算机现代字体).

%\TeX\ itself uses only the metrics file, since it doesn't care what the
%characters look like but only how much space they occupy.  The device
%driver ordinarily uses the shape file, since it's responsible for
%creating the printed image of each typeset character.  Some device
%drivers need to use the metrics file as well.  Some device drivers can
%utilize fonts that are resident in a printer and don't need shape files
%for those fonts.
%\secondprinting{\vfill\eject}
\TeX\ 本身仅使用字体度量文件, 因为它只关心这个字体的字符占用了多大的空间,
而并不关心这个字体的外形是什么样子的.
设备驱动则往往需要使用字体轮廓文件,
因为它需要负责把每个排出的字符变成印出的图形.
某些设备驱动也需要使用字体度量文件.
一些设备驱动可以使用打印机中包含的字体,
所以可以不需要这些字体的字体轮廓文件.
%\secondprinting{\vfill\eject}


%\subsection Running {\TeX}
\subsection{运行 \TeX}

%\bix^^{running \TeX}
%You can run \TeX\ on an input file |screed.tex| by typing
%^^{input files}
%something like `|run tex|' or just `|tex|' (check your local information).
%\TeX\ will respond with something like:
%% 4/23/90 is Shakespeare's 426th birthday, and Karl's 26th.
%\csdisplay
%This is TeX, Version 3.0 (preloaded format=plain 90.4.23)
%**
%|
%The ``preloaded format'' here refers to a predigested form of the
%^{\plainTeX} macros that come with \TeX.
%You can now type `|screed|' to get \TeX\ to process your file.
%When it's done, you'll see something like:
%\csdisplay
%(screed.tex [1] [2] [3] )
%Output written on screed.dvi (3 pages, 400 bytes).
%Transcript written on screed.log.
%|
%displayed on your terminal, or printed in the record of your
%run if you're not working at a terminal.  Most of this output is
%self-explanatory.
%The numbers in brackets are page numbers that \TeX\ displays when it
%ships out each page of your document to the \dvifile.
%\TeX\ will usually assume an
%extension `|.tex|' to an input file name
%if the input file name you gave doesn't
%have an extension.  For some forms of \TeX\ you may be able to
%invoke \TeX\ directly for an input file by typing:
%\csdisplay
%tex screed
%|
%or something like this.
\bix^^{运行 \TeX}
你可以通过输入类似 `|run tex|' 或 `|tex|' (查看你的本地信息) 来运行 \TeX\ 来处理一个输入文件 |screed.tex|.
^^{输入文件}
\TeX\ 会作出如下显示
% 4/23/90 is Shakespeare's 426th birthday, and Karl's 26th.
\csdisplay
This is TeX, Version 3.0 (preloaded format=plain 90.4.23)
**
|
在这里, ``preloaded format'' (已预先载入的格式) 指的是 \TeX\ 提供的 ^{\plainTeX} 宏集的一个己预先产生的形式.
现在你可以输入 `|screed|' 来让 \TeX\ 来处理你的文件.
当处理结束时, 你可以在终端或者打印机印出的记录上 (如果你不使用终端) 得到类似下面的输出:
\csdisplay
(screed.tex [1] [2] [3] )
Output written on screed.dvi (3 pages, 400 bytes).
Transcript written on screed.log.
|
这个输出大部分是不言自明的.
在括号中的数字是 \TeX\ 在把每页内容输入 \dvifile\ 时所显示的当前页码.
当你提供的名件名不含扩展名时, \TeX\ 往往会把你的文件扩展名当作 `|.tex|'.
在某些 \TeX\ 环境中, 你可能可以直接执行类似下面的语句:
\csdisplay
tex screed
|
来直接处理文档.

%Instead of providing your \TeX\ input from a file, you can type it directly at
%your terminal.  To do so, type `^|\relax|' instead of `|screed|' at the
%`|**|' prompt.
%\TeX\ will now prompt you with a `|*|' for each line of input and interpret
%each line of input as it sees it.
%To terminate the input, type a command such as `|\bye|' that tells \TeX\
%you're done.
%Direct input is sometimes a handy way of experimenting with \TeX.
\TeX\ 不仅可以读取文件输入,也可以直接读取你在终端中的输入:
只需将 `|**|' 提示后的输入从 `|screed|' 改为 `^|\relax|'。
\TeX\ 将在你输入的各行前面给出 `|*|' 提示,
并且逐行解释你的输入。
输入类似 `|\bye|' 这样的命令可以结束输入。
在试验 \TeX\ 时直接输入有时更方便。

%When your input file contains other embedded input files, the displayed
%information indicates when \TeX\ begins and ends processing each
%embedded file.
%^^{input files//embedded}
%\xrdef{infiles}
%\TeX\ displays a left parenthesis and the file name
%when it starts working on a file and displays the corresponding right
%parenthesis when it's done with the file.
%If you get any ^{error messages} in the displayed output, you can match
%them with a file by looking for the most recent unclosed left parenthesis.
当你的输入文件包含其他嵌入的输入文件时,
显示的信息中表明了 \TeX\ 何时开始和结束处理每个嵌入文件。
^^{输入文件//嵌入输入文件}
\xrdef{infiles}
当 \TeX\ 开始处理一个文件时它显示一个左圆括号和该文件名,
而结束时显示一个对应的右圆括号。
如果在所显示的输出中有任何^{错误信息},
你就能从最接近的未配对左圆括号中确定是哪个文件引起的。

%For a more complete explanation of how to run \TeX,
%see \knuth{Chapter~6} and your ^{local information}.
%\eix^^{running \TeX}
要得到如何运行\TeX\ 的更完整解释,
可以见\knuth{第~6~章}以及你的^{本地信息}.
\eix^^{运行 \TeX}


%\section Preparing an input file
\section 准备输入文件

%In this section we explain some of the conventions that you must follow in
%preparing input for \TeX\null.  Some of the information given here also
%appears in the examples in \chapterref{examples} of this book.
%^^{input, preparing}
在这一节中,我们解释准备 \TeX\ 输入文件时必须遵循的一些约定。
其中的有些信息同样出现在本书\chapterref{examples}的例子中。
^^{输入文件//准备输入文件}

%\subsection Commands and control sequences
\subsection 命令和控制序列

%\bix^^{commands}
%\bix^^{control sequences}
%Input to \TeX\ consists of a sequence of commands that tell \TeX\ how to
%typeset your document.  Most characters act as commands of a particularly
%simple kind: ``typeset me''.  The letter `|a|', for instance, is a
%command to typeset an `a'.  But there's another kind of command---a
%\emph{control sequence}---that gives \TeX\ a more elaborate
%instruction.  A control sequence ordinarily starts with a backslash
%(|\|), though you can change that convention if you need to.
%\xrdef{@backslash}
%For instance, the input:
\bix^^{命令}
\bix^^{控制序列}
\TeX\ 的输入由一系列指引 \TeX\ 如何排版文档的命令组成。
大部分字符如同一种特别简单的命令:``排版我''。
比如,字母 `|a|' 是作为命令将排版出 `a'。
但有另一种命令——\emph{控制序列}(control sequence)——将给 \TeX\ 提供更细致的指令。
控制序列通常以反斜杠(|\|)开头,但必要时你也可以更改此约定。
\xrdef{@backslash}
例如,下列输入:

%\csdisplay
%She plunged a dagger (\dag) into the villain's heart.
%|
%contains the control sequence |\dag|; it produces the typeset output:
%\display{%
%She plunged a dagger (\dag) into the villain's heart.
%}
%\noindent Everything in this example except for the |\dag| and the spaces
%acts like a ``typeset me'' command.  We'll explain more about spaces
%on \xrefpg{spaces}.
\csdisplay
She plunged a dagger (\dag) into the villain's heart.
|
包含一个控制序列|\dag|;它排版出下列的输出:
\display{%
She plunged a dagger (\dag) into the villain's heart.
}
\noindent 在这个例子中,
除 |\dag| 和空格之外的每个字符都如同一个``排版我''命令。
在\xrefpg{spaces}中我们将更详细地解释空格的作用。

%There are two kinds of control sequences: \emph{control words}
%^^{control words}
%and \emph{control symbols}:
%^^{control symbols}
%\ulist\compact
%\li A control word consists of a
%backslash followed by one or more letters, e.g., `|\dag|'.
%The first character that isn't a letter marks the end
%of the control word.
%\li A control symbol consists of a backslash followed by a single character
%that isn't a letter, e.g., `|\$|'.
%The character can be a space or even the end of a line (which is a perfectly
%legitimate character).
%\endulist
控制序列可分为\emph{控制词}(control word)^^{控制词}%
和\emph{控制符}(control symbol)^^{控制符}这两种类型:
\ulist\compact
\li 控制词由反斜杠后跟一个或多个字母组成,例如`|\dag|'。
其后用一个非字母字符结束该控制词。
\li 控制符由反斜杠后跟一个单个的非字母字符组成,例如`|\$|'。
该非字母字符可以是空格符,甚或是行结束符(这也是合乎规则的字符)。
\endulist
%\noindent
%A control word (but not a control symbol)
%absorbs any spaces or ends of line that follow it.
%^^{control sequences//absorbing spaces}
%If you don't want to lose a space after a control word,
%follow the control sequence with a ^{control space}
%(|\!visiblespace|) or with `|{}|'.  Thus either:
%\csdisplay
%The wonders of \TeX\!visiblespace!.shall never cease!!
%|
%or:\hfil\
%\csdisplay
%The wonders of \TeX{} shall never cease!!
%|
%produces:
%\display{%
%The wonders of \TeX{} shall never cease!
%}
%\noindent rather than:
%\display{%
%The wonders of \TeX shall never cease!
%}
%\noindent
%which is what you'd get if you left out the `|\|\visiblespace'
%or the `|{}|'.
\noindent
控制词后的任何空格或行结束符将被忽略(但控制符后不会这样)。
^^{控制序列//忽略空格}
如果你真要在控制词后得到一个空格,
可以在该控制词后键入一个^{控制空格}(|\!visiblespace|)或者`|{}|'。
因而这样输入:
\csdisplay
The wonders of \TeX\!visiblespace!.shall never cease!!
|
或者这样输入:\hfil\
\csdisplay
The wonders of \TeX{} shall never cease!!
|
将得到同样的结果:
\display{%
The wonders of \TeX{} shall never cease!
}
\noindent 作为对比,下面的结果:
\display{%
The wonders of \TeX shall never cease!
}
\noindent
是你去掉上面输入中的`|\|\visiblespace'或者`|{}|'后得到的。

%Don't run a control word together with the text that follows it---\TeX\
%won't know where the control word ends.  For instance, the |\c| control
%sequence places a cedilla accent on the character that follows it.  The
%French word {\it gar\c con\/} must be typed as
%`|gar\c!visiblespace!.con|', not `|gar\ccon|'; if you write the latter,
%\TeX\ will complain about an undefined control sequence |\ccon|.
不要将控制词和其后的文本连在一起---这样\TeX\ 无法知道控制词在哪里结束。
比如,控制词|\c|给其后的字符加上软音符。
要得到法文单词{\it gar\c con\/},你必须输入`|gar\c!visiblespace!.con|',
而不是`|gar\ccon|';若你输入后者,\TeX\ 将抱怨有未定义的控制序列|\ccon|。

%A control symbol, on the other hand, doesn't absorb anything that
%follows it.  Thus you must type `\$13.56' as `|\$13.56|', not
%`|\$!vs13.56|'; the latter form would produce `\hbox{\$ 13.56}'.
%However, those accenting commands that are named by control symbols are
%defined in such a way that they produce the effect of absorbing a
%following space.  Thus, for example, you can type the French word {\it
%d\'eshabiller\/} either as `|d\'eshabiller|' or as
%`|d\'!visiblespace!.eshabiller|'.
另一方面,控制符后面的任何字符都不会被忽略。
因此,要得到`\$13.56'你必须键入`|\$13.56|',
而不是`|\$!vs13.56|';后者将生成`\hbox{\$ 13.56}'。
然而,那些作为控制符的重音命令,是以忽略其后空格的方式定义的。
例如,要得到法文单词{\it d\'eshabiller\/},键入`|d\'eshabiller|',
或者`|d\'!visiblespace!.eshabiller|'都可以。

%Every control sequence is also a command,
%but not the other way around.
%^^{commands//versus control sequences}
%^^{control sequences//versus commands}
%For instance, the letter `|N|'
%is a command, but it isn't a control sequence.
%In this book we ordinarily use ``command'' rather than
%``control sequence'' when either term would do.
%We use ``control sequence'' when we want to emphasize aspects of \TeX\
%syntax that don't apply to commands in general.
每个控制序列同时也是一个命令,但反之未必。
^^{命令//与控制序列对比}
^^{控制序列//与命令对比}
例如字母`|N|'是一个命令,但它不是一个控制序列。
在本书里面,当两者都适用时我们用``命令''而不用``控制序列''。
在强调 \TeX\ 的语法层面时,我们用``控制序列'',
因为此时用``命令''一般并不合适。

%\eix^^{control sequences}
%\eix^^{commands}
\eix^^{控制序列}
\eix^^{命令}

%\subsection Arguments
\subsection 参量

%\xrdef{arg1}
%Some commands need to be followed by one or more
%\emph{arguments} ^^{arguments}
%that help to determine what the command does.
%For instance, the |\vskip| command, which
%tells \TeX\ to skip down (or up) the page,
%expects an argument specifying how much space to skip.  To skip
%down two inches, you would type `|\vskip 2in|', where |2in|
%is the argument of |\vskip|.
\xrdef{arg1}
有些命令的运行依赖于其后的一个或多个\emph{参量}(argument)^^{参量}。
例如,|\vskip|命令告诉 \TeX\ 在页面中往上(或往下)跳过一定间距,
它需要一个参量来指定所跳过间距的大小。
要往下跳过两英寸,你需要键入`|\vskip 2in|',
其中的|2in|就是|\vskip|的参量。

%Different commands expect different kinds of arguments.  Many commands
%expect dimensions, such as the |2in| in the example above.
%Some commands, particularly those defined by macros,
%expect arguments that are either a single character or some
%text enclosed in braces.
%Yet others require that their arguments be enclosed in braces, i.e.,
%they don't accept single-character arguments.
%The description of each command in this book tells you what kinds of arguments,
%if any, the command expects.
%In some cases, required braces define a group (see \xref{bracegroup}).
不同的命令要求不同类型的参量。
很多命令的参量要求是长度值,比如上面例子中的|2in|。
有些命令,特别是用宏定义的命令,
其参量可以是单个字符或者放在花括号中的文本。
而对其他有些命令,其参量只能放在花括号中,而不能是单个字符。
在本书对各个命令的描述中,若一个命令需要参量,
我们都说明了该命令所需参量的类型。
在有些情形中,命令的参量所需的花括号定义了一个编组(见\xref{bracegroup})。

%\secondprinting{\vfill\eject}


%\subsection Parameters
\subsection 参数

%\xrdef{introparms}
%Some commands are parameters (\xref{parameter}).
%^^{parameters//as commands}
%You can use a parameter in either of two ways:
%\olist
%\li You can use the value of a parameter
%as an argument to another command.  For example, the command
%\hbox{|\vskip\parskip|}
%causes a vertical skip by the value of the |\parskip| (paragraph skip)
%glue parameter.
%\li You can change the value of the parameter by assigning
%something to it.  For example, the assignment \hbox{|\hbadness=200|}
%causes the value of the |\hbadness| number parameter to be $200$.
%\endolist
%\noindent
%We also use the term ``parameter'' to refer to entities such as |\pageno|
%that are actually registers but behave just like parameters.
%^^{registers//parameters as}
\xrdef{introparms}
有些命令实际上是参数(parameter,见\xref{参数})。
^^{参数//作为命令}
你可以按下面两种方式使用这种参数:
\olist
\li 将该参数的值用作其他命令的参量。
例如 \hbox{|\vskip\parskip|} 这个命令生成大小%
等于粘连参数|\parskip|(段落间距)的竖直间距。
\li 通过赋值修改该参数的值。
例如 \hbox{|\hbadness=200|} 这个赋值使得|\hbadness|参数的值变为$200$.
\endolist
\noindent
有些命令,例如|\pageno|,虽然是寄存器但和参数的用法一样。
因此我们同样使用``参数''这个术语来指代这些命令。
^^{寄存器//作为参数}

%Some commands are names of tables.  These commands are used like
%parameters, except that they require an additional argument that
%specifies a particular entry in the table. For example, |\catcode| names
%a table of category codes (\xref{category code}). Thus
%the command
%\hbox{|\catcode`~=13|} sets the category code of the `|~|'
%character to $13$.
有些命令是表名。这些命令的用法和参数类似,
只是它们需要额外的参量指定表格项。
例如,|\catcode| 是类别码表格的名称(\xref{类别码})。
因此,\hbox{|\catcode`~=13|} 这个命令设置字符`|~|'的类别码为$13$。


%\subsection Spaces
\subsection 空格

%\xrdef{spaces}
%\bix^^{spaces}
%You can freely use extra spaces in your input.  Under nearly all circumstances
%\TeX\ treats several spaces in a row as being equivalent to a
%single space.  For instance, it doesn't matter whether you put one space
%or two spaces after a ^{period} in your input.  Whichever you do, \TeX\
%performs its end-of-sentence maneuvers and leaves the appropriate
%(in most cases) amount of space after the period.
%\TeX\ also treats the end of an input line as equivalent to a space.
%Thus you can end your input lines wherever it's convenient---%
%\TeX\ makes input
%lines into
%paragraphs in the same way no matter where the line breaks are in your
%input.
\xrdef{spaces}
\bix^^{空格}
在输入文件中你可以任意加上额外的空格。
在几乎所有情形中 \TeX\ 都将同一行的连续多个空格看成一个空格。
例如,在^{句号}后加上一个或两个空格是没有差别的。
不论按照哪种写法,\TeX\ 都会执行其句子结束操作,
并在句号后加上适当(在多数情形下)大小的间隔。
\TeX\ 同样将输入中的行结束符视为一个空格。
因此在输入时你可以在方便的地方换行——%
\TeX\ 按照此方式将各输入行转换为段落,而不在乎输入中的换行位置。

%A blank line in your input marks the end of a paragraph.
%^^{paragraphs//ending}
%Several blank lines are equivalent to a single one.
输入中的一个空行结束当前段落。
^^{段落//结束段落}
多个空行也等同于一个空行。

%\TeX\ ignores input spaces within math formulas (see below).  Thus you can
%include or omit spaces anywhere within a math formula---\TeX\ doesn't care.
%Even within a math formula, however,
%you must not run a control word together with a following letter.
\TeX\ 忽略数学公式中的空格(见后面)。
因此,你可以加上或者省去公式中的空格——\TeX\ 不在乎这个。
但即使在数学公式中,你也不要将控制词和其后的字母连在一起。

%If you are defining your own macros, you need to be particularly careful about
%where you put ends of line in their definitions.
%It's all too easy to define a macro that produces an
%^{unwanted space} in addition to whatever else it's supposed to produce.
%We discuss this problem elsewhere since it's somewhat
%technical; see \xrefpg{unwantedspace}.
在定义自己的宏时,你必须小心定义中的行结束符位置。
因为一不小心该宏就在你希望的空格之外生成了^{不需要的空格}。
因为这是个有些技术性的问题,我们在其他地方也讨论到此问题;见\xrefpg{unwantedspace}。

%A space or its equivalent between two words in your input doesn't simply turn
%into a space character in your output.
%A few of these input spaces turn into ends of lines
%in the output,
%since input lines generally don't correspond to output lines.
%The others turn into spaces of variable width called ``glue'' (\xref{glue}),
%which has a natural size (the size it ``wants to be'')
%but can stretch or shrink.
%When \TeX\ is typesetting a paragraph
%that is supposed to have an even right margin (the usual
%case), it adjusts the widths of the glue in each line
%to get the lines to end at the margin.
%(The last line of a paragraph is an exception, since it isn't ordinarily
%required to end at the right margin.)
输入文件中两单词间的一个空格或其等价字符,
并不简单地变成输出中的一个空格。
由于输入行和输出行通常不会正好对应,
输入中的这些空格有的变成输出中的行结束符,
另外有些空格则变成宽度可变的间隔即``粘连'' (\xref{粘连});
粘连有其自然宽度(它希望的宽度),但也可以伸展或者收缩。
\TeX\ 排版要求右页边对齐(通常如此)的段落时,
将调整各行的粘连的宽度以让各行正好在页边结束。%
(段落的最后一行是个例外,因为一般不要求它在右页边结束。)

%You can prevent an input space from turning into an end of line by using a
%^{tie} (^|~|).
%For example, you wouldn't want \TeX\ to put a line break between the
%`Fig.' and `8' of `Fig.~8'.
%By typing `|Fig.~8|' you can prevent such a line break.
%\eix^^{spaces}
%\needspace{2in}
要阻止输入中的空格在输出中变成行结束符,你可以用^{带子} (^|~|)。
比如,你不会希望 \TeX\ 在`Fig.~8'的 `Fig.' 和 `8' 之间断行。
输入 `|Fig.~8|' 就可以禁止这样断行。
\eix^^{空格}
%\needspace{2in}

%\subsection Comments
\subsection 注释

%\xrdef{comments}
%\pix\bix^^{comments}
%You can include comments in your \TeX\ input.
%When \TeX\ sees a comment it just passes over it, so
%what's in a comment doesn't affect your typeset document in any way.
%Comments are useful for
%providing extra information about what's in your input file.
%For example:
%\csdisplay
%% ========= Start of Section `Hedgehog' =========
%|
\xrdef{comments}
\pix\bix^^{注释}
在\TeX\ 输入中可以包含注释。\TeX\ 跳过它所碰到的注释,
因此注释中的内容不会以任何方式影响到排版的文档。
注释用于在输入文件中提供更多的信息。
例如:
\csdisplay
% ========= Start of Section `Hedgehog' =========
|

%{\indexchar % }%
%A comment starts with a percent sign (|%|) and extends to the end of the
%input line.
%\TeX\ ignores not just the comment but the end of the line as well, so
%comments have another very
%important use: connecting two lines so that the end of line
%^^{line breaks//deleting}
%between them is invisible to \TeX\ and doesn't generate
%an output space or an end of line.
%For instance, if you type:
%\csdisplay
%A fool with a spread%
%sheet is still a fool.
%|
%you'll get:
%\display{
%A fool with a spread%
%sheet is still a fool.
%}
%\eix^^{comments}
{\indexchar % }%
注释以百分号(|%|)开始,一直延续到该输入行结束。
\TeX\ 既忽略注释也忽略该行的结束符,
因此注释有另外的重要用法:将两个输入行连接起来,
^^{断行//删除断行}
使得 \TeX\ 看不到两行间的行结束符,
从而不会在输出中生成空格或换行。
例如,对如下的输入:
\csdisplay
A fool with a spread%
sheet is still a fool.
|
你得到的输出为:
\display{
A fool with a spread%
sheet is still a fool.
}
\eix^^{注释}

%\subsection Punctuation
\subsection 标点

%\null
%\xrdef{periodspacing}
%\TeX\ normally adds some extra space after what it thinks is a
%^{punctuation} mark at the end of a sentence,
%namely, `^|.|', `^|?|', or `|!!|' \indexchar !
%^^{period} ^^{question mark} ^^{exclamation point}
%followed by an input space.
%\TeX\ doesn't add
%the extra space if the punctuation mark follows
%a capital letter, though, because it assumes the capital
%letter to be an initial in someone's name.
%You can force the extra space where it wouldn't otherwise occur by
%typing something like:
%\csdisplay
%A computer from IBM\null?
%|
\null
\xrdef{periodspacing}
若 `^|.|',`^|?|' 或 `|!!|' 之后有个输入空格,\indexchar !
^^{句号} ^^{问号} ^^{感叹号}
\TeX\ 认为它是句末^{标点}符号,并在其后添加额外间隔。
但若该标点符号之前为大写字母,\TeX\ 就不会添加额外间隔,
因为它认为大写字母是人名的首字母。
如若不然你可以强制生成额外间隔,例如:
\csdisplay
A computer from IBM\null?
|
%The |\null| doesn't produce any output, but it does prevent \TeX\
%from associating the capital `M' with the question mark.
%On the other hand, you can cancel the
%extra space where it doesn't belong by typing a control space
%after the punctuation mark, e.g.:
%\csdisplay
%Proc.\!visiblespace!.Royal Acad.\!visiblespace!.of Twits
%|
%so that you'll get:
%\display{Proc.\ Royal Acad.\ of Twits}
%\noindent rather than:
%\display{Proc. Royal Acad. of Twits}
命令 |\null| 不生成任何输出,
但它阻止 \TeX\ 将大写字母`M'和问号结合在一起。
反过来,如果某标点符号不为句末标点,
你也能够取消该标点之后的额外间隔;
这可以通过在其后键入控制空格达到,例如:
\csdisplay
Proc.\!visiblespace!.Royal Acad.\!visiblespace!.of Twits
|
得到的正确结果为:
\display{Proc.\ Royal Acad.\ of Twits}
\noindent 而去掉控制空格后的结果是:
\display{Proc. Royal Acad. of Twits}

%Some people prefer not to leave more space after punctuation at the
%end of a sentence.  You can get this effect with the
%^|\frenchspacing| command (\xref\frenchspacing).
%|\frenchspacing| is often recommended for ^{bibliographies}.
有些人更喜欢不在句末标点后留下更多间隔。
利用 ^|\frenchspacing| 命令(\xref\frenchspacing )你就可达成此效果。
在^{参考文献}中通常建议使用这个 |\frenchspacing| 命令。

%For single ^{quotation marks}, you should use the left and right
%single quotes
%(|`| and |'|) on your keyboard.  For left and right
%double quotation marks, use two left single
%quotes or two right single quotes (|``| or |''|) rather
%than the double quote (|"|) on your keyboard.
%The keyboard double quote
%will in fact give you a right double quotation mark in
%many fonts, but the two right single quotes
%are the preferred \TeX\ style.
%For example:
对于单^{引号}, 用键盘上的左右单引号(|`| 和 |'|)就可以得到。
对于左双引号或右双引号,
需要用两个左单引号(|``|)或右单引号(|''|)得到,
不能直接用键盘上的双引号(|"|)。
键盘上的双引号在多数字体中将得到一个右双引号,
但两个右单引号是 \TeX\ 中首选的写法。例如:

%\vbox{%
%\csdisplay
%There is no `q' in this sentence.
%``Talk, child,'' said the Unicorn.
%She said, ``\thinspace`Enough!!', he said.''
%|
%}%
%These three lines yield:
%\display{\par\restoreplainTeX
%There is no `q' in this sentence.
%\par ``Talk, child,'' said the Unicorn.
%\par She said, ``\thinspace`Enough!', he said.''
%}
%\noindent
%The |\thinspace| in the third input line prevents
%the single quotation mark from coming
%too close to the double quotation marks.
%Without it, you'd just see three
%nearly equally spaced quotation marks in a row.
\vbox{%
\csdisplay
There is no `q' in this sentence.
``Talk, child,'' said the Unicorn.
She said, ``\thinspace`Enough!!', he said.''
|
}%
这三行输入得到的结果为:
\display{\par\restoreplainTeX
There is no `q' in this sentence.
\par ``Talk, child,'' said the Unicorn.
\par She said, ``\thinspace`Enough!', he said.''
}
\noindent
第三个输入行中的 |\thinspace| 避免单引号和双引号太过靠近。
否则,你将看到三个并排的间隔相近的引号。

%\TeX\ has three kinds of ^{dashes}:
%\ulist\compact
%\li Short ones (hyphens) like this ( - ). You get them by typing~`^|-|'.
%\li Medium ones (en-dashes) like this ( -- ). You get them by typing~`^|--|'.
%\li Long ones (em-dashes) like this ( --- ). You get them by typing~`^|---|'.
%\endulist
%\noindent
%Typically you'd use hyphens to indicate compound words like
%``will-o'-the-wisp'',
%en-dashes to indicate
%page ranges such as ``pages~81--87'', and em-dashes to indicate
%a break in continuity---like this.
在 \TeX\ 中有三种^{横线号}(dash):
\ulist\compact
\li 短横线即连字号(hyphen)如这样( - )。键入~`^|-|'~可以得到。
\li 中横线即连接号(en-dash)如这样( -- )。键入~`^|--|'~可以得到。
\li 长横线即破折号(em-dash)如这样( --- )。键入~`^|---|'~可以得到。
\endulist
\noindent
一般地,你将用连字号表示像“will-o'-the-wisp”(鬼火)这样的复合词,
用连接号表示像“第~81--87~页”这样的页码起止,
而用破折号表示语气转折\null{---}像这样。
% 用 \null 和 {} 绕开 xecjk 的问题


%\subsection Special characters
\subsection 特殊字符

%Certain characters have special meaning to \TeX, so you shouldn't use them
%in ordinary text.  They are:
在 \TeX\ 中某些字符有特殊的含义,
因而不能在普通文本中使用。这些特殊字符为:

%\csdisplay
%    $  #  &  %  _  ^  ~  {  }  \
%|
%^^|$//in ordinary text|
%^^|#//in ordinary text|
%^^|&//in ordinary text|
%^^|_//in ordinary text|
%^^|^//in ordinary text|
%^^|~//in ordinary text|
%^^|%//in ordinary text|
%^^|{//in ordinary text|
%^^|}//in ordinary text|
%{\recat!ttidxref[\//in ordinary text]]
%\noindent
%In order to produce them in your typeset document,
%you need to use circumlocutions.  For the first five,
%you should instead type:
%^^|\$|
%^^|\#|
%^^|\&|
%^^|\%|
%^^|\_|
%\csdisplay
%    \$  \#  \&  \%  \_
%|
\csdisplay
    $  #  &  %  _  ^  ~  {  }  \
|
\ifoldeplain
^^|$//在普通文本中|
^^|#//在普通文本中|
^^|&//在普通文本中|
^^|_//在普通文本中|
^^|^//在普通文本中|
^^|~//在普通文本中|
^^|%//在普通文本中|
^^|{//在普通文本中|
^^|}//在普通文本中|
{\recat!ttidxref[\//在普通文本中]]
\fi
\noindent
要在排版的文档中生成它们,你需要使用间接的方法。
对前面五个字符,你需要键入:
\ifoldeplain
^^|\$|
^^|\#|
^^|\&|
^^|\%|
^^|\_|
\fi
\csdisplay
    \$  \#  \&  \%  \_
|

%\noindent
%For the others, you need something more elaborate:
\noindent
而对另外五个字符,你需要键入更多:

\csdisplay
   \^{!visiblespace}   \~{!visiblespace}   $\{$   $\}$   $\backslash$
|


%\subsection Groups
\subsection 编组

%\bix^^{groups}
%A \emph{group}
%consists of material enclosed in matching left and right braces (|{| and
%|}|).
%^^|{//starting a group|
%^^|}//ending a group|
%By placing a command within a group, you can limit its effects to
%the material within the group.
%For instance, the |\bf| command tells \TeX\ to set
%something in {\bf boldface} type.  If you were to put |\bf| into your input
%and do nothing else to counteract it, everything in your document following the
%|\bf| would be set in boldface.
%By enclosing |\bf| in a group,
%you limit its effect to the group.  For example, if you type:
%\csdisplay
%We have {\bf a few boldface words} in this sentence.
%|
%\noindent you'll get:
%\display{We have {\bf a few boldface words} in this sentence.}
\bix^^{编组}
包含在配对的左右花括号(|{| 和 |}|)中的内容组成一个\emph{编组}。
^^|{//开始编组|
^^|}//结束编组|
编组内部的命令,其作用范围被限制在该编组内部。
例如,|\bf| 命令让 \TeX\ 将某些内容用粗体显示。
如果你将 |\bf| 放在你的输入中,而没有用任何方式取消它,
|\bf| 之后的所有内容都将以粗体显示。
如果将 |\bf| 包含在一个编组中,
就可以将它的作用限制在此编组中。
比如,如果你键入:
\csdisplay
We have {\bf a few boldface words} in this sentence.
|
\noindent 你将会得到如下的结果:
\display{We have {\bf a few boldface words} in this sentence.}

%\noindent You can also use a group to limit the effect of
%an assignment to one of \TeX's parameters.
%These parameters contain values that affect how \TeX\ typesets your document.
%For example, the value of the |\parindent|
%parameter specifies the indentation at the beginning of a paragraph.
%The assignment |\parindent = 15pt|
%sets the indentation to $15$ printer's points.
%By placing this assignment at the beginning
%of a group containing a few paragraphs, you can change
%the indentation of just those paragraphs.  If you don't enclose
%the assignment in a group,
%the changed indentation will apply to the rest of the document (or up to the
%next assignment to |\parindent|, if there's a later one).
\noindent 利用编组,你可以限制 \TeX\ 某个参数的取值的作用范围。
这些参数的取值影响 \TeX\ 对文档的排版。
例如,|\parindent| 参数的值指定了段首缩进量,
赋值 |\parindent = 15pt| 将设定该缩进量为 $15$ 点。
若将该赋值放在包含几个段落的编组开始处,
你就可以仅仅改变这几个段落的段首缩进。
若不将该赋值包含在编组中,
对缩进量的修改将作用到文档的剩余部分%
(或者作用到下个对|\parindent|的赋值为止,若还有下个赋值的话)。

%\xrdef{bracegroup}
%Not all pairs of braces indicate a group.
%In particular, the braces associated with an argument for which the
%braces are \emph{not} required don't indicate a group---they just
%serve to delimit the argument.
%Of those commands that do require braces for their arguments,
%some treat the braces as defining a group
%and the others interpret the argument in some special way that depends on
%the command.\footnote
%{More precisely, for primitive commands either
%the braces define a group or they enclose tokens that aren't processed in
%\TeX's stomach.
%For |\halign| and |\valign| the group has a trivial
%effect because everything within the braces either doesn't reach the stomach
%(because it's in the template) or is enclosed in a further inner group.
%^^|\halign//grouping for|
%^^|\valign//grouping for|
%}
%\eix^^{groups}
\xrdef{bracegroup}
并非所有配对花括号都表示一个编组。
特别地,若参量\emph{不要求}有花括号,
则与该参量一起的花括号并不表示一个编组%
——它们仅用于确定该参量的界限。
对于那些参量中确实需要花括号的命令,
有些命令将花括号作为编组的定义,
而其他命令用各自的特殊方式解释该参量。\footnote
{更准确地说,对于原始命令,
花括号要么定义了一个编组,要么不会在 \TeX\ 的胃中被处理。
对于 |\halign| 和 |\valign| 命令,编组只有平凡的效果;
这是由于编组中的一切东西要么不会到达胃里(因为它在模板中),
要么包含在更深一层的编组中。
^^|\halign//阵列中的编组|
^^|\valign//阵列中的编组|
}
\eix^^{编组}


%\subsection Math formulas
\subsection 数学公式

%\bix^^{math}
%\xrdef{mathform}
%A math formula can appear in text (\emph{text math})
%^^{text math}
%or set off on a line by itself
%with extra vertical space around it (\emph{display math}).
%^^{display math}
%You enclose a text formula in single dollar signs (|$|)
%and a displayed formula in double dollar signs (|$$|).
%\ttidxref{$}\ttidxref{$$}
%For example:
\bix^^{数学}
\xrdef{mathform}
数学公式可以出现在正文中(\emph{文内公式}),
^^{文内公式}
或者出现在留出额外上下间距的单独一行中(\emph{陈列公式})。
^^{陈列公式}
文内公式两边用单个美元符(|$|)括起来,
而陈列公式两边用双个美元符(|$$|)括起来。
\ifoldeplain\ttidxref{$}\ttidxref{$$}\fi
例如:

%\csdisplay
%If $a<b$, then the relation $$e^a < e^b$$ holds.
%|
%\noindent This input produces:
%\display{\centereddisplays
%If $a<b$, then the relation $$e^a < e^b$$ holds.}
%\smallskip
%\noindent \chapterref{math} describes the commands that are useful
%in math formulas.
%\eix^^{math}
\csdisplay
If $a<b$, then the relation $$e^a < e^b$$ holds.
|
\noindent 此输入得到的结果是:
\display{\centereddisplays
If $a<b$, then the relation $$e^a < e^b$$ holds.}
\smallskip
\noindent \chapterref{math}描述了数学公式中常用的命令。
\eix^^{数学}


%\section How \TeX\ works
\section \TeX\ 如何工作

%In order to use \TeX\ effectively, it helps to
%have some idea of how \TeX\ goes about
%its activity of transmuting input into output.
%You can imagine \TeX\ as a kind of organism with ``eyes'',
%``mouth'', ``gullet'',
%``stomach'', and ``intestines''.
%Each part of the organism transforms its input in some way and passes
%the transformed input to the next stage.
为了更有效地使用 \TeX ,
对 \TeX\ 着手将输入转换为输出的活动有所了解是有益的。
你可以将 \TeX\ 想象为一种有“眼睛”、“嘴巴”、“食道”、“胃”以及“肠道”的生物体。
该生物体的每个器官都以某种方式转换它的输入,并将转换的结果送到下个器官中。

%The ^{eyes} transform an input file into a sequence of characters.
%The ^{mouth} transforms the sequence of characters into a sequence of
%\emph{tokens},
%^^{tokens}
%where each token is either a single character or a control sequence.
%^^{control sequences//as tokens}
%The gullet expands the tokens into a sequence of
%\emph{primitive commands}, which are also tokens.
%^^{expanding tokens}
%The ^{stomach} carries out the operations specified by the primitive commands,
%producing a sequence of pages.
%Finally, the ^{intestines} transform each page into the form required
%for the \dvifile\ and send it there.
%^^{\dvifile//created by \TeX's intestines}
%These actions are described in more detail
%in \chapterref{concepts} under \conceptcit{\anatomy}.
%^^{\anatomy}
^{眼睛}将输入文件转换为一系列字符。
^{嘴巴}将这串字符转换为一系列\emph{记号},
^^{记号}
其中每个记号是单个字符或者一个控制序列。
^^{控制序列//作为记号}
食道将该串记号展开为一系列\emph{原始命令}记号。
^^{展开记号}
^{胃}执行原始命令指定的操作,生成一些页面。
最后,^{肠道}将每个页面转换为 \dvifile 所需的形式并发送给它。
^^{\dvifile//由 \TeX\ 的肠道生成}
这些活动在\chapterref{concepts}的\conceptcit{\anatomy}中有更详细的描述。
^^{\anatomy}

%The real typesetting goes on in the stomach.
%The commands instruct \TeX\ to typeset such-and-such a character in
%such-and-such a font, to insert an interword space, to end a paragraph, and
%so on.
%Starting with individual typeset characters and other simple typographic
%elements, \TeX\ builds up a page ^^{pages} as a nest of
%^{boxes} within boxes within boxes \seeconcept{box}.
%Each typeset character occupies a box, and so does an entire page.
%A box can contain not just smaller boxes but also \emph{glue} ^^{glue}
%(\xref{glue}) and a few other things.
%The glue produces
%space between the smaller boxes.
%An important property of glue is that it can stretch and shrink;
%thus \TeX\ can make a box
%larger or smaller by stretching or shrinking
%the glue within~it.
实际的排版活动出现在胃中。
这些命令指导 \TeX\ 用这样或那样的字体排版这个或那个字符,
插入单词间空隔,结束一个段落,等等。
从单个排版字符及其他简单排印元素开始,
\TeX\ 将^{盒子}一层层放入其他盒子中\seeconcept{盒子},
这一整套盒子构造了一个页面^^{页面}。
每个排版字符占用一个盒子,整个页面也一样。
盒子中不仅包含更小的盒子,也包含\emph{粘连}^^{粘连}
(\xref{粘连})和一些其他东西。
粘连生成小盒子之间的间距。
粘连的一个重要属性在于它可以伸展和收缩;
因此通过伸缩盒子中的粘连,\TeX\ 可以让盒子变大或变小。

%Roughly speaking, a line is a box containing a sequence of character boxes,
%and a page is a box containing a sequence of line boxes.
%There's glue between the words of a line and between the lines of a page.
%\TeX\ stretches or shrinks
%the glue on each line so as to make the right margin
%of the page come out even and the glue on each page
%so as to make the bottom margins of different pages be equal.
%Other kinds of typographical elements can also appear in a line or in a page,
%but we won't go into them here.
粗略说来,文本行是一个包含一系列字符盒子的盒子,
而页面是一个包含一系列行盒子的盒子。
一行的各单词之间以及页面的各行之间都有粘连。
\TeX\ 伸缩各行中的粘连以让右页边对齐,
伸缩各页中的粘连以让各页的下边距相等。
其他类型的排印元素也可以出现在行内或页内,但我们不在这里讨论。

%As part of the process of assembling pages, \TeX\ needs to break paragraphs
%into lines and lines into pages.  The stomach first sees a paragraph as one
%long line, in effect.  It inserts \emph{line breaks}
%^^{line breaking}
%in order to transform
%the paragraph into a sequence of lines of the right length, performing a
%rather elaborate analysis in order to choose the set of breaks
%that makes the paragraph look best
%\seeconcept{line break}.
%The stomach carries out a similar
%but simpler process in order to transform a sequence of lines into a page.
%Essentially the stomach accumulates lines until no more lines can fit on the
%page.  It then chooses a single place to break the page, putting the lines
%before the break on the current page
%and saving the lines after the break for the
%next page \seeconcept{page break}. ^^{page breaks//inserted by \TeX's stomach}
作为组装页面过程的一部分,
\TeX\ 需要将段落分为多行,将多行组成页面。
事实上,\TeX\ 的胃刚开始将段落看成一个很长的行。
它插入\emph{断行点}以将段落分为长度合适的一些行,
^^{断行}
通过复杂的分析以找到最佳的断行点的集合\seeconcept{断行点}。
\TeX\ 的胃也执行类似但简单些的分析以将一些行组成页面。
基本上,胃累加各行直到页面上放不下更多的行。
然后它选择一个分页位置,将该分页位置之前的各行放在当前页面,
将之后的各行留给下个页面\seeconcept{分页点}。%
^^{分页//由 \TeX\ 的胃插入}

%When \TeX\ is assembling an entity from a list of items (boxes, glue, etc.),
%it is in one of six
%\emph{modes} ^^{modes} (\xref{mode}).
%The kind of entity it is assembling defines the mode that it is in.
%There are two ordinary modes: ordinary horizontal mode for assembling
%paragraphs (before they are broken into lines)
%and ordinary vertical mode for assembling pages.
%There are two restricted modes:
%restricted horizontal mode for appending items horizontally to form
%a horizontal box
%and internal vertical mode for appending items vertically to form
%a vertical box (other than a page).
%Finally, there are two math modes: text math mode for assembling math formulas
%within a paragraph and display math mode for assembling math formulas that are
%displayed on lines by themselves (see ``Math formulas'', \xref{mathform}).
在 \TeX\ 组装一个列表项(盒子,粘连等)中的某个东西时,
它总处于六种\emph{模式}^^{模式}(\xref{模式})的其中一种模式中。
它所组装的东西的类型定义了它所在的模式。
首先有两种普通模式:普通水平模式用于组装段落(在它们分为多行之前),
普通竖直模式用于组装页面。
其次有两种限制模式:受限水平模式用于水平地添加元素形成水平盒子,
内部竖直模式用于竖直地添加元素以形成竖直盒子(非页面盒子)。
最后还有两种数学模式:文内数学模式用于组装段落内部的数学公式,
陈列数学模式用于组装单独一行显示的数学公式(见“数学公式”,\xref{mathform})。


%\section New \TeX\ versus old {\TeX}
\section{新 \TeX\ 和老 \TeX}

%\xrdef{newtex}
%In 1989 Knuth made a major revision to \TeX\ in order to
%adapt it to the
%character sets needed to support typesetting for languages other than
%English.\space ^^{foreign languages}
%The revision included a few minor extra features that could be added
%without disturbing anything else.
%This book describes ``^{\newTeX}''.
%If you're still using an older version
%of \TeX\ (version $2.991$ or earlier),
%you'll want to know what features of {\newTeX} you can't use.
%The following features aren't available in the older versions:
%\ulist\compact
%\li ^|\badness| (\xref\badness)
%\li ^|\emergencystretch| (\xref\emergencystretch)
%\li ^|\errorcontextlines| (\xref\errorcontextlines)
%\li ^|\holdinginserts| (\xref\holdinginserts)
%\li ^|\language|, ^|\setlanguage|, and |\new!-lan!-guage|
%(\pp\xrefn\language, \xrefn{\@newlanguage}) ^^|\newlanguage|
%\li ^|\lefthyphenmin| and ^|\righthyphenmin| (\xref\lefthyphenmin)
%\li ^|\noboundary| (\xref\noboundary)
%\li ^|\topglue| (\xref\topglue)
%\li The |^^|$xy$ notation for hexadecimal digits (\xref{hexchars})
%\endulist
%\noindent
%We recommend that you obtain new \TeX\ if you can.
\xrdef{newtex}
在 1989 年,Knuth 对 \TeX\ 做了一次大幅修改,
使得它能够处理非英语排版所需的字符集。^^{外语}
这次修改还包括了一些额外的不干扰其它东西的小功能,
这本书介绍 ``^{\newTeX}''。
如果你仍使用旧版本的 \TeX ($2.991$ 版或以前的版本),
你可能会想知道哪些 {\newTeX} 的功能你是没法使用的。
以下这些功能无法在旧版中使用:
\ulist\compact
\li ^|\badness|(\xref\badness )
\li ^|\emergencystretch|(\xref\emergencystretch )
\li ^|\errorcontextlines|(\xref\errorcontextlines )
\li ^|\holdinginserts|(\xref\holdinginserts )
\li ^|\language|、^|\setlanguage| 和 |\new!-lan!-guage|%
(第 \xrefn\language 和 \xrefn{\@newlanguage}页)^^|\newlanguage|
\li ^|\lefthyphenmin| 和 ^|\righthyphenmin|(\xref\lefthyphenmin )
\li ^|\noboundary|(\xref\noboundary )
\li ^|\topglue|( \xref\topglue )
\li 十六进制的数字表达方式 |^^|$xy$(\xref{hexchars})
\endulist
\noindent
我们建议你尽可能地使用新版本的 \TeX 。

%\section Resources
\section 资源

%\xrdef{resources}
%A number of resources are available to help you in using \TeX.
%\texbook\ is the definitive source of information on \TeX:
\xrdef{resources}
有很多资源可以帮助你使用 \TeX ,其中 \texbook\ 是最可靠的 \TeX\ 信息来源。

%\smallskip
%{\narrower\noindent
%^{Knuth, Donald E.}, \texbook.  Reading, Mass.: Addison-Wesley, 1984.\par}
%\smallskip
%\noindent
%Be sure to get the seventeenth printing (January 1990) or later;
%the earlier printings don't cover the features of new \TeX.
\smallskip
{\narrower\noindent
^{Knuth, Donald E.}, \texbook.  Reading, Mass.: Addison-Wesley, 1984.\par}
\smallskip
\noindent
请务必使用第十七次(1990 年 1 月)及以后的印刷版本。
早先的印刷版本不包括新 \TeX\ 的许多功能。

%^{\LaTeX} is a very popular collection of commands designed to simplify the use
%of \TeX.  It is described in:
%\smallskip
%{\narrower\noindent\frenchspacing\spaceskip = 3.33pt plus 2pt minus 1.2pt
%^{Lamport, Leslie}, {\sl The \LaTeX\ Document Preparation System}.
%Reading, Mass.: Addison-Wesley, 1986.\par}
%\smallskip
%\noindent
%^{\AMSTeX} is the collection of commands adopted by the American Mathematical
%Society as a standard for submitting mathematical man\-u\-scripts
%electronically.
%It is described in:
%\smallskip
%{\narrower\noindent
%^{Spivak, Michael~D.}, {\sl The Joy of \TeX}. Providence, R.I.:
%American Mathematical Society, 1986.
%\par}
%\smallskip
%\noindent
%You can join the ^{\TUG} (TUG), which publishes a newsletter
%called {\it ^{TUGBoat}}.
%TUG is an excellent source not only for information about \TeX\ but also
%for collections of macros, including \AMSTeX.
%Its address is:
%\smallskip
%{\obeylines
%^{\TUG}
%c/o American Mathematical Society
%P.O. Box 9506
%Providence,  RI  02940
%U.S.A.
%}
%\smallskip
%\noindent
%Finally, you can obtain copies of the ^|eplain.tex| macros
%described in \chapterref{eplain} as well as the macros used in typesetting
%this book.
%They are available through the Internet network by anonymous \ftp\ from the
%following hosts:
%{\obeylines\display{\tt
%labrea.stanford.edu [36.8.0.47]
%ics.uci.edu [128.195.1.1]
%june.cs.washington.edu [128.95.1.4]}}
^{\LaTeX} 是一套为了简化 \TeX\ 的使用而设计的命令集,
它在这本书里面有介绍\footnote{译注: 这本书中介绍的 \LaTeX\ 己经过时。}:
\smallskip
{\narrower\noindent\frenchspacing\spaceskip = 3.33pt plus 2pt minus 1.2pt
^{Lamport, Leslie}, {\sl The \LaTeX\ Document Preparation System}.
Reading, Mass.: Addison-Wesley, 1986.\par}
\smallskip
\noindent
^{\AMSTeX} 是一套美国数学学会定义的提交电子数学手稿的命令标准,
它在这里有介绍:
\smallskip
{\narrower\noindent
^{Spivak, Michael D.}, {\sl The Joy of \TeX}. Providence, R.I.:
American Mathematical Society, 1986.
\par}
\smallskip
\noindent
你可以加入 ^{\TUG}(TUG),这个组织出版一本名为 {\it ^{TUGBoad}} 的通讯。
TUG 是一个很好的信息来源渠道,它同时也提供包括 \AMSTeX\ 在内的很多 \TeX\ 宏包资源。
它的地址是:
\smallskip
{\obeylines
^{\TUG}
c/o American Mathematical Society
P.O. Box 9506
Providence,  RI  02940
U.S.A.
}
\smallskip
\noindent
最后,你可以获取在\chapterref{eplain}介绍的用来排版本书的宏包 ^|eplain.tex|。
它可以通过使用 \ftp 匿名登录获取:
{\obeylines\display{\tt
labrea.stanford.edu [36.8.0.47]
ics.uci.edu [128.195.1.1]
june.cs.washington.edu [128.95.1.4]}}

%The electronic version includes additional macros
%that format input for the
%^{\BibTeX}\ computer program, written by Oren Patashnik at Stanford
%University, ^^{Patashnik, Oren}
%and print the output from that program.
%If you find bugs in the macros, or think of improvements, you can send
%electronic mail to Karl at {\tt karl@cs.umb.edu}.
电子版本还包括了为 ^{\BibTeX} 程序处理输入并排印其输出的宏——
\thinspace\BibTeX 程序是斯坦弗大学的 Oren Patashnik ^^{Patashnik, Oren} 编写的。
如果你在这个宏包中发现了错误,或者想改进它,
你可以给 Karl 发电子邮件。Karl 的电子邮箱地址是 {\tt karl@cs.umb.edu}。

%The macros are also available for US \$10.00 on $5\frac1/4$\inches\
%or $3\frac1/2$\inches\ PC-format diskettes from:
%\smallskip
%{\obeylines
%Paul Abrahams
%214 River Road
%Deerfield,  MA  01342
%\vskip\tinyskipamount
%Email: {\tt Abrahams\%Wayne-MTS@um.cc.umich.edu}
%}
%\smallskip
%\noindent
%These addresses are correct as of June 1990; please be aware that they may
%change after that, particularly the electronic addresses.
这个宏包还提供了 $5\frac1/4$\inches\ 或 $3\frac1/2$\inches\ 软盘,
你可以向以下地址邮寄 10.00 美元购买:
\smallskip
{\obeylines
Paul Abrahams
214 River Road
Deerfield,  MA  01342
\vskip\tinyskipamount
Email: {\tt Abrahams\%Wayne-MTS@um.cc.umich.edu}
}
\smallskip
\noindent
这个地址在 1990 年 6 月是正确的;请注意在这之后它可能发生改变,尤其是电子邮件地址。

\ifoldeplain\else\ifcompletebook\else
\vskip4em{\sectionfonts\leftline{本章索引}}
\readindexfile{i}
\fi\fi

\endchapter\byebye
